\usepackage{amsthm, thmtools, thm-restate, xpatch}

% No more extra space around proof env.
% https://tex.stackexchange.com/questions/232655/mysterious-vertical-space-after-theorem-proof-environments 
\let\proof\undefined
\declaretheoremstyle[%
  spaceabove=.5em,%
  headfont=\normalfont\itshape,%
  postheadspace=1em,%
  qed=\qedsymbol%
]{proofstyle} 
\declaretheorem[name={Proof},style=proofstyle,unnumbered]{proof}

\declaretheoremstyle[spaceabove=.5em,bodyfont=\normalfont]{noitalics}
\declaretheoremstyle[headfont=\normalfont\itshape,spaceabove=.5em,bodyfont=\normalfont]{noitalicsminor}

\declaretheorem[name=Theorem, refname={Theorem,Theorems}, numberwithin=section, style=noitalics]{theorem}
\declaretheorem[name=Proposition, refname={Proposition,Propositions}, sibling=theorem, style=noitalics]{proposition}
\declaretheorem[name=Property, refname={Property,Properties}, sibling=theorem, style=noitalics]{property}
\declaretheorem[name=Definition, refname={Definition,Definitions}, sibling=theorem, style=noitalics]{definition}
\declaretheorem[name=Corollary, refname={Corollary,Corollaries}, sibling=theorem, style=noitalics]{corollary}
\declaretheorem[name=Conjecture, refname={Conjecture,Conjectures}, sibling=theorem, style=noitalics]{conjecture}
\declaretheorem[name={Open Problem}, refname={Open Problem,Open Problems}, sibling=theorem, style=noitalics]{openproblem}
\declaretheorem[name=Question, refname={Question,Questions}, sibling=theorem, style=noitalics]{question}
\declaretheorem[name=Lemma, refname={Lemma,Lemmas}, sibling=theorem, style=noitalics]{lemma}
\declaretheorem[name=Example, refname={Example,Examples}, sibling=theorem, style=noitalics]{example}
\declaretheorem[name=Remark, refname={Remark,Remarks}, sibling=theorem, style=noitalics]{remark}
\declaretheorem[name=Fact, refname={Fact,Facts}, sibling=theorem, style=noitalics]{fact}
\declaretheorem[name=Claim, refname={Claim,Claims}, sibling=theorem, style=noitalicsminor]{claim}