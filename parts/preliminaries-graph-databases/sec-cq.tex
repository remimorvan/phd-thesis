\section{Relational Databases}

\subsection{Query Languages for Relational Databases}

The most common model of databases is by far that of "relational databases",
in which data is stored in \emph{tables}: an example is depicted
in \Cref{fig:relational-database-cinema}.

\begin{table}
	\centering%
	{%
	\footnotesize%
	\begin{tabular}{cccc}
		\multicolumn{4}{c}{\textsc{Movies}} \\ \toprule
		id & title & length & director \\ \midrule
		197 & Eyes Wide Shut & 159 & Stanley Kubrick \\ 
		205 & J'ai tué ma mère & 96 & Xavier Dolan \\
		304 & Amadeus & 161 & Miloš Forman \\
		321 & 120 Battements par minute & 143 & Robin Campillo \\ \bottomrule
	\end{tabular}
	\\\bigskip%
	\begin{tabular}{cc}
		\multicolumn{2}{c}{\textsc{Rooms}} \\ \toprule
		id & capacity  \\ \midrule
		1 & 108 \\
		2 & 124 \\
		3 & 96 \\
		4 & 102 \\ \bottomrule
	\end{tabular}%
	\hspace{1cm}%
	\begin{tabular}{ccc}
		\multicolumn{3}{c}{\textsc{Projections}} \\ \toprule
		movie\_id & room\_id & time \\ \midrule
		197 & 2 & 2025-03-28 14:00 \\
		205 & 3 & 2025-03-28 14:30 \\
		321 & 4 & 2025-03-28 14:30 \\
		197 & 1 & 2025-03-28 17:00 \\ \bottomrule
	\end{tabular}
}
	\caption{
		\AP\label{fig:relational-database-cinema}
		A "relational database" consisting of three tables, representing data
		stored by a cinema. (Replica of \Cref{fig:example-db-as-rel}.)
	}
\end{table}

\todo{we will not mention relational algebra, see \cite[\S\!\S~2--7]{ArenasBarceloLibkinMartensPieris2022DatabaseTheory}.}

\begin{itemize}
	\item duality w/ table of equivalence (vertex <-> variable ; disjoint union <-> disjoint conjunctions)
	\itemAP $\intro*\underlying{\gamma}$ and ""underlying graph""
	\itemAP ""relational database""
	\itemAP ""equality atom""
	\itemAP ""evaluation map@@cq"" (explain why notion is necessary by pointing to "evaluation map" for CRPQs)
	\itemAP ""disjunct""
	\itemAP ""conjunctive query evaluation""
\end{itemize}

\subsection{Static Analysis of Conjunctive Queries}

\subsection{Intermezzo: Parametrized Complexity Classes}

\begin{itemize}
	\itemAP ""FPT""
	\itemAP ""W1""
	\itemAP ""XP""
\end{itemize}

\subsection{Beyond Conjunctive Queries}