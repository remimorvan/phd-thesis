\section{Discussion}
\label{sec:algebra-discussion}

\subsection{Path-Algebras and the Lifting Theorem}

A natural next step is to generalize \Cref{quest:V-relations} by replacing
$\WellFormed$ by a fixed "regular language" $\Omega$.

\begin{question}
	Given a class of regular languages $\+V$, can we characterize the class $\+V^\Omega$
	of all languages of the form $L \cap \Omega$ for some $L \in \+V$, in a way that preserves decidability?
\end{question}

\begin{remark}
	\label{rk:no-algo-lifting}
	There is no (meta$^2$-)algorithm taking as input a "regular language" $\Omega$,
	and returning a (meta-)algorithm "st", given
	a membership algorithm for $\+V$, returns a membership algorithm
	for $\+V^\Omega$.

	Let $\+V$ be a "pseudovariety of regular languages" with
	decidable membership but undecidable separation---see
	\cite[Corollary 1.6, p.~478]{Rhode2011Pointlike} and \Cref{footnote:undecidability-separation}.
	We reduce the $\+V$-separability problem to our problem.

	Given two regular languages $L_1$ and $L_2$, to decide if they are $\+V$-separable,
	we first test if $L_1 \cap L_2 = \emptyset$: if not, we reject.
	Otherwise, we let $\Omega \defeq L_1 \cap L_2$, and ask whether
	$L_1$ is in $\+V^\Omega$.
	By definition, this happens "iff" $L_1$ can be written as
	$S \cap \Omega = S \cap (L_1 \cup L_2)$ for some $S \in \+V$.
	Since $L_1$ and $L_2$ are disjoint, the equality $L_1 = S \cap (L_1 \cup L_2)$
	is precisely equivalent to having $L_1 \subseteq S$ and $S \cap L_2 = \emptyset$.
	Hence, $L_1$ is in $\+V^\Omega$ if, and only if, $L_1$ and $L_2$ are $\+V$-separable.
\end{remark}

What this remark shows is that actually the language $\WellFormed$ is special,
in the sense that we relied on some of its specific properties to obtain the "Lifting theorem".

We claim that the construction of "synchronous algebras"
can be generalized for any $\Omega$, giving rise to the notion of ``path algebras''.%
\footnote{In short, they are the adaptation of
the free category generated by a graph to "dependent sets". See also \Cref{apdx:monads}.}
The "lifting theorem for monoids" can be shown to hold for some $\Omega$, including "well-formed words" for $n$-ary relations with $n\geq 3$, and that it cannot effectively hold for all $\Omega$.
We believe that a necessary condition for the "Lifting theorem" to hold
would be that $\Omega$ is ""fully-preordered@@lang"",
in the sense that there exists a preorder $\preccurlyeq$ on the alphabet $\Sigma$
"st" $\Omega = \{u_1 \dotsc u_n \in \Sigma^* \mid u_1 \preccurlyeq \dotsc \preccurlyeq u_n\}$.%
\footnote{The important property about these languages is that
the monad defining their path-algebras are strongly acyclic, like the
monad of \Cref{apdx:monads}.}

\subsection{Path-Algebras and Restricted Head Movements}

A natural next step would then be to study the relationship between ``path algebras'' and
Figueira \& Libkin's "$T$-controlled relations" defined
in \Cref{sec:preliminaries-automatic-structures-relations-restricted-head-movements}.
For any regular language $T \subseteq \lBrack 1,k\rBrack^*$, we claim that we can define a regular language $\Omega_T$ such that finite $\Omega_T$-path algebras exactly recognize 
"$T$-controlled relations".

\begin{restatable}{conjecture}{conjAdjunctionControlledRelations}
	\label{conj:adjunction-controlled-relations}
	There is a way of defining a map $T \mapsto \Omega_T$ "st"
	for any regular languages $T_1, T_2 \subseteq \lBrack 1,k \rBrack^*$,
	"$T_1$-controlled relations" are included in "$T_2$-controlled relations" if, and only if,
	there is an adjunction from the category of $\Omega_{T_1}$-path algebras to the category of
	$\Omega_{T_2}$-path algebras.
\end{restatable}

% \subsection{Beyond Synchronous Relations: Path algebras \& the Figueira-Libkin Synchronization Problem}
% \label{sec:discussion-path}

% The construction of "synchronous algebras"
% can be generalized to any "type" system defined by a finite graph
% giving rise to the notion of ``path algebras''\footnote{In short, they are the adaptation of
% the free category generated by a graph to "dependent sets". See also \Cref{apdx:monads}.},
% and the "syntactic morphism theorem@@sync",
% the "Eilenberg correspondence theorem@@sync" and the "Reiterman-style characterization of 
% pseudovarieties@@sync" hold. Similarly, the "lifting theorem for monoids"
% can be proven provided that the graph defining the type system is acyclic.
% In particular, this extends the results of this paper to
% "synchronous relations" of arbitrary arity,
% or to co-synchronous ("aka" right synchronous) relations.

% In \cite[\S 3]{FigueiraLibkin2015SynchronizingRelations}, Figueira and Libkin introduced a ``systematic way of defining classes of relations on words'', which generalized the classes of "recognizable relations", "synchronous relations", co-syn\-chronous relations or "rational relations".
% % Rational $k$-ary relations can be defined as the "relations" recognized by a $k$-tape automata, with
% % $k$ independent heads (one per tape), moving from left to right. Binary "synchronous relations" of 
% % arity are obtained by adding the constraint that the moves of the two heads must alternate, except 
% % if you of the two heads reached the end: in other words, the sequence of movements of heads must 
% % belong to $(12)^*(1^*+2^*)$.
% Any regular language $L$ over $\lBrack 1,k \rBrack$ 
% gives rise to a class of relations, called ``$L$-controlled''. They showed that, given $L \subseteq \{1,2\}^*$, it is decidable if $L$-controlled relations correspond exactly to "recognizable 
% relations" (resp.~"synchronous relations", resp.~"rational relations")
% \cite[\S, Theorem 1, p.~6]{DescotteFigueiraPuppis2018ResynchronizingClasses}. Descotte, Figueira \& Puppis then showed
% that the problem of given $K, L \subseteq \{1,2\}^*$, it was decidable if all $K$-controlled relations are also $L$-controlled relations \cite[Corollary 18, p.~12]{DescotteFigueiraPuppis2018ResynchronizingClasses}. The case of 
% $k$-ary relations for $k > 2$ was left wide open \cite[\S 9, p.~13]{DescotteFigueiraPuppis2018ResynchronizingClasses}: no 
% simple characterization (even without putting any decidability constraint) is known.

% For any regular language $L \subseteq \lBrack 1,k\rBrack^*$, we claim that we can define a type system $\+T_L$ such that finite $\+T_L$-path algebras exactly recognize 
% $L$-controlled relations.

% \begin{question}
% 	Is it true that for any regular languages $K, L \subseteq \lBrack 1,k \rBrack^*$,
% 	$K$-controlled relations are included in $L$-controlled relations if, and only if,
% 	there is an adjunction from the category of $\+T_K$-path algebras to the category of
% 	$\+T_L$-path algebras?
% \end{question}

% \subsection{Beyond Synchronous Relations: Dependencies in Other Algebras}

% To our knowledge, this is the first time notion of "dependency" was introduced: we believe it
% could be useful in other settings, and more specifically whenever the algebras at hand are "typed".
% The typical example is $\omega$-semigroups, the algebras for ${\leq}\omega$-regular languages
% \cite[\S II.4, p.~92]{Perrin2004InfiniteWords}.
% Consider the stream of ${\leq}\omega$-regular languages corresponding to the infamous
% Büchi conditions
% \begin{align*}
% 	\textrm{Büchi}_{\Sigma} \defeq\; &
% 	\big\{ (L_+, L_\omega) \subseteq \Sigma^+\times \Sigma^\omega \mid L_\omega \text{ is the set of} \\
% 	& \hphantom{\big\{}\text{ $\omega$-word containing infinitely many prefixes in } L_+ \big\}.
% \end{align*}
% This does not form a pseudovariety of ${\leq}\omega$-regular languages, not even
% when considering Pin's ordered variants \cite[\S VI.2, p.~267]{Perrin2004InfiniteWords}, precisely because of the algebras inability to define
% constraints between elements of different "type".
% Another promising application of the "dependency relation" would be for 
% forest algebras \cite[\S 1.3, p.~4]{Bojanczyk2008Forest} \cite[\S 5, p.~159]{Bojanczyk2020MSO}, which have two types: one for forests and one
% for contexts (forests with holes).