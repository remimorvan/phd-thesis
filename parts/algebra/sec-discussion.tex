\section{Discussion}
\label{sec:algebra-discussion}

\subsection{Path Algebras and the Lifting Theorem}

A natural next step is to generalize \Cref{quest:V-relations} by replacing
$\WellFormed$ by a fixed "regular language" $\Omega$.

\begin{question}
	Given a class of regular languages $\+V$, can we characterize the class $\+V^\Omega$
	of all languages of the form $L \cap \Omega$ for some $L \in \+V$, in a way that preserves decidability?
\end{question}

\begin{remark}
	\label{rk:no-algo-lifting}
	There is no (meta$^2$-)algorithm taking as input a "regular language" $\Omega$,
	and returning a (meta-)algorithm "st", given
	a membership algorithm for $\+V$, returns a membership algorithm
	for $\+V^\Omega$.

	Let $\+V$ be a "pseudovariety of regular languages" with
	decidable membership but undecidable separation---see
	\cite[Corollary 1.6, p.~478]{Rhode2011Pointlike} and \Cref{footnote:undecidability-separation}.
	We reduce the $\+V$-separability problem to our problem.

	Given two regular languages $L_1$ and $L_2$, to decide if they are $\+V$-separable,
	we first test if $L_1 \cap L_2 = \emptyset$: if not, we reject.
	Otherwise, we let $\Omega \defeq L_1 \cap L_2$, and ask whether
	$L_1$ is in $\+V^\Omega$.
	By definition, this happens "iff" $L_1$ can be written as
	$S \cap \Omega = S \cap (L_1 \cup L_2)$ for some $S \in \+V$.
	Since $L_1$ and $L_2$ are disjoint, the equality $L_1 = S \cap (L_1 \cup L_2)$
	is precisely equivalent to having $L_1 \subseteq S$ and $S \cap L_2 = \emptyset$.
	Hence, $L_1$ is in $\+V^\Omega$ if, and only if, $L_1$ and $L_2$ are $\+V$-separable.
\end{remark}

What this remark shows is that actually the language $\WellFormed$ is special,
in the sense that we relied on some of its specific properties to obtain the "Lifting theorem".

We claim that the construction of "synchronous algebras"
can be generalized for any $\Omega$, giving rise to the notion of ``path algebras''.%
\footnote{In short, they are the adaptation of
the free category generated by a graph to "dependent sets". See also \Cref{apdx:monads}.}
The "lifting theorem for monoids" can be shown to hold for some $\Omega$, including "well-formed words" for $n$-ary relations with $n\geq 3$, and that it cannot effectively hold for all $\Omega$.
We believe that a necessary condition for the "Lifting theorem" to hold
would be that $\Omega$ is ""fully-preordered@@lang"",
in the sense that there exists a preorder $\preccurlyeq$ on the alphabet $\Sigma$
"st" $\Omega = \{u_1 \dotsc u_n \in \Sigma^* \mid u_1 \preccurlyeq \dotsc \preccurlyeq u_n\}$.%
\footnote{The important property about these languages is that
the monad defining their path-algebras are strongly acyclic, like the
monad of \Cref{apdx:monads}.}

\subsection{Path Algebras and Restricted Head Movements}

A natural next step would then be to study the relationship between ``path algebras'' and
Figueira \& Libkin's "$T$-controlled relations" defined
in \Cref{sec:preliminaries-automatic-structures-relations-restricted-head-movements}.
For any regular language $T \subseteq \intInt{1,k}^*$, we claim that we can define a regular language $\Omega_T$ such that finite $\Omega_T$-path algebras exactly recognize 
"$T$-controlled relations".

\begin{restatable}{conjecture}{conjAdjunctionControlledRelations}
	\AP\label{conj:controlled-relations-category-equivalence}
	There is a way of defining a map $T \mapsto \Omega_T$ "st"
	for any regular languages $T_1, T_2 \subseteq \intInt{1,k}^*$,
	"$T_1$-controlled relations" are included in "$T_2$-controlled relations" if, and only if,
	there is an adjunction from the category of $\Omega_{T_1}$-path algebras to the category of
	$\Omega_{T_2}$-path algebras.
\end{restatable}

% \subsection{Separability by Recognizable Relations}
% \label{sec:algebra-discussion-recognizable-pseudovar}

% The algebraic framework developed here could provide a foundation
% to tackle the "$\AUT$/$\REC$-separability problem",
% provided that $\REC$ has some nice closure properties.
% We start that, unfortunately, by showing that $\REC$ is not a "pseudovariety of automatic relations".

% \begin{proposition}
% 	The class of "recognizable relations" is not closed under "residuals".
% \end{proposition}

% Before proving this point, we actually want to point out that it is less trivial
% than one could \emph{a priori} expect: of course, with a naive definition of "residual",
% we would get that $\proj{a^* \times a^*}\cdot \residual{\pair{a}{a}}$ equals $\proj{\{\tup{a^n, a^n} \mid n\in\N\}}$ which is not "recognizable@@rel". However, using the proper definition of "residuals",
% it turns out that $\proj{a^* \times a^*}\cdot \residual{\pair{a}{a}} = \proj{a^* \times a^*}$
% is a "recognizable relation"…

% \begin{proof}
% 	\todo{todo!}
% \end{proof}

\subsection{Theorem Projection via Monad Adjunction}
\label{sec:theorem-projection-monad-adjunction}

In fact, \Cref{conj:controlled-relations-category-equivalence}
is part of a larger idea we have, about relating the expressiveness of
universe---formalized by a monad---and its categorical properties. 
In algebraic language theory, monads play an essential role---see \Cref{apdx:monads}.
Informally, a monad describes a universe.
For instance, the monad
\[
	\Sigma \mapsto \Sigma^*
\]
describes the universe of finite words, and gives rise to the notion of
"monoids". Similarly, weighted words over a field $\symbb{K}$ can be described by
\[
	\Sigma \mapsto \symbb{K}\Sigma^*,
\]
where $\symbb{K}X$ denotes the free vector space over set $X$.
Again, this monad gives rise to Reutenauer's ``algèbres associatives''
\cite{Reutenauer1980SeriesFormelles}, "aka" ``$\symbb{K}$-weighted monoids'',
which is essentially an algebraic structure that is both a monoid and a
$\symbb{K}$-vector space.
Our "synchronous algebras" are no exception---see \Cref{apdx:monads}---,
and we refer the reader to \cite[\S~4]{Bojanczyk2020MSO} for more examples.

For well-behaved monads, ``languages'' recognized by their finite algebras
coincide with "monadic second-order"-definable languages.%
\footnote{We refer here to \emph{external} logics in the sense of
\Cref{rk:internal-vs-external-logic}.}
This gives rise to a natural set of questions, consisting
in finding algebraic characterizations
of fragments of "monadic second-order logic".
The perhaps most celebrated example is that of
the Schützenberger-McNaughton-Papert theorem,
which proves that "first-order definable" languages of finite words
are exactly those recognized by aperiodic monoids
\cite{Schutzenberger1965FiniteMonoidsAperiodic,McNaughtonPapert1971CounterFreeAutomata}.
The statement was extended to $\omega$-words \cite{Perrin1984Recent},
to countable ordinal words \cite{Bedon2001Logic},
and to countable scattered words \cite{BesCarton2011AlgebraicFO}.
In the latter case, asking only for aperiodicity is not enough:
the algebra should also be ``gap-insensitive''.%
\footnote{In the more general case of words over countable orderings,
Colcombet and Sreejith also provides a characterization for "first-order"
definable languages, although it is more complex \cite{ColcombetSreejith2015Limited}.
See also \cite{ColcombetSreejith2025RegularExpressionsCountableWords} for other
classes of languages.}
This can be explained as follows: by going from the simpler setting
of finite words (or countable ordinal words) to countable scattered words,
there are new phenomena that appear and that "first-order logic" is 
not able to describe.

However, intuitively, going in the converse direction should be easy!
Given an algebraic characterization of first-order logic over a ``richer'' monad
("eg" countable scatter words), it should be easy to reprove the
Schützenberger-McNaughton-Papert theorem over finite words, using the
``rich'' algebraic characterization as a black box.

\begin{marginfigure}
	\centering
	\begin{tikzpicture}
		\node at (0,0) (base) {\small``base monad''};
\node[right=3em of base] (rich) {\small``richer monad''};

\draw[edge] (base) to[bend left=30] node[midway, above] {\small encoding} (rich);
\draw[edge] (rich) to[bend left=30] node[midway, below] {\small projection} (base);
	\end{tikzpicture}
	\caption{
		\AP\label{fig:adjunction-informal}
		Tools to project algebraic characterizations from one universe to another.
	}
\end{marginfigure}
Surprisingly, while it is easy to ``project'' the algebraic characterization
of first-order logic over $\omega$-words to get its counterpart for finite words,
the task becomes less straightforward---but still possible---when projecting from
countable ordinal words or countable scattered words to finite ones.
In both cases, the proof relies on two ingredients:
\begin{itemize}
	\item a function, encoding a language of finite words into
		a language in the richer setting;
	\item a function, projecting a language in the richer setting
		to a language of finite words,
\end{itemize}
see \Cref{fig:adjunction-informal}.
Moreover, both functions should preserve the properties that we are studying---either
being "first-order" definable, or satisfying the algebraic characterization.

In the case of $\omega$-words, finding these functions is easy: for instance, 
the projection simply consists in restricting a language of words of
length at most $\omega$ to the finite words.
On the other hand, the projection from countable scattered words
to finite ones cannot simply do this, since finite words are not
"first-order" definable amongst countable scattered ones.%
\footnote{This can be proven by using the algebraic characterization
of "first-order definable" languages over scattered words of
Bès \& Carton\cite[Theorem~21]{BesCarton2011AlgebraicFO}.}
Hence, the projection in question needs to be tailor-made to 
preserve first-order logic.\footnote{This construction is an unpublished joint
work with Thomas Colcombet.}

In fact, not only does the projection of $\omega$-words to finite words preserve
the property of being "first-order definable", it also preserve membership in any
"pseudovariety@@reglang"! In this sense, words of length at most $\omega$
form a \emph{conservative extension} of finite words, while
countable scattered words do not: this is why we can find a single construction
to project algebraic characterizations%
\footnote{By ``algebraic characterization'' we formally mean an "Eilenberg-Schützenberger correspondence@@lang".}
from $\omega$-words to finite ones, but not from countable scattered words to finite ones.

This begs the question: how can we formalize this idea of a monad being
a \emph{conservative extension} of another one?
In light of \Cref{fig:adjunction-informal}, we propose to look at the notion of "monad adjunction".
Before introducing it, we need a preliminary definition.

Let $\Monad[S] = (S,\unit,\mult)$ be a $\cat{C}$-monad
and $\Monad[T] = (T,\unit,\mult)$ be a $\cat{D}$-monad.%
\footnote{Following Street, we use the same notation for the unit and multiplication in $\Monad[S]$ and $\Monad[T]$.}
We assume $\cat C$ and $\cat D$ to be both categories of typed sets.

A \AP""monad functor"",%
\footnote{The notion was introduced by Street \cite[\S 1]{Street1972Monads}
in the slightly more general context of 2-categories.
We found this reference thanks to \cite{Rezk2012Functors}. "Monad functors" are also known as
``lax maps of monads''.}
from $\Monad[S]$ to $\Monad[T]$ is a pair $(\Phi, F)$, where:
\begin{itemize}
\item $F\colon \cat{C} \to \cat{D}$ is a functor,
\item $\Phi\colon TF \Rightarrow FS$ is a natural transformation,
\end{itemize}
such that, for every object $X \in \cat{C}$, the following diagrams commute:\\
\begin{center}
	\intro*\textup{\MFunit:}
	\begin{tikzcd}
		FX \ar[d,"{\unit[FX]}" swap] \ar[dr, "{F\unit[X]}"] & \\
		TFX \ar[r,"\phi_{X}" swap] & FSX,\\
	\end{tikzcd}
	\quad\intro*\textup{\MFmult:}
	\begin{tikzcd}[row sep=large, column sep=-.2em]
		& & TFSX \ar[rr, "\phi_{SX}"] & &  FSSX \ar[drr, "{F\mult[X]}", bend left=0] \\
		TTFX \ar[urr, "T\phi_X", bend left=0] \ar[drrr, "{\mult[FX]}" swap, bend right=0] & &
			& & & & FSX. \\[-.5em]
		& & & TFX \ar[urrr, "\phi_X" swap, bend right=0]
	\end{tikzcd}
\end{center}

Such a pair is denoted by $\intro*\MF{\Phi}{F}\colon \Monad[S] \Rightarrow \Monad[T]$.%
\footnote{Note that, somewhat surprisingly, one element of a monad functor
from $\Monad[S]$ to $\Monad[T]$ is covariant---namely the functor
$F\colon \cat{C} \to \cat{D}$---while the other one is contravariant---namely
the natural transformation $\Phi\colon TF \Rightarrow FS$. The next example will
illustrate why this definition makes sense.}
% \begin{marginfigure}
% 	\centering
% 	\intro*\textup{\MFunit:}
% 	\begin{tikzcd}
% 		FX \ar[d,"\unit_{FX}" swap] \ar[dr, "F\unit_{X}"] & \\
% 		TFX \ar[r,"\phi_{X}" swap] & FSX.\\
% 	\end{tikzcd}\\
% 	\intro*\textup{\MFmult:}
% 	\begin{tikzcd}[row sep=large, column sep=0em]
% 		& & TFSX \ar[rr, "\phi_{SX}"] & &  FSSX \ar[drr, "F\mult_X", bend left=0] \\
% 		TTFX \ar[urr, "T\phi_X", bend left=0] \ar[drrr, "\mult_{FX}" swap, bend right=0] & &
% 			& & & & FSX. \\
% 		& & & TFX \ar[urrr, "\phi_X" swap, bend right=0]
% 	\end{tikzcd}
% 	\caption{
% 		\AP\label{fig:axioms-monad-functor}
% 		Axioms of a "monad functor".}
% \end{marginfigure}

\begin{marginfigure}
	\centering
	\begin{tikzpicture}
		\node at (0,0) (base) {$\MonadPlus$};
\node[right=3em of base] (rich) {$\MonadWilke$};

\draw[edge] (base) to[bend left=30] node[midway, above] {$\MFzero$} (rich);
\draw[edge] (rich) to[bend left=30] node[midway, below] {$\MFproj$} (base);
	\end{tikzpicture}
	\caption{
		\AP\label{fig:adjunction-informal-finite-wilke}
		"Monads functors" between "Kleene's monad" and "Wilke's monad".
	}
\end{marginfigure}
\begin{example}[Finite words and $\omega$-words]
	\AP\label{ex:finite-word-to-omega-word}
	Let $\intro*\MonadPlus$ be ""Kleene's monad"" over the category $\Set[]$
	of non-empty finite words,
	defined by $X \mapsto X^+$. Moreover, let $\intro*\MonadWilke$ be ""Wilke's monad"",
	defined over the category of "2-typed sets" $\Set[+,\omega]$, by
	$\tup{F,I} \mapsto \tup{F^+, F^+I \cup F^{\omegafin}}$, where \AP$F^{\intro*\omegafin}$
	denotes the set of ultimately-periodic words of length $\omega$ over $F$.

	We start by defining a "monad functor" from $\MonadWilke$
	to $\MonadPlus$: it represents the natural surjection
	from words of length at most $\omega$ to non-empty finite words.
	Let $\intro*\projMon\colon \Set[+,\omega] \to \Set$ be the functor that only keeps the first component, defined by $\tup{F,I} \mapsto F$, and $\Phi$ be the natural transformation
	$\MonadPlus\projMon \Rightarrow \projMon\MonadWilke$, defined for $\tup{F,I} \in \Set[+,\omega]$
	by letting
	\[
		\begin{tikzcd}[column sep=small]
			\phi_{F,I}\colon
			\underbrace{\MonadPlus\projMon \tup{F,I}}_{= F^+} \rar &
			\underbrace{\projMon\MonadWilke \tup{F,I}}_{= F^+}
		\end{tikzcd}
	\]
	be the identity. Then $\intro*\MFproj$ is a "monad functor" from
	"Wilke's monad" to "Kleene's monad".

	Dually, we then define a "monad functor" from $\MonadPlus$
	to $\MonadWilke$ that corresponds to the
	natural embedding of non-empty finite words into words of length at most $\omega$.
	We let \AP$\intro*\zero\colon \Set[] \to \Set[+,\omega]$ be the functor that transforms a
	set $X$ into a pair $\tup{X,\set{\bullet}}$, and $\Psi$ be the natural transformation
	$\MonadWilke\zero \Rightarrow \zero\MonadPlus$, defined for $X \in \Set[]$ by letting
	\[
		\begin{tikzcd}[column sep=small]
			\psi_{X}\colon
			\underbrace{\MonadWilke\zero X\vphantom{\MonadPlus}}%
				_{=\tup{X^+, X^+\set{\bullet} \cup X^{\omegafin}}} \rar &
			\underbrace{\zero\MonadPlus X}%
				_{=\tup{X^+, \set{\bullet}}}
		\end{tikzcd}
	\]
	send $u \in X^+$ to itself, and $u \in X^+\set{\bullet} \,\cup\, X^{\omegafin}$ to $\bullet \in \set{\bullet}$. Then $\intro*\MFzero$ is a "monad functor" from
	"Kleene's monad" to "Wilke's monad", see \Cref{fig:adjunction-informal-finite-wilke}.
\end{example}

Unsurprisingly, "monad functors" can be composed. Moreover, 
it can be shown that any "monad functor" from $\Monad[S]$ to $\Monad[T]$ induces
a functor from the category of $\Monad[S]$-algebras to $\Monad[T]$-algebras.
For instance, $\MFproj\colon \MonadWilke \Rightarrow \MonadPlus$ induces
the functor from Wilke's algebras to semigroups that sends $\tup{S^+,S^\omega}$
to $S^+$.
Dually, $\MFzero\colon \MonadPlus \Rightarrow \MonadWilke$ induces the functor
from semigroups to Wilke's algebras that sends $S$ to $\tup{S, \set{\bullet}}$.

We conjecture that we can naturally define a notion of ``monad (functor) adjunction'',
which generalizes the notion of ``functor adjunctions'',
in such a way that a \AP""monad adjunction"" between $\Monad[S]$ and $\Monad[T]$
induces an adjunction between $\Monad[S]$-algebras and $\Monad[T]$-algebras.
A typical example of such a "monad adjunction" would be
\Cref{fig:adjunction-informal-finite-wilke}.

\begin{marginfigure}
	\centering
	\begin{tikzpicture}
		\node[font=\footnotesize] at (0,0) (base) {semigroups};
\node[right=3em of base, font=\footnotesize] (rich) {Wilke's algebras};

\draw[edge] (base) to[bend left=30] node[midway, above, font=\footnotesize] {$R \mapsto \tup{R,\set{\bullet}}$} (rich);
\draw[edge] (rich) to[bend left=30] node[midway, below, font=\footnotesize] {$\tup{L^+, L^\omega} \mapsto L^+$} (base);
	\end{tikzpicture}
	\caption{
		\AP\label{fig:adjunction-formal-finite-wilke}
		An adjunction between the category of
		$\MonadWilke$-algebras and the category of $\MonadPlus$-algebras.
	}
\end{marginfigure}
``Monad (functor) adjunctions'' are high-order objects, and so, for the sake
of simplicity, we will work with the adjunction they induce between the categories
algebras. Going back to \Cref{ex:finite-word-to-omega-word,fig:adjunction-informal-finite-wilke},
we represent in \Cref{fig:adjunction-formal-finite-wilke} the two functors induced
between their Eilenberg-Moore categories.
It is indeed an adjunction, with $\tup{L^+, L^\omega} \mapsto L^+$ acting as the left
adjoint and $R \mapsto \tup{R,\set{\bullet}}$ as the right adjoint, and there
are \emph{natural} bijections
\[
	\textrm{hom}_{\cat{Sgp}}(L^+, R)
	\cong \textrm{hom}_{\cat{Wilke}}(\tup{L^+,L^\omega}, \tup{R,\bullet}).
\]

We believe that this adjunction is precisely what allows us to automatically project
any algebraic characterization of $\omega$-regular languages to an algebraic
characterization of "regular languages", for the good reason that
the two maps $\tup{L^+, L^\omega} \mapsto L^+$ and $R \mapsto \tup{R,\set{\bullet}}$
are the key ingredients of the proof!

Interestingly, sending a semigroup $S$ to the ordinal semigroup
$S \dcup \{\textbullet\}$, defined by letting any infinite product
be equal to the zero element $\textbullet$ also defines a functor
from the category of semigroups to the category of ordinal semigroups.
Dually, the canonical surjection sending an ordinal semigroup to
its underlying semigroup is a functor from
the category of ordinal semigroups to the category of semigroups.
However, they do not form an adjunction!
We actually believe that there are no adjunction between these categories.

These observations lead us to the last conjecture of this thesis.
\begin{restatable}[Projecting Algebraic Characterizations]{conjecture}{projectingAlgebraicCharacterization}
	\AP\label{conj:projecting-algebraic-characterizations}
	Let $\Monad[S]$ and $\Monad[T]$ be two monads.
	Any monad adjunction between them induces
	a ``natural function'' from "Eilenberg-Schützenberger correspondences@@lang" over $\Monad[T]$
	to "Eilenberg-Schützenberger correspondences@@lang" over $\Monad[S]$.
\end{restatable}

In the case of finite words and $\omega$-words, this
surjection sends for instance the theorem
``an $\omega$-regular language is first-order definable "iff" its
syntactic Wilke's algebra is aperiodic'' onto the theorem
``a "regular language" is first-order definable "iff" its
syntactic semigroup is aperiodic''!
We believe that the "lifting theorem" (\Cref{thm:lifting-theorem-monoids-pseudovarieties})
could be a consequence of
\Cref{conj:projecting-algebraic-characterizations}.