\section{Discussion}
\label{sec:discussion}

A natural next step is to generalize \Cref{quest:V-relations} by replacing
$\WellFormed$ by a fixed regular language $\Omega$.

\begin{question}
	Given a class of regular languages $\+V$, can we characterize (and decide) all
	languages of the form $L \cap \Omega$ for some $L \in \+V$?
\end{question}

We claim that the construction of "synchronous algebras"
can be generalized for any $\Omega$, giving rise to the notion of ``path algebras''.
The "lifting theorem for monoids" can be shown to hold for some $\Omega$, including "well-formed words" for $n$-ary relations with $n\geq 3$, and that it cannot effectively hold for all $\Omega$.

A natural next step would then be to study the relationship between ``path algebras'' and
Figueira \& Libkin's $L$-controlled relations \cite[\S 3]{FigueiraLibkin2015SynchronizingRelations},
see also \cite{DescotteFigueiraPuppis2018ResynchronizingClasses}.

Lastly, it would be interesting to extend the results on algebras to automata: for instance, can
we adapt our proof to show the existence of a minimal \emph{synchronous} automaton for each relation?
% \subsection{Beyond Synchronous Relations: Path algebras \& the Figueira-Libkin Synchronization Problem}
% \label{sec:discussion-path}

% The construction of "synchronous algebras"
% can be generalized to any "type" system defined by a finite graph
% giving rise to the notion of ``path algebras''\footnote{In short, they are the adaptation of
% the free category generated by a graph to "dependent sets". See also \Cref{apdx:monads}.},
% and the "syntactic morphism theorem@@sync",
% the "Eilenberg correspondence theorem@@sync" and the "Reiterman-style characterization of 
% pseudovarieties@@sync" hold. Similarly, the "lifting theorem for monoids"
% can be proven provided that the graph defining the type system is acyclic.
% In particular, this extends the results of this paper to
% "synchronous relations" of arbitrary arity,
% or to co-synchronous ("aka" right synchronous) relations.

% In \cite[\S 3]{FigueiraLibkin2015SynchronizingRelations}, Figueira and Libkin introduced a ``systematic way of defining classes of relations on words'', which generalized the classes of "recognizable relations", "synchronous relations", co-syn\-chronous relations or "rational relations".
% % Rational $k$-ary relations can be defined as the "relations" recognized by a $k$-tape automata, with
% % $k$ independent heads (one per tape), moving from left to right. Binary "synchronous relations" of 
% % arity are obtained by adding the constraint that the moves of the two heads must alternate, except 
% % if you of the two heads reached the end: in other words, the sequence of movements of heads must 
% % belong to $(12)^*(1^*+2^*)$.
% Any regular language $L$ over $\lBrack 1,k \rBrack$ 
% gives rise to a class of relations, called ``$L$-controlled''. They showed that, given $L \subseteq \{1,2\}^*$, it is decidable if $L$-controlled relations correspond exactly to "recognizable 
% relations" (resp.~"synchronous relations", resp.~"rational relations")
% \cite[\S, Theorem 1, p.~6]{DescotteFigueiraPuppis2018ResynchronizingClasses}. Descotte, Figueira \& Puppis then showed
% that the problem of given $K, L \subseteq \{1,2\}^*$, it was decidable if all $K$-controlled relations are also $L$-controlled relations \cite[Corollary 18, p.~12]{DescotteFigueiraPuppis2018ResynchronizingClasses}. The case of 
% $k$-ary relations for $k > 2$ was left wide open \cite[\S 9, p.~13]{DescotteFigueiraPuppis2018ResynchronizingClasses}: no 
% simple characterization (even without putting any decidability constraint) is known.

% For any regular language $L \subseteq \lBrack 1,k\rBrack^*$, we claim that we can define a type system $\+T_L$ such that finite $\+T_L$-path algebras exactly recognize 
% $L$-controlled relations.

% \begin{question}
% 	Is it true that for any regular languages $K, L \subseteq \lBrack 1,k \rBrack^*$,
% 	$K$-controlled relations are included in $L$-controlled relations if, and only if,
% 	there is an adjunction from the category of $\+T_K$-path algebras to the category of
% 	$\+T_L$-path algebras?
% \end{question}

% \subsection{Beyond Synchronous Relations: Dependencies in Other Algebras}

% To our knowledge, this is the first time notion of "dependency" was introduced: we believe it
% could be useful in other settings, and more specifically whenever the algebras at hand are "typed".
% The typical example is $\omega$-semigroups, the algebras for ${\leq}\omega$-regular languages
% \cite[\S II.4, p.~92]{Perrin2004InfiniteWords}.
% Consider the stream of ${\leq}\omega$-regular languages corresponding to the infamous
% Büchi conditions
% \begin{align*}
% 	\textrm{Büchi}_{\Sigma} \defeq\; &
% 	\big\{ (L_+, L_\omega) \subseteq \Sigma^+\times \Sigma^\omega \mid L_\omega \text{ is the set of} \\
% 	& \hphantom{\big\{}\text{ $\omega$-word containing infinitely many prefixes in } L_+ \big\}.
% \end{align*}
% This does not form a pseudovariety of ${\leq}\omega$-regular languages, not even
% when considering Pin's ordered variants \cite[\S VI.2, p.~267]{Perrin2004InfiniteWords}, precisely because of the algebras inability to define
% constraints between elements of different "type".
% Another promising application of the "dependency relation" would be for 
% forest algebras \cite[\S 1.3, p.~4]{Bojanczyk2008Forest} \cite[\S 5, p.~159]{Bojanczyk2020MSO}, which have two types: one for forests and one
% for contexts (forests with holes).