\section{Preliminaries}
\label{sec:preliminaries}

\subsection{Automata \& Relations}

We assume familiarity with basic algebraic language theory over finite words, see \cite[\S 1, 2, 4, pp.~3--66 \& pp.~107--156]{Bojanczyk2020MSO} for a succinct and monad-driven approach, or \cite[\S I--XIV, pp.~3--247]{Pin2022MathematicalFoundations} for a more detailed presentation of the domain.
We also refer to \cite{StraubingWeil2021Varieties} for a presentation on pseudovarieties.\footnote{``Pseudovarieties of \emph{foo}'' and ``varieties of finite \emph{foo}''---where \emph{foo} is "eg" ``groups''
or ``semigroups''---are used interchangeably in the literature.}
More precise pointers are given in \Cref{apdx:pointers-pin}.

Recall that relation is a subset of $\Sigma^*\times\Sigma^*$,
where $\Sigma$ is an alphabet---"ie" a non-empty finite set.
We define its \AP""complement@@rel"" \AP$\intro*\negrel \+R$ as the relation $\{(u,v) \in \Sigma^*\times \Sigma^* \mid (u,v) \not\in \+R\}$.
Letting
$\SigmaPair \defeq
(\Sigma \times \Sigma) \,\cup\,
(\Sigma \times \{\pad\}) \,\cup\,
(\{\pad\} \times \Sigma)$, a "synchronous automaton" is a finite-state machine with initial states, final states, and
non-deterministic transitions labelled by elements of $\SigmaPair$.
As mentioned in \Cref{ch:preliminaries-automatic-structures}, we denote by $\WellFormed$ the set of "well-formed" words over $\SigmaPair$ where the padding symbols are placed consistently, namely: if some padding symbol occurs on a tape/component, then the following symbols of this tape/component must all be padding symbols.
From this constraint, and since $\pair{\pad}{\pad} \not\in \SigmaPair$,
there can never be padding symbols on both tapes.

Note that elements of $\WellFormed$ are in natural bijection with $\Sigma^*\times\Sigma^*$---see
\Cref{fig:ex-sync-auto}.
The relation recognized by a "synchronous automaton" is the set of pairs $(u,v) \in \Sigma^*\times\Sigma^*$ such that their corresponding element in $\WellFormed$ is the label of
an accepting run of the automaton. Recall that we say that a relation is \AP"automatic@@rel" if it is recognized 
by such a machine.

\begin{remark}
	\AP\label{rk:semantic}
	Crucially, in the semantics of "synchronous automata" we \emph{never}
	try to feed them inputs where the padding symbols are not consistent: for instance, while
	\[
		\pair{aab}{b\pad a},
		\text{ or }
		\pair{aba\pad}{a\pad\pad b}	
	\]
	are sequences in $(\SigmaPair)^*$, the behaviour of a "synchronous automaton"
	on such sequences is completely disregarded to define the relation it recognizes. 
\end{remark}

We can then reformulate the definition of the semantics of a "synchronous automaton",
to make the connection with "$\+V$-relations"---see the next subsection---explicit.

\begin{fact}
	\AP\label{fact:synchronous-is-regular}
	Given a "synchronous automaton", its semantics as a "synchronous automaton"
	can be written as the intersection of its semantics as a classical automaton over $\SigmaPair$
	with $\WellFormed$.
\end{fact}

In particular a relation $\+R$ is "automatic@@rel" if, and only if, it is a regular language when seen as a subset of $(\SigmaPair)^*$.
% However, note that building a deterministic complete
% automaton over $\SigmaPair$ from a deterministic complete "synchronous automaton" over $\Sigma$ 
% does not preserve many properties of the form ``the automaton belongs to some well-behaved class $\+V$ of automata'', as we will show in \Cref{ex:group-languages}.

\subsection{Induced Relations}

Given a class $\+V$ of regular languages,
the class of \AP""$\+V$-relations"" over $\Sigma$ consists of all relations
of the form $L \cap \WellFormed$ for some $L \in \+V_{\SigmaPair}$---see \Cref{fig:projection}.\footnote{The notation $L \in \+V_{\SigmaPair}$ means that $L$ is a language over
the alphabet $\SigmaPair$. See \cite[introduction of \S{}XIII.1]{Pin2022MathematicalFoundations}
for why classes of regular languages are defined in such a way.}

\begin{figure}[tbp]
	\begin{center}
		\scalebox{1}{
		\begin{tikzpicture}[shorten >= 1pt, node distance = 2cm, on grid, baseline]
			\AP\small
			\node[state, initial left, accepting] (q0) {}; 
			\node[state] (q1) [right =of q0] {};
			\node[state, accepting] (q0') [below left = 1.5cm and 1cm of q0] {};
			\node[state, accepting] (q0'') [below right = 1.5cm and 1cm of q0] {};
			\node[state] (qbot) [below right = 1.5cm and 1cm of q0''] {};
			\path[->]
				(q0) edge[loop above] node[font=\scriptsize] {$\pair{a}{a}, \pair{b}{b}$} (q0) 
				(q0) edge[bend left=20] node[above=0pt, font=\scriptsize] {$\pair{a}{b},\pair{b}{a}$} (q1)
				(q1) edge[bend left=20] node[above=0pt, font=\scriptsize] {$\pair{a}{b},\pair{b}{a}$} (q0)
				(q1) edge[loop above] node[font=\scriptsize] {$\pair{a}{a}, \pair{b}{b}$} (q1)
				(q0) edge node[left] {$\pair{a}{\pad}, \pair{b}{\pad}$} (q0')
				(q0) edge node[right] {$\pair{\pad}{a}, \pair{\pad}{b}$} (q0'')
				(q0') edge[loop left] node[left] {$\pair{a}{\pad}, \pair{b}{\pad}$} (q0')
				(q0'') edge[loop right] node[right] {$\pair{\pad}{a}, \pair{\pad}{b}$} (q0'')
				(q0') edge[bend right=20] node[left] {$*$} (qbot)
				(q0'') edge node[right] {$*$} (qbot)
				(q1) edge[bend left=90] node[right] {$*$} (qbot)
				(qbot) edge[loop below] node[below] {$*$} (qbot);
		\end{tikzpicture}
		}
	\end{center}
	\caption{
		\AP\label{fig:min-auto}
		Minimal (deterministic complete) ``classical'' automaton for 
		the binary relation
		of pairs $(u,v)$ such that the number of $a$'s in $u_1\hdots u_k$
		and in $v_1\hdots v_k$ are the same mod $2$, where $k = \min(|u|, |v|)$,
		seen as a language over $\SigmaPair$.
		Said otherwise, this is automaton rejects exactly all words in $(\SigmaPair)^*$ which (1)
		are not the valid encoding of a pair of words and (2) are the encoding of a pair
		which does not satisfy the property above.
		Each label $*$ is defined so that the automaton is deterministic and complete.
	} 
\end{figure}
For instance, if $\+V$ is the class of all regular languages, then by
\Cref{fact:synchronous-is-regular}, "$\+V$-relations" are exactly the "automatic relations"!
However, because of
\Cref{rk:semantic}, the minimal automaton for a relation, seen as a language over $\SigmaPair$,
can be significantly more complex than a deterministic complete "synchronous automaton" recognizing it, see \Cref{fig:min-auto} in page~\pageref{fig:min-auto}---while the size blow-up is only polynomial, it breaks many of the structural properties of the automaton, such as the property of being a permutation automaton.

Note that if $\+R$ belongs to $\+V$ when $\+R$ is seen as a language over $\SigmaPair$,
then $\+R$ is a "$\+V$-relation".
The converse implication holds under some strong assumption on $\+V$ (\Cref{fact:a-triviality-on-trivial-relations}),
but is not true in general (\Cref{ex:group-languages}).

\begin{fact}
	\AP\label{fact:a-triviality-on-trivial-relations}
	If $\+V$ is a class of languages closed under intersection and that contains $\WellFormed$, then
	a relation $\+R$ is a "$\+V$-relation" if, and only if, it belongs to $\+V$ when seen
	as a language over $\SigmaPair$.
\end{fact}

Classes of languages $\+V$ satisfying the previous assumption ("eg" first-order definable languages, piecewise-testable languages, etc.) are easy to capture when
it comes to "$\+V$-relations" since this class reduces to $\+V$-languages.
So, in the remaining of the paper, we will focus on classes $\+V$ which do not satisfy
the assumptions of \Cref{fact:a-triviality-on-trivial-relations}, such as group languages.

\begin{example}[Group relations]
	\AP\label{ex:group-languages}
	If $\+V$ is the class
	of group languages, namely languages recognized by permutation automata\footnote{A permutation 
	automaton is a finite-state deterministic complete automaton whose transition functions are all 
	permutations of states.} or equivalently by a finite group, then we call
	"$\+V$-relations" \AP``""group relations""''. They can be characterized 
	as relations recognized by permutation "synchronous automata". For instance, the relation
	of \Cref{fig:ex-sync-auto} is a "group relation" as witnessed by the permutation synchronous automaton of \Cref{fig:ex-sync-auto}. Note however that it is not a group language, when seen as a language over $\SigmaPair$, since its minimal automaton over $\SigmaPair$ is not
	a permutation automaton, see \Cref{fig:min-auto} on \cpageref{fig:min-auto}.
\end{example}

\begin{fact}
	\AP\label{fact:separability}
	Given a relation $\+R$ and a class $\+V$ of languages, the following are equivalent:
	\begin{enumerate}
		\item $\+R$ is a "$\+V$-relation";
		\item $\+R$ and $\negrel \+R$ are $\+V$-separable as languages over $\SigmaPair$,
		"ie" there is a language in $\+V$ which contains $\+R$ and does not intersect $\negrel \+R$.
	\end{enumerate}
\end{fact}

\begin{proof}
	By definition, see \Cref{fig:projection}, on page \pageref{fig:projection}.
\end{proof}

And so, if the $\+V$-separability problem is decidable, then the class of "$\+V$-relations"
is decidable. However, there are pseudovarieties $\+V$ with decidable membership but 
undecidable separability problem \cite[Corollary 1.6, p.~478]{Rhode2011Pointlike}.%
\footnote{\AP\label{footnote:undecidability-separation}The paper cited only claims undecidability of pointlikes, but it was noted in \cite[\S 1, pp.~1--2]{Gool2019Pointlike} that undecidability of the 2-pointlikes also holds, which is a problem 
equivalent to separability by \cite[Proposition 3.4, p.~6]{Almeida1999Algorithmic}.}
Moreover, some of these classes do not contain $\WellFormed$ \cite[Corollary 1.7, p.~478]{Rhode2011Pointlike}. But beyond 
this, even when a separation algorithm exists, it can be conceptually much harder than its
membership counterpart: for instance, deciding membership for group languages is trivial---it boils down to checking if a monoid is a group---, yet the decidability of the 
separation problem for group languages is considered to be one of the major results in semigroup theory:
it follows from Ash's infamous type II theorem \cite[Theorem 2.1, p.~129]{Ash1991Inevitable}, see \cite[Theorem 1.1, p.~3]{Henckell1991Ash} for a presentation of the result in terms of pointlike sets, see also \cite[\S III, Theorem 8, p.~5]{PlaceZeitoun2023Group} for an elegant automata-theoretic reformulation.

% \begin{example}[Nilpotent relations]
% 	\AP\label{ex:nilpotent}
% 	Say that a finite semigroup $S$ is \AP""nilpotent"" if it contains a unique "idempotent" element 
% 	$e$, and this "idempotent" is a zero---"ie" $ex = e = xe$ for all $x\in S$. Equationally, "nilpotent 
% 	semigroups", whose class is denoted by \AP$\intro*\NilS$,
% 	are defined by the "profinite equalities" $x^{\idem} y = x^{\idem}$ and
% 	$y x^{\idem} = x^{\idem}$. It is well known \cite[Proposition XI.4.14, p.~196 \& Proposition XIV.1.18, p.~238]{Pin2022MathematicalFoundations} that $\NilS$ corresponds
% 	to the "$+$-pseudovariety@@lang" \AP$\intro*\Nil$ of regular languages which are either finite 
% 	or cofinite. It is routine to check that "$\Nil$-relations" consists of relations
% 	$\+R$ which are finite, or such that $\negrel \+R$ is finite. On the other hand,
% 	note that if $\negrel \+R$ is finite, then $\+R$ is not cofinite as a subset of $(\SigmaPair)^+$,
% 	but only as a subset of $\WellFormed$. In particular, in this case, the syntactic semigroup of
% 	$\+R$, seen as a subset of $(\SigmaPair)^+$ will not be "nilpotent".
% \end{example}