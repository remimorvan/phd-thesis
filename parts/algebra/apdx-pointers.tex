\section{Pointers to Notions of Algebraic Language Theory}
\label{apdx:pointers-pin}

\begin{itemize}
	\itemAP ""idempotent"" elements:
		see \cite[\S II.1.2, p.~14]{Pin2022MathematicalFoundations};
	\itemAP semigroup division and ""monoid division"":
		see \cite[\S II.3.3, p.~21]{Pin2022MathematicalFoundations};
	\itemAP ""pseudovarieties of semigroups"" and ""pseudovarieties of monoids"":
		see \cite[\S XI.1, p.~189]{Pin2022MathematicalFoundations} under the name ``variety'';
	\itemAP ""$\ast $-pseudovarieties of regular languages""
		see \cite[\S XIII.3, p.~226]{Pin2022MathematicalFoundations};
	\itemAP ""$+$-pseudovarieties of regular languages"":
		see \cite[\S XIII.4, ``Eilenberg’s $+$-varieties'', p.~229]{Pin2022MathematicalFoundations};
	\itemAP ""Eilenberg correspondence@@lang"":
		see \cite[Theorem XIII.4.10, p.~228]{Pin2022MathematicalFoundations};
	\itemAP profinite topology and the $\intro*\profSg{-}$ notation:
		see \cite[\S X.2, p.~178]{Pin2022MathematicalFoundations};
	\itemAP ""profinite equality"" and the $\intro*\profeq$ notation:
		see \cite[\S XI.3, p.~193]{Pin2022MathematicalFoundations} under
		the name ``profinite identity for semigroups/monoids'';
	\itemAP ""profinite implication"" and the $\intro*\profimp$ notation:
		see \cite[\S XIII.1, p.~223]{Pin2022MathematicalFoundations};
	% \itemAP ""profinite equivalence"" and the $\intro*\profequiv$ notation:
	% 		see \cite[\S XIII.1, p.~223]{Pin2022MathematicalFoundations}.
	\itemAP ""satisfiability@@profeq"" of profinite equality:
		see \cite[\S XI.3, p.~193]{Pin2022MathematicalFoundations};
	\itemAP equations ""defining@@profeq"" a pseudovariety:
		see \cite[\S XI.3.3, p.~194]{Pin2022MathematicalFoundations}.
\end{itemize}