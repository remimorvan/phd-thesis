\section{The Lifting Theorem \& Pseudovarieties}
\label{sec:lifting-theorem}

\subsection{Elementary Formulation}

\begin{example}[Group relations: {\Cref{ex:algebra-Zpq} cont'd}]
	\AP\label{ex:charac-group-relation-monoids}
	We want to decide when the relation
	\begin{align*}
		\+R^{I,J} \defeq \big\{
			(u,v) \;\big\vert\; & |u| > |v| \text{ and } (|u| - |v| \bmod{p}) \in I, \text{ or} \\
			& |u| < |v| \text{ and } (|v| - |u| \bmod{q}) \in J.
		\hphantom{\text{ or}}\big\}	
	\end{align*}
	from \Cref{ex:algebra-Zpq} is a \AP"group relation".
	By definition this happens if and only if there exists a finite group $G$,
	together with a monoid morphism $\phi\colon (\SigmaPair)^* \to G$ and a subset
	$\Acc \subseteq G$ "st" $\forall u \in \WellFormed$, $u\in \+R^{I,J}$ "iff" $\phi(u) \in \Acc$.
	We claim:
	\begin{equation}
		\+R^{I,J} \text{ is a "group relation"}\quad\text{"iff"}\quad
		\big(
			\bar 0 \not\in I \text{ and } \bar 0 \not\in J
		\big).
	\end{equation}

	The right-to-left implication was shown in \Cref{ex:algebra-Zpq}.
	We prove the implication from left to right:
	let $n$ be the order of $G$ so that $x^n = 1$ for all $x\in G$. In particular, we have:
	$\phi\bigl(\pair{a}{\pad}^{pqn}\bigr) = 1 = \phi\bigl(\pair{a}{a}^{pqn}\bigr)$.
	Since $\phi\bigl(\pair{a}{a}^{pqn}\bigr) \not\in \+R^{I,J}$, it follows that 
	$\pair{a}{\pad}^{pqn} \not\in \+R^{I,J}$
	"ie" $\bar 0 \not\in I$. Also, $\bar 0 \not\in J$ by symmetry, which concludes the proof.
\end{example}

\AP\phantomintro{Lifting Theorem}
Even more generally, we can decide if a relation $\+R$ is a "group relation"
by simply looking at the "syntactic synchronous algebra" of $\+R$.

\liftingtheoremmonoids

\begin{proof}
	\proofcase{(1) $\Rightarrow$ (2).} Since $\+R$ is a "$\+V$-relation", there exists
	$\+L \in \+V_{\SigmaPair}$ such that $\proj{\+R} = \+L \cap \WellFormed$.
	Hence, there exists a morphism of monoids $\phi\colon (\SigmaPair)^* \to M$ such 
	that $M \in \symbb{V}$ and $\+L = \phi^{-1}[\Acc]$ for some $\Acc \subseteq M$.
	It follows that $\proj{\+R} = \+L \cap \WellFormed[\Sigma]$ rewrites as
	``for all $u \in \WellFormed$, $\phi(u) \in \Acc$ "iff" $u \in \proj{\+R}$''.
	Letting $\inducedalg{M}$ be the "synchronous algebra induced by the monoid" $M$, define
	$\psi\colon \Sync\Sigma \to \inducedalg{M}$ by $\psi(\type{u}{\sigma}) \defeq
	\type{(\phi(u))}{\sigma}$ for $\type{u}{\sigma} \in \Sync\Sigma$.
	Let $\Acc' \defeq \{\type{x}{\sigma} \mid x \in \Acc \land \sigma \in \types\}$.
	We claim that $\psi^{-1}[\Acc'] = \proj{\+R}$. Indeed, for $\type{u}{\sigma} \in \Sync\Sigma$,
	$\type{u}{\sigma} \in \proj{\+R}$ "iff" $u \in \+L$,
	"ie" $\phi(u) \in \Acc$,
	that is $\psi(\type{u}{\sigma}) = \type{(\phi(u))}{\sigma} \in \Acc'$.
	Notice then that all "underlying monoids" of $\inducedalg{M}$ are $M$,
	and hence they belong to $\symbb{V}$.

	\proofcase{(2) $\Rightarrow$ (3).} By \Cref{lem:syntactic-morphism-theorem}, 
	the "syntactic synchronous algebra" of $\+R$ "divides@@sync"
	any "algebra@@sync" $\?B$ "recognizing@@sync" $\+R$.
	In particular, its "underlying monoids" "divide@@monoid" the "underlying monoids" of $\?B$. The conclusion follows since $\symbb{V}$ is closed under "division@@monoid". 
	
	\proofcase{(3) $\Rightarrow$ (1).} Denote by $M_{\LL}, M_{\LB}$ and $M_{\BL}$ the 
	"underlying monoids" of $\SyntSA{\+R}$. Let $\Acc \subseteq \SyntSA{\+R}$ be the accepting set 
	such that $\proj{\+R} = \SyntSAM[-1]{\+R}[\Acc]$.
	Define $M \defeq M_{\LL} \times M_{\LB} \times M_{\BL}$, and
	\begin{center}
		\begin{tabular}{rccc}
			$\phi\colon$
			& $(\SigmaPair)^*$
			& $\to$
			& $M$\\
			& $\pair{a}{b}$
			& $\mapsto$
			& $\langle \SyntSAM{\+R}\type{\pair{a}{b}}{\LL},\, \type{1}{\LB},\, \type{1}{\BL} \rangle$ \\
			& $\pair{a}{\pad}$
			& $\mapsto$
			& $\langle \type{1}{\LL},\, \SyntSAM{\+R}\type{\pair{a}{\pad}}{\LB},\, \type{1}{\BL} \rangle$ \\
			& $\pair{\pad}{a}$
			& $\mapsto$
			& $\langle \type{1}{\LL},\, \type{1}{\LB},\, \SyntSAM{\+R}\type{\pair{\pad}{a}}{\BL} \rangle$
		\end{tabular}		
	\end{center}
	and finally, let
	\begin{align*}
		\Acc'
		& \cup \bigl\{
			\langle\type{1}{\LL},\, \type{1}{\LB},\, \type{z}{\BL}\rangle
			\mid \type{z}{\BL} \in \Acc
		\bigr\} \\
		& \cup \bigl\{
			\langle\type{x}{\LL},\, \type{y}{\LB},\, \type{1}{\BL}\rangle
			\mid \type{x}{\LL}\cdot \type{y}{\LB} \in \Acc
		\bigr\} \\
		& \cup \bigl\{
			\langle\type{x}{\LL},\, \type{1}{\LB},\, \type{z}{\BL}\rangle
			\mid \type{x}{\LL}\cdot\type{z}{\BL} \in \Acc
		\bigr\}.
	\end{align*}
	We first claim that
	\begin{align}
		\text{
			For every $\type{u}{\LL\to\LB} \in \Sync\Sigma$,
		} & \notag \\
		& 
		\hspace{-2cm}\text{
			$\phi(u)$ is of the form $\langle a,\, b,\, 1 \rangle$
			and moreover,
		}
		\label{eq:relationship-monoid-with-synchronous-algebra}
		\\
		& \hspace{-2cm} \text{
			$\SyntSAM{\+R}(\type{u}{\LL\to\LB}) = a \cdot b$,
		}\notag
	\end{align}
	which can trivially be proven by induction on $u$. Analogous results
	hold for words of different "type".
	We then prove that for each $\type{u}{\sigma} \in \Sync\Sigma$,
	\begin{equation}
		\SyntSAM{\+R}(\type{u}{\sigma}) \in \Acc
		\quad\text{"iff"}\quad
		\phi(u) \in \Acc'.
		\label{eq:accepting-set-is-preserved}
	\end{equation}
	The direct implication is straightforward, using
	\Cref{eq:relationship-monoid-with-synchronous-algebra}.
	The converse implication is more tricky: assume "eg" that $\sigma = \LL\to\LB$,
	say $\type{t}{\sigma} = \type{u}{\LL}\type{v}{\LB}$.
	If $\phi(t) \in \Acc'$, using \Cref{eq:relationship-monoid-with-synchronous-algebra} then it implies either that
	\begin{enumerate}
		\item $\SyntSAM{\+R}(\type{u}{\LL}) = \type{1}{\LL}$ and $\SyntSAM{\+R}(\type{v}{\LB}) = \type{1}{\LL}$, and $\type{1}{\BL} \in \Acc$, or
		\item $\SyntSAM{\+R}(\type{u}{\LL})\cdot \SyntSAM{\+R}(\type{v}{\LB}) \in \Acc$, or even
		\item $\SyntSAM{\+R}(\type{v}{\LB}) = \type{1}{\LB}$ and
			$\SyntSAM{\+R}(\type{u}{\LL})\cdot \type{1}{\BL} \in \Acc$.
	\end{enumerate}
	Clearly, $(2)$ implies the desired conclusion, namely that
	\[\SyntSAM{\+R}(\type{t}{\sigma}) = \SyntSAM{\+R}(\type{u}{\LL})\SyntSAM{\+R}(\type{v}{\LB}) \in \Acc.\]
	In all other cases, we will make heavy use of the "dependency relation".
	For case $(1)$, we have that $\SyntSAM{\+R}(\type{t}{\sigma}) = \type{1}{\LL\to\LB}$.
	From $\type{1}{\BL} \in \Acc$, \Cref{fact:closed-subset-units} yields $\type{1}{\LL} \cdot \type{1}{\LB} = \type{1}{\LL\to\LB} \in \Acc$, since $\Acc$ is "closed".
	Lastly, in case $(3)$, $\SyntSAM{\+R}(\type{u}{\LL}) \dep
	\SyntSAM{\+R}(\type{u}{\LL})\cdot \type{1}{\BL} \in \Acc$ so
	$\SyntSAM{\+R}(\type{u}{\LL}) \in \Acc$ and hence
	$\SyntSAM{\+R}(\type{t}{\LL\to\LB}) = \SyntSAM{\+R}(\type{u}{\LL}) \cdot \type{1}{\LB}
	\dep \SyntSAM{\+R}(\type{u}{\LL}) \in \Acc$ which yields
	$\SyntSAM{\+R}(\type{t}{\LL\to\LB}) \in \Acc$. This concludes the proof
	of \eqref{eq:accepting-set-is-preserved} for "type" $\sigma = \LL\to\LB$.
	Other "types" are handled similarly, and hence
	$\proj{\+R} = \phi^{-1}[\Acc']\cap\WellFormed$.
\end{proof}

\begin{remark}
	In light of \Cref{thm:lifting-theorem-monoids},
	one can wonder whether the notion of "synchronous algebra" is necessary to
	characterize "$\+V$-relations", or if it is enough to look at the languages corresponding to
	the "underlying monoids".
	Said otherwise, is the membership of $\+R$ in the class of "$\+V$-relations" uniquely
	determined by the regular languages $\+R \cap (\Sigma\times\Sigma)^*$,
	$\+R \cap (\Sigma\times\{\pad\})^*$ and $\+R \cap (\{\pad\}\times\Sigma)^*$?
	Unsurprisingly, "synchronous algebras" are indeed necessary, as 
	there are relations $\+R$ such that:
	\begin{equation}
		\AP\label{eq:naive-characterization}
		\proj{\+R} \cap (\Sigma\times \Sigma)^* \in \+V_{\Sigma\times \Sigma}, \quad
		\proj{\+R} \cap (\Sigma\times \pad)^* \in \+V_{\Sigma\times \pad}\quad\text{and}\quad
		\proj{\+R} \cap (\pad\times \Sigma)^* \in \+V_{\pad\times \Sigma},
	\end{equation}
	but $\+R$ is \emph{not} a "$\+V$-relation". This can happen even if
	$\+V$ is the "$\ast$-pseudovariety@@reglang" of all regular languages: for instance for the
	relation
	\[
		\+R \defeq \{(u,v) \mid |u| > |v| > 0 \text{ and } |u| - |v| \text{ is prime}\}.
	\]
	Notice that there is a subtle but crucially important difference between
	\eqref{eq:naive-characterization} and the second item of the "Lifting Theorem":
	while the "underlying monoids" of a "synchronous algebra" $\?A$ "recognizing@@sync" $\+R$
	only accept words of the form $(\Sigma\times \Sigma)^*$, $(\Sigma\times \pad)^*$
	or $(\pad\times \Sigma)^*$, elements of $(\Sigma\times \Sigma)^+(\Sigma\times \pad)^+$
	or $(\Sigma\times \Sigma)^+(\pad\times \Sigma)^+$ influence the "underlying monoids" of $\?A$
	via the axioms of "synchronous algebras".
\end{remark}

Also, note that the existence the "Lifting Theorem" follows from the careful
definition of "synchronous algebras": more naive definitions of these algebras
simply cannot characterize "$\+V$-relations", see \Cref{sec:counterexample}.

From \Cref{thm:lifting-theorem-monoids}
and the implicit fact that all our constructions are effective,
we obtain a decidability (meta-)result for
"$\+V$-relations".
\begin{corollary}
	\AP\label{coro:decidability}
	The class of "$\+V$-relations" has decidable membership if, and only if, $\+V$ has decidable membership.
\end{corollary}

For instance, a relation is a "group relation" if, and only if, all
"underlying monoids" of its "syntactic synchronous algebra" are
groups. 
% Beyond "group relations", the "lifting theorem" captures "eg" the case of "commutative relations".

\subsection{Synchronous Algebras Require a Dependency Relation}
\label{sec:counterexample}

In this part, we introduce the notion of "synchronous algebra" with no dependency relation, 
called "naive synchronous algebra". This notion is more natural---or naive---than
\Cref{def:synchronous-algebra}, and share some of its enjoyable properties, such as the existence of syntactic algebras. Yet, we show that these algebras cannot characterize some
natural classes of automatic relation. More precisely, we show that there
is a "$\ast$-pseudovariety of regular languages" $\+V$ and two automatic relations
$\+R_0$ and $\+R_1$, such that:
\begin{itemize}
	\item $\+R_0$ is a $\+V$-relation,
	\item $\+R_1$ is not a $\+V$-relation,
	\item $\+R_0$ and $\+R_1$ have the same syntactic "naive synchronous algebra".
\end{itemize}

\begin{definition}[Naive synchronous algebra]
	Let $\intro*\typesEps \defeq \types \cup \{1\}$. We extend the notion of "compatibility@@type" so 
	that every $\sigma \in \typesEps$ is "compatible@@type" with $1$ and
	$1$ is "compatible@@type" with $\sigma$.
	A \AP""naive synchronous algebra"" $\?A$ consists of a "$\typesEps$-typed sets",
	together with a partial binary operator $\cdot$ such that:
	\begin{itemize}
		\item $\cdot$ is defined exactly on "compatible@@type" elements and is associative, and
		\item there is a unique element of type $1$, denoted by $1$, and it satisfies
			$\type{x}{\sigma} \cdot 1 = \type{x}{\sigma} = 1 \cdot \type{x}{\sigma}$
			for all $\type{x}{\sigma} \in \?A$.
	\end{itemize}
\end{definition}

The set of all "synchronous words" is naturally a "naive synchronous algebra" under concatenation.
Moreover, any "automatic relation" admits a syntactic "naive synchronous algebra"---this can be shown in the same fashion as \Cref{lem:syntactic-morphism-theorem}.

\begin{example}[Group relations: \Cref{ex:charac-group-relation-monoids} cont'd]
	Consider the relations
	\begin{align*}
		\+R_0 \defeq \big\{
			(u,v) \;\big\vert\; & |u| > |v| \text{ and } (|u| - |v| \bmod{p}) = 0
		\big\} \\
		\+R_1 \defeq \big\{
			(u,v) \;\big\vert\; & |u| > |v| \text{ and } (|u| - |v| \bmod{p}) = 1
		\big\}.
	\end{align*}
	Then by \Cref{ex:charac-group-relation-monoids}, $\+R_0$ is not a "group relation" but
	$\+R_1$ is.
	Yet, we claim that both relations have the same syntactic "naive synchronous algebra" $\?A$, described as follows:
	\begin{itemize}
		\item it has a unit, denoted by $0$, of type $1$,
		\item $\?A_{\LL}$, $\?A_{\BL}$ and $\?A_{\LL\to\BL}$ are all reduced to a single element,
			denoted by $\type{0}{\LL}$, $\type{0}{\BL}$ and $\type{0}{\LL\to\BL}$,
		\item $\?A_{\LL\to\LB}$ and $\?A_{\LB}$ contain the elements $\ZnZ{p}$,
		\item $\cdot$ is defined as the addition over $\ZnZ{p}$, by identifying
			$\type{0}{1}$, $\type{0}{\LL}$, $\type{0}{\LL\to\BL}$ and $\type{0}{\BL}$
			with the zero of $\ZnZ{p}$.
	\end{itemize}
	Then $\+R_0$ and $\+R_1$ are the preimages of $\{\type{0}{\LL\to\LB}, \type{0}{\LB}\}$
	and $\{\type{1}{\LL\to\LB}, \type{1}{\LB}\}$, respectively, by the natural morphism onto $\?A$.
	And hence $\+R_0$ and $\+R_1$ are recognized by $\?A$. It is easy to show that it is in fact the syntactic "naive synchronous algebra" of these relations: by surjectivity of the morphism 
	above, it suffices to show that no two elements of $\?A$ can be identified and still recognize 
	the same relation.
\end{example}

And so, from this example is follows that ``being a $\+V$-relations'' cannot be characterized
by the syntactic "naive synchronous algebra" of the relation, which shows how crucial the "dependency relation" of \Cref{def:synchronous-algebra} is in order to get \Cref{thm:lifting-theorem-monoids}.

The same result can be used to prove that ``naive positive synchronous algebras''---defined analogously to "naive synchronous algebra" except that there is no "type" 1 and no unit, and hence no empty word in the free algebra---are also unable to capture the property of ``being a $\+V$-relations''.

\subsection{Pseudovarieties of Automatic Relations}
\label{sec:varieties}

We introduce the notion of "pseudovariety of synchronous algebras" 
and "$\ast$-pseudovariety of automatic relations". We show an "Eilenberg-Schützenberger correspondence@@sync" between these two notions. We then reformulate the "Lifting Theorem"
to show that any "Eilenberg-Schützenberger correspondence@@lang" between monoids and regular languages
lifts to an "Eilenberg-Schützenberger correspondence@@sync" between "synchronous algebras" and "automatic relations".

Recall that a "synchronous algebra" $\?A$ is a "quotient@@sync" of $\?B$
when there exists a surjective "synchronous algebra morphism" from $\?B$ to $\?A$.
A ""subalgebra@@sync"" of $\?B$ is any "closed subset" of $\?B$ closed under "product"
and containing the units.
We then say that "synchronous algebra" $\?A$ \AP""divides@@sync"" $\?B$
when $\?A$ is a "quotient@@sync" of a "subalgebra@@sync" of $\?B$.

Observe that $\Sync\Sigma$ admits the following property:
elements of type $\LL\to\LB$ and $\LL\to\BL$ are generated by the "underlying monoids".
Since "syntactic synchronous algebras" are homomorphic images of $\Sync\Sigma$, they also
satisfy this property. In general, we say that a "synchronous algebra" $\?A$ is \AP""locally 
generated@@sync"" if every element of type $\LL\to\LB$ (resp.~$\LL\to\BL$)
can be written as the product of an element of type $\LL$ with an element of type $\LB$ (resp.~$\BL$).

A \AP""pseudovariety of synchronous algebras"" is any class $\symbb{V}$
of "locally generated@@sync" finite "synchronous algebras" closed under
\begin{itemize}
	\item \emph{finite product:} if $\?A, \?B \in \symbb{V}$ then $\?A \times \?B \in \symbb{V}$,
	\item \emph{division:} if some finite "locally generated" algebra $\?A$ "divides@@sync" $\?B$ for some $\?B \in \symbb{V}$, then $\?A \in \symbb{V}$.
\end{itemize}

Because of \Cref{lem:syntactic-morphism-theorem}, an "automatic relation" is "recognized@@sync"
by a finite synchronous algebra of a "pseudovariety@@syncalg" $\symbb{V}$ "iff"
its "syntactic synchronous algebra" belongs to $\symbb{V}$.

A \AP""$\ast$-pseudovariety of automatic relations"" is a function $\+V \colon \Sigma \mapsto \+V_\Sigma$
such that for any finite alphabet $\Sigma$, $\+V_\Sigma$ is a set of "automatic relations" over 
$\Sigma$ such that $\+V$ is closed under
\begin{itemize}
	\itemAP ""Boolean combinations""\emph{:} if $\+R, \+S \in \+V_{\Sigma}$, then
		$\negrel\+R$, $\+R \cup \+S$ and $\+R \cap \+S$ belong to $\+V_{\Sigma}$ too,
	% \item \emph{Partial quotients:} if $\+R \in \+V_{\Sigma}$, then
	% 	$\pair{a}{b}^{-1}\+R$,
	% 	$\+R\pair{a}{\pad}^{-1}$
	% 	and $\+R\pair{\pad}{a}^{-1}$ all belong to $\+V_\Sigma$, where $a,b \in \Sigma$,
	\itemAP ""Syntactic derivatives""\emph{:} if $\+R \in \+V_{\Sigma}$, then any relation
	"recognized@@sync" by the "syntactic synchronous algebra morphism" of $\+R$ also belongs
	to $\+V_{\Sigma}$.
	\itemAP ""Inverse morphisms""\emph{:} if $\phi\colon \Sync\Gamma \to \Sync\Sigma$ is
		a "synchronous algebra morphism" and $\+R \in \+V_{\Sigma}$ then
		$\phi^{-1}[\+R] \in \+V_{\Gamma}$. 
\end{itemize}

To recover a more traditional definition (of the form ``closure under Boolean operations, residuals\footnote{Also called ``quotient'' "eg" in \cite[\S III.1.3, p.~39]{Pin2022MathematicalFoundations}, or ``polynomial derivative'' in \cite[\S 4, p.~19]{Bojanczyk2015Recognisable}.} and inverse morphisms''), we need to properly define what are the residuals of a relation. It 
turns out that the answer is quite surprising and less trivial than what one would expect.

\begin{definition}[Residuals]
	\AP\label{def:residuals}
	Let $\?A$ be a "synchronous algebra", $\type{x}{\sigma} \in \?A$,
	and $C \subseteq \?A$ be a "closed subset".
	The \emph{left residual} and \emph{right residual} of $C$ by $\type{x}{\sigma}$ are defined by
	$\phantomintro{\residual}$
	\begin{align*}
		\reintro*\residual[\sigma]{x} C & \defeq
		\bigl\{
			\type{y}{\tau} \in \?A \mid
				\exists \type{y'}{\tau'} \congr{C} \type{y}{\tau},\;
				\type{x}{\sigma} \type{y'}{\tau'} \in C
		\bigr\}, \text{ and} \\
		C\reintro*\residual[\sigma]{x} & \defeq
		\bigl\{
			\type{y}{\tau} \in \?A \mid
				\exists \type{y'}{\tau'} \congr{C} \type{y}{\tau},\;
				\type{y'}{\tau'}\type{x}{\sigma} \in C
		\bigr\},
	\end{align*}
	respectively. We refer indiscriminately to both these notions as \AP""residuals"".
	We extend these notions to sets, by letting
	$\residual{X}{}C \defeq \bigcup_{x\in X}\residual{x}{}C$
	and $C\residual{X}{} \defeq \bigcup_{x\in X} C\residual{x}{}$.
\end{definition}

For the sake of readability, we will sometimes drop the "type" of elements when dealing
with "residuals".
It is routine to check that "residuals" are always "closed subsets" (since $\congr{C}$ is coarser than the "dependency relation"), or that $(\residual{x} C)\residual{y} =
\residual{x} (C\residual{y})$.
Equivalently, $C\residual[\sigma]{x}$ can be defined as the smallest "closed subset"
containing the ``naive residual''
$\bigl\{
	\type{y}{\tau} \in \?A \mid
		\type{y}{\tau}\type{x}{\sigma} \in C
\bigr\}$.
This latter set is always contained in $C\residual[\sigma]{x}$ (by reflexivity of $\congr{C}$),
and moreover, if it is empty, then so is $C\residual[\sigma]{x}$.

As an example, consider the relation $\+R$ from \Cref{ex:last_letter_is_a_if_big_diff}.
Then the ``naive right residual'' of $\proj{\+R}$ by $\type{\pair{a}{\pad}}{\LB}$
consists of $\type{\varepsilon}{\LL}$ and all elements of type $\LB$ and $\LL\to\LB$.
But it does not contain any element of type $\BL$ or $\LL\to\BL$ because such elements cannot be concatenated with $\type{\pair{a}{\pad}}{\LB}$ on the right.
Yet, the "residual" 
$\proj{\+R} \residual[\LB]{\pair{a}{\pad}}$ contains all elements of type $\BL$ (and also $\LL\to\BL$): for instance, $\type{\pair{\pad}{a}}{\BL} \in
\proj{\+R} \residual[\LB]{\pair{a}{\pad}}$ since $\type{\pair{\pad}{a}}{\BL} \congr{\+R} \type{\pair{a}{\pad}}{\LB}$
and $\type{\pair{a}{\pad}}{\LB} \type{\pair{a}{\pad}}{\LB} \in \+R$.

On the other hand, in the "algebra@@sync" $\Sync{a}$ consider the relation
$\+S = (aa)^*\times a(aa)^*$.
Then $\proj{\+S}\residual[\LL]{\pair{a}{a}}$ is empty since its 
``naive residual'' $\{\type{y}{\tau} \in \Sync{a} \mid \type{y}{\tau}\cdot\pair{a}{a} \in \+S\}$
is empty. Indeed, for $\type{y}{\tau}\cdot\type{\pair{a}{a}}{\LL}$ to
be well-defined, one needs $\tau$ to be $\LL$, "ie" $y$ encodes a pair of
two words $(u,v)$ of the same length. But then $(ua, va) \not\in \+S$.

\begin{restatable}{lemma}{lemmaCharacterizationPseudovar}
	\AP\label{lemma:characterization-pseudovarieties-syncrel}
	A class $\+V\colon \Sigma \mapsto \+V_\Sigma$ is a "$\ast$-pseudovariety of automatic relations" if, and only if, it is closed under "Boolean combinations", "residuals" and
	"inverse morphisms".
\end{restatable}

\begin{proof}
	We first need two propositions.

	\begin{claim}
		\label{claim:inverse-morphism-preserve-congruence}
		Let $\phi\colon \?A \surj \?B$ be a surjective "morphism@@sync",
		and $\Acc$ be a "closed subset" of $\?B$. Let $a, a' \in \?A$.
		Then
		\[
			a \congr{\phi^{-1}[\Acc]} a'
			\quad\text{"iff"}\quad
			\phi(a) \congr{\Acc} \phi(a').
		\]
	\end{claim}
	
	\proofcase{Direct implication.}
	Pick any $b_\ell, b_r \in \?B$ such that both
	$b_\ell \phi(a) b_r$ and
	$b_\ell \phi(a') b_r$
	are well-defined. By surjectivity of $\phi$, there exists
	$a_\ell, a_r \in \?A$
	such that $\phi(a_\ell) = b_\ell$
	and $\phi(a_r) = b_r$.
	Then both $a_\ell a a_r$
	and $a_\ell a' a_r$ are well-defined since they have the same "type"
	as $b_\ell \phi(a) b_r$ and $b_\ell \phi(a') b_r$, respectively.
	From $a \congr{\phi^{-1}[\Acc]} a'$,
	it follows that $a_\ell a a_r$ belongs to $\phi^{-1}[\Acc]$ "iff" $a_\ell a' a_r$ does.
	And hence
	\[
		b_\ell \phi(a) b_r \in \Acc 
			\quad\text{"iff"}\quad
		b_\ell \phi(a') b_r \in \Acc.
	\]

	\proofcase{Converse implication.}
	Dually, pick any $a_\ell, a_r \in \?A$ such that both
	$a_\ell a a_r$ and $a_\ell a' a_r$ are well-defined.
	Then $\phi(a_\ell) \phi(a) \phi(a_r)$ and $\phi(a_\ell) \phi(a') \phi(a_r)$
	are also well-defined since they have the same "type" as their preimage,
	and $\phi(a) \congr{\Acc} \phi(a')$ implies that the element $\phi(a_\ell) \phi(a) \phi(a_r)$ belongs
	to $\Acc$ "iff" $\phi(a_\ell) \phi(a') \phi(a_r)$ does. It follows
	that $a_\ell a a_r \in \phi^{-1}[\Acc]$ "iff" $a_\ell a' a_r \in \phi^{-1}[\Acc]$.
	This concludes the proof of \Cref{claim:inverse-morphism-preserve-congruence}.
	
	
	\begin{claim}[Inverse images of surjective "morphisms@@sync" preserve "residuals"]
		\label{claim:inverse-morphism-preserve-residuals}
		Let $\phi\colon \?A \surj \?B$ be a surjective "morphism@@sync", and $\Acc \subseteq \?B$
		be a "closed subset". Let $u \in \?A$. Then
		\[\residual{u}\phi^{-1}[\Acc] = \phi^{-1}[\residual{\phi(u)}\Acc].\]
	\end{claim}
	
	\proofcase{Left-to-right inclusion.}
	Let $a\in \residual{u}\phi^{-1}[\Acc]$.
	Then there exists $a' \in \?A$ such that $a \congr{\phi^{-1}[\Acc]} a'$
	and $ua' \in \phi^{-1}[\Acc]$.
	By \Cref{claim:inverse-morphism-preserve-congruence}
	$a \congr{\phi^{-1}[\Acc]} a'$ implies $\phi(a) \congr{\Acc} \phi(a')$,
	and $ua' \in \phi^{-1}[\Acc]$ yields $\phi(u)\phi(a') \in \Acc$.
	Overall, this shows that $a \in \phi^{-1}[\residual{\phi(u)}\Acc]$.

	\proofcase{Right-to-left inclusion.}
	Let $a \in \phi^{-1}[\residual{\phi(u)}\Acc]$. Then
	$\phi(a) \in \residual{\phi(u)}\Acc$, so there
	exists $b' \in \?B$ such that $\phi(a) \congr{\Acc} b'$
	and $\phi(u)b' \in \Acc$. By surjectivity of $\phi$ and
	\Cref{claim:inverse-morphism-preserve-congruence}, there exists $a' \in \?A$ 
	such that $\phi(a') = b'$ and $a \congr{\phi^{-1}[\Acc]} a'$.

	Being done with the proof of \Cref{claim:inverse-morphism-preserve-residuals},
	we now proceed to prove \Cref{lemma:characterization-pseudovarieties-syncrel}.
	\proofcase{Direct implication.} By \Cref{claim:inverse-morphism-preserve-residuals},
	the "residual" of any relation "recognized@@sync" by some "morphism@@sync"
	$\phi$ is also "recognized@@sync" by $\phi$. Hence, being closed under
	"syntactic derivatives" implies being closed under "residuals".

	\proofcase{Converse implication.} Consider some relation $\+R$.
	We will show that any relation "recognized@@sync" by $\SyntSAM{\+R}$
	can be expressed as a "Boolean combination" of "residuals" of $\+R$.\footnote{This result can be 
	put in perspective with \cite[Lemma XIII.4.11, p.~229]{Pin2022MathematicalFoundations} which proves a similar result in the context of monoids.}
	Let $\Acc$ be the "closed subset" of $\SyntSA{\+R}$ such that
	$\proj{\+R} = \SyntSAM{\+R}^{-1}[\Acc]$. Pick $x \in \SyntSA{\+R}$.
	Let $\Lambda \defeq \{s, t \in \SyntSA{\+R} \mid \exists x'\in\SyntSA{\+R},\; x' \dep x \text{ and } sx't \in \Acc\}$.
	We claim that
	\begin{equation}
		\equivclass{x}{\dep_{\SyntSA{\+R}}} =
		\left(
			\bigcap_{(s,t) \in \Lambda} \residual{s} \Acc\,\residual{t}
		\right)
		\smallsetminus
		\left(
			\bigcup_{(s, t) \not\in \Lambda} \residual{s} \Acc\,\residual{t}
		\right).
		\label{eq:boolean-combination-of-residuals}
	\end{equation}
	To prove the inclusion from left-to-right, first
	notice that $x \in \residual{s} \Acc\,\residual{t}$ for all $(s, t) \in \Lambda$.
	Then, assume by contradiction that there exists $(s, t) \not\in \Lambda$ "st"
	$x \in \residual{s} \Acc\,\residual{t}$. Then there would exist $x' \congr{\Acc} x$ such that
	such that $sx't \in \Acc$. But since $\SyntSAM{\+R}$ is the "syntactic synchronous algebra"
	of $\+R$, $\congr{\Acc}$ is precisely the relation $\dep$ by \Cref{coro:syntactic-congruence-is-syntactic-dependency}. Contradiction.
	Hence, $x$ belongs to the right-hand side (RHS). But then, this latter set is
	a Boolean combination of "residuals" of a "closed subset", so it is also
	"closed", and hence $\equivclass{x}{\dep_{\SyntSA{\+R}}}$ is included in the RHS.

	Dually, any element $y$ of the RHS satisfies that for all $s, t \in \SyntSA{\+R}$,
	$x \in \residual{s} \Acc\,\residual{t}$ "iff" $y \in \residual{s} \Acc\,\residual{t}$.
	We claim that $x \congr{\Acc} y$. Pick $s, t \in \?B$ and assume that
	both $sxt$ and $syt$ are well-defined. If $sxt \in \Acc$ then $x \in \residual{s} \Acc\,\residual{t}$ so $y \in \residual{s} \Acc\,\residual{t}$ and hence, there
	exists $y' \dep_{\SyntSA{\+R}} y$ "st" $sy't \in \Acc$. But $syt$ is also well-defined
	and $y \dep_{\SyntSA{\+R}} y'$ so $syt \in \Acc$. By symmetry, we have shown that
	$sxt \in \Acc$ "iff" $syt \in \Acc$, and hence $x \congr{\Acc} y$.
	Using again the fact that $\SyntSA{\+R}$ is the "syntactic algebra@@sync" of $\+R$, it 
	follows that $x \dep_{\SyntSA{\+R}} y$. This concludes the proof of
	\eqref{eq:boolean-combination-of-residuals}. By taking the union, it
	follows that any "closed subset" of $\SyntSA{\+R}$ is a Boolean combination
	of "residuals" of $\Acc$. Applying \Cref{claim:inverse-morphism-preserve-residuals}
	then yields that any relation "recognized@@sync" by $\phi$ is a Boolean combination of 
	"residuals" of $\+R$. Hence, any class closed under "Boolean combinations" and
	"residuals" is also closed under "syntactic derivatives".
\end{proof}

Let \AP$\symbb{V} \intro*\corrAR \+V$ denote the map (called \emph{correspondence}) that takes a 
"pseudovariety of synchronous algebras" and maps it to
\[\+V\colon \Sigma \mapsto \{\+R \subseteq \Sigma^*\times \Sigma^* \mid \SyntSA{\+R} \in \symbb{V}\}.\]

Dually, let \AP$\+V \intro*\corrRA \symbb{V}$ denote the \emph{correspondence} that takes
a "$\ast$-pseudo\-variety of automatic relations" $\+V$
and maps it to the "pseudovariety of synchronous algebras" "generated@@var" by
all $\SyntSA{\+R}$ for some $\+R \in \+V_{\Sigma}$.
Here, the ""pseudovariety generated"" by a class $C$
of finite "locally generated" "synchronous algebras"
is the smallest "pseudovariety@@syncalg" containing
all finite "locally generated" "algebras@@sync" of $C$,
or equivalently,\footnote{The proof is straightforward,
see "eg" \cite[Proposition XI.1.1, p.~190]{Pin2022MathematicalFoundations} for a proof in the context of semigroups.} the class of all finite "locally generated" "synchronous algebras" 
that "divide@@sync" a finite product of "algebras@@sync" of $C$.\footnote{Note that ``being "locally generated"'' is not preserved by taking "subalgebras", but this is not an issue: we restrict the construction to (finite) "locally generated" "algebras@@sync".}

\AP\phantomintro(sync){Eilenberg-Schützenberger correspondence theorem}\vspace{-1em}
\begin{restatable}[\reintro{An Eilenberg theorem for automatic relations}]{lemma}{thmeilenberg}
	\AP\label{lem:eilenberg-sy}
	The correspondences $\symbb{V} \corrAR \+V$ and $\+V \corrRA \symbb{V}$ define
	mutually inverse bijections between "pseudovarieties of
	synchronous algebras" and "$\ast$-pseudovarieties of automatic relations".
\end{restatable}

\begin{proof}
	We very roughly follow the proof schema of \cite[\S XIII.4, pp.~226--229]{Pin2022MathematicalFoundations}, which is
	a proof of Eilenberg's theorem in the context of monoids.

	\proofcase{The correspondence $\symbb{V} \corrAR \+V$ produces varieties.}
	First we have to show that if $\symbb{V}$ is a "pseudovariety
	of synchronous algebras" and $\symbb{V} \corrAR \+V$, then $\+V$ is a
	"$\ast$-pseudovarieties of automatic relations".
	Since $\symbb{V}$ is closed under finite products, $\+V$ is closed under Boolean operations.

	\emph{"Syntactic derivatives":} Then let $\+R \in \+V_\Sigma$, and let $\+S$ be any other relation
	"recognized@@sync" by $\SyntSA{\+R}$. This implies that $\SyntSA{\+S}$
	"divides@@sync" $\SyntSA{\+R}$, and so $\SyntSA{\+S} \in \symbb{V}$, from which
	we have $\SyntSA{\+S} \in \+V_\Sigma$.

	\emph{"Inverse morphisms":} Lastly, if $\+R \in \+V_\Sigma$, say $\proj{\+R} = \SyntSAM[-1]{\+R}[\Acc]$,
	if $\psi\colon \Sync\Gamma \to \Sync\Sigma$ is a "synchronous algebra morphism",
	then $\psi^{-1}[\+R] = (\SyntSAM{\+R} \circ \psi)^{-1}[\Acc]$, so
	$\psi^{-1}[\+R]$ is "recognized@@sync" by $\SyntSA{\+R}$, that is
	$\SyntSA{\psi^{-1}[\+R]}$ "divides@@sync" $\SyntSA{\+R}$. Since $\SyntSA{\+R} \in \symbb{V}$
	and $\symbb{V}$ is closed by "division@@sync", it follows that $\SyntSA{\psi^{-1}[\+R]} \in \symbb{V}$
	and hence $\psi^{-1}[\+R] \in \+V_{\Gamma}$. 
	This concludes the proof that $\+V$ is a "$\ast$-pseudovariety of automatic relations".

	\proofcase{Inverse bijections: part 1.} Assume that $\symbb{V} \corrAR \+V$ and $\+V \corrRA \symbb{W}$.
	Then
	\[\+V\colon \Sigma \mapsto \{\+R \subseteq \Sigma^*\times \Sigma^* \mid \SyntSA{\+R} \in \symbb{V}\},\]
	and so $\symbb{W}$ is the "pseudovariety@@syncalg" generated by all "syntactic synchronous algebras" that 
	belong to $\symbb{V}$.
	It follows that $\symbb{W} \subseteq \symbb{V}$.
	To prove that $\symbb{V} \subseteq \symbb{W}$, let $\?A \in \symbb{V}$.
	Let $\Sigma_{\?A}$ be an alphabet big enough so that
	there are injections from $\?A_{\LL}$ to $\Sigma_{\?A}\times \Sigma_{\?A}$,
	and from $\?A_{\LB}$ and $\?A_{\BL}$ to $\Sigma_{\?A} \times \pad$ and $\pad \times \Sigma_{\?A}$,
	respectively. Since $\?A$ is "locally generated@@sync",
	this allows us to define a surjective "synchronous algebra morphism"
	$\phi\colon \Sync{\Sigma_{\?A}} \surj \?A$.
	We then claim that $\?A$ "divides@@sync"
	$\?B \defeq \prod_{\type{x}{\tau} \in \?A} \?B_{\type{x}{\tau}}$
	where $\?B_{\type{x}{\tau}}$ is the "syntactic synchronous algebra"
	of $\phi^{-1}[\type{x}{\tau}]$.
	Indeed, let $\psi_{\type{x}{\tau}} \colon \Sync{\Sigma_{\?A}} \surj \?B_{\type{x}{\tau}}$
	be the "syntactic synchronous algebra morphism" of $\phi^{-1}[\type{x}{\tau}]$, say
	$\phi^{-1}[\type{x}{\tau}] = \psi_{\type{x}{\tau}}^{-1}[\Acc_{\type{x}{\tau}}]$.
	Then consider
	\[\begin{tabular}{rccc}
		$\Psi\colon$ & $\Sync{\Sigma_{\?A}}$ & $\to$ & $\?B$ \\
		& $\type{u}{\sigma}$ & $\mapsto$ &
			$\langle \psi_{\type{x}{\tau}}(\type{u}{\sigma}) \rangle_{\type{x}{\tau} \in \?A}$,
	\end{tabular}\]
	and let $\?B_0$ be its image. Observe that for each $\type{u}{\sigma} \in \Sync{\Sigma_{\?A}}$,
	$\psi_{\type{x}{\tau}}(\type{u}{\sigma}) \in \Acc_{\type{x}{\tau}}$ "iff"
	$\type{u}{\sigma} \in \phi^{-1}[\type{x}{\tau}]$ "ie" $\phi(\type{u}{\sigma}) = \type{x}{\tau}$---note by the way that it implies $\sigma = \tau$.
	This implies that for any $\type{(\langle y_{\type{x}{\tau}} \rangle_{\type{x}{\tau} \in \?A})}{\sigma} \in \?B_0$, there exists a unique $\type{x}{\tau}$ "st"
	$y_{\type{x}{\tau}} \in \Acc_{\type{x}{\tau}}$. This defines a map
	$\chi\colon \?B_0 \to \?A$. Since moreover it makes the following diagram commute
	
	\begin{center}\begin{tikzcd}
		\Sync{\Sigma_{\?A}} \ar[r, twoheadrightarrow, "\Psi"]
			\ar[dr, twoheadrightarrow, "\phi" swap]
		& \?B_0 \dar["\chi"] \\
		& \?A
	\end{tikzcd}\end{center}
	it follows that $\chi$ is in fact a surjective "synchronous algebra morphism".\footnote{See "eg" \cite[Lemma 3.2, p.~10]{Bojanczyk2015Recognisable} for a proof in a similar (but different) context.} Hence, $\?A$ is a "quotient@@sync" of $\?B_0$, which is a "subalgebra@@sync" of
	$\?B$, which in turns in a product of "algebras@@sync" from $\symbb{W}$, and so $\?A \in \symbb{W}$.
	It concludes the proof that $\symbb{V} = \symbb{W}$.

	\proofcase{Inverse bijections: part 2.} Assume now that $\+V \corrRA \symbb{V}$
	and $\symbb{V} \corrAR \+W$. Then for each $\Sigma$, for each $\+R \in \+V_{\Sigma}$,
	$\SyntSA{\+R} \in \symbb{V}$ so $\+R \in \+W_{\Sigma}$, and hence $\+V \subseteq \+W$.

	We then want to show the converse inclusion, namely $\+W \subseteq \+V$. Let $\+R \in \+W_{\Sigma}$ for some $\Sigma$, "ie" $\SyntSA{\+R} \in \symbb{V}$.
	Hence there exists $\Gamma$ and relations $\+S_1 \in \+V_{\Gamma_1}, \dotsc \+S_k \in \+V_{\Gamma_k}$
	such that $\SyntSA{\+R}$ "divides@@sync"
	$\?B \defeq \SyntSA{\+S_1} \times \cdots \times \SyntSA{\+S_k}$,
	"ie" there is a "subalgebra@@sync" $\?C \subseteq \?B$ which is a "quotient@@sync" of $\?B$.
	Then $\?C$ also "recognizes@@sync" $\+R$, say $\+R = \phi^{-1}[\Acc]$ for some
	"morphism@@sync" $\phi\colon \Sync\Sigma \surj \?C$ and $\Acc \subseteq \?C$.
	Let $\iota\colon \?C \to \?B$ be the canonical embedding,
	$\pi_i\colon \?B \surj \SyntSA{\+S_i}$ be the canonical projection,
	and $\phi_i \defeq \pi_i \circ \iota \circ \phi\colon \Sync\Sigma \to \SyntSA{\+S_i}$
	for $i \in \intInt{1,k}$. Then notice that since $\SyntSAM{\+S_i}\colon \Sync{\Gamma_i} \surj \SyntSA{\+S_i}$ is surjective, then there exists $\psi_i \colon \Sync\Sigma \to \Sync{\Gamma_i}$ such that $\SyntSAM{\+S_i} \circ \psi_i = \phi_i$. Indeed, it suffices
	to send $\pair{a}{b}$ (resp.~$\pair{a}{\pad}$, resp.~$\pair{\pad}{a}$)
	on any element $\type{u}{\LL} \in \Sync{\Gamma_i}$ (resp.~$\type{u}{\LB}$,
	resp.~$\type{u}{\BL}$) such that $\SyntSAM{\+S_i}(\type{u}{\LL}) = \phi\pair{a}{b}$
	(resp.~$\SyntSAM{\+S_i}(\type{u}{\LB}) = \phi\pair{a}{\pad}$,
	resp.~$\SyntSAM{\+S_i}(\type{u}{\BL}) = \phi\pair{\pad}{a}$). Overall, the following diagram
	commutes
	\begin{center}
	\begin{tikzcd}
		\Sync\Sigma
			\ar[rr, "\psi_i"]
			\dar["\phi" swap]
			\ar[ddrr, "\phi_i"]
		&[-2em]
		& \Sync{\Gamma_i}
			\ar[dd, "\SyntSAM{\+S_i}"] \\
		\?C
			\ar[dr, "\iota" swap]
		&
		& \\[-1.5em]
		& \?B
			\ar[r, "\pi_i" swap]
		& \SyntSA{\+S_i}.
	\end{tikzcd}
	\end{center}
	Our goal is to show that $\+R \in \+V_{\Sigma}$. Observe that:
	\[
		\proj{\+R} =
		\phi^{-1}[\Acc] = \bigcup_{x \in \Acc} \phi^{-1}[x]
	\]
	but then $\Acc \subseteq \?B$, so $x$ is a tuple $\langle x_1, \dotsc, x_n \rangle$
	(all elements having the same type), and by definition:
	\[
		\phi^{-1}[x] = \bigcap_{i=1}^n \phi^{-1}[\iota^{-1}[\pi_i^{-1}[x_i]]]
		= \bigcap_{i=1}^n \phi_i^{-1}[x_i].
	\]
	But then $\phi_i^{-1}[x_i] = \psi_i^{-1}[\SyntSAM{\+S_i}^{-1}[x_i]]$.
	Since $\+V$ is closed under "syntactic derivatives" and $\+S_i \in \+V_{\Gamma_i}$
	we have $\SyntSAM{\+S_i}^{-1}[x_i] \in \+V_{\Gamma_i}$, and then since $\+V$ is closed under
	"Inverse morphisms" and $\psi_i\colon \Sync\Sigma \to \Sync{\Gamma_i}$ is a "morphism@@sync" between free algebras,
	$\psi_i^{-1}[\SyntSAM{\+S_i}^{-1}[x_i]] \in \+V_{\Sigma}$.
	Thus $\proj{\+R}$ is a "Boolean combination" of elements of $\+V_{\Sigma}$, and hence
	it also belongs to $\+V_{\Sigma}$. This concludes the proof of $\+W \subseteq \+V$.
\end{proof}

As consequence of \Cref{lem:eilenberg-sy}, if
$\+V$ is a "$\ast$-pseudovariety of automatic relations"
and $\symbb{V}$ is a "pseudovariety of synchronous algebras",
we write \AP$\+V \intro*\corr \symbb{V}$
to mean that either $\+V \corrRA \symbb{V}$ or, equivalently, $\symbb{V} \corrAR \+V$.
This relation is called an \AP""Eilenberg-Schützenberger correspondence@@sync"".

\begin{proposition}
	If~~$\symbb{V}$ is a "pseudovariety of monoids", then \AP$\phantomintro{\projA}$
	\begin{align*}
		\reintro*\projA{\symbb{V}} & \defeq
		\{\?A \text{ "locally generated@@sync" finite "synchronous algebra"} \\
		& \qquad\qquad \text{ "st" all "underlying monoids" of $\?A$ are in $\symbb{V}$}\}
	\end{align*}
	is a "pseudovariety of synchronous algebras". Similarly,
	if~~$\+V$ is an "$\ast$-pseudovariety of regular languages", then
	the class of "$\+V$-relations", namely
	\[
		\intro*\projL{\+V} \colon
		\Sigma \mapsto \{\+R \subseteq \Sigma^* \times \Sigma^* \mid \exists L \in \+V_{\SigmaPair},\, \proj{\+R} = L \cap \WellFormed \},
	\]
	is a "$\ast$-pseudovariety of automatic relations".
\end{proposition}

\begin{proof}
	The first point is straightforward. The second one follows from it and \Cref{lem:eilenberg-sy}and \Cref{thm:lifting-theorem-monoids}.
\end{proof}

\AP\phantomintro{Lifting Theorem: Pseudovariety Formulation}
Finally, \Cref{thm:lifting-theorem-monoids} can be elegantly rephrased
by saying that correspondences between "pseudovarieties of monoids"
and "$\ast$-pseudovarieties of regular languages" lift to correspondences
between "pseudovarieties of synchronous algebras" and
"$\ast$-pseudovarieties of automatic relations".

\begin{theorem}[\reintro{Lifting Theorem: Pseudovariety Formulation}]
	\AP\label{thm:lifting-theorem-monoids-pseudovarieties}
	If, in the "Eilenberg-Schützenberger correspondence@@lang"
	between "pseudovarieties of mon\-oids" and "$\ast$-pseudovarieties of regular languages"
	we have~$\+V \corr \symbb{V}$,
	then in the "Eilenberg-Schützenberger correspondence@@sync"
	between the "pseudovariety of synchronous algebras" $\projA{\symbb{V}}$ and
	the "$\ast$-pseudovariety of automatic relations" $\projL{\+V}$,
	we have~$\projL{\+V} \corr \projA{\symbb{V}}$.
\end{theorem}