\section{The Lifting Theorem \& Pseudovarieties}
\label{sec:lifting-theorem}

\subsection{Elementary Formulation}

\begin{example}[Group relations: {\Cref{ex:algebra-Zpq} cont'd}]
	\label{ex:charac-group-relation-monoids}
	We want to decide when the relation
	\begin{align*}
		\+R^{I,J} \defeq \big\{
			(u,v) \;\big\vert\; & |u| > |v| \text{ and } (|u| - |v| \bmod{p}) \in I, \text{ or} \\
			& |u| < |v| \text{ and } (|v| - |u| \bmod{q}) \in J.
		\hphantom{\text{ or}}\big\}	
	\end{align*}
	from \Cref{ex:algebra-Zpq} is a \AP"group relation".
	By definition this happens if and only if there exists a finite group $G$,
	together with a monoid morphism $\phi\colon (\SigmaPair)^* \to G$ and a subset
	$\Acc \subseteq G$ "st" $\forall u \in \WellFormed$, $u\in \+R^{I,J}$ "iff" $\phi(u) \in \Acc$.
	We claim:
	\begin{equation}
		\+R^{I,J} \text{ is a "group relation"}\quad\text{"iff"}\quad
		\big(
			\bar 0 \not\in I \text{ and } \bar 0 \not\in J
		\big).
		\tag{{\Large\adforn{10}}}
	\end{equation}

	The right-to-left implication was shown in \Cref{ex:algebra-Zpq}.
	We prove the implication from left to right:
	let $n$ be the order of $G$ so that $x^n = 1$ for all $x\in G$. In particular, we have:
	$\phi\bigl(\pair{a}{\pad}^{pqn}\bigr) = 1 = \phi\bigl(\pair{a}{a}^{pqn}\bigr)$.
	Since $\phi\bigl(\pair{a}{a}^{pqn}\bigr) \not\in \+R^{I,J}$, it follows that 
	$\pair{a}{\pad}^{pqn} \not\in \+R^{I,J}$
	"ie" $\bar 0 \not\in I$. Also, $\bar 0 \not\in J$ by symmetry, which concludes the proof.
\end{example}

\AP\phantomintro{Lifting Theorem}
Even more generally, we can decide if a "relation" $\+R$ is a "group relation"
by simply looking at the "syntactic synchronous algebra" of $\+R$.

\liftingtheoremmonoids

See the proof in "the full version".

\begin{remark}
	In light of \Cref{thm:lifting-theorem-monoids},
	one can wonder whether the notion of "synchronous algebra" is necessary to
	characterize "$\+V$-relations", or if it is enough to look at the languages corresponding to
	the "underlying monoids".
	Said otherwise, is the membership of $\+R$ in the class of "$\+V$-relations" uniquely
	determined by the regular languages $\+R \cap (\Sigma\times\Sigma)^*$,
	$\+R \cap (\Sigma\times\{\pad\})^*$ and $\+R \cap (\{\pad\}\times\Sigma)^*$?
	Unsurprisingly, "synchronous algebras" are indeed necessary, as 
	there are relations $\+R$ such that:
	\begin{equation}
		\label{eq:naive-characterization} \tag{\adforn{12}}
		\proj{\+R} \cap (\Sigma\times \Sigma)^* \in \+V_{\Sigma\times \Sigma}, \quad
		\proj{\+R} \cap (\Sigma\times \pad)^* \in \+V_{\Sigma\times \pad}\quad\text{and}\quad
		\proj{\+R} \cap (\pad\times \Sigma)^* \in \+V_{\pad\times \Sigma},
	\end{equation}
	but $\+R$ is \emph{not} a "$\+V$-relation". This can happen even if
	$\+V$ is the "$\ast$-pseudovariety@@reglang" of all regular languages: for instance for the
	"relation"
	\[
		\+R \defeq \{(u,v) \mid |u| > |v| > 0 \text{ and } |u| - |v| \text{ is prime}\}.
	\]
	Notice that there is a subtle but crucially important difference between
	\eqref{eq:naive-characterization} and the second item of the "Lifting Theorem":
	while the "underlying monoids" of a "synchronous algebra" $\?A$ "recognizing@@sync" $\+R$
	only accept words of the form $(\Sigma\times \Sigma)^*$, $(\Sigma\times \pad)^*$
	or $(\pad\times \Sigma)^*$, elements of $(\Sigma\times \Sigma)^+(\Sigma\times \pad)^+$
	or $(\Sigma\times \Sigma)^+(\pad\times \Sigma)^+$ influence the "underlying monoids" of $\?A$
	via the axioms of "synchronous algebras".
\end{remark}

Also, note that the existence the "Lifting Theorem" follows from the careful
definition of "synchronous algebras": more naive definitions of these algebras
simply cannot characterize "$\+V$-relations", see "the full version".

From \Cref{thm:lifting-theorem-monoids}
and the implicit fact that all our constructions are effective,
we obtain a decidability (meta-)result for
"$\+V$-relations".
\begin{corollary}
	\label{coro:decidability}
	The class of "$\+V$-relations" has decidable membership if, and only if, $\+V$ has decidable membership.
\end{corollary}

For instance, a "relation" is a "group relation" if, and only if, all
"underlying monoids" of its "syntactic synchronous algebra" are
groups. 
% Beyond "group relations", the "lifting theorem" captures "e.g." the case of "commutative relations".

\subsection{Pseudovarieties of Synchronous Relations}
\label{sec:varieties}

We introduce the notion of "pseudovariety of synchronous algebras" 
and "$\ast$-pseudovariety of synchronous relations". We show an "Eilenberg correspondence@@sync" between these two notions. We then reformulate the "Lifting Theorem"
to show that any "Eilenberg correspondence@@lang" between monoids and regular languages
lifts to an "Eilenberg correspondence@@sync" between "synchronous algebras" and "synchronous relations".

Say that a "synchronous algebra" $\?A$ is a \AP""quotient@@sync"" of $\?B$
when there exists a surjective "synchronous algebra morphism" from $\?B$ to $\?A$.
A ""subalgebra@@sync"" of $\?B$ is any "closed subset" of $\?B$ closed under "product"
and containing the units.
We then say that "synchronous algebra" $\?A$ \AP""divides@@sync"" $\?B$
when $\?A$ is a "quotient@@sync" of a "subalgebra@@sync" of $\?B$.

Observe that $\Sync\Sigma$ admits the following property:
elements of type $\LL\to\LB$ and $\LL\to\BL$ are generated by the "underlying monoids".
Since "syntactic synchronous algebras" are homomorphic images of $\Sync\Sigma$, they also
satisfy this property. In general, we say that a "synchronous algebra" $\?A$ is \AP""locally 
generated@@sync"" if every element of type $\LL\to\LB$ (resp.~$\LL\to\BL$)
can be written as the product of an element of type $\LL$ with an element of type $\LB$ (resp.~$\BL$).

A \AP""pseudovariety of synchronous algebras"" is any class $\B{V}$
of "locally generated@@sync" finite "synchronous algebras" closed under
\begin{itemize}
	\item \emph{finite product:} if $\?A, \?B \in \B{V}$ then $\?A \times \?B \in \B{V}$,
	\item \emph{division:} if some finite "locally generated" algebra $\?A$ "divides@@sync" $\?B$ for some $\?B \in \B{V}$, then $\?A \in \B{V}$.
\end{itemize}

Because of \Cref{lem:syntactic-morphism-theorem}, a "synchronous relation" is "recognized@@sync"
by a finite synchronous algebra of a "pseudovariety@@syncalg" $\B{V}$ "iff"
its "syntactic synchronous algebra" belongs to $\B{V}$.

A \AP""$\ast$-pseudovariety of synchronous relations"" is a function $\+V \colon \Sigma \mapsto \+V_\Sigma$
such that for any finite alphabet $\Sigma$, $\+V_\Sigma$ is a set of "synchronous relations" over 
$\Sigma$ such that $\+V$ is closed under
\begin{itemize}
	\itemAP ""Boolean combinations""\emph{:} if $\+R, \+S \in \+V_{\Sigma}$, then
		$\negrel\+R$, $\+R \cup \+S$ and $\+R \cap \+S$ belong to $\+V_{\Sigma}$ too,
	% \item \emph{Partial quotients:} if $\+R \in \+V_{\Sigma}$, then
	% 	$\pair{a}{b}^{-1}\+R$,
	% 	$\+R\pair{a}{\pad}^{-1}$
	% 	and $\+R\pair{\pad}{a}^{-1}$ all belong to $\+V_\Sigma$, where $a,b \in \Sigma$,
	\itemAP ""Syntactic derivatives""\emph{:} if $\+R \in \+V_{\Sigma}$, then any "relation"
	"recognized@@sync" by the "syntactic synchronous algebra morphism" of $\+R$ also belongs
	to $\+V_{\Sigma}$.
	\itemAP ""Inverse morphisms""\emph{:} if $\phi\colon \Sync\Gamma \to \Sync\Sigma$ is
		a "synchronous algebra morphism" and $\+R \in \+V_{\Sigma}$ then
		$\phi^{-1}[\+R] \in \+V_{\Gamma}$. 
\end{itemize}

To recover a more traditional definition (of the form ``closure under Boolean operations, residuals\footnote{Also called ``quotient'' "eg" in \cite[\S III.1.3, p.~39]{pin_mathematical_2022}, or ``polynomial derivative'' in \cite[\S 4, p.~19]{bojanczyk_recognisable_2015}.} and inverse morphisms''), we need to properly define what are the residuals of a "relation". It 
turns out that the answer is quite surprising and less trivial than what one would expect.

\begin{definition}[Residuals]
	\label{def:residuals}
	Let $\?A$ be a "synchronous algebra", $\type{x}{\sigma} \in \?A$,
	and $C \subseteq \?A$ be a "closed subset".
	The \emph{left residual} and \emph{right residual} of $C$ by $\type{x}{\sigma}$ are defined by
	$\phantomintro{\residual}$
	\begin{align*}
		\reintro*\residual[\sigma]{x} C & \defeq
		\bigl\{
			\type{y}{\tau} \in \?A \mid
				\exists \type{y'}{\tau'} \congr{C} \type{y}{\tau},\;
				\type{x}{\sigma} \type{y'}{\tau'} \in C
		\bigr\}, \text{ and} \\
		C\reintro*\residual[\sigma]{x} & \defeq
		\bigl\{
			\type{y}{\tau} \in \?A \mid
				\exists \type{y'}{\tau'} \congr{C} \type{y}{\tau},\;
				\type{y'}{\tau'}\type{x}{\sigma} \in C
		\bigr\},
	\end{align*}
	respectively. We refer indiscriminately to both these notions as \AP""residuals"".
	We extend these notions to sets, by letting
	$\residual{X}{}C \defeq \bigcup_{x\in X}\residual{x}{}C$
	and $C\residual{X}{} \defeq \bigcup_{x\in X} C\residual{x}{}$.
\end{definition}

For the sake of readability, we will sometimes drop the "type" of elements when dealing
with "residuals".
It is routine to check that "residuals" are always "closed subsets" (since $\congr{C}$ is coarser than the "dependency relation"), or that $(\residual{x} C)\residual{y} =
\residual{x} (C\residual{y})$.
Equivalently, $C\residual[\sigma]{x}$ can be defined as the smallest "closed subset"
containing the ``naive residual''
$\bigl\{
	\type{y}{\tau} \in \?A \mid
		\type{y}{\tau}\type{x}{\sigma} \in C
\bigr\}$.
This latter set is always contained in $C\residual[\sigma]{x}$ (by reflexivity of $\congr{C}$),
and moreover, if it is empty, then so is $C\residual[\sigma]{x}$.

As an example, consider the relation $\+R$ from "the full version".
Then the ``naive right residual'' of $\proj{\+R}$ by $\type{\pair{a}{\pad}}{\LB}$
consists of $\type{\varepsilon}{\LL}$ and all elements of type $\LB$ and $\LL\to\LB$.
But it does not contain any element of type $\BL$ or $\LL\to\BL$ because such elements cannot be concatenated with $\type{\pair{a}{\pad}}{\LB}$ on the right.
Yet, the "residual" 
$\proj{\+R} \residual[\LB]{\pair{a}{\pad}}$ contains all elements of type $\BL$ (and also $\LL\to\BL$): for instance, $\type{\pair{\pad}{a}}{\BL} \in
\proj{\+R} \residual[\LB]{\pair{a}{\pad}}$ since $\type{\pair{\pad}{a}}{\BL} \congr{\+R} \type{\pair{a}{\pad}}{\LB}$
and $\type{\pair{a}{\pad}}{\LB} \type{\pair{a}{\pad}}{\LB} \in \+R$.

On the other hand, in the "algebra@@sync" $\Sync{a}$ consider the relation
$\+S = (aa)^*\times a(aa)^*$.
Then $\proj{\+S}\residual[\LL]{\pair{a}{a}}$ is empty since its 
``naive residual'' $\{\type{y}{\tau} \in \Sync{a} \mid \type{y}{\tau}\cdot\pair{a}{a} \in \+S\}$
is empty. Indeed, for $\type{y}{\tau}\cdot\type{\pair{a}{a}}{\LL}$ to
be well-defined, one needs $\tau$ to be $\LL$, "ie" $y$ encodes a pair of
two words $(u,v)$ of the same length. But then $(ua, va) \not\in \+S$.

\begin{restatable}{lemma}{lemmaCharacterizationPseudovar}
	\label{lemma:characterization-pseudovarieties-syncrel}
	A class $\+V\colon \Sigma \mapsto \+V_\Sigma$ is a "$\ast$-pseudovariety of synchronous relations" if, and only if, it is closed under "Boolean combinations", "residuals" and
	"inverse morphisms".
\end{restatable}
See the proof in "the full version".

Let \AP$\B{V} \intro*\corrAR \+V$ denote the map (called \emph{correspondence}) that takes a 
"pseudovariety of synchronous algebras" and maps it to
\[\+V\colon \Sigma \mapsto \{\+R \subseteq \Sigma^*\times \Sigma^* \mid \SyntSA{\+R} \in \B{V}\}.\]

Dually, let \AP$\+V \intro*\corrRA \B{V}$ denote the \emph{correspondence} that takes
a "$\ast$-pseudo\-variety of synchronous relations" $\+V$
and maps it to the "pseudovariety of synchronous algebras" "generated@@var" by
all $\SyntSA{\+R}$ for some $\+R \in \+V_{\Sigma}$.
Here, the ""pseudovariety generated"" by a class $C$
of finite "locally generated" "synchronous algebras"
is the smallest "pseudovariety@@syncalg" containing
all finite "locally generated" "algebras@@sync" of $C$,
or equivalently,\footnote{The proof is straightforward,
see "eg" \cite[Proposition XI.1.1, p.~190]{pin_mathematical_2022} for a proof in the context of semigroups.} the class of all finite "locally generated" "synchronous algebras" 
that "divide@@sync" a finite product of "algebras@@sync" of $C$.\footnote{Note that ``being "locally generated"'' is not preserved by taking "subalgebras", but this is not an issue: we restrict the construction to (finite) "locally generated" "algebras@@sync".}

\AP\phantomintro(sync){Eilenberg correspondence theorem}\vspace{-1em}
\begin{restatable}[\reintro{An Eilenberg theorem for synchronous relations}]{lemma}{thmeilenberg}
	\label{lem:eilenberg-sy}
	The correspondences $\B{V} \corrAR \+V$ and $\+V \corrRA \B{V}$ define
	mutually inverse bijections between "pseudovarieties of
	synchronous algebras" and "$\ast$-pseudovarieties of synchronous relations".
\end{restatable}
See the proof in "the full version".

As consequence of \Cref{lem:eilenberg-sy}, if
$\+V$ is a "$\ast$-pseudovariety of synchronous relations"
and $\B{V}$ is a "pseudovariety of synchronous algebras",
we write \AP$\+V \intro*\corr \B{V}$
to mean that either $\+V \corrRA \B{V}$ or, equivalently, $\B{V} \corrAR \+V$.
This relation is called an \AP""Eilenberg-Schützenberger correspondence@@sync"".

\begin{proposition}
	If~~$\B{V}$ is a "pseudovariety of monoids", then \AP$\phantomintro{\projA}$
	\begin{align*}
		\reintro*\projA{\B{V}} & \defeq
		\{\?A \text{ "locally generated@@sync" finite "synchronous algebra"} \\
		& \qquad\qquad \text{ "st" all "underlying monoids" of $\?A$ are in $\B{V}$}\}
	\end{align*}
	is a "pseudovariety of synchronous algebras". Similarly,
	if~~$\+V$ is an "$\ast$-pseudovariety of regular languages", then
	the class of "$\+V$-relations", namely
	\[
		\intro*\projL{\+V} \colon
		\Sigma \mapsto \{\+R \subseteq \Sigma^* \times \Sigma^* \mid \exists L \in \+V_{\SigmaPair},\, \proj{\+R} = L \cap \WellFormed \},
	\]
	is a "$\ast$-pseudovariety of synchronous relations".
\end{proposition}

\begin{proof}
	The first point is straightforward. The second one follows from it and \Cref{lem:eilenberg-sy,thm:lifting-theorem-monoids}.
\end{proof}

\AP\phantomintro{Lifting Theorem: Pseudovariety Formulation}
Finally, \Cref{thm:lifting-theorem-monoids} can be elegantly rephrased
by saying that correspondences between "pseudovarieties of monoids"
and "$\ast$-pseudovarieties of regular languages" lift to correspondences
between "pseudovarieties of synchronous algebras" and
"$\ast$-pseudovarieties of synchronous relations".

\begin{theorem}[\reintro{Lifting Theorem: Pseudovariety Formulation}]
	\label{thm:lifting-theorem-monoids-pseudovarieties}
	If, in the "Eilenberg correspondence@@lang"
	between "pseudovarieties of mon\-oids" and "$\ast$-pseudovarieties of regular languages"
	we have~$\+V \corr \B{V}$,
	then in the "Eilenberg correspondence@@sync"
	between the "pseudovariety of synchronous algebras" $\projA{\B{V}}$ and
	the "$\ast$-pseudovariety of synchronous relations" $\projL{\+V}$,
	we have~$\projL{\+V} \corr \projA{\B{V}}$.
\end{theorem}