\pagestyle{empty}
\cleartoevenpage
\pagestyle{empty}
\small

\begin{chapterpresentation}
\begin{abstract}[Abstract]
	This thesis investigates the central role of homomorphism problems---structure-preserving maps---in two complementary domains: database querying over finite, graph-shaped data, and constraint solving over (potentially infinite) structures.

	\par Building on the well-known equivalence between conjunctive query evaluation and homomorphism existence, the first part focuses on conjunctive regular path queries, a standard extension of conjunctive queries that incorporates regular-path predicates. We study the fundamental problem of query minimization under two measures: the number of atoms (constraints) and the tree-width of the query graph. In both cases, we prove the problem to be decidable, and provide efficient algorithms for a large fragment of queries used in practice.

	\par The second part of the thesis lifts homomorphism problems to automatic structures, which are infinite structures describable by finite automata. We highlight a dichotomy, between homomorphism problems over automatic structures that are decidable in non-deterministic logarithmic space, and those that are undecidable---proving to be the more common case. In contrast to this prevalence of undecidability, we then focus on the language-theoretic properties of these structures, and show, relying on a novel algebraic language theory, that for any well-behaved logic (a pseudovariety), whether an automatic structure can be described in this logic is decidable.
\end{abstract}

\vspace{.3cm}
\begin{center}
\pgfornament[color=maincolor,height=.3cm]{3}
\end{center}
\vspace{.3cm}

\begin{abstract}[Résumé]
	Cette thèse étudie le rôle central joué par les problèmes d'homomorphismes---c'est-à-dire des fonctions préservant les structures relationnelles---dans deux domaines complémentaires : le requêtage de données sous forme de graphes, et la résolution de contraintes sur des structures potentiellement infinies.

	\hspace{\parindent}\ignorespaces En se fondant sur l'équivalence entre l'évaluation des requêtes conjonctives et l'existence d'homomorphismes, la première partie étudie les requêtes conjonctives à chemins réguliers, une extension du langage précédent incorporant des prédicats exprimant l'existence de chemins réguliers. Nous étudions le problème fondamental de la minimisation de ces requêtes selon deux métriques : le nombre d'atomes (contraintes) et la largeur arborescente du graphe sous-jacent. Dans les deux cas, nous montrons que le problème est décidable et proposons des algorithmes efficaces pour un fragment substantiel des requêtes utilisées en pratique.

	\hspace{\parindent}\ignorespaces La seconde partie de la thèse généralise les problèmes d'homomorphismes aux structures automatiques, structures infinies décrites par des automates finis. Nous dessinons une dichotomie entre les problèmes résolubles en espace logarithmique non-déterministe et ceux qui sont indécidables---ces derniers étant malheureusement majoritaires. \emph{A contrario}, nous mettons en avant une large classe de problèmes plus syntaxiques : à l'aide d'une nouvelle théorie algébrique des langages, nous montrons que, pour toute logique raisonnable (une pseudovariété), il est décidable de savoir si une structure automatique peut être spécifiée dans cette logique.
\end{abstract}
\end{chapterpresentation}