\subsection{Bounded Recognizable Relations}

\paragraph*{Separability for Bounded Recognizable Relations.}

In this part, we capitalize on the undecidability result of \Cref{sec:dichotomy-k-regular-colourability}, showing how this implies the undecidability for the separability problem on two natural classes of bounded "recognizable relations", namely $\kREC$ and $\kPROD$.
For any $k$, $\reintro*\kPROD$ is the subclass of $\REC$ consisting of unions of $k$ Cartesian products of "regular languages" (which is a subclass of $\kREC[2^{2k}]$).

First, observe that the "$\kREC[1]$-separability problem" is trivially decidable, since the only possible "separator@@rel" is $\Sigma^* \times \Sigma^*$. However, for any other $k>1$, the problem is undecidable.

\begin{corollary}
    The "$\kREC$-separability problem" is undecidable, for every $k>1$.
\end{corollary}

\begin{proof}
    This is a consequence of the reduction from the "$k$-regular colourability problem" of \Cref{thm:reg-colourability-equiv-separability}, combined with the undecidability of the latter for every $k>1$ (\Cref{thm:k-reg-col-undec}).
\end{proof}

On the $\kPROD$ hierarchy we will find the same phenomenon. In particular the case $k=1$ is also trivially decidable.

\begin{proposition}
    The "$\kPROD[1]$-separability problem" is decidable.
\end{proposition}
\begin{proof}
    Given two "synchronous relations" $\+R_1, \+R_2$, there exists $S \in $ \kPROD[1]
    that "separates@@rel" $\+R_1$ from $\+R_2$ if and only if $\pi_1(\+R_1)\times \pi_2(\+R_1)$
    "separates@@rel" $\+R_1$ from $\+R_2$.\footnote{Here,
    \(
        \pi_1(\+R_1) \defeq 
        \{u \in \Sigma^* \mid \exists v\in \Sigma^*,\,\tup{u,v} \in \+R_1\},
    \)
    and similarly,
    \(
        \pi_2(\+R_1) \defeq 
        \{v \in \Sigma^* \mid \exists u\in \Sigma^*,\,\tup{u,v} \in \+R_1\}.
    \)
    Both languages can be effectively computed from $\+R_1$.}
\end{proof}

As soon as $k>1$, the "$\kPROD$-separability problem" becomes undecidable.

\begin{lemma}
    A "symmetric@@rel" "synchronous relation" $\+R$ and the identity $\Id$ are "separable@@rel" by a relation in $\kPROD[2]$ iff they have a "separator@@rel" of the form $(A \times B) \cup (B \times A)$.
\end{lemma}
\begin{proof}
    Assume that $\+S \in \kPROD[2]$ "separates@@rel" $\+R$ from $\Id$.
    Then $\+R \subseteq \+S$, but since $\+R$ is symmetric, $\+R = \+R^{-1} \subseteq S^{-1}$,
    and so $\+R \subseteq \+S \cap \+S^{-1}$.
    Moreover, since $\+S$ has empty intersection with $\Id$, so does $\+S \cap \+S^{-1}$.
    Hence, $\+S \cap \+S^{-1}$ "separates@@rel" $\+R$ from $\Id$.

    Since $\+S \in \kPROD[2]$, there exists $A_1,A_2,B_1,B_2 \subseteq \Sigma^*$ such that
    $\+S = A_1 \times B_1 \cup B_2 \times A_2$.
    Note that $\+S \cap \Id = \emptyset$ yields $A_i \cap B_i = \emptyset$ for each $i \in \{1,2\}$.
    Finally:
    \begin{align*}
        \+S \cap \+S^{-1} &
        =
            \bigl( A_1 \times B_1 \cup B_2 \times A_2 \bigr)
            \cap \bigl( B_1 \times A_1 \cup A_2 \times B_2 \bigr) \\
        &
        =
            \bigl( (A_1 \times B_1) \cap (B_1 \times A_1) \bigr)
            \cup \bigl( (A_1 \times B_1) \cap (A_2 \times B_2) \bigr) \\
        &
        \hphantom{=\;} \cup \bigl( (B_2 \times A_2) \cap (B_1 \times A_1) \bigr)
            \cup \bigl( (B_2 \times A_2) \cap (A_2 \times B_2) \bigr) \\
        &
        =
            \bigl( \overbrace{(A_1 \cap B_1) \times (A_1 \cap B_1)}^{= \emptyset} \bigr)
            \cup \bigl( (A_1 \cap A_2) \times (B_1 \cap B_2) \bigr) \\
        &
        \hphantom{=\;} \cup \bigl( (B_1 \cap B_2) \times (A_1 \cap A_2) \bigr)
            \cup \bigl( \underbrace{(A_2 \cap B_2) \times (A_2 \cap B_2)}_{= \emptyset} \bigr)\\
        &
        = \bigl( (A_1 \cap A_2) \times (B_1 \cap B_2) \bigr)
            \cup \bigl( (B_1 \cap B_2) \times (A_1 \cap A_2) \bigr).
    \end{align*}
    Therefore, $\+S \cap \+S^{-1}$ is a "separator@@rel" of $\+R$ and $\Id$ of the desired shape.
\end{proof}

\begin{corollary}\AP\label{cor:2reg-2prod}
    A "symmetric@@rel" "synchronous relation" $\+R$ and $\Id$ are "separable@@rel" by a relation in $\kPROD[2]$ "iff" $\AutGraph{\Sigma^*}{\+R}$ is "$2$-regular colourable".
\end{corollary}

\begin{proof}
    By observing that for any "symmetric relation" $\+R \subseteq \Sigma^* \times \Sigma^*$, we have that $A,B \subseteq \Sigma^*$ is a "colouring" of $\AutGraph{\Sigma^*}{\+R}$ if, and only if, $(A \times B) \cup (B \times A)$ "separates@@rel" $\+R$ from $\Id$.
\end{proof}

We can now easily show undecidability for the "$\kPROD[2]$-separability problem" by reduction from the "$2$-regular colourability problem".
\begin{lemma}\AP\label{lem:aut-2prod-sep-undec}
    The "$\kPROD[2]$-separability problem" is undecidable.
\end{lemma}
\begin{proof}
    By reduction from the "$2$-regular colourability problem" on "rational graphs", which is undecidable by \Cref{thm:k-reg-col-undec}. Let $\AutGraph{V}{\+E}$ be an "rational graph" and $\AutGraph{V}{\+E'}$ the symmetric closure of $\AutGraph{V}{\+E}$. It follows that $\AutGraph{V}{\+E'}$ is still "rational@@struct" and that there is a "$2$-regular colouring" for $\AutGraph{V}{\+E'}$ "iff" there is a "$2$-regular colouring" for $\AutGraph{V}{\+E}$---take the same colouring.
    Thus, by \Cref{cor:2reg-2prod}, $\AutGraph{V}{\+E}$ is "$2$-regular colourable" "iff" 
    there is a $\kPROD[2]$ relation that "separates@@rel" $\+E'$ from $\Id$.
\end{proof}

Further, this implies undecidability for every larger $k$.
\begin{theorem}
    \AP\label{thm:kprod-undecidable}
    The "$\kPROD$-separability problem" is undecidable, for every $k \geq 2$.
\end{theorem}

\begin{figure}
    \centering
    \begin{tikzpicture}
        \tikzset{use Hobby shortcut, font=\small}

\newcommand{\curveA}{(0,0) .. (.5,-.3) .. (2,2) .. (1.1,.8)}
\newcommand{\curveB}{(.65,.9) .. (.55, 2) .. (-.1, 1.6) .. (-.9, 1.4) .. (-1.3, 1) .. (-1.1, .5) .. (0, 1) .. (.45, .9)};

% Separator
\draw[rounded corners=2pt, draw=cGrey, fill=cGrey, opacity=.5] (1.14,.85) rectangle (-.15,2.1);
\draw[rounded corners=2pt, draw=cGrey, fill=cGrey, opacity=.5] (-.15,1.7) rectangle (-1.4,.4);

% R2
\draw[
	closed,
	fill=cRed,
	draw=cRed,
	opacity=.5
] \curveA;

% R1
\draw[
	closed,
	fill=cBlue,
	draw=cBlue,
	opacity=.5
] \curveB;

% Labels
\node[cRed] at (1.5, -.6) {$\+R_2$};
\node[cBlue] at (-.7, 2) {$\+R_1$};
\node[cGrey] at (-1.65, 1.05) {$\+S$};

% Extra nodes
\foreach \x in {0,...,2} {
	\coordinate (ca\x) at ($(4.25,1.5)+(1*\x,0)$);
	\coordinate (cb\x) at ($(4.25,.5)+(1*\x,0)$);
}

% New separator
\draw[rounded corners=2pt, draw=cGrey, fill=cGrey, opacity=.5]
	($(ca0)+(0,-.15)$) rectangle ($(cb0)+(-.15,.15)$)
	($(ca1)+(0,-.15)$) rectangle ($(cb1)+(-.15,.15)$)
	($(ca2)+(0,-.15)$) rectangle ($(cb2)+(-.15,.15)$);
\node[cGrey, right] at (6.5, 1) {$\+S'\setminus \+S$};

% Extra edges
\foreach \x in {0,...,2} {
	\node[vertex] (a\x) at (ca\x) {};
		\node[above = 0cm of a\x] {$a_{\x}$};
	\node[vertex] (b\x) at (cb\x) {};
		\node[below = 0cm of b\x] {$b_{\x}$};
}

\foreach \x in {1,2} {
	\draw[edge, <->, cRed] (a0) to (b\x);
}
\foreach \x in {0,2} {
	\draw[edge, <->, cRed] (a1) to (b\x);
}
\foreach \x in {0,1} {
	\draw[edge, <->, cRed] (a2) to (b\x);
}
\foreach \x in {0,...,2} {
	\draw[edge, cBlue, bend right=20] (a\x) to (b\x);
	\draw[edge, cRed, bend right=20] (b\x) to (a\x);
}

\node[vertex] at (1.47, 2.55) (u) {};
\node[below=-.1em, align=center] at (1.47, 2.55) {$v \in \Sigma^*$};
\draw[edge, <-, cRed] (u) .. (1.64, 2.92) .. (2.3, 3.02) ..  ($(a0)+(-.1,.2)$) .. ($(a0)+(-.04,.08)$);
\foreach \x in {1,2} {
	\draw[edge, -, cRed] (u) .. (1.64, 2.92) .. (2.3, 3.02) ..  ($(a\x)+(-.1,.2)$) .. ($(a\x)+(-.04,.08)$);
}

\node[vertex] at (2.6, -.6) (v) {};
\node[below=-.1em, align=center] at (2.6, -.6) {$u \in \Sigma^*$};
\foreach \x in {0,...,2} {
	\draw[edge, <-, cRed] ($(b\x)+(-.08,-.05)$) .. ($(b\x)+(-.2,-.25)$) .. ($(b\x)+(-.3,-.7)$) .. (3.2, -.65) .. (2.9, -.6) ..  (v);
}
\node[vertex, fill=white] at (2.6, -.6) {};
    \end{tikzpicture}
    \caption{
        \AP\label{fig:2prod-to-kprod}
        Construction in the proof of \Cref{thm:kprod-undecidable} for $k = 5$. $\+S$ is depicted as the union of two grey rectangles since $\+S \in \kPROD[2]$.
        The relation $\+R'_1$ is obtained from $\+R_1$ (blue shape) by adding all blue edges,
        namely $a_i \to b_i$ for $1\leq i \leq k-2$. The relation $\+R'_2$ is obtained from $\+R_2$ (red shape) by adding
        all red edges, namely every other non-self-loop edge involving a vertex $a_i$ or $b_i$.
        Finally, $\+S'$ ($\+S$ plus three grey rectangles) is obtained from $S$ by adding
        each $\{a_i\} \times \{b_i\}$.
    }
\end{figure}

\begin{proof}
    The case $k=2$ is shown in \Cref{lem:aut-2prod-sep-undec}, so suppose $k>2$.
    The proof goes by reduction from the "$\kPROD[2]$-separability problem". Let $\+R_1,\+R_2$ be a pair of "synchronous relations" over an alphabet $\Sigma$. Consider the alphabet extended with $2(k-2)$ fresh symbols $\Sigma' = \Sigma \dcup \set{a_1, \cdots, a_{k-2}, b_1, \cdots, b_{k-2}}$. We build "synchronous relations" $\+R'_1,\+R'_2$ over $\Sigma'$ such that $\tup{\+R_1, \+R_2}$ are $\kPROD[2]$ "separable@@rel" over $\Sigma$ "iff" $\tup{\+R'_1, \+R'_2}$ are $\kPROD$ "separable@@rel" over $\Sigma'$.

    Let $\+R'_1 \defeq \+R_1 \dcup \set{\tup{a_i,b_i} : 1 \leq i \leq k-2}$ and 
    \begin{align*}
    \+R'_2 =  \+R_2 \dcup~&\set{\tup{a_i,v} \mid v \in \Sigma^*,\; i \in \lBrack 1,k-2\rBrack} \\
    \dcup~&\set{\tup{u,b_i} \mid u \in \Sigma^*,\; i \in \lBrack 1,k-2\rBrack}\\
    \dcup~&\set{\tup{a_i,b_j} \mid i,j \in \lBrack 1,k-2\rBrack \text{ and } i \neq j}\\
    \dcup~&\set{\tup{b_i,a_j} \mid i,j \in \lBrack 1,k-2\rBrack}
    \end{align*}
    
    If $\+R_1$ and $\+R_2$ have a $\kPROD[2]$ "separator@@rel" $\+S$, then $\+S \dcup \set{\tup{a_i,b_i} \mid i \in \lBrack 1,k-2\rBrack}$ is a $\kPROD$ "separator@@rel" of $\+R'_1$ and $\+R'_2$.
    

    Conversely, if $\+S' = (A_1 \times B_1) \cup \dotsb \cup (A_k \times B_k)$ is a $\kPROD$ "separator@@rel" of $\+R'_1$ and $\+R'_2$, then for every $i$ there must be some $j_i$ such that $A_{j_i} \times B_{j_i}$ contains $(a_i,b_i)$. From $\+S' \cap \+R'_2 = \emptyset$, we get:
    \begin{itemize}
        \item $A_{j_i} \cup B_{j_i}$ cannot contain any $a_{i'}$ or $b_{i'}$ for $i' \neq i$, and
        \item $A_{j_i} \cup B_{j_i}$ cannot contain any $w \in \Sigma^*$;
    \end{itemize}
    since otherwise we would have $(A_{j_i} \times B_{j_i}) \cap \+R'_2 \neq \emptyset$.
    Hence, $\set{i \mapsto j_i}_i$ is injective, and thus $\+S'$ is of the form $\+S' = (A_1 \times B_1) \cup (A_2 \times B_2)  \cup (\set{a_1} \times \set{b_1}) \cup \dotsb \cup (\set{a_{k-2}} \times \set{b_{k-2}})$. We can further assume that $A_1,B_1,A_2,B_2$ do not contain any $a_i$ or $b_i$ since otherwise we can remove them preserving the property of being a $\kPROD$ "separator@@rel" of $\+R'_1$ and $\+R'_2$.
    Hence, $\+S \defeq (A_1 \times B_1) \cup (A_2 \times B_2)$ must cover $\+R_1$ and be disjoint from $\+R_2$, obtaining that $\+S$ is a $\kPROD[2]$ "separator@@rel" of $\+R_1$ and $\+R_2$.
\end{proof}


\paragraph*{Membership for Bounded Recognizable Relations.}
Up until now, we have examined two hierarchies of bounded recognizable relations, namely $\kPROD$ and $\kREC$. 
Our previous analysis demonstrated that, for any element in these hierarchies (where $k>1$), their separability problem is undecidable. Nevertheless, 
we will now establish that their membership problem are decidable.

\AP Given an "synchronous" relation $\+R \subseteq \Sigma^* \times \Sigma^*$, consider the "synchronous" equivalence relation $\intro*\autequiv \subseteq \Sigma^* \times \Sigma^*$, defined as $w \mathrel{\autequiv} w'$ if for every $v \in \Sigma^*$ we have 
\begin{enumerate}
    \item $(w,v) \in \+R$ "iff" $(w',v) \in \+R$, and
    \item $(v,w) \in \+R$ "iff" $(v,w') \in \+R$.
\end{enumerate}

It turns out that equivalence classes of $\autequiv$ define the coarsest partition onto which $\+R$ can be recognized in terms of $\kREC$.

\begin{lemma}\AP\label{lem:krec-characterization}
    For every "synchronous" $\+R \subseteq \Sigma^* \times \Sigma^*$, $\autequiv$ has index at most $k$ if, and only if, $\+R$ is in $\kREC$.\footnote{Recall that the index of an equivalence relation is its number of equivalence classes.}
\end{lemma}
\begin{proof}
    \proofcase{Left-to-right.}
    Assume that $\autequiv$ has the equivalence classes $C_1, \cdots, C_k$. Consider the set $P \subseteq \lBrack 1,k\rBrack^2$ of all pairs $\tup{i,j}$ such that there are $u_i \in C_i$ and $u_j \in C_j$ with $\tup{u_i,u_j} \in \+R$. Define the $\kREC$ relation $\+R' = \bigcup_{(i,j) \in P} C_i \times C_j$. We claim that $\+R=\+R'$. 
    In fact, by definition of $\autequiv$, note that if there are $u_i \in C_i$ and $u_j \in C_j$ with $\tup{u_i,u_j} \in \+R$, then $C_i \times C_j \subseteq \+R$. Hence, $\+R' \subseteq \+R$.
    On the other hand, for every pair $\tup{u,v} \in \+R$ there exists $\tup{i,j} \in P$ such that $u \in C_i$, $v \in C_j$ implying $\tup{u,v} \in \+R'$.
    Hence, $\+R \subseteq \+R'$.

    \proofcase{Right-to-left.}
    If $\+R$ is a union of products of sets from the partition $C_1 \dcup \hdots \dcup C_k = \Sigma^*$, then every two elements of each $C_i$ are $\autequiv$-related, and thus $\autequiv$ has index at most $k$.
\end{proof}

We can then conclude that the membership problem for $\kREC$ is decidable. 

\begin{corollary}
    The "$\kREC$-membership problem" is decidable, for every $k \in \Np$.
\end{corollary}
\begin{proof}
    An "synchronous relation" $\+R$ is in $\kREC$ "iff" $\autequiv$ has at most $k$ equivalence classes by \Cref{lem:krec-characterization}. 
    % This can be tested with the formula $\forall w_1, \dotsc, w_{k+1} ~ \bigvee_{i \neq j} w_i \mathrel{\autequiv} w_j$.
    %    
    In other words, a "synchronous relation" $\+R$ is not in $\kREC$ "iff" the complement of $\autequiv$ contains a $(k+1)$-clique, which can be easily tested.
\end{proof}

The relation $\autequiv$ can also be used to characterize which synchronous relations are definable in the class $\kPROD$.

\begin{proposition}
    An "synchronous relation" $\+R$ is in $\kPROD$ if, and only if, $\+R=(A_1 \times B_1) \cup \hdots \cup (A_k \times B_k)$ where each $A_i$ and $B_i$ is a union of equivalence classes of $\autequiv$.
\end{proposition}
\begin{proof}
    The right-to-left implication is trivial. For the converse implication,
    assume that $\+R$ is in $\kPROD$, say
    \[
        \+R=(A_1 \times B_1) \cup \hdots \cup (A_k \times B_k)
    \]
    for some arbitrary "regular languages" $A_1,\hdots,A_k$ and $B_1,\hdots,B_k$.
    By definition of $\autequiv$, we also have
    \[
        \+R=(\equivclass{A_1}{\autequiv} \times \equivclass{B_1}{\autequiv})
        \cup \hdots \cup (\equivclass{A_k}{\autequiv} \times \equivclass{B_k}{\autequiv}).\qedhere
    \]
\end{proof}

Again, this characterization allows us to show that membership in the class $\kPROD$ is decidable. 

\begin{corollary}
    The "$\kPROD$-membership problem" is decidable, for every $k > 0$.
\end{corollary}

\begin{proof}
    By brute force testing whether the "synchronous relation" $\+R$ is equivalent to $(A_1 \times B_1) \cup \hdots \cup (A_k \times B_k)$ for every possible $A_i,B_i$ which is a union of equivalence classes of $\autequiv$.
\end{proof}
