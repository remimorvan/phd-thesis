\section{Undecidability of the Homomorphism Problems}
\label{sec:dichotomy-undecidability}

Recall that \Cref{thm:k-reg-col-undec} show that the "regular $k$-colourability problem"
is undecidable for all "automatic graphs". This problem can be rephrased in terms
of "homomorphisms", and naturally leads us to conjecture that
for most "target structure" $\?B$, the problem of given an "automatic presentation" $\+A$,
to decide if there is a "regular homomorphism" from $\+A$ to $\?B$ is undecidable.
This is formalized in \Cref{thm:dichotomy-theorem-automatic-structures},
and we devote the rest of this chapter to proving it.

\subsection{Overview \& Easy Implications of the Dichotomy Theorem}
\label{sec:dichotomy-overview}

\begin{mainstatement}
	\DichotomyThmDichotomyAutomatic
\end{mainstatement}

\begin{remark}[Beyond automatic relations]
	In the statement of \Cref{thm:dichotomy-theorem-automatic-structures},
	"automatic structures" can be replaced by
	"$\omega$-tree-automatic structures", or even
	"higher-order automatic structures"
	and the statement of the theorem would remain true.
	The undecidability results obviously remain true when defined on a larger class.
	Moreover, to prove decidability, notice that the key lemma of \Cref{sec:uniformly-first-order-definable-hom},
	namely \Cref{lemma:finite-duality-uniformly-definable-homomorphisms},
	actually deals with $\HomAll{\?B}$ and not $\HomAut{\?B}$.
	The decidability follows from the fact that the first-order theory of any
	"higher-order automatic structure" is decidable.
\end{remark}


\begin{marginfigure}
	\centering
	\begin{tikzpicture}
		\node (1) at (0,0) {\itemDTFinDual};
\node[below=2em of 1] (5) {\itemDTFirstOrder};
\node[below=2em of 5] (4) {\itemDTEqual};
\node[below left=2em and -1em of 4] (3) {\itemDTHomRegDec};
\node[below right=2em and -1em of 4] (2) {\itemDTHomDec};

% ---
% Transitions
% ---
\draw[implication] (1) to (5);
\draw[implication] (5) to (4);
\draw[implication] (4) to (3);
\draw[implication] (4) to (2);
\draw[implication, bend left=30] (3) to (1.west);
\draw[implication, bend right=30] (2) to (1.east);
	\end{tikzpicture}
	\caption{\AP\label{fig:dichotomy-overview}Implications shown in the chapter to prove
	\Cref{thm:dichotomy-theorem-automatic-structures}.}
\end{marginfigure}
We prove \Cref{thm:dichotomy-theorem-automatic-structures} by showing the
implications depicted in \Cref{fig:dichotomy-overview}.
The most difficult implications are $\itemDTFinDual \Rightarrow \itemDTFirstOrder$,
which we prove in \Cref{sec:dichotomy-decidability}, and the implications
$\itemDTHomDec \Rightarrow \itemDTFinDual$ and $\itemDTHomRegDec \Rightarrow \itemDTFinDual$,
which we prove by contraposition in \Cref{sec:undecidability-hom,sec:undecidability-homreg}.

On the other hand, the implications $\itemDTFirstOrder \Rightarrow \itemDTEqual$,
$\itemDTEqual \Rightarrow \itemDTHomRegDec$ and $\itemDTEqual \Rightarrow \itemDTHomDec$ are straightforward: we prove the first one in \Cref{sec:dichotomy-decidability},
and the last two in this section. 
Before showing these implications, we start by proving $\itemDTFinDual \Rightarrow \itemDTHomDec$.\footnote{While it is redundant with the implications of \Cref{fig:dichotomy-overview}, 
we prove this implication since not only is it straightforward, but it is also
the implication which, together with the fact that both $\HomAut{\clique{2}}$
and $\HomRegAut{\clique{2}}$ are undecidable by
\cite[Proposition 6.5]{Kocher2014AutomatischenGraphen}
and \Cref{thm:k-reg-col-undec}, that lead us to conjecture
\Cref{thm:dichotomy-theorem-automatic-structures}.}

\paragraph*{Decidability of the Homomorphism Problem.}

\begin{proposition}
	\!\footnote{This corresponds to the implication $\itemDTFinDual \Rightarrow \itemDTHomDec$
	of \Cref{thm:dichotomy-theorem-automatic-structures}.}
	\AP\label{prop:dichotomy-FinDual-implies-HomDec}
	Let $\?B$ be a "finite $\sigma$-structure".
	If $\?B$ has finite duality, then $\HomAut{\?B}$ is decidable in "NL".
\end{proposition}

\begin{proof}
	Given a "finite $\sigma$-structure" $\?D$ with domain $\{d_1,\dotsc,d_n\}$,
	we build the "first-order sentence"
	\[
		\phi_{\?D} \defeq \exists x_1.\; \cdots \exists x_n.\;
		\bigwedge_{\+R_{(k)} \in \sigma} \bigwedge_{\substack{\tup{i_1,\dotsc,i_k} \in \intInt{1,n}^k\\\text{"st" }\tup{d_{i_1},\dotsc,d_{i_k}} \in \+R(\?D)}}
		\+R(x_{i_1},\dotsc,x_{i_k}).
	\]
	By construction, for any arbitrary "$\sigma$-structure" $\?A$, we have $\?A \FOmodels \phi_{\?D}$
	"iff" $\?D \homto \?A$.
	Then, since $\?B$ has "finite duality", it admits a finite "dual"
	$\?D_1,\dotsc,\?D_m$.
	Then
	\[
		\?A \FOmodels \bigwedge_{i=1}^m \neg \phi_{\?D_i}
		\quad\text{"iff"}\quad 
		\?A \homto \?B.
	\]
	The conclusion follows from the fact that the "data complexity"
	of "first-order model checking of automatic structures" is "NL"
	by \Cref{prop:first-order-model-checking-automatic-structures}.
\end{proof}

Note that \Cref{prop:dichotomy-FinDual-implies-HomDec} still holds
when the "homomorphism problem" takes as input "higher-order automatic
structures" since such structures have a decidable first-order theory.

\paragraph*{Equality of the Homomorphism Problems Imply their Decidability.}

\begin{proposition}
	\!\footnote{This corresponds to the implication $\itemDTEqual \Rightarrow \itemDTHomDec$
	of \Cref{thm:dichotomy-theorem-automatic-structures}.}
	\AP\label{prop:dichotomy-Equal-implies-HomDec}
	Let $\?B$ be a "finite $\sigma$-structure".
	If $\HomAut{\?B} = \HomRegAut{\?B}$ then $\HomAut{\?B}$ and $\HomRegAut{\?B}$
	are decidable.
\end{proposition}

To prove this, we first give an upper bound on the homomorphism problems independently
of any assumption on $\?B$.
\begin{proposition}
	\AP\label{prop:dichotomy-general-upper-bounds}
	Let $\?B$ be a "finite $\sigma$-structure".
	Then $\HomAut{\?B}$ is "coRE" and $\HomRegAut{\?B}$ is "RE".
\end{proposition}

\begin{proof}
	\proofcase{$\HomAut{\?B}$ is "coRE".}
	By the "De Bruijn–Erdős Theorem", for any arbitrary "$\sigma$-structure" $\?A$,
	we have $\?A \nothomto \?B$ "iff" there exists a finite "substructure" $\?A'$ of $\?A$
	"st" $\?A' \nothomto \?B$.
	Given a "finite $\sigma$-structure" $\?A'$ and an "automatic $\sigma$-structure",
	it is decidable to test whether $\?A'$ is a "substructure" of $\?A$: indeed, it suffices
	to check, using \Cref{prop:first-order-model-checking-automatic-structures} if
	\[
		\?A
		\FOmodels^{?}
		\Bigl(
			\exists x_1.\; \cdots \exists x_n.\;
			\bigwedge_{\+R_{(k)} \in \sigma} \bigwedge_{\substack{\tup{i_1,\dotsc,i_k} \in \intInt{1,n}^k\\\text{"st" }\tup{a_{i_1},\dotsc,a_{i_k}} \in \+R(\?A')}}
			\+R(x_{i_1},\dotsc,x_{i_k})
		\Bigr),
	\]
	by letting $\{a_1,\dotsc,a_n\} = A'$. 
	Moreover, whether $\?A' \nothomto^? \?B$ is also decidable in "coNP".
	Overall, this provides a co-semi-algorithm for $\HomAut{\?B}$: we enumerate finite
	"$\sigma$-structure" $\?A'$, and test if (1) $\?A'$ is a "substructure" of $\?A$ and if (2)
	$\?A' \homto \?B$. And hence $\HomAut{\?B}$ is "coRE".

	\proofcase{$\HomRegAut{\?B}$ is "RE".} This is an easy generalization of
	\Cref{thm:k-reg-col-undec}: instead of having "$k$-coloured automaton", we define the notion of
	``$\?B$-automata'': the notion of accepting state is replaced by a partitition $\langle C_{b} \rangle_{b \in B}$ of the states.
	The semantics of such an automaton is a partial function $f\colon \Sigma^* \to B$. 
	Given an "automatic structure" $\+A$, we can then effectively test if $f$ is defined on $\domainPres{\+A}$, and if $f$ defines a "homomorphism" from $\?A$ to $\?B$---see
	the proof of \Cref{thm:k-reg-col-undec}. If so, $\+A \homregto \?B$. Dually, any "regular homomorphism" $\+A \homregto \?B$ can be described by such a $\?B$-automaton.
	Therefore, $\HomRegAut{\?B}$ is "RE".
\end{proof}

\begin{proof}[Proof of \Cref{prop:dichotomy-Equal-implies-HomDec}.]
	If $\HomAut{\?B} = \HomRegAut{\?B}$, then by \Cref{prop:dichotomy-general-upper-bounds},
	these problems are both "RE" and "coRE", and are hence decidable.
\end{proof}

\subsection{Undecidability of \,$\HomAut{\?B}$}
\label{sec:undecidability-hom}

We now prove the undecidability of $\HomAut{\?B}$ and $\HomRegAut{\?B}$
when $\?B$ does not have "finite duality". Both reductions are
direct adaptations of the proof that $\HomFin{\?B}$ is "L"-hard when $\?B$ does not
have finite duality by \textcite[Theorem 3.2]{LaroseTesson2009UniversalAlgebraCSP}.
However, proving the undecidability of the problem that is reduced
to $\HomRegAut{\?B}$ is not entirely trivial and requires some work.
\begin{itemize}
	\item For $\HomFin{\?B}$, we reduce the complement of "Connectivity in automatic graphs",
		providing a "coRE"-lowerbound.
	\item For $\HomRegAut{\?B}$, we reduce "regular unconnectivity in automatic graphs",
		which in turn is reduced from the "regular reachability problem".
\end{itemize}

For $n\in\N$, we define the \AP""$n$-link"" $\intro*\link{n}$ be the "$\sigma$-structure"\sidenote{From \cite[\S~2]{LaroseLotenTardif2007CharacterisationFOCSP}.} 
whose domain is $\intInt{0,n}$, and every "predicate" $\+R$
of arity $k$, is interpreted as the set of tuples $\langle a_1,\, \dotsc,\, a_k \rangle$
"st" $|a_i-a_j| \leq 1$ for all $i,j \in \intInt{0,n}$. See \Cref{fig:n-link}.
\begin{marginfigure}
	\centering
	\begin{tikzpicture}
		\node[vertex] (0) at (0,0) {};
\node[vertex, right=of 0] (1) {};
\node[vertex, right=of 1] (2) {};
\node[vertex, right=of 2, opacity=0] (3) {};
\node[vertex, right=of 3] (4) {};
\node[yshift=-.1em] at (3) (3-label)  {$\cdots$};

% ---
% Transitions
% ---
\foreach \i in {0,1,2,3} {
	\pgfmathtruncatemacro{\next}{\i + 1}
	\draw[edge, bend right=30] (\i) to (\next);
	\draw[edge, bend right=30] (\next) to (\i);
};
\foreach \i in {0,1,2,4} {
	\draw[edge, loop, out=60,in=120,looseness=6] (\i) to (\i);
}
	\end{tikzpicture}
	\caption{\AP\label{fig:n-link}The "$n$-link" $\link{n}$ over the "graph signature".}
\end{marginfigure}
Given a "$\sigma$-structure" $\?B$, say that $b \in \?B$ and $b'$ are
\AP""$n$-linked"" if there exists a "homomorphism" from $\link{n}$ to $\?B$
that sends $0$ to $b$ and $n$ to $b'$. We say that $b$ and $b'$ are \AP\reintro{linked} if
they are "$n$-linked" for some $n \in \N$.

Note that the fact that $k \mapsto n-k$
defines an "automorphism" of $\link{n}$ implies that the relation of being "$n$-linked"---%
and to a greater extent of being "linked"---is symmetric.
Moreover, being "linked" is transitive, but not necessarily reflexive.

% \begin{fact}[Link Composition]
% 	\AP\label{fact:link-composition}
% 	Let $n,m \in \N$, let $\?B$ be a "$\sigma$-structure", and let $b,b',b'' \in B$.
% 	If $b$ and $b'$ are "$n$-linked" and $b'$ and $b''$ are "$m$-linked", then
% 	$b$ and $b''$ are "$(n+m)$-linked".
% \end{fact}
% We define the binary relation $\sim_n$
% on $\link{n} \prodstruct \iterstruct{\?B}{2}$ as follows:\sidenote{From \cite[\S~4.3]{LaroseLotenTardif2007CharacterisationFOCSP}.}%
% \[
% 	\langle k,\, b_1,\, b_2\rangle \sim_n \langle k',\, b'_1,\, b'_2\rangle
% 	\quad\text{when}\quad
% 	\begin{cases*}
% 		\;\langle k,\,  b_1,\, b_2\rangle = \langle k',\, b'_1,\, b'_2\rangle\text{, or}\\
% 		\;k = k' = 0 \text{ and } b_1 = b'_1\text{, or}\\
% 		\;k = k' = n \text{ and } b_2 = b'_2.
% 	\end{cases*}
% \]

\begin{proposition}[{\cite[Theorem 4.7]{LaroseLotenTardif2007CharacterisationFOCSP}}]%
	\!\footnote{Actually \cite[Theorem 4.7]{LaroseLotenTardif2007CharacterisationFOCSP} assumes
	that $\HomFin{\?B}$ is "first-order definable", but this condition
	is equivalent to $\?B$ having "finite duality" by Atserias' result
	\cite[Corollary 4]{Atserias2008DigraphColoring}.}%
	%%%
	\AP\label{prop:characterization-finite-duality-path-projections}
	An arbitrary "$\sigma$-structure" $\?B$ has "finite duality" "iff"
	$\projHom{1}$ and $\projHom{2}$ are "linked" in $\powstruct{\?B}{(\iterstruct{\?B}{2})}$.
\end{proposition}

Equipped with the previous proposition, we can now show the undecidability 
of $\HomAut{\?B}$ by reduction from the following problem.

\decisionproblem{""Connectivity in Automatic Graphs""}{
	An "automatic presentation" $\•G$ of a "directed graph",
	and two elements $s,t \in \Sigma^*$.
}{
	Are $\•G(s)$ and $\•G(t)$ "connected" in $\?G$?
}
\begin{property}
	\AP\label{prop:undecidability-connectivity}
	"Connectivity in automatic graphs" is "RE"-complete.
\end{property}

\begin{proof}
	This follows from the fact that the "configuration graph" of
	a "Turing machine" is always "automatic@@struct" by \Cref{ex:turing-machine-are-automatic}.
	Indeed, a "Turing machine" halts on the empty word "iff" there is, in its
	"configuration graph", a path from the "initial configuration"
	to $\textbullet$, where $\textbullet$ is a newly added node,
	"st" we add an edge from any accepting "configuration@@TM" to $\textbullet$. 
\end{proof}

\begin{lemma}
	\AP\label{lem:reduction-hom}
	Assume that $\sigma$ contains at least one "predicate" of arity at least 2,
	and let $\?B$ be a "finite $\sigma$-structure".
	If $\?B$ does not have "finite duality", then there is a "first-order reduction" 
	from the complement of "connectivity in automatic graphs" to $\HomAut{\marked{\?B}}$.
\end{lemma}

\begin{marginfigure}
	\centering
	\begin{tikzpicture}
		\node[vertex] (0) at (0,0) {};
\node[vertex, right = 2.5 em of 0] (1) {};
\node[vertex, above right = 1em and 2em of 1] (2) {};
\node[vertex, below right = 1em and 2em of 1] (3) {};

\draw[edge] (1) to (2);
\draw[edge, bend right=15] (2) to (3);
\draw[edge, bend right=15] (3) to (2);
\draw[edge] (1) to (3);
	\end{tikzpicture}
	\caption{
		\AP\label{fig:graph-to-struct-graph}
		A "graph@@dir" $\?G$.
	}
\end{marginfigure}
\begin{marginfigure}
	\centering
	\begin{tikzpicture}
		\node[vertex] (0) at (0,0) {};
\node[vertex, right = 2.5 em of 0] (1) {};
\node[vertex, above right = 1em and 2em of 1] (2) {};
\node[vertex, below right = 1em and 2em of 1] (3) {};

\draw[edge, bend right=15] (1) to (2);
\draw[edge, bend right=15] (2) to (1);
\draw[edge, bend right=15] (2) to (3);
\draw[edge, bend right=15] (3) to (2);
\draw[edge, bend right=15] (1) to (3);
\draw[edge, bend right=15] (3) to (1);

\draw[edge, loop, out=150,in=210,looseness=6] (1) to (1);
\draw[edge, loop, out=60,in=120,looseness=6] (2) to (2);
\draw[edge, loop, out=-120,in=-60,looseness=6] (3) to (3);
	\end{tikzpicture}
	\caption{
		\AP\label{fig:graph-to-struct-struct}
		The "structure" $\?A$ defined from $\?G$ (in \Cref{fig:graph-to-struct-graph}),
		using the construction done in the proof of \Cref{lem:reduction-hom},
		when $\sigma$ consists of a single binary relation.
	}
\end{marginfigure}

\begin{proof}
	Given an instance $\langle \•G, s, t \rangle$ of "connectivity in automatic graphs",
	we first define the $\sigma$-structure $\?A$ with "automatic presentation" $\•A$
	obtained by replacing every edge of $\•G$ by a "$1$-link".
	Formally, $\?A$ has the same domain as $\?G$, and for any
	"predicate" $\+R \in \sigma$ of arity $k$,
	$\langle g_1,\, \dotsc,\, g_k \rangle \in \+R(\?A)$ "iff"
	$\{g_1, \dotsc, g_k\} = \{g,g'\}$ for some $g,g' \in G$ "st"
	there is an edge from $g$ to $g'$ in $\?G$.
	See \Cref{fig:graph-to-struct-graph,fig:graph-to-struct-struct}.

	\begin{claim}
		\AP\label{claim:reduction-hom-from-graph-to-link}
		$\•G(s)$ and $\•G(t)$ are "connected" "iff"
		$\•A(s)$ and $\•A(t)$ are "linked".
	\end{claim}
	For the left-to-right implication: if there is an edge between two elements
	in $\?G$, then they are "$1$-linked" in $\?A$. Since being "linked" is
	reflexive and transitive, the conclusion follows.
	Conversely, if two elements $a$ and $a'$ of $\?A$ are "$1$-linked", 
	then pick a "predicate" $\+R \in \sigma$ of arity at least 2.
	Then $\langle a,\, \dotsc,\, a,\, a' \rangle \in \+R(\?A)$,
	and so by definition of $\?A$ there is either an edge from $a$ to $a'$
	or from $a'$ to $a$ in $\?G$.\sidenote{Note that the proof of this claim
	is the only part of the proof of \Cref{lem:reduction-hom} that requires
	the assumption that $\sigma$ contains at least one "predicate" of arity at least 2.}

	We then consider the "automatic $\sigma$-structure" $\?A\prodstruct \iterstruct{\?B}{2}$,
	and extend it to a
	"$\extendedSignature{\sigma}{\?B}$-structure" \AP\(\intro*\ConstrUndecHom{(\?A\prodstruct \iterstruct{\?B}{2})}\)
	in which for each $b_0 \in B$,
	we "interpret@@predicate" the unary "predicate" $\unarypred{b_0}$ as
	\[
		\big\{
			\langle \•A(s),\, b_0,\, b\rangle \;\big\vert\; b \in B
			\big\}
		\cup
		\big\{
			\langle \•A(t),\, b,\, b_0\rangle \;\big\vert\; b \in B
		\big\}
	\]
	To construct an "automatic presentation" for this structure, see \Cref{sec:construction-automatic-presentations}.
	\begin{claim}
		\AP\label{claim:reduction-hom-direct}
		If there is a "homomorphism" from
		$\ConstrUndecHom{(\?A\prodstruct \iterstruct{\?B}{2})}$
		to $\marked{\?B}$,
		then $\•G(s)$ and $\•G(t)$ are not "connected" in $\?G$.
	\end{claim}
	Let $f\colon \ConstrUndecHom{(\?A\prodstruct \iterstruct{\?B}{2})} \homto \marked{\?B}$
	be a "homomorphism".\sidenote{Recall that both sides are
	"$\extendedSignature{\sigma}{\?B}$-structures".}
	It induces a "homomorphism"
	\[
		\overbar f\colon \?A\prodstruct \iterstruct{\?B}{2} \to \?B
	\]
	between "$\sigma$-structures", and by currying (\Cref{prop:currying-hom}),
	$\overbar f$ can be seen as a "homomorphism"
	\[
		F\colon \?A \to \powstruct{\?B}{(\iterstruct{\?B}{2})}.
	\]
	Note moreover that because $\overbar f$ comes from a "homomorphism" between
	"$\extendedSignature{\sigma}{\?B}$-structures" then we must have  
	$f(\•A(s),\, b,\, b') = b$
	and $f(\•A(t),\, b,\, b') = b'$ for all $b,b' \in B$.
	This implies that $F(\•A(s)) = \projHom{1}$ and $F(\•A(t)) = \projHom{2}$.
	
	We now assume by contradiction that
	$\+G(s)$ and $\+G(t)$ are "connected", and hence by
	\Cref{claim:reduction-hom-from-graph-to-link}
	there is some $n \in \N$
	"st" there is a "homomorphism" $g\colon \link{n} \to \?A$
	with $g(0) = \•A(s)$ and $g(n) = \•A(t)$.
	Then by composition, we obtain a "homomorphism"
	\[
		F \circ g\colon
		\link{n} \to \powstruct{\?B}{(\iterstruct{\?B}{2})},
 	\]	
	which sends $0$ to $F(g(0)) = F(\•A(s)) = \projHom{1}$
	and sends $n$ to $F(g(n)) = F(\•A(t)) = \projHom{2}$.
	So, by \Cref{prop:characterization-finite-duality-path-projections},
	$\?B$ would have "finite duality", which is a contradiction.
	Hence, $\•A(s)$ and $\•A(t)$ are not "linked",
	and so by \Cref{claim:reduction-hom-from-graph-to-link}, $\•G(s)$ and $\•G(t)$
	are not "connected".

	\begin{claim}
		\AP\label{claim:reduction-hom-converse}
		If $\•G(s)$ and $\•G(t)$ are not "connected" in $\?G$,
		then there is a "homomorphism" from
		$\ConstrUndecHom{(\?A\prodstruct \iterstruct{\?B}{2})}$ to $\marked{\?B}$.
	\end{claim}
	We define a "homomorphism" $f\colon \ConstrUndecHom{(\?A\prodstruct \iterstruct{\?B}{2})} \to \marked{\?B}$ by:
	\[
		f(a, b, b') \defeq \begin{cases*}
			\;b & \text{ if $\•A(s)$ and $a$ are "linked",} \\
			\;b' & \text{ otherwise.}
		\end{cases*}
	\]
	We show that this is indeed a "homomorphism": for any "predicate" $\+R$
	of arity $k$ in $\sigma$, if
	\[
		\langle a_1,\, b_1,\, b'_1 \rangle,\;
		\langle a_2,\, b_2,\, b'_2 \rangle,\;
		\dotsc,\;
		\langle a_k,\, b_k,\, b'_k \rangle
	\]
	are all "$\+R$-tuples" of $\ConstrUndecHom{(\?A\prodstruct \iterstruct{\?B}{2})}$,
	then by definition of $\?A$, we have that either (1) all $a_i$'s are equal,
	or (2) $\{a_1,\, \dotsc,\, a_k\} = {a,a'}$ for some $a \neq a' \in A$
	and there is an edge from $a$ to $a'$ or from $a'$ to $a$ in $\?G$.
	In both cases, it follows that $\•A(s)$ and $a_i$ are "linked"
	"iff" $\•A(s)$ and $a_j$ are "linked", for all $i,j\in \intInt{1,k}$.
	Hence, either:
	\begin{itemize}
		\item $f(a_i,\, b_i,\, b'_i) = b_i$ for all $i\in \intInt{1,k}$
			(if all $a_i$'s are "connected" to $\+A(s)$), or
		\item $f(a_i,\, b_i,\, b'_i) = b'_i$ for all $i\in \intInt{1,k}$ (otherwise).
	\end{itemize}
	In both cases, we get that
	\[
		\Big\langle
			f(a_1,\, b_1,\, b'_1),\;
			f(a_2,\, b_2,\, b'_2),\;
			\dotsc,\;
			f(a_k,\, b_k,\, b'_k)
		\Big\rangle
		\in \+R(\?B).
	\]
	We also need to show that this map preserves the new unary "predicates" of
	$\extendedSignature{\sigma}{\?B}$: this follows from---and is in fact equivalent to---the
	fact that $\•A(s)$ and $\•A(t)$ are not "linked" by \Cref{claim:reduction-hom-from-graph-to-link}:
	indeed, the currying of $f$ behaves like $\projHom{1}$ on the "connected component"
	of $\+A(s)$ and like $\projHom{2}$ on its complement.
	Overall, this proves that $\ConstrUndecHom{(\?A\prodstruct \iterstruct{\?B}{2})} \homto \marked{\?B}$.

	Putting \Cref{claim:reduction-hom-direct,claim:reduction-hom-converse} together,
	we get that the reduction is correct.
	Lastly, note that it is a "first-order reduction":
	clearly, one can go from $\?G$ to $\?A$ via a (one-dimensional) "first-order reduction",
	and then from $\?A$ to $\?A\prodstruct \iterstruct{\?B}{2}$
	via a (multi-dimensional) "first-order reduction" since $\?B$ is finite,
	and lastly 
	$\ConstrUndecHom{(\?A\prodstruct \iterstruct{\?B}{2})}$ can be obtained
	by a "first-order reduction" from the latter "structure" since
	"first-order logic" can test equality with a fixed element.
\end{proof}

By \Cref{prop:undecidability-connectivity}, the complement of "Connectivity in automatic graphs"
is "coRE"-complete, and assuming that $\sigma$ contains at least one "predicate" of arity 2,
it reduces by \Cref{lem:reduction-hom} to any problem $\HomAut{\marked{\?B}}$ when $\?B$ has "finite duality". In turn, by \Cref{prop:idempotent-core-preserves-csp-complexity}, it reduces to
$\HomAut{\?B}$, which is thus "coRE"-hard. It remains to deal with "signatures" consisting of only
unary "predicates".\sidenote{It is not clear to us whether this case was properly handled in
\cite{LaroseLotenTardif2007CharacterisationFOCSP}.}

\begin{property}
	\AP\label{prop:finite-duality-unary-predicates}
	If $\sigma$ only consists of unary "predicates", then all "$\sigma$-structures"
	have "finite duality".
\end{property}	

\begin{proof}
	Fix a $\sigma$-structure $\?B$. We define the \AP""unary type""
	$\intro*\unaryType{b}{\?B}$ of $b \in \?B$
	to be the set of "predicates" $\+P$ "st" $b \in \+P(\?B)$.
	
	Given $\tau \subseteq \sigma$, define \AP$\intro*\structOfUnaryType{\tau}$
	to be the "$\sigma$-structure"
	consisting of a single element $*$, and "st" $* \in \+P(\?1_\tau)$ "iff"
	$\+P \in \tau$.
	We say that $\tau$ is \AP""obstructing@@unarytype"" if
	$\tau \not\subseteq \unaryType{b}{\?B}$ for all $b \in \?B$.

	\begin{claim}
		\AP\label{claim:finite-duality-unary-predicates-direct}
		If $\tau$ is "obstructing@@unarytype",
		then $\structOfUnaryType{\tau} \nothomto \?B$.
	\end{claim}
	We prove the result by contraposition.
	Any "homomorphism" from $\structOfUnaryType{\tau}$ to $\?B$
	should send $*$ on some element $b$ of $\?B$
	"st" $b \in \+P(\?B)$ for all $\+P \in \tau$, and
	hence $\tau \subseteq \unaryType{b}{\?B}$.

	\begin{claim}
		\AP\label{claim:finite-duality-unary-predicates-converse}
		If $\?A \nothomto \?B$ then there exists an "obstructing@@unarytype"
		$\tau \subseteq \sigma$ "st" $\structOfUnaryType{\tau} \homto \?A$.
	\end{claim}
	We define a partial homomorphism $f$ from $A$ to $B$,
	by sending $a \in A$ to any $b \in B$ "st" the "unary type" of $a$
	is included in the "unary type" of $b$. This is clearly a (partial) "homomorphism",
	and so since $\?A \nothomto \?B$, it follows that it must be partial,
	"ie" that some element $a \in \?A$ "st" $\unaryType{a}{\?A} \not\subseteq
	\unaryType{b}{\?B}$ for every $b \in B$. It follows that $\unaryType{a}{\?A}$
	is "obstructing@@unarytype". Since $\structOfUnaryType{\unaryType{a}{\?A}} \homto \?A$
	"via" $* \mapsto a$, the conclusion follows.

	Putting \Cref{claim:finite-duality-unary-predicates-direct,claim:finite-duality-unary-predicates-converse} together, we get that
	\[
		\big\{\;
			\structOfUnaryType{\tau}
			\;\big\vert\;
			\text{ $\tau \subseteq \sigma$ is "obstructing@@unarytype"} 
		\;\big\}
	\]
	is a finite "dual" for $\?B$.
\end{proof}

\begin{corollary}
	\!\sidenote[][-15em]{In the case of Larose and Tesson, they study the problem
	$\HomFin{-}$, and prove in \cite[Theorem 3.2]{LaroseTesson2009UniversalAlgebraCSP}
	that there is a "first-order reduction" from "Connectivity in Finite Graphs" to
	$\HomFin{\marked{\?B}}$ for any $\?B$ that does not have "finite duality".
	Moreover, "Connectivity in Finite Graphs" is "L"-hard under "first-order reductions" since
	Etessami \cite[Theorem 3.2]{Etessami1997CountingLogSpace} proved that the problem
	of given a directed path and two vertices $s$, $t$ to decide if there is a path from
	$s$ to $t$ is "L"-hard under "first-order reductions"-in fact even under quantifier-free reductions. It turn, this problem can be reduced in "first-order@@reduction"
	to "Connectivity in Finite Graphs" \cite{SamiD2015USTCONNLogspace}.
	Overall, and together with \Cref{prop:idempotent-core-preserves-csp-complexity},
	this shows that $\HomFin{\?B}$ is "L"-hard under "first-order reductions".}%
	%%%
	\AP\label{coro:lowerbound-hom}
	If $\?B$ does not have "finite duality", then $\HomAut{\?B}$
	is "coRE"-hard.
\end{corollary}

\begin{proof}
	By \Cref{prop:finite-duality-unary-predicates}, since $\?B$ does not have "finite duality",
	then $\sigma$ has at least one "predicate" of arity at least 2.
	The conclusion follows from \Cref{prop:idempotent-core-preserves-csp-complexity,prop:undecidability-connectivity,lem:reduction-hom}.
\end{proof}

\subsection{Undecidability of \,$\HomRegAut{\?B}$}
\label{sec:undecidability-homreg}

The reduction to show undecidability of is nearly identical to \Cref{lem:reduction-hom},
but the input problem differs quite a lot.
\decisionproblem{""Regular Unconnectivity in Automatic Graphs""}{
	An "automatic presentation" $\•G$ of a "directed graph" $\?G$,
	and two elements $s,t \in \Sigma^*$.
}{
	Is there a regular language $L \subseteq \Sigma^*$ 
	such that $s \in L$, $t\not\in L$ and $L$ is a union of "connected components"
	of $\•G$?\footnotemark{}
	In this case we say that $s$ and $t$ are \AP""regularly unconnected"".
}
\footnotetext{Formally, we mean that $L = \•G^{-1}[U]$ for some union $U$ of "connected components" of $\?G$.}

We will first reduce this problem to $\HomRegAut{\?B}$,
and will later settle its complexity.

\begin{lemma}
	\AP\label{lem:reduction-hom-reg}
	Assume that $\sigma$ contains at least one "predicate" of arity at least 2.
	If $\?B$ does not have "finite duality", then there is a "first-order reduction"
	from "regular unconnectivity in automatic graphs"
	to $\HomRegAut{\marked{\?B}}$.
\end{lemma}

\begin{proof}
	Given an instance $\langle \•G, s, t \rangle$ of "regular unconnectivity in automatic graphs",
	we first define the $\sigma$-structure $\?A$ with "automatic presentation" $\•A$
	obtained by replacing every edge by a "$1$-link", as in \Cref{lem:reduction-hom}.

	\begin{claim}
		\!\footnote{While ``being "linked"'' is not reflexive in general, it is over the
		structure $\?A$, by reflexivity of ``being "connected"'' in $\•G$.}%
		\AP\label{claim:reduction-homreg-from-graph-to-link}
		$\•G(s)$ and $\•G(t)$ are "regularly unconnected" "iff"
		there is a "regular language" $L \subseteq \Sigma^*$ "st" $\•A(s)\in L$ and $t \not\in L$,
		and $L$ is a union of equivalences classes of $\domainPres{\•A}$
		under ``being "linked"''.
	\end{claim}
	The proof is similar to \Cref{claim:reduction-homreg-from-graph-to-link}.
	Then again, we reduce the instance $\langle \•G, s, t \rangle$
	to an "automatic presentation" of \(\ConstrUndecHom{(\?A\prodstruct \iterstruct{\?B}{2})}\),
	as in \Cref{lem:reduction-hom}.
	\begin{claim}
		\AP\label{claim:reduction-homreg-direct}
		If $\ConstrUndecHom{(\•A\prodpres \iterstruct{\?B}{2})} \homregto \marked{\?B}$,
		then $\•G(s)$ and $\•G(t)$ are "regularly unconnected" in $\?G$.
	\end{claim}
	
	Let \(f\colon \ConstrUndecHom{(\•A\prodpres \iterstruct{\?B}{2})} \to \marked{\?B}\)
	be a "regular homomorphism".
	By currying---see \Cref{coro:homreg-currying}---of the underlying "homomorphism"
	between "\(\sigma\)-structures", we obtain a "regular homomorphism"
	\[
		F\colon \•A \to \powstruct{\?B}{(\iterstruct{\?B}{2})}.
	\]
	Moreover, using the "predicates" \(\unarypred{b}\), \(b \in B\),
	we get that $F(\•A(s)) = \projHom{1}$ and $F(\•A(s)) = \projHom{2}$.

	We then define \[\+X \defeq \{g \in \powstruct{\?B}{(\iterstruct{\?B}{2})} \mid \text{ $g$ and $\projHom{1}$ are "linked" or $g = \projHom{1}$}\}.\]
	We claim that ${F}^{-1}[\+X]$ witnesses the fact that
	$\•G(s)$ and $\•G(t)$ are "regularly unconnected".
	First, $\projHom{1} \in \+X$ so $\•A(s) \in {F}^{-1}[\+X]$.
	Since $\?B$ has "finite duality", by \Cref{prop:characterization-finite-duality-path-projections}, $\projHom{2} \not\in \+X$
	and so $\•A(t) \not\in {F}^{-1}[\+X]$.
	Then, ${F}^{-1}[\+X]$ is "regular@@lang" since $F$ is a "regular homomorphism". Finally, ${F}^{-1}[\+X]$ is a union of
	equivalences classes of $\domainPres{\•A}$ under ``being "linked"''.\footnote{Indeed,
	if $c_1, c_2 \in \?C$ are "linked" in some "structure" $\?C$ and if $f\colon \?C \to \?D$ is a "homomorphism", then $f(c_1)$ and $f(c_2)$ are "linked" in $\?D$.}
	Hence, by \Cref{claim:reduction-homreg-from-graph-to-link}, $\•G(s)$ and $\•G(t)$ are "regularly unconnected".

	\begin{claim}
		\AP\label{claim:reduction-homreg-converse}
		If $\•G(s)$ and $\•G(t)$ are "regularly unconnected" in $\?G$,
		then $\•A\prodpres \iterstruct{\?B}{2} \homregto \marked{\?B}$.
	\end{claim}

	Since $\•G(s)$ and $\•G(t)$ are "regularly unconnected" in $\?G$,
	by \Cref{claim:reduction-homreg-from-graph-to-link} there is a "regular language" $L \subseteq \Sigma^*$ "st" $\•A(s)\in L$ and $\•A(t) \not\in L$,
	and $L$ is a union of equivalences classes of $\domainPres{\•A}$
	under ``being "linked"''.
	We define a function $f\colon \domainPres{\•A}\times B^2 \to B$ by 
	\[
		f(a, b, b') \defeq \begin{cases*}
			\;b & \text{ if $\•A(s) \in L$,} \\
			\;b' & \text{ otherwise,}
		\end{cases*}
	\]
	and we claim that $f$ is a "regular homomorphism" from
	\(\•A\prodpres \iterstruct{\?B}{2}\) to \(\marked{\?B}\).
	The proof that it is a "homomorphism" is similar to \Cref{claim:reduction-hom-converse}:
	in particular, we use the fact that $\•G(s)$ and $\•G(t)$ are not "connected" in $\?G$,
	which is a consequence of the fact that they are "regularly unconnected".
	"Regularity@@hom" follows from the "regularity@@lang" of $L$. 
	Hence, $\•A\prodpres \iterstruct{\?B}{2} \homregto \marked{\?B}$.

	Putting \Cref{claim:reduction-homreg-direct,claim:reduction-homreg-converse} together,
	we get that the reduction is correct.
	Lastly, note that this is a first-order reduction for the same reason
	as \Cref{lem:reduction-hom}.
\end{proof}

We then prove a lower bound on the complexity of "regular unconnectivity in automatic graphs".

\begin{lemma}
	\AP\label{lemma:regular-unconnectivity-lowerbound}
	"Regular unconnectivity in automatic graphs" is "RE"-hard.
\end{lemma}

\begin{marginfigure}%
	\centering
	\begin{tikzpicture}
		% Reachable configuration
\fill[rounded corners, draw=cGreen, fill=cGreen, opacity=.3]
	(-.3,-.3) rectangle (2.75, .3);
% Initial configuration
\fill[rounded corners, draw=cYellow, fill=cYellow, opacity=.3]
	(-.3,.3) rectangle (.3, -1.95);

% First line
\node[vertex] (a0) at (0,0) {};
\foreach \i in {0,...,2} {
	\pgfmathtruncatemacro{\next}{\i + 1}
	\node[vertex, right = of a\i] (a\next) {};
	\draw[edge] (a\i) to (a\next);
}

% Second line
\node[vertex, below = of a0] (b0) {};
\foreach \i in {0,...,3} {
	\pgfmathtruncatemacro{\next}{\i + 1}
	\node[vertex, right = of b\i] (b\next) {};
	\draw[edge] (b\i) to (b\next);
}
\node[draw=none, fill=none, right = 0cm of b4] (binf) {$\cdots$};

% Third line
\node[vertex, below = of b0] (c0) {};
\foreach \i in {0,1} {
	\pgfmathtruncatemacro{\next}{\i + 1}
	\node[vertex, right = of c\i] (c\next) {};
	\draw[edge] (c\i) to (c\next);
}

% Labels
\node[draw=none, fill=none, font=\small] [below = 1em of c0] {$\GermC{cYellow}$};
\node[draw=none, fill=none, font=\small, align=center] [above = 1em of $(a1)!0.5!(a2)$]
	{$\ReachC{\+T}{cGreen}$};
	\end{tikzpicture}
	\caption{
		\AP\label{fig:reduction-regreach-to-regunconnect-lhs}
		"Configuration graph" of a "linear Turing machine".
		(Replica of \Cref{fig:reduction-wf-RTM-to-colouring-config-graph-wf-RTM}.)
	}
\end{marginfigure}%
\begin{marginfigure}%
	\centering
	\begin{tikzpicture}
		% First line
\node[vertex, draw=c0, fill=c0, fill opacity=.4] (a0) at (0,0) {};
\foreach \i in {0,...,2} {
	\pgfmathtruncatemacro{\next}{\i + 1}
	\node[vertex, right = of a\i, draw=c0, fill=c0, fill opacity=.4] (a\next) {};
	\draw[edge] (a\i) to (a\next);
}

% Second line
\node[vertex, below = of a0, draw=c1, fill=c1, fill opacity=.4] (b0) {};
\foreach \i in {0,...,3} {
	\pgfmathtruncatemacro{\next}{\i + 1}
	\node[vertex, right = of b\i, draw=c1, fill=c1, fill opacity=.4] (b\next) {};
	\draw[edge] (b\i) to (b\next);
}
\node[draw=none, fill=none, right = 0cm of b4] (binf) {$\cdots$};

% Third line
\node[vertex, below = of b0, draw=c1, fill=c1, fill opacity=.4] (c0) {};
\foreach \i in {0,1} {
	\pgfmathtruncatemacro{\next}{\i + 1}
	\node[vertex, right = of c\i, draw=c1, fill=c1, fill opacity=.4] (c\next) {};
	\draw[edge] (c\i) to (c\next);
}

\node[vertex, below = of c0, draw=c1, fill=c1, fill opacity=.4] (star) {};

\draw[edge, <->] (b0) to (c0);
\draw[edge, <->] (c0) to (star);
\draw[edge, <->, bend right=45] (b0) to (star);

\node[above=0em of a0] {$s$};
\node[below=0em of star] {$t$};


% Labels
% \node[draw=none, fill=none, font=\small] [below = 1em of c0] {$\GermC{\+T}{cYellow}$};
% \node[draw=none, fill=none, font=\small, align=center] [above = 1em of $(a1)!0.5!(a2)$]
% 	{$\ReachC{\+T}{cGreen}$};
	\end{tikzpicture}
	\caption{
		\AP\label{fig:reduction-regreach-to-regunconnect-rhs}
		The instance of "regular unconnectivity in automatic graphs"
		to which the "Turing machine" of \Cref{fig:reduction-regreach-to-regunconnect-lhs}
		is reduced. Colours indicate the different "connected components".
	}
\end{marginfigure}%

\begin{proof}
	We reduce the "regular reachability problem" on "linear Turing machines",
	which is "RE"-hard by \Cref{lem:regular-reachability},
	to "regular unconnectivity in automatic graphs".

	Given a "linear Turing machine" $\+T$ with
	configuration graph $\confGraph = \tup{V, \+E}$,
	we reduce it to the "automatic graph" $\+G = \tup{V', \+E'}$ where:
	$V' \defeq V \dcup \{\textbullet\}$ and
	$\+E'$ is the union of $\+E$ with the "clique" that
	puts in relation all vertices that are either of the form $\textbullet$
	or that are "germinal" but not the "initial configuration".
	We then pick $s$ to be the "initial configuration", and $t$ to be $\textbullet$.

	By construction, $\+G$ is "automatic@@struct" and has exactly two connected components:
	$\Reach{\+T}$ (containing $s$) and its complement (containing $t$).
	Hence, the only union of "connected components" of $\+G$ that contains $s$
	but not $t$ is $\Reach{\+T}$.
	And hence, $\+T$ is a positive instance of the "regular reachability problem"
	if, and only if, $\Reach{\+T}$ is "regular@@lang", "ie" $\tup{\+G,s,t}$
	is a positive instance of "regular unconnectivity in automatic graphs".
\end{proof}


\begin{corollary}
	\AP\label{coro:lowerbound-homreg}
	If $\?B$ does not have "finite duality", then $\HomAut{\?B}$
	is "RE"-hard.
\end{corollary}

\begin{proof}
	Recall that $\HomAut{\?B} = \HomAut{\core{\?B}}$, so we assume "wlog"
	that $\?B$ is a "core".
	By \Cref{lemma:regular-unconnectivity-lowerbound},
	"regular unconnectivity in automatic graphs" is "RE"-hard.
	Then by \Cref{prop:finite-duality-unary-predicates}, since $\?B$ does not have "finite
	duality", $\sigma$ does not consist only of unary "predicates",
	and hence by \Cref{lem:reduction-hom-reg}, we get
	a reduction from "regular unconnectivity in automatic graphs" to
	$\HomRegAut{\marked{\?B}}$, which in turns
	reduces to $\HomRegAut{\?B}$ by \Cref{prop:idempotent-core-preserves-csp-complexity}
	since $\?B$ was assumed to be a "core".
	Indeed, "first-order reductions" preserves "regularity@@hom", by 
	\Cref{prop:automatic-first-order}.
\end{proof}