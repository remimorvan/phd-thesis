\section{Preliminaries}

For \AP""first-order reductions"", see
\cite[Definition 2.11 \& Definition 1.26]{Immerman1998DescriptiveComplexity}.

\subsection{Undirected Paths}

An \AP""undirected path"" in a "$\sigma$-structure" $\?A$ consists of a sequence
\[\big\langle a_0,\, e_0,\, a_1,\, \hdots,\, e_{n-1},\, a_n\big\rangle, \text{ with } n \in \N,\]
where $a_i \in \?A$ and each $e_i$ is a "hyperedge" of $\?A$ "st" both
$a_i$ and $a_{i+1}$ occur in $e_i$. When such an "undirected path" exists, we say that
there is an "undirected path" between $a_0$ and $a_n$, or equivalently
that $a_0$ and $a_n$ are \AP""connected"".\sidenote{Note that this relation is reflexive
and symmetric.} A \AP"connected component" of $\?A$ consists of an equivalence class of the 
transitive closure of this relation.

\subsection{Constructions on Structures}

Given two "structures" $\?A$ and $\?B$, we define the structure \AP$\intro*\powstruct{\?B}{\?A}$ as follows:
\begin{itemize}
  \item its domain are "homomorphisms" $\?A \to \?B$,
  \item for every "relation symbol" $\+R$ of arity $k$, for any homomorphism $f_1,\hdots,f_k$,
  we have $\langle f_1,\hdots,f_k\rangle \in \+R$ when for every
  $\langle a_1,\hdots,a_k\rangle \in \+R(\?A)$
  then $\langle f_1(a_1), \hdots, f_k(a_k) \rangle \in \+R(\?B)$.
\end{itemize}


\begin{proposition}[Folklore: Currying Homomorphisms]
	\AP\label{prop:currying-hom}
	Given "structures" $\?A$, $\?B$ and $\?C$, if $f\colon \?A\prodstruct \?B \to \?C$
	is a "homomorphism", then $F\colon \?A \to \powstruct{\?C}{\?B}$,
	defined by $a \mapsto (b \mapsto f(a,b))$, is a "homomorphism".
	In fact, this mapping $f \mapsto F$ is a bijection
	between "homomorphisms" $\?A\prodstruct \?B \to \?C$
	and "homomorphisms" $\?A \to \powstruct{\?C}{\?B}$.
\end{proposition}

\begin{proof}
Let $\+R$ be a relational symbol of arity $k$, and let
$\langle a_1,\hdots,a_k \rangle \in \+R(\?A)$.
We want to show that $\langle F(a_1),\hdots,F(a_k) \rangle \in \+R(\powstruct{\?C}{\?B})$:
for any $\langle b_1,\hdots,b_k \rangle \in \+R(\?B)$, we have
\[\langle F(a_1)(b_1), \hdots, F(a_k)(b_k)\rangle = \langle f(a_1,b_1),\hdots,f(a_k,b_k) \rangle \in \+R(\?C)\] since $f$ is a "homomorphism" from $\?A\prodstruct \?B$ to $\?C$.
Hence, $F$ is indeed a "homomorphism" from $\?A$ to $\powstruct{\?C}{\?B}$.

Dually, if $F$ is a "homomorphism" from $\?A$ to $\powstruct{\?C}{\?B}$,
we define $f\colon \?A\prodstruct \?B \to \?C$ by $\langle a,b \rangle \mapsto F(a)(b)$,
and claim that $f$ is a "homomorphism". Indeed, if $\+R$ be a relational symbol of arity $k$,
for any $\langle a_1, \hdots, a_k \rangle \in \+R(\?A)$
and $\langle b_1, \hdots, b_k \rangle \in \+R(\?B)$,
we have $\langle f(a_1,b_1), \hdots, f(a_k,b_k) \rangle
= \langle F(a_1)(b_1), \hdots, F(a_k)(b_k) \rangle$.
Since $\langle F(a_1), \hdots, F(a_k) \rangle \in \+R(\powstruct{\?C}{\?B})$
and $\langle b_1, \hdots,b_k \rangle \in \+R(\?B)$ 
it follows that $\langle F(a_1)(b_1), \hdots, F(a_k)(b_k) \rangle \in \+R(\?C)$.
Therefore, $f$ is a "homomorphism" from $\?A \prodstruct \?B$ to $\?C$.

It is then routine to check that the maps $f \mapsto F$ and $F \mapsto f$ defined
in the two previous paragraphs are mutually inverse bijections.
\end{proof}

\subsection{Constructions on Automatic Presentations}

Let $\•A$ and $\•B$ be "automatic presentations" of some "$\sigma$-structures"
$\?A$ and $\?B$. We define \AP$\•A \intro*\prodpres \•B$ to be the "presentation@@auto"
"st":
\begin{align*}
	\domainPres{\•A\prodpres \•B} & \defeq \{u\#v \mid u \in \domainPres{\•A} \land v \in \domainPres{\•B}\}\\
	\relPres{\•A\prodpres \•B}{\+R} & \defeq \{(u_1\#v_1) \convol \cdots \convol (u_k\#v_k) \mid
		u_1 \convol \cdots \convol u_k \in \relPres{\•A}{\+R} \land
		v_1 \convol \cdots \convol v_k \in \relPres{\•B}{\+R}
	\}
\end{align*}
for each "relation symbol" $\+R$ of arity $k$ in $\sigma$.
It is an "automatic presentation" of $\?A \prodstruct \?B$.

$\•A \prodpres \•B \homregto \•C$ does not depend on the "automatic presentation"
of $\?B$ nor of $\?C$: this is used implicitly in \Cref{sec:undecidability-homreg}.
Also add reminder there + notation.
Curryfying in this case.


\subsection{Idempotent Core}

We fix a "purely relational signature" $\sigma$.
Given a "$\sigma$-structure" $\?B$,
we denote by \AP$\intro*\extendedSignature{\sigma}{\?B}$
the signature obtained from $\sigma$ by adding
a unary predicate \AP$\intro*\unarypred{b}$ for each $b\in B$.
The "marked structure" \AP$\intro*\marked{\?B}$ of $\?B$ is the
"$\extendedSignature{\sigma}{\?B}$-structure"
obtained from $\?B$ by "interpreting@@symbol" each predicate $\unarypred{b}$ as the
singleton $\{b\}$.\sidenote{TODO: ASIA'S REMARK: this is called the ``idempotent core''.}

\begin{proposition}[Folklore]
\!\sidenote{The non-easy part is to reduce $\HomFin{\marked{\?B}}$ to $\HomFin{\?B}$: only this reduction
requires the assumption that $\?B$ is a core. This reduction is folklore, see "eg"
\cite[Lemma 2.5]{LaroseTesson2009UniversalAlgebraCSP}.
TODO: understand if the proof is indentical to us. We provide here a self-contained proof.}%
\sidenote{The "marked structure" $\marked{\core{\?B}}$ of the "core" of $\?B$ is
usually called the \AP""idempotent core""
of $\?B$. By TODO:addref and this proposition, $\HomFin{\?B}$ and
$\HomFin{\marked{\core{\?B}}}$ are equivalent under "first-order reductions".
This reduction is a central tool
in the algebraic approach to understand "constraint satisfaction
problem" since the algebra associated to the "CSP" over an "idempotent core"
only has idempotent operations, making it much easier to work with. See TODO:addref for
more details.}%
\AP\label{prop:marking-preserves-csp-complexity}
If $\?B$ is a finite "core", then the problems $\HomFin{\marked{\?B}}$ and 
$\HomFin{\?B}$ are equivalent under "first-order reductions".
\end{proposition}

\begin{proof}[Proof of \Cref{prop:marking-preserves-csp-complexity}]
\proofcase{Reduction from $\HomFin{\?B}$ to $\HomFin{\marked{\?B}}$.}
We reduce a "$\sigma$-structure" $\?A$ to the
"$\extendedSignature{\sigma}{\?B}$-structure" $\?A'$ obtained
from $\?A$ by "interpreting@@symbol" each predicate $\unarypred{b}$ as the emptyset.
Clearly, a function from $A$ to $B$ is a "homomorphism" from $\?A$ to $\?B$
"iff" it is a "homomorphism" from $\?A'$ to $\marked{\?B}$, proving the correctness
of the reduction. It is, by definition, "first-order@@reduction".

\proofcase{Reduction from $\HomFin{\marked{\?B}}$ to $\HomFin{\?B}$.}
We first define the reduction $\Phi$ and show its correctness, we will show later that it
is a "first-order reduction". We reduce a "$\extendedSignature{\sigma}{\?B}$-structure" $\?A$ to the "$\sigma$-structure"
$\Phi(\?A)$ defined as follows:
\begin{itemize}
	\item its underlying universe is the disjoint union $A \sqcup B$,
	\item given a "relation symbol" $\+R$ of arity $k$, its "hyperedges" are:
	\begin{itemize}
	\item all $\+R$-"hyperedges" of $A$,
	\item all $\+R$-"hyperedges" of $B$, and
	\item all $\+R$-"hyperedges" $(b_1,\hdots,b_{i-1}, a_i, b_{i+1},\hdots,b_k)$
		"st" there exists $b_i$ for which the $\+R$-"hyperedge"
		$(b_1,\hdots,b_{i-1}, b_i, b_{i+1},\hdots,b_k)$
		is in $\+R(\?A)$, and $a_i$ belongs to the "interpretation@@symbol" of 
		$\unarypred{b_i}$ in $\?A$.
	\end{itemize}
\end{itemize}
Note that by construction, the neighbourhood (TODO: TO BE DEFINED) of $a \in A$ in $\Phi(\?A)$ is
the union of its neighbourhood in $\?A$, and the union of the neighbourhoods of
$b$ in $\?B$ for all $b$ "st" $a \in \unarypred{b}(\?A)$.
To be concluded.  
\end{proof}