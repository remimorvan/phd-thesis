\section{Preliminaries}
\label{sec:dichotomy-preliminaries}

\subsection{Regular Homomorphisms}

Given two regular languages $K$ and $L$,%
\footnote{Note that whether $K$ and $L$ share the same alphabet is irrelevant since we can always work on the union of their alphabet.}
a \AP""regular function"" from
$K$ to $L$ is a function $f\colon K \to L$ "st" the relation
\[
	\{\tup{u, f(u)} \mid u \in K\}
\]
is "automatic@@rel".
A \AP""regular homomorphism"" between two "presentations of automatic $\sigma$-structures"
$\•A$ and $\•B$ is a "regular function" from $\domainPres{\•A}$ to $\domainPres{\•B}$
that defines a "homomorphism" from $\?A$ to $\?B$.\footnote{We use the terminology ``regular''
instead of ``automatic'' simply because ``automatic homomorphism'' sounds somewhat weird.}\footnote{Note that "regular homomorphisms" are to "automatic structures" what ``definable homomorphisms''
are to ``definable structures'' in \cite{KlinLasotaOchremiakTorunczyk2016HomomorphismProblems}: they form a restriction on the notion of "homomorphisms" to make then finitely describable---although the "source structure" might be infinite---, using the same logical framework as the one used to describe the "source structure".}%
\footnote{The related notion of ``regular isomorphism''
(under the name ``automatic isomorphism'') was studied in
\cite[Definition~6.10]{KhoussainovNerode1995AutomaticPresentations}. They showed that for any
"$\sigma$-structure" $\?A$,
there are either zero, one, or $\omega$ many "automatic presentations" of $\?A$
up to ``regular isomorphism'' \cite[Theorem~6.8]{KhoussainovNerode1995AutomaticPresentations}.}
We denote by \AP $\•A \intro*\homregto \•B$ the existence of a "regular homomorphism" from
$\•A$ to $\•B$.

\begin{property}
	\label{prop:regular-function-finite-domain}
	Let $f\colon K \to L$ be a function, where $L$ is finite.
	Then $f$ is a "regular function" "iff" for every $v \in L$,
	$f^{-1}[v]$ is a "regular language".
\end{property}

\begin{proof}
	For the left-to-right implication, if $f$ is a "regular function",
	then there exists a first-order formula $\phi(x,y)$ "st"
	for all $\tup{u,v} \in K\times L$, then
	\[
		\univStructSynchronous{\Sigma}, u, v \FOmodels \phi(x,y)
		\text{ "iff" }
		v = f(u).
	\]
	Then given $v\in L$,
	the formula\footnote{``$y = v$'' is not properly defined in the syntax of "first-order logic" but it is straightforward to come up with a formula expressing this property.}
	\[
		\exists y,\, \phi(x,y) \land y = v
	\]
	describes $\{ u\in K \mid f(u) = v \}$, which is hence "regular@@lang".

	Conversely, we consider "first-order formulas" $\phi_{v}$ describing each set
	$\{ u\in K \mid f(u) = v \}$, with $v\in L$. Then
	\[
		\phi(x) \defeq \bigwedge_{v\in L} 
			\phi_{b}(x) \land y = v
	\]
	is a "first-order formula"---since $L$ is finite---describing $f$.
\end{proof}

As a consequence, when $\?B$ is finite, the existence of a "regular homomorphism" to $\?B$
is independent of its "presentation@@auto", and we will hence write $\•A \reintro*\homregto \?B$ 
to mean that $\•A \homregto \•B$.%
\footnote{Unsurprisingly, this property cannot be extended to all "automatic structures":
for instance letting $\+A$ be the "automatic presentation" $\tup{0^*, \set{\tup{0^n,0^{n+1} \mid n \in \N}}}$ of $\tup{\N, \textrm{succ}}$,
and letting $\+A'$ be the "automatic presentation"
$\tup{(00)^*, \set{\tup{0^{2n},0^{2n+2} \mid n \in \N}}}$ of the same "structure",
it can be shown that $\+A \homregto \+A$ but $\+A \nothomregto \+A'$.}
In particular, a \AP""regular $k$-colouring"" of a "presentation of an automatic graph"
$\•G$ is equivalently a "regular homomorphism" from $\•G$ to $\?K_k$,
or a "$k$-colouring" of $\?G$ "st" for any colour, the set of words of $\domainPres{\•G}$
sharing this colour is a "regular language". 

\begin{figure}
	\centering
	\begin{tikzpicture}
		\tikzset{every node/.style={fill opacity=.4}}

\node[vertex, draw=c1, fill=c1] at (0,0) (0-0) {};
\foreach \j in {1,...,4} {
	\pgfmathtruncatemacro{\prev}{\j - 1}
	\pgfmathtruncatemacro{\jc}{mod(\j,2) ? 2 : 1}
	\node[vertex, draw=c\jc, fill=c\jc, right=2em of 0-\prev] (0-\j) {};
}
\foreach \i in {1,...,4} {
	\pgfmathtruncatemacro{\prev}{\i - 1}
	\foreach \j in {0,...,4} {
		\pgfmathtruncatemacro{\jc}{mod(\i+\j,2) ? 2 : 1}
		\node[vertex, draw=c\jc, fill=c\jc, above=2em of \prev-\j] (\i-\j) {};
	}
}
\foreach \i in {0,...,4} {
	\node[right=2em of \i-4] (\i-5) {};
}
\foreach \j in {0,...,4} {
	\node[above=2em of 4-\j] (5-\j) {};
}

\foreach \i in {0,...,4} {
	\foreach \j in {0,...,4} {
		\node[below right=-.25em of \i-\j, font=\tiny, color=cDarkGrey] {$x^\j y^\i$};
	}
}

\foreach \i in {0,...,4} {
	\foreach \j in {0,...,4} {
		\draw[edge] (\i-\j) to (\i-\the\numexpr\j+1\relax);
		\draw[edge] (\i-\j) to (\the\numexpr\i+1\relax-\j);
	}
}
	\end{tikzpicture}
	\caption{
		\AP\label{fig:grid-2-reg-colouring}
		A "regular $2$-colouring" of an "automatic presentation" of the infinite quarter-grid.
	}
\end{figure}
In \Cref{fig:grid-2-reg-colouring}, we provide an example of a "regular $2$-colouring":
we let $\+G = \tup{V,\+E}$ be the "automatic presentation" of the infinite quarter-grid,
defined over the alphabet $\{x,y\}$ by $V \defeq x^*y^*$ and
\[
	\+E = \set{\tup{x^p y^q, x^{p+1}y^q} \mid p,q \in \N}
	\cup
	\set{\tup{x^p y^q, x^p y^{q+1}} \mid p,q \in \N}.
\]
Then the unique "$2$-colouring" of $\+G$ assigns one colour%
\footnote{In yellow in \Cref{fig:grid-2-reg-colouring}.} to the
vertices of the form $x^p y^q$ "st" $p-q$ is even.
This "colouring" is of course "regular@@colouring".

Given a fixed "signature" $\sigma$ and a $\sigma$-structure $\?B$,
we denote by:
\begin{itemize}
	\itemAP $\intro*\HomFin{\?B}$ (resp. $\intro*\HomAut{\?B}$, resp.
	$\intro*\HomAll{\?B}$) the class of all "finite $\sigma$-structure"
	(resp. "automatic $\sigma$-structure", resp. arbitrary "$\sigma$-structures")
	that admit a "homomorphism" to $\?B$,
	\itemAP $\intro*\HomRegAut{\?B}$ is the class of all "automatic presentations of  $\sigma$-structures" that admit a "regular homomorphism" to $\?B$.
\end{itemize}
Somewhat abusively, we identify these classes with the associated decision problems---except
for $\HomAll{\?B}$ since arbitrary "$\sigma$-structures" cannot be encoded as finite strings.
For finite structures, we assume the input to be given using adjacency lists, and for "automatic 
structures", we assume the input to be described by an "automatic presentation".
We call $\HomRegAut{\?B}$ the \AP""regular homomorphism problem"" over $\?B$.


\subsection{Constructions on Structures}

Given two "structures" $\?A$ and $\?B$, we define the structure \AP$\intro*\powstruct{\?B}{\?A}$ as follows:
\begin{itemize}
  \item its domain are "homomorphisms" $\?A \to \?B$,
  \item for every "predicate" $\+R$ of arity $k$, for any homomorphism $f_1,\dotsc,f_k$,
  we have $\langle f_1,\dotsc,f_k\rangle \in \+R(\powstruct{\?B}{\?A})$ when 
  \[
	\text{for every }
	\langle a_1,\dotsc,a_k\rangle \in \+R(\?A)
	\text{, we have }
	\langle f_1(a_1), \dotsc, f_k(a_k) \rangle \in \+R(\?B).
  \]
\end{itemize}


\begin{proposition}[Folklore: Currying Homomorphisms]
	\AP\label{prop:currying-hom}
	Given "structures" $\?A$, $\?B$ and $\?C$, if $f\colon \?A\prodstruct \?B \to \?C$
	is a "homomorphism", then $F\colon \?A \to \powstruct{\?C}{\?B}$,
	defined by $a \mapsto (b \mapsto f(a,b))$, is a "homomorphism".
	In fact, this mapping $f \mapsto F$ is a bijection
	between "homomorphisms" $\?A\prodstruct \?B \to \?C$
	and "homomorphisms" $\?A \to \powstruct{\?C}{\?B}$.
\end{proposition}

\begin{proof}
Let $\+R$ be a "predicate" of arity $k$, and let
$\langle a_1,\dotsc,a_k \rangle \in \+R(\?A)$.
We want to show that $\langle F(a_1),\dotsc,F(a_k) \rangle \in \+R(\powstruct{\?C}{\?B})$:
for any $\langle b_1,\dotsc,b_k \rangle \in \+R(\?B)$, we have
\[\langle F(a_1)(b_1), \dotsc, F(a_k)(b_k)\rangle = \langle f(a_1,b_1),\dotsc,f(a_k,b_k) \rangle \in \+R(\?C)\] since $f$ is a "homomorphism" from $\?A\prodstruct \?B$ to $\?C$.
Hence, $F$ is indeed a "homomorphism" from $\?A$ to $\powstruct{\?C}{\?B}$.

Dually, if $F$ is a "homomorphism" from $\?A$ to $\powstruct{\?C}{\?B}$,
we define $f\colon \?A\prodstruct \?B \to \?C$ by $\langle a,b \rangle \mapsto F(a)(b)$,
and claim that $f$ is a "homomorphism". Indeed, if $\+R$ be a predicate of arity $k$,
for any $\langle a_1, \dotsc, a_k \rangle \in \+R(\?A)$
and $\langle b_1, \dotsc, b_k \rangle \in \+R(\?B)$,
we have $\langle f(a_1,b_1), \dotsc, f(a_k,b_k) \rangle
= \langle F(a_1)(b_1), \dotsc, F(a_k)(b_k) \rangle$.
Since $\langle F(a_1), \dotsc, F(a_k) \rangle \in \+R(\powstruct{\?C}{\?B})$
and $\langle b_1, \dotsc,b_k \rangle \in \+R(\?B)$ 
it follows that $\langle F(a_1)(b_1), \dotsc, F(a_k)(b_k) \rangle \in \+R(\?C)$.
Therefore, $f$ is a "homomorphism" from $\?A \prodstruct \?B$ to $\?C$.

It is then routine to check that the maps $f \mapsto F$ and $F \mapsto f$ defined
in the two previous paragraphs are mutually inverse bijections.
\end{proof}

\subsection{Constructions on Automatic Presentations}
\label{sec:construction-automatic-presentations}

Let $\•A$ and $\•B$ be "automatic presentations" of some "$\sigma$-structures"
$\?A$ and $\?B$, over alphabets $\Sigma$ and $\Gamma$, respectively.
We define \AP$\•A \intro*\prodpres \•B$ to be the "presentation@@auto"
over the alphabet $(\Sigma \times \Gamma) \dcup (\Sigma \times \{\pad\}) \dcup (\{\pad\} \times 
\Gamma)$  "st":
\begin{align*}
	\domainPres{\•A\prodpres \•B} & \defeq \{u\convol v \mid u \in \domainPres{\•A} \land v \in \domainPres{\•B}\}\\
	\relPres{\+R}{\•A\prodpres \•B} & \defeq
		\{\tup{u_1\convol v_1,\, \dotsc,\, u_k\convol v_k} \mid
		\tup{u_1,\, \dotsc,\, u_k} \in \relPres{\+R}{\•A} \land
		\tup{v_1,\, \dotsc,\, v_k} \in \relPres{\+R}{\•B}
	\}
\end{align*}
for each "predicate" $\+R$ of arity $k$ in $\sigma$.
It is an "automatic presentation" of $\?A \prodstruct \?B$.
Indeed, given a "first-order formula" $\phi(x_1,\dotsc,x_k)$
over $\signatureSynchronous{\Sigma}$, describing $\relPres{\+R}{\•A}$,
and a "first-order formula" $\psi(x_1,\dotsc,x_k)$
over $\signatureSynchronous{\Sigma}$, describing $\relPres{\+R}{\•B}$,
we let $\phi^*$ (resp. $\psi^*$) be the "formula@@FO" obtained from $\phi$ (resp. $\psi$)
by substituting $\lastLetter{a}(x)$ for $\bigvee_{b \in \Gamma \dcup \{\pad\}} \lastLetter{\tup{a,b}}(x)$ (resp. $\bigvee_{b \in \Sigma \dcup \{\pad\}} \lastLetter{\tup{b,a}}(x)$).
Then $\phi^* \land \psi^*$ is a "first-order formula", over the product alphabet,
that describes $\relPres{\+R}{\•A\prodpres \•B}$. The same construction
works for $\domainPres{\•A\prodpres \•B}$.
This shows that if $\?A$ and $\?B$ are "automatic $\sigma$-structures",
then so is $\?A \times \?B$.\footnote{Note that the "block product", that is defined similarly to the "Cartesian product" except that we ask one tuple to be in the relation, and the other to be in the equality relation---"ie" $\tup{u_1,\, \dotsc,\, u_k} \in \relPres{\+R}{\•A}$ and
$v_1 = \cdots = v_k$, or its symmetric version---is also "automatic@@struct", using the same construction.
For instance, the infinite quarter-grid of \Cref{fig:grid-2-reg-colouring}
can also be constructed as the "block product" of $\tup{\N,+1}$ with itself.}

\begin{proposition}
	\label{prop:homreg-prod-finite}
	Let $\?A$, $\?B$ and $\?C$ be "automatic $\sigma$-structures", such that
	$\?B$ and $\?C$ are finite.
	Let $\•A$ (resp. $\•B$ and $\•B'$, resp. $\•C$ and $\•C'$) be an "automatic presentation"
	of $\?A$ (resp. $\?B$, resp. $\?C$).
	Then $\•A \prodpres \•B \homregto \•C$ "iff" $\•A \prodpres \•B' \homregto \•C'$.
\end{proposition}

\begin{proof}
	The proof follows from the following claim, which can be proven
	exactly in the same fashion as \Cref{prop:regular-function-finite-domain}.

	\begin{claim}
		\label{claim:homreg-prod-finite}
		Assuming again that $\?B$ and $\?C$ are finite,
		a function $f\colon\•A \prodpres \•B \to \•C$ is a "regular homomorphism"
		"iff" for every $b\in \domainPres{\•B}$, for every $c\in \domainPres{\•C}$,
		\(\{
			a\in \domainPres{\•A} \mid f(a,b) = c
		\}\)
		is a "regular language".\qedhere
	\end{claim}
\end{proof}

In other words, the existence of a "regular homomorphism" does not depend on the
"automatic presentation" of the \emph{finite} "structures" that are involved, but only
on the "structure" they represent.
As a consequence of \Cref{prop:homreg-prod-finite}, we write
\(\•A \prodpres \?B \homregto \?C\) as a synonym for \(\•A \prodpres \•B \homregto \•C\).

\begin{corollary}[Currying]
	\label{coro:homreg-currying}
	Let $\?A$, $\?B$ and $\?C$ be "automatic $\sigma$-structures",
	and let $\•A$ be an "automatic presentation" of $\?A$.
	Then $\•A \prodpres \?B \homregto \?C$ "iff" $\•A \homregto \powstruct{\?C}{\?B}$.
\end{corollary}

\begin{proof}
	This also follows from \Cref{claim:homreg-prod-finite}.
\end{proof}

\subsection{Idempotent Core}

\begin{figure}
	\centering
	\begin{tikzpicture}
		\node[vertex, draw=c0, fill=c0, fill opacity=.4] (9) at (0,0) {};
\node[vertex, below right=of 9] (10) {};
\node[vertex, above right=of 10, draw=c1, fill=c1, fill opacity=.4] (11) {};
\node[vertex, below right=of 11] (12) {};
\node[vertex, below right=of 12] (13) {};

% ---
% Transitions
% ---
\foreach \i in {9,11,12} {
	\pgfmathtruncatemacro{\next}{\i + 1}
	\draw[edge] (\i) to (\next);
};
\foreach \i in {10} {
	\pgfmathtruncatemacro{\next}{\i + 1}
	\draw[edge] (\next) to (\i);
}
	\end{tikzpicture}
	\hspace{1cm}
	\begin{tikzpicture}
		\node[vertex] (9) at (0,0) {};
\node[vertex, below right=of 9] (10) {};
\node[vertex, above right=of 10] (11) {};
\node[vertex, below right=of 11] (12) {};
\node[vertex, below right=of 12] (13) {};

\node[vertex, above=of 11, draw=c1, fill=c1, fill opacity=.4] (t2-1) {};
\node[vertex, right=of t2-1, draw=c2, fill=c2, fill opacity=.4] (t2-2) {};
\node[vertex, left=of t2-1, draw=c0, fill=c0, fill opacity=.4] (t2-0) {};

% ---
% Transitions
% ---
\foreach \i in {9,11,12} {
	\pgfmathtruncatemacro{\next}{\i + 1}
	\draw[edge] (\i) to (\next);
};
\foreach \i in {10} {
	\pgfmathtruncatemacro{\next}{\i + 1}
	\draw[edge] (\next) to (\i);
}

\draw[edge] (t2-0) to (t2-1);
\draw[edge] (t2-1) to (t2-2);
\draw[edge] (t2-0) to[bend left=60] (t2-2);

\draw[edge] (9) to[bend left=30] (t2-1);
\draw[edge] (t2-0) to[bend left=30] (11);
\draw[edge] (11) to[bend left=30] (t2-2);
	\end{tikzpicture}
	\caption{\AP\label{fig:dichotomy-idempotent-core}
	Reduction from $\,\HomAll{\marked{\?B}}$ to $\,\HomAll{\?B}$ when $\?B$ 
	is the "$2$-transitive tournament":
	we depict on the left-hand side, the "$\extendedSignature{\sigma}{\?B}$-structure"
	$\?A \not\in \HomAll{\marked{\transitiveTournament{2}}}$,
	and on the right-hand side, the "$\sigma$-structure" $\Phi(\?A)
	\not\in \HomAll{\transitiveTournament{2}}$ to which it is reduced.
	The "interpretation@@predicate" of unary predicates in $\?A$
	are described using colours.}
\end{figure}
We fix a "purely relational signature" $\sigma$.
Given a "$\sigma$-structure" $\?B$,
we denote by \AP$\intro*\extendedSignature{\sigma}{\?B}$
the signature obtained from $\sigma$ by adding
a unary predicate \AP$\intro*\unarypred{b}$ for each $b\in B$.
The \AP""marked structure"" \AP$\intro*\marked{\?B}$ of $\?B$ is the
"$\extendedSignature{\sigma}{\?B}$-structure"
obtained from $\?B$ by "interpreting@@predicate" each predicate $\unarypred{b}$ as the
singleton $\{b\}$.

\begin{proposition}[Folklore]
	\!\sidenote{The non-easy part is to reduce $\HomFin{\marked{\?B}}$ to $\HomFin{\?B}$: only this reduction
	requires the assumption that $\?B$ is a core. This reduction is folklore, see "eg"
	\cite[Lemma 2.5]{LaroseTesson2009UniversalAlgebraCSP}.
	We provide here a self-contained proof.}%
	\sidenote{The "marked structure" $\marked{\core{\?B}}$ of the "core" of $\?B$ is
	usually called the \AP""idempotent core""
	of $\?B$. From this proposition, we get that $\HomFin{\?B}$ and
	$\HomFin{\marked{\core{\?B}}}$ are "first-order equivalent".
	This reduction is a central tool
	in the algebraic approach to understand "constraint satisfaction
	problem" since the algebra associated to the "CSP" over an "idempotent core"
	only has idempotent operations, making it much easier to work with.}%
	%%%
	\AP\label{prop:idempotent-core-preserves-csp-complexity}
	If $\?B$ is a finite "core", then the problems $\HomAll{\marked{\?B}}$ and 
	$\HomAll{\?B}$ are "first-order equivalent".
	Moreover, this equivalence preserves "finiteness@@structure",
	in the sense that "finite structures" are mapped to "finite structures".  
	Hence, by restricting this equivalence, we also obtain that
	$\HomFin{\marked{\?B}}$ and $\HomFin{\?B}$ are "first-order equivalent".
\end{proposition}

\begin{proof}[Proof of \Cref{prop:idempotent-core-preserves-csp-complexity}]
	\proofcase{Reduction from $\,\HomAll{\?B}$ to $\,\HomAll{\marked{\?B}}$.}\\
	We reduce a "$\sigma$-structure" $\?A$ to the
	"$\extendedSignature{\sigma}{\?B}$-structure" $\?A'$ obtained
	from $\?A$ by "interpreting@@predicate" each predicate $\unarypred{b}$ as the empty set.
	Clearly, a function from $A$ to $B$ is a "homomorphism" from $\?A$ to $\?B$
	"iff" it is a "homomorphism" from $\?A'$ to $\marked{\?B}$, proving the correctness
	of the reduction. It is, by definition, "first-order@@reduction".

	\proofcase{Reduction from $\,\HomAll{\marked{\?B}}$ to $\,\HomAll{\?B}$.}
	We first define the reduction $\Phi$ and show its correctness; the fact that it
	is a "first-order reduction" is straightforward.
	We reduce a "$\extendedSignature{\sigma}{\?B}$-structure" $\?A$ to the "$\sigma$-structure"
	$\Phi(\?A)$ illustrated on \Cref{fig:dichotomy-idempotent-core} and defined as follows:
	\begin{itemize}
		\item its underlying universe is the disjoint union $A \dcup B$,
		\item given a "predicate" $\+R$ of arity $k$, its "hyperedges" are:
		\begin{itemize}
		\item all "$\+R$-tuple" of $\?A$,
		\item all "$\+R$-tuple" of $\?B$, and
		\item all "$\+R$-tuple" $\tup{b_1,\dotsc,b_{i-1}, a_i, b_{i+1},\dotsc,b_k}$
			"st" there exists $b_i$ for which the $\+R$-"hyperedge"
			$\tup{b_1,\dotsc,b_{i-1}, b_i, b_{i+1},\dotsc,b_k}$
			is in $\+R(\?B)$, and $a_i$ belongs to the "interpretation@@predicate" of 
			$\unarypred{b_i}$ in $\?A$.
		\end{itemize}
	\end{itemize}
	Note that by construction, the "adjacency" of $a \in A$ in $\Phi(\?A)$ is
	the union of its "adjacency" in $\?A$, and the union of the "adjacencies" of
	$b$ in $\?B$ for all $b$ "st" $a \in \unarypred{b}(\?A)$.

	We show that $\?A \in \HomAll{\marked{\?B}}$ "iff" $\Phi(\?A) \in \HomAll{\?B}$.
	So, assume that there exists a "homomorphism" $f\colon \?A \to \marked{\?B}$.
	Then we let $f'\colon A \dcup B \to B$ be defined by $f'(a) = f(a)$ for all $a\in A$ and
	$f'(b) = b$ for all $b \in B$. We claim that $f'$ is a "homomorphism" from
	$\Phi(\?A)$ to $\?B$. Indeed, consider a "hyperedge" of $\Phi(\?A)$:
	\begin{itemize}
		\item if it is a "hyperedge" of $\?A$, its image by $f'$ is still a "hyperedge"
			of $\?B$ since $f$ is a "homomorphism" from $\?A$ to $\marked{\?B}$;
		\item if it is a "hyperedge" of $\?B$, then its image by $f'$ is itself, and is hence
			a "hyperedge" of $\?B$;
		\item otherwise, it must be of the form 
			\[\tup{b_1,\dotsc,b_{i-1}, a_i, b_{i+1},\dotsc,b_k}\]
			"st" there exists $b_i$ for which
			$\tup{b_1,\dotsc,b_{i-1}, b_i, b_{i+1},\dotsc,b_k} \in \+R(\?B)$
			and $a_i \in \unarypred{b_i}(\?A)$:
			in this case, its image by $f'$ is 
			\[f'(\tup{b_1,\dotsc,b_{i-1}, a_i, b_{i+1},\dotsc,b_k})
			= \tup{b_1,\dotsc,b_{i-1}, b_i, b_{i+1},\dotsc,b_k} \in \+R(\?B)\]
			since $f'(b) = b$ for all $b\in B$ and $f'(a_i) = f(a_i) = b_i$ since
			$a_i \in \unarypred{b_i}(\?A)$ and $f$ is a "homomorphism" from
			$\?A$ to $\marked{\?B}$.
	\end{itemize}
	And hence, $\Phi(\?A) \homto \?B$.

	Conversely, now let $g\colon \Phi(\?A) \to \?B$ be a "homomorphism".
	Its restriction to $\?B$, namely $\restr{g}{B}$ is a "homomorphism" from $\?B$
	to itself, and since $\?B$ is a "core", it must be an "automorphism" over $\?B$
	by \Cref{prop:core-iff-hom-are-auto}.
	We then define a map $g' \colon A \to B$ by sending
	$a$ to $(\restr{g}{B})^{-1}\circ g(a)$, and claim that it is a "homomorphism"
	from $\?A$ to $\marked{\?B}$. As a matter of fact, it clearly preserves "$\+R$-tuple"
	for any $\+R$ in $\sigma$, since $g$ and $(\restr{g}{B})^{-1}$ are "homomorphisms".
	We must then show that it preserves all unary "predicates" $\unarypred{b}$, with $b\in B$:
	let $a \in A$ "st" $\unarypred{b}$ holds, "ie" $a \in \unarypred{b}(\?A)$.
	Now, by construction of $\Phi(\?A)$, the "adjacency" of $g(a)$ in $\?B$
	and the "adjacency" of $g(b)$ in $\?B$ are equal.
	Since $\?B$ is a "core", it follows by \Cref{prop:adjacency-core} that $g(a) = g(b)$.
	By definition of $g'$, this rewrites as $g'(a) = b$, "ie" $g'(a) = \unarypred{b}(\marked{\?B})$.
	Therefore, we have built a "homomorphism" from $\?A$ to $\marked{\?B}$.

	Overall, this proves that $\Phi$ is correct.
	It is trivially a "first-order reduction" and moreover,
	it preserves "finiteness@@structure" since $\?B$
	is "finite@@structure".
\end{proof}


\subsection{De Bruijn–Erdős Theorem}

\begin{proposition}[""De Bruijn–Erdős Theorem""]
	\!\footnote{It is straightforward to note that
	one can replace ``every finite "substructure"'' by
	``every finite "induced substructure"'' in the statement of the theorem.
	The original theorem is about graph colouring, but the generalization is straightforward.}%
	\AP\label{prop:de-bruijn-erdos}
	Let $\?A$ be an arbitrary "$\sigma$-structure" and $\?B$ a "finite $\sigma$-structure".
	There is a "homomorphism" from $\?A$ to $\?B$ "iff" for every finite "substructure" $\?A'$
	of $\?A$, there is a "homomorphism" from $\?A$ to $\?B$.
\end{proposition}

\begin{proof}
	The left-to-right implication is direct.
	We prove the converse by using the "Tychonoff's compactness theorem".\footnote{This is a direct adaptation from \cite[\S~``Proof'']{Wikipedia2024DeBruijnErdos}.}
	So, assume that for every finite "substructure" $\?A'$ of $\?A$,
	there is a "homomorphism" from $\?A$ to $\?B$. 
	Consider the topological space $B^A$, consisting of all functions from $A$ to $B$,
	together with the product topology.\footnote{We equip $B$ with the discrete topology,
	making it compact since $B$ is finite.} By "Tychonoff's compactness theorem",
	$B^A$ is compact. For each finite subset $X$ of $A$, let
	$H_X$ denote the set of all $f \in B^A$ "st" $\restr{f}{X}$ is a "homomorphism"
	from the "substructure" of $\?A$ "induced@@structure" by $X$ to $\?B$.
	Then, each $H_X$ is closed---indeed, whether $f\in B^A$ belongs to $H_X$ only depends
	on finitely many $f(x)$'s---, and moreover the intersection of finitely many
	$H_X$'s, say $H_{X_1} \cap \cdots \cap H_{X_n}$, is non-empty since
	$H_{X_1} \cap \cdots \cap H_{X_n} \supseteq H_{X_1\cup \dotsc \cup X_n}$
	and by assumption $H_{X_1\cup \dotsc \cup X_n}$ is non-empty since $X_1 \cup \cdots \cup X_n$ is finite. Hence, by compactness of $B^A$ and the "finite intersection property", it follows
	that $\bigcap_X H_X$ is non-empty, which means that there is a "homomorphism" from $\?A$ to $\?B$.
\end{proof}

\begin{corollary}
	\!\footnote{Another important consequence of the "De Bruijn–Erdős Theorem" is that,
	for instance, the notion of "dual" does not depend on whether we are considering finite or
	arbitrary "$\sigma$-structures".}
	\AP\label{coro:de-bruijn-erdos}
	Given arbitrary $\sigma$-structures $\?B_1$ and $\?B_2$, the following are equivalent:
	\begin{enumerate}
		\itemAP\label{item:de-bruijn-erdos-finite} for every finite "$\sigma$-structure" $\?A$, we have $\?A \homto \?B_1$
		"iff" $\?A \homto \?B_2$;
		\itemAP\label{item:de-bruijn-erdos-arbitrary} for every arbitrary "$\sigma$-structure" $\?A$, we have $\?A \homto \?B_1$
			"iff" $\?A \homto \?B_2$;
		\itemAP\label{item:de-bruijn-erdos-hom} $\?B_1$ and $\?B_2$ are "homomorphically equivalent".
	\end{enumerate}
\end{corollary}
\begin{proof}
	\eqref{item:de-bruijn-erdos-arbitrary} $\Rightarrow$ \eqref{item:de-bruijn-erdos-hom}
	and \eqref{item:de-bruijn-erdos-hom} $\Rightarrow$ \eqref{item:de-bruijn-erdos-finite}
	are trivial.
	For \eqref{item:de-bruijn-erdos-finite} $\Rightarrow$ \eqref{item:de-bruijn-erdos-arbitrary},
	we assume "wlog" by contradiction that there is an arbitrary "$\sigma$-structure" $\?A$ "st" $\?A \homto \?B_1$ but $\?A \nothomto \?B_2$. Then by \Cref{prop:de-bruijn-erdos},
	there exists a finite "substructure" $\?A_0$ of $\?A$ "st" $\?A_0 \nothomto \?B_2$.
	But then $\?A_0 \homto \?A \homto \?B_1$, which contradicts \eqref{item:de-bruijn-erdos-finite}.
\end{proof}


\subsection{Obstructions and Finite Duality}

Let $\?B$ and $\?D$ be "finite $\sigma$-structures".
We say that $\?D$ is an \AP""obstruction"" of $\?B$ when $\?D \nothomto \?B$.
In this case, note that finding $\?D$ inside $\?A$---in the sense that $\?D \homto \?A$---implies that $\?A$ cannot have a "homomorphism" to $\?B$: in this sense, the presence of
$\?D$ in an \emph{obstruction} to the existence of a "homomorphism" to $\?B$.

A \AP""dual"" of $\?B$ is any arbitrary set $\+D$ of "finite $\sigma$-structures" "st"
for any "finite $\sigma$-structure" $\?A$:
\[
	\?A \homto \?B
	\text{ "iff" }
	\forall \?D \in \+D,\, \?D \nothomto \?A.
\]
Note that any "dual" must only contain "obstructions" of $\?B$.%
\footnote{The notion of dual was implicitly introduced
by Ne\v{s}et\v{r}il and Pultr, who proved that for "\emph{undirected} graph", no
non-trivial "finite duality" existed, in the sense that the
only "cores" $\?B$ having a finite "dual" where the empty graph, and the graph having
a single vertex and no edge \cite[Corollary 4.1]{NesetrilPultr1978Duality}.}
The set of all "obstructions" of $\?B$ is a "dual" of $\?B$.

\begin{example}
	\AP\label{ex:dual-T2} Let $k\in\N$. The set
	$\{\pathGraph{k+1}\}$ is a "dual" of the
	"$k$-transitive tournament" $\transitiveTournament{k}$---see \Cref{fig:T2-P2}.
	
	Indeed, $\pathGraph{k+1} \nothomto \transitiveTournament{k}$. Dually,
	if $\?G$ if a "finite graph" "st" $\?G \homto \transitiveTournament{k}$ then
	letting $f$ be a "homomorphism" from $\?G$ to $\transitiveTournament{k}$,
	we have that any edge from $u \in G$ to $v \in G$, we must have $f(u) < f(v)$,
	and so $\pathGraph{k} \nothomto \?G$.
\end{example}

Moreover, if $\+D$ and $\+D'$ are sets of "obstructions" of $\?B$ and $\+D \subseteq \+D'$,
then if $\+D$ is a "dual", so is $\+D'$: hence, the goal is to find \emph{small} duals.
For this reason, "duals" are also called \emph{complete sets of obstructions}.
We say that $\?B$ has \AP""finite duality"" if is admits a finite "dual", "ie"
consisting of finitely many "structures". For instance, $\transitiveTournament{k}$ has "finite duality".

An "obstruction" $\?D$ of $\?B$ is \AP""critical@@obs""\footnote{We borrow the terminology
from \cite{LaroseLotenTardif2007CharacterisationFOCSP}.} when for every
"proper substructure" $\?D'$ of $\?D$, we have $\?D' \homto \?B$.
Clearly, every "critical obstruction" must be a "core".

Note first that the set of all "critical obstructions" of $\?B$ is a "dual" of $\?B$.
Indeed, if $\?A \nothomto \?B$, then $\?A$ is an "obstruction" and so it must contain---by well-foundedness of $\N$---a "critical obstruction" $\?D$ as a "substructure".

\begin{proposition}
	\label{prop:finite-duality-iff-critical-obstructions}
	Let $\?B$ be a "finite $\sigma$-structure". $\?B$ has "finite duality"
	"iff" it has finitely many "critical obstructions".
\end{proposition}

\begin{proof}
	The right-to-left implication is trivial. For the converse one,
	let $\+D = \{D_1,\dotsc,D_m\}$ be a finite "dual" of $\?B$.
	If $\?C$ is a "critical obstruction" of $\?B$ then in particular $\?C \nothomto \?B$
	and so there exists $i \in \intInt{1,m}$ "st" there is a "homomorphism" $f$ 
	from $\?D_i$ to $\?C$. Now the image of $f$ is again an "obstruction" of $\?B$,
	and since $\?C$ is "critical@@obs", it follows that it must be $\?C$ itself. In other
	words, $f$ is "strong onto". In particular, we obtain that $\?C$ is a "quotient@@struct" of
	$\?D_i$. Hence, there are only finitely many "critical obstructions" of $\?B$.
\end{proof}

One of the key interest of \Cref{prop:finite-duality-iff-critical-obstructions} is
to prove that some structures don't have "finite duality".

\begin{figure}
	\centering
	\begin{tikzpicture}
		\node[vertex] (0) at (0,0) {};
\node[vertex, below right=of 0] (1) {};
\foreach \i in {1, 3, 5, 7, 9} {
	\pgfmathtruncatemacro{\next}{\i + 1}
	\pgfmathtruncatemacro{\nnext}{\i + 2}
	\node[vertex, below right=of \i] (\next) {};
	\node[vertex, above right=of \next] (\nnext) {};
};
\node[vertex, below right=of 11] (12) {};
\node[vertex, below right=of 12] (13) {};

% ---
% Transitions
% ---
\foreach \i in {0,1,3,5,7,9,11,12} {
	\pgfmathtruncatemacro{\next}{\i + 1}
	\draw[edge] (\i) to (\next);
};
\foreach \i in {2,4,6,8,10} {
	\pgfmathtruncatemacro{\next}{\i + 1}
	\draw[edge] (\next) to (\i);
}
		
\foreach \i in {1,3,5,7,9,11} {
	\pgfmathtruncatemacro{\half}{\i / 2}
	\node[draw=none, above right=-.6em of \i] {$a_{\half}$};
};
\foreach \i in {2,4,6, 8, 10, 12} {
	\pgfmathtruncatemacro{\half}{(\i / 2) - 1}
	\node[draw=none, below left=-.6em of \i] {$b_{\half}$};
};
\node[draw=none, below left=-.6em of 0] {$a'_0$};
\node[draw=none, above right=-.6em of 13] {$b'_5$};

\draw[decoration={brace}, decorate, transform canvas={yshift=1.5em}] (3.north west) to node[midway, above] {$\+O(n)$ nodes} (11.north east);
	\end{tikzpicture}
	\caption{\AP\label{fig:zigzag-graph}The "zigzag graph" $\zigzag{5}{2}$.}
\end{figure}
\begin{example}[{\Cref{ex:dual-T2}, continued.}]
	\AP\label{ex:zigzag-defn}
	Let $n\in\?N$.
	We define the \AP""zigzag graph"" $\intro*\zigzag{n}{2}$ of width $n$ and length 2
	to be "graph@@dir" whose vertices are $a_0, \dotsc, a_n$, $b_0, \dotsc, b_{n}$,
	$a'_0$ and $b'_n$, with edges from $a_i$ to $b_{i-1}$ and to $b_{i}$ (for $i \in \intInt{0,n}$, whenever the nodes exist), and with an edge from $a'_0$ to $a_0$ and from $b_n$
	to $b'_n$. See \Cref{fig:zigzag-graph} for an illustration.
	
	Note that $\zigzag{n}{2}$ does not admit a "homomorphism" to the "$2$-path"---indeed, such a homomorphism should send $a'_0$, $a_0$ and $b_0$ onto $0$, $1$, and $2$, respectively, 
	and so all $a_i$'s (resp. $b_i$'s) must be sent onto $1$ (resp. $2$), but then $b'_n$ cannot be mapped anywhere.

	We claim that each $\zigzag{n}{2}$ ($n\in\N$) is a "critical obstruction" of $\pathGraph{2}$.
	We have already seen that $\zigzag{n}{2}$ is an "obstruction" of $\pathGraph{2}$.
	But then notice that to obtain a "proper substructure" of $\zigzag{n}{2}$,
	we must either:
	\begin{itemize}
		\item remove the edge from $a'_0$ to $a_0$ or the edge from $b_n$ to $b'_n$,
		in which case it admits a "homomorphism" to $\pathGraph{2}$, or
		\item remove any other edge, in which case the resulting "substructure" is not "connected",
		and both parts admit a "homomorphism" to $\pathGraph{2}$.
	\end{itemize}
	And hence, by \Cref{prop:finite-duality-iff-critical-obstructions}, it follows that
	$\pathGraph{2}$ does not have "finite duality".

	On the other hand, each $\zigzag{n}{2}$ with $n\in\Np$ admits a "homomorphism" to the "$2$-transitive tournament", as witnessed by \Cref{fig:zigzag-graph-hom-T2}.%
	\sidenote[][-11em]{However, observe that $\zigzag{0}{2} = \pathGraph{3}$ is an "obstruction" of
	$\transitiveTournament{2}$.}
	In fact, this "homomorphism" is far from being unique:
	each vertex $a_1,\,a_2,\,\dotsc,\,a_{n-1}$ can be sent on either $0$ or $1$
	(the red and yellow vertices), 
	and similarly, each vertex $b_1,\,b_2,\,\dotsc,\,b_{n-1}$ can be sent on either $1$ or $2$
	(the yellow and blue vertices).%
	\sidenote[][-11em]{
		Note that it is straightforward to extend the fact that
		\[\zigzag{n}{2} \homto \transitiveTournament{2}
		\text{ and }
		\zigzag{n}{2} \nothomto \pathGraph{2}\]
		to arbitrary values of $k\in \N$, in the sense that
		\[\zigzag{n}{k} \homto \transitiveTournament{k}
		\text{ and }
		\zigzag{n}{k} \nothomto \pathGraph{k},\]
		by letting
		$\zigzag{n}{k}$ be the graph obtained from $\zigzag{n}{2}$ by
		replacing the path leading to $a_0$ by a path of length $k-1$.
	}
	\begin{figure}
		\centering 
		\begin{tikzpicture}
			\node[vertex, draw=c0, fill=c0, fill opacity=.4] (0) at (0,0) {};
\node[vertex, below right=of 0, draw=c1, fill=c1, fill opacity=.4] (1) {};
\foreach \i in {1, 3, 5, 7, 9} {
	\pgfmathtruncatemacro{\next}{\i + 1}
	\pgfmathtruncatemacro{\nnext}{\i + 2}
	\node[vertex, below right=of \i, draw=c2, fill=c2, fill opacity=.4] (\next) {};
	\node[vertex, above right=of \next, draw=c0, fill=c0, fill opacity=.4] (\nnext) {};
};
\node[vertex, below right=of 11, draw=c1, fill=c1, fill opacity=.4] (12) {};
\node[vertex, below right=of 12, draw=c2, fill=c2, fill opacity=.4] (13) {};

% ---
% Transitions
% ---
\foreach \i in {0,1,3,5,7,9,11,12} {
	\pgfmathtruncatemacro{\next}{\i + 1}
	\draw[edge] (\i) to (\next);
};
\foreach \i in {2,4,6,8,10} {
	\pgfmathtruncatemacro{\next}{\i + 1}
	\draw[edge] (\next) to (\i);
}

% ---
% 2-transitive tournament
% ---
\node[vertex, above right=-.25em and 5em of 13, draw=c2, fill=c2, fill opacity=.4] (t2-2) {};
\node[vertex, above=of t2-2, draw=c1, fill=c1, fill opacity=.4] (t2-1) {};
\node[vertex, above=of t2-1, draw=c0, fill=c0, fill opacity=.4] (t2-0) {};

\draw[edge] (t2-0) to (t2-1) to (t2-2);
\draw[edge] (t2-0) to[bend left=60] (t2-2);
		\end{tikzpicture}
		\caption{\AP\label{fig:zigzag-graph-hom-T2}A "homomorphism" from the "zigzag graph" (left-hand side) to the "$2$-transitive tournament" (right-hand side).}
	\end{figure}
\end{example}

While $\pathGraph{k}$ and $\transitiveTournament{k}$ are similar structures,
one has "finite duality" and the other does not.

Recall that by "Atserias' theorem", this implies that
$\HomFin{\transitiveTournament{k}}$ is "first-order definable"\footnote{By the formula saying that
there are no variables $x_0$, $\dotsc$, $x_{k+1}$ "st" for all $i$ there is an edge from $x_i$ to $x_{i+1}$.} but $\HomFin{\pathGraph{k}}$ is not.%
\footnote{This can also be proven by hand: for 
instance we let $\tilde{\?Z}^{(n)}_2$ be defined analogously to $\zigzag{n}{2}$ except
that we remove the edge from $a_{k}$ to $b_k$ for $k = \lceil \frac{n}{2} \rceil$. 
Then for a well-chosen single-exponential function $f\colon \N \to \N$, for $n\in\N$,
we have that Duplicator wins the Ehrenfeucht-Fraïssé game on $\zigzag{f(n)}{2}$ and
$\tilde{\?Z}^{(f(n))}_2$. Since $\zigzag{f(n)}{2} \nothomto \pathGraph{2}$ but
$\tilde{\?Z}^{(n)}_2 \homto \pathGraph{2}$, this implies that $\HomFin{\pathGraph{2}}$ is not 
"first-order definable". See "eg" \cite[\S~3]{Kolaitis2007FiniteModelTheory} for details on
Ehrenfeucht-Fraïssé games.}


\begin{remark}[From finite to infinite structures]
	In the property
	\[
		\?A \homto \?B
		\;\text{ "iff" }\;
		\forall \?D \in \+D,\, \?D \nothomto \?A
	\]
	defining a "dual", we quantified $\?A$ over "finite $\sigma$-structures". 
	We could equally quantify over all "$\sigma$-structures" without changing the notion of "dual" by "De Bruijn-Erdős theorem".
\end{remark}

We say that a "$\sigma$-structure" is \AP""rigid"" if its only "automorphism"
is the identity.
\begin{proposition}[{\cite[Lemma~4.1]{LaroseLotenTardif2007CharacterisationFOCSP}}]
	\!\footnote{In fact
	only assumes that $\?B$ has "tree duality": as we will see in 
	\Cref{prop:finite-duality-implies-tree-duality}, this is a weaker condition
	than having "finite duality".}%
	\label{prop:finite-duality-implies-rigid}
	If a finite "core" has "finite duality", then it is "rigid".
\end{proposition}

\subsection{Trees and Tree Duality}

A "$\sigma$-structure" is \AP""strongly acyclic@@struct"" if its "incidence graph" is acyclic.
A \AP""$\sigma$-tree"" is a "$\sigma$-structure" that is both "connected" and "strongly acyclic@@struct".
Do not confuse this notion with the classical notion of "directed trees":
every "directed tree" is a "$\sigma$-tree" for the "graph signature" $\sigma$,
but $\zigzag{n}{2}$---see \Cref{fig:zigzag-graph}---is a "$\sigma$-tree"
while it is not a "directed tree". 

Given a "$\sigma$-tree" $\?T$ and a vertex $t\in T$, the \AP""height@@struct"" of $\?T$
when rooted at $t$ is the maximal "distance@@struct" between $t$ and any other vertex of $\?T$.

We say that a finite "$\sigma$-structure" $\?B$ has \AP""tree duality""
if it admits a (potentially infinite) "dual" consisting only of "$\sigma$-trees".
Somewhat surprisingly, Ne\v{s}et\v{r}il and Tardif showed that
this notion generalizes "finite duality".

\begin{proposition}[{\cite[Theorem~3.1]{NesetrilTardif2000DualityTheorems}}]
	\!\footnote{The statement of \cite[Theorem~3.1]{NesetrilTardif2000DualityTheorems}
	is somewhat cryptic: the relationship with "duals" is given by
	\cite[Lemma~2.5]{NesetrilTardif2000DualityTheorems}.
	We refer the reader to Foniok's Ph.D. for statements that use a terminology closer to ours:
	\cite[Theorem~2.1.12]{Foniok2007PhD} shows that if $\{\?A\}$ is a "dual" of $\?B$,
	then $\?A$ is "homomorphically equivalent" to a tree, "ie" if a "structure" has
	``singleton duality'', then it has "tree duality".
	The generalization to "finite duality" then follows from
	\cite[Theorem~2.4.4]{Foniok2007PhD}.}
	\AP\label{prop:finite-duality-implies-tree-duality}
	If a finite "$\sigma$-structure" $\?B$ has "finite duality",
	then it has "tree duality".
\end{proposition}

The converse does not hold.
\begin{proposition}
	\AP\label{prop:2-path-tree-duality}
	The "$2$-path" $\pathGraph{2}$ has "tree duality".
\end{proposition}

\begin{proof}[Proof sketch]
	It can be shown that $\{\zigzag{n} \mid n\in\N\}$ is a "dual" of
	$\pathGraph{2}$. Moreover, each $\zigzag{n}{2}$ ($n\in\N$) is a "$\sigma$-tree".
\end{proof}

Feder and Vardi introduced a construction to decide if a "finite structure" has "tree duality":
given a "$\sigma$-structure" $\?B$, we let \AP$\intro*\FederVardi{\?B}$ be
the $\sigma$-structure whose domain is $\psetp{B}$, and for every $\+R_{(k)} \in \sigma$,
we have $\tup{Y_1, \dotsc, Y_k} \in \+R(\FederVardi{\?B})$ 
(with $Y_1, \dotsc, Y_k \in \psetp{B}$) precisely when 
for every $i\in \intInt{1,k}$, 
for every $b_i \in Y_i$,
there exists $b_j \in Y_j$ for every $j \neq i$ "st" 
$\tup{y_1, \dotsc, y_k} \in \+R(\?B)$.%
\footnote{In fact $\FederVardi{-}$ can
easily be extended to be a functor in the category of $\sigma$-structures, see
"eg" \cite[\S~9.2.2]{NesetrilPOM2012FirstOrderCSPs}.}
In the case of "graphs@@dir", the nodes of $\FederVardi{\?H}$ are non-empty
subsets of vertices of $\?H$, and there is an edge from $X$ to $Y$ when:
\begin{itemize}
	\item for every $x \in X$, there exists $y\in Y$ "st" $\tup{x,y}$ is an edge of $\?H$, and
	\item for every $y\in Y$, there exists $x\in X$ "st" $\tup{x,y}$ is an edge of $\?H$.
\end{itemize}

By construction, note that $b \mapsto \set{B}$ defines a "homomorphism" from
$\?B$ to $\FederVardi{\?B}$.

\begin{proposition}[{\cite[Theorem 21]{FederVardi1998ComputationalStructure}}]
	\label{prop:charac-Feder-Vardi}
	A "finite $\sigma$-structure" $\?B$ has "tree duality" if, and only if,
	$\FederVardi{\?B} \homto \?B$, or, equivalently, if $\?B$ and $\FederVardi{\?B}$
	are "homomorphically equivalent".
\end{proposition}

Note that $\FederVardi{\FederVardi{\?B}}$ is always "homomorphically equivalent" to
$\FederVardi{\?B}$---see "eg" \cite[\S~9.2.2, Proposition 9.1]{NesetrilPOM2012FirstOrderCSPs}---
and so $\FederVardi{\?B}$ \emph{always} has "tree duality", no matter if $\?B$ has this property.

\begin{example}[{\Cref{ex:zigzag-defn}, continued}]
	\label{ex:Feder-Vardi-P2}
	Using \Cref{prop:charac-Feder-Vardi}, we can prove for instance that
	if a "graph@@dir" $\?G$ has "tree duality", then either it is a "DAG", or $\?G$ contains
	a self-loop.

	Indeed, let $v_0 \to v_1 \to \cdots \to v_m$ be a "directed cycle" in $\?G$,
	with $v_0 = v_m$. Then in $\FederVardi{\?G}$, there is an edge from
	$\{v_0,v_1,\dotsc,v_m\}$ to itself.
	Since $\?G$ has "tree duality", it follows by \Cref{prop:charac-Feder-Vardi}
	that $\FederVardi{\?G} \homto \?G$ and hence $\?G$ also contains a self-loop.
	
	Note that, if $\?G$ contains a self-loop, then actually it is "homomorphically equivalent" to the graph consisting of a single self-loop---that admits $\emptyset$ as a "dual" and hence has
	"tree duality". In other words, what we showed can be rephrased as: any
	non-trivial "graph@@dir" with "tree duality" is a "DAG".
	For instance, this implies that $\clique{2}$ does not have
	"tree duality".
	
	\begin{marginfigure}[-20em]
		\centering
		\begin{tikzpicture}
			% ---
% 2-path graph
% ---
\node[vertex, draw=c2, fill=c2, fill opacity=.4] at (0,0) (p2-2) {};
\node[vertex, above=of p2-2, draw=c1, fill=c1, fill opacity=.4] (p2-1) {};
\node[vertex, above=of p2-1, draw=c0, fill=c0, fill opacity=.4] (p2-0) {};

\node[vertex, inner sep=4pt, right=2.5em of $(p2-0)!0.5!(p2-1)$] (p2-01) {};
\node[tiny vertex, draw=c0, fill=c0, fill opacity=.4, above=-7pt of p2-01] {};
\node[tiny vertex, draw=c1, fill=c1, fill opacity=.4, below=-7pt of p2-01] {};

\node[vertex, inner sep=4pt, right=2.5em of $(p2-1)!0.5!(p2-2)$] (p2-12) {};
\node[tiny vertex, draw=c1, fill=c1, fill opacity=.4, above=-7pt of p2-12] {};
\node[tiny vertex, draw=c2, fill=c2, fill opacity=.4, below=-7pt of p2-12] {};

\node[vertex, inner sep=4pt, right=2em of p2-01] (p2-02) {};
\node[tiny vertex, draw=c0, fill=c0, fill opacity=.4, above=-7pt of p2-02] {};
\node[tiny vertex, draw=c2, fill=c2, fill opacity=.4, below=-7pt of p2-02] {};

\node[vertex, inner sep=4pt, right=2em of p2-12] (p2-012) {};
\node[tiny vertex, draw=c0, fill=c0, fill opacity=.4, above=-8pt of p2-012] {};
\node[tiny vertex, draw=c1, fill=c1, fill opacity=.4, below left=-7pt of p2-012] {};
\node[tiny vertex, draw=c2, fill=c2, fill opacity=.4, below right=-7pt of p2-012] {};

\draw[edge] (p2-0) to (p2-1);
\draw[edge] (p2-1) to (p2-2);
\draw[edge] (p2-01) to (p2-12);
		\end{tikzpicture}
		\caption{
			\AP\label{fig:P2-Feder-Vardi}
			The Feder-Vardi construction $\FederVardi{\pathGraph{2}}$
			on the "$2$-path".
		}
	\end{marginfigure}
	On the other hand, we saw in \Cref{ex:zigzag-defn} that $\pathGraph{2}$ does not have
	"finite duality". However, it has "tree duality": indeed, see \Cref{fig:P2-Feder-Vardi}
	and observe that $\FederVardi{\pathGraph{2}} \homto \pathGraph{2}$.
\end{example}