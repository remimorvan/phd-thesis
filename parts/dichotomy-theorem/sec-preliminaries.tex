\section{Preliminaries}

For \AP""first-order reductions"", see
\cite[Definition 2.11 \& Definition 1.26]{Immerman1998DescriptiveComplexity}.

Given two "structures" $\?A$ and $\?B$, we define the structure \AP$\intro*\powstruct{\?B}{\?A}$ as follows:
\begin{itemize}
  \item its domain are "homomorphisms" $\?A \to \?B$,
  \item for every "relation symbol" $\+R$ of arity $k$, for any homomorphism $f_1,\hdots,f_k$,
  we have $\langle f_1,\hdots,f_k\rangle \in \+R$ when for every
  $\langle a_1,\hdots,a_k\rangle \in \+R(\?A)$
  then $\langle f_1(a_1), \hdots, f_k(a_k) \rangle \in \+R(\?B)$.
\end{itemize}

\begin{proposition}[Folklore]
	Given "structures" $\?A$, $\?B$ and $\?C$, if $f\colon \?A\prodstruct \?B \to \?C$
	is a "homomorphism", then $F\colon \?A \to \powstruct{\?C}{\?B}$,
	defined by $a \mapsto (b \mapsto f(a,b))$, is a "homomorphism".
	In fact, this mapping $f \mapsto F$ is a bijection
	between "homomorphisms" $\?A\prodstruct \?B \to \?C$
	and "homomorphisms" $\?A \to \powstruct{\?C}{\?B}$.
  \end{proposition}
  
  \begin{proof}
	Let $\+R$ be a relational symbol of arity $k$, and let
	$\langle a_1,\hdots,a_k \rangle \in \+R(\?A)$.
	We want to show that $\langle F(a_1),\hdots,F(a_k) \rangle \in \+R(\powstruct{\?C}{\?B})$:
	for any $\langle b_1,\hdots,b_k \rangle \in \+R(\?B)$, we have
	\[\langle F(a_1)(b_1), \hdots, F(a_k)(b_k)\rangle = \langle f(a_1,b_1),\hdots,f(a_k,b_k) \rangle \in \+R(\?C)\] since $f$ is a "homomorphism" from $\?A\prodstruct \?B$ to $\?C$.
	Hence, $F$ is indeed a "homomorphism" from $\?A$ to $\powstruct{\?C}{\?B}$.
  
	Dually, if $F$ is a "homomorphism" from $\?A$ to $\powstruct{\?C}{\?B}$,
	we define $f\colon \?A\prodstruct \?B \to \?C$ by $\langle a,b \rangle \mapsto F(a)(b)$,
	and claim that $f$ is a "homomorphism". Indeed, if $\+R$ be a relational symbol of arity $k$,
	for any $\langle a_1, \hdots, a_k \rangle \in \+R(\?A)$
	and $\langle b_1, \hdots, b_k \rangle \in \+R(\?B)$,
	we have $\langle f(a_1,b_1), \hdots, f(a_k,b_k) \rangle
	= \langle F(a_1)(b_1), \hdots, F(a_k)(b_k) \rangle$.
	Since $\langle F(a_1), \hdots, F(a_k) \rangle \in \+R(\powstruct{\?C}{\?B})$
	and $\langle b_1, \hdots,b_k \rangle \in \+R(\?B)$ 
	it follows that $\langle F(a_1)(b_1), \hdots, F(a_k)(b_k) \rangle \in \+R(\?C)$.
	Therefore, $f$ is a "homomorphism" from $\?A \prodstruct \?B$ to $\?C$.
  
	It is then routine to check that the maps $f \mapsto F$ and $F \mapsto f$ defined
	in the two previous paragraphs are mutually inverse bijections.
  \end{proof}