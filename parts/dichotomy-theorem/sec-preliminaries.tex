\section{\AP\label{sec:dichotomy-preliminaries}%
	Preliminaries}

\subsection{Constructions on Structures}

Given two "structures" $\?A$ and $\?B$, we define the structure \AP$\intro*\powstruct{\?B}{\?A}$ as follows:
\begin{itemize}
  \item its domain are "homomorphisms" $\?A \to \?B$,
  \item for every "predicate" $\+R$ of arity $k$, for any homomorphism $f_1,\hdots,f_k$,
  we have $\langle f_1,\hdots,f_k\rangle \in \+R$ when for every
  $\langle a_1,\hdots,a_k\rangle \in \+R(\?A)$
  then $\langle f_1(a_1), \hdots, f_k(a_k) \rangle \in \+R(\?B)$.
\end{itemize}


\begin{proposition}[Folklore: Currying Homomorphisms]
	\AP\label{prop:currying-hom}
	Given "structures" $\?A$, $\?B$ and $\?C$, if $f\colon \?A\prodstruct \?B \to \?C$
	is a "homomorphism", then $F\colon \?A \to \powstruct{\?C}{\?B}$,
	defined by $a \mapsto (b \mapsto f(a,b))$, is a "homomorphism".
	In fact, this mapping $f \mapsto F$ is a bijection
	between "homomorphisms" $\?A\prodstruct \?B \to \?C$
	and "homomorphisms" $\?A \to \powstruct{\?C}{\?B}$.
\end{proposition}

\begin{proof}
Let $\+R$ be a predicate of arity $k$, and let
$\langle a_1,\hdots,a_k \rangle \in \+R(\?A)$.
We want to show that $\langle F(a_1),\hdots,F(a_k) \rangle \in \+R(\powstruct{\?C}{\?B})$:
for any $\langle b_1,\hdots,b_k \rangle \in \+R(\?B)$, we have
\[\langle F(a_1)(b_1), \hdots, F(a_k)(b_k)\rangle = \langle f(a_1,b_1),\hdots,f(a_k,b_k) \rangle \in \+R(\?C)\] since $f$ is a "homomorphism" from $\?A\prodstruct \?B$ to $\?C$.
Hence, $F$ is indeed a "homomorphism" from $\?A$ to $\powstruct{\?C}{\?B}$.

Dually, if $F$ is a "homomorphism" from $\?A$ to $\powstruct{\?C}{\?B}$,
we define $f\colon \?A\prodstruct \?B \to \?C$ by $\langle a,b \rangle \mapsto F(a)(b)$,
and claim that $f$ is a "homomorphism". Indeed, if $\+R$ be a predicate of arity $k$,
for any $\langle a_1, \hdots, a_k \rangle \in \+R(\?A)$
and $\langle b_1, \hdots, b_k \rangle \in \+R(\?B)$,
we have $\langle f(a_1,b_1), \hdots, f(a_k,b_k) \rangle
= \langle F(a_1)(b_1), \hdots, F(a_k)(b_k) \rangle$.
Since $\langle F(a_1), \hdots, F(a_k) \rangle \in \+R(\powstruct{\?C}{\?B})$
and $\langle b_1, \hdots,b_k \rangle \in \+R(\?B)$ 
it follows that $\langle F(a_1)(b_1), \hdots, F(a_k)(b_k) \rangle \in \+R(\?C)$.
Therefore, $f$ is a "homomorphism" from $\?A \prodstruct \?B$ to $\?C$.

It is then routine to check that the maps $f \mapsto F$ and $F \mapsto f$ defined
in the two previous paragraphs are mutually inverse bijections.
\end{proof}

\subsection{Constructions on rational presentations}
\label{sec:construction-automatic-presentations}

Let $\•A$ and $\•B$ be "rational presentations" of some "$\sigma$-structures"
$\?A$ and $\?B$. We define \AP$\•A \intro*\prodpres \•B$ to be the "presentation@@auto"
"st":
\begin{align*}
	\domainPres{\•A\prodpres \•B} & \defeq \{u\#v \mid u \in \domainPres{\•A} \land v \in \domainPres{\•B}\}\\
	\relPres{\•A\prodpres \•B}{\+R} & \defeq \{(u_1\#v_1) \convol \cdots \convol (u_k\#v_k) \mid
		u_1 \convol \cdots \convol u_k \in \relPres{\•A}{\+R} \land
		v_1 \convol \cdots \convol v_k \in \relPres{\•B}{\+R}
	\}
\end{align*}
for each "predicate" $\+R$ of arity $k$ in $\sigma$.
It is an "rational presentation" of $\?A \prodstruct \?B$. TODO:proof.

\begin{proposition}
	\label{prop:homreg-prod-finite}
	Let $\?A$, $\?B$ and $\?C$ be "automatic $\sigma$-structures", such that
	$\?B$ and $\?C$ are finite.
	Let $\•A$ (resp. $\•B$ and $\•B'$, resp. $\•C$ and $\•C'$) be an "rational presentation"
	of $\?A$ (resp. $\?B$, resp. $\?C$).
	Then $\•A \prodpres \•B \homregto \•C$ "iff" $\•A \prodpres \•B' \homregto \•C'$.
\end{proposition}

\begin{proof}
	The proof will follow from the following claim.
	\begin{claim}
		\label{claim:homreg-prod-finite}
		A function $f\colon\•A \prodpres \•B \homregto \•C$ is a "regular homomorphism"
		"iff" for every $b\in \domainPres{\•B}$, for every $c\in \domainPres{\•C}$,
		\(\{
			a\in \domainPres{\•A} \mid f(a,b) = c
		\}\)
		is a "regular language".
	\end{claim}
	TODO.
\end{proof}

In other words, the existence of a "regular homomorphism" does not depend on the
"rational presentation" of the \emph{finite} "structures" that are involved, but only
on the "structure" they represent.
As a consequence of \Cref{prop:homreg-prod-finite}, we write
\(\•A \prodpres \?B \homregto \?C\) as a synonym for \(\•A \prodpres \•B \homregto \•C\).

\begin{corollary}[Currying]
	\label{coro:homreg-currying}
	Let $\?A$, $\?B$ and $\?C$ be "automatic $\sigma$-structures",
	and let $\•A$ be an "rational presentation" of $\?A$.
	Then $\•A \prodpres \?B \homregto \?C$ "iff" $\•A \homregto \powstruct{\?C}{\?B}$.
\end{corollary}

\begin{proof}
	By \Cref{claim:homreg-prod-finite}… TODO.
\end{proof}

\subsection{Idempotent Core}

We fix a "purely relational signature" $\sigma$.
Given a "$\sigma$-structure" $\?B$,
we denote by \AP$\intro*\extendedSignature{\sigma}{\?B}$
the signature obtained from $\sigma$ by adding
a unary predicate \AP$\intro*\unarypred{b}$ for each $b\in B$.
The \AP""marked structure"" \AP$\intro*\marked{\?B}$ of $\?B$ is the
"$\extendedSignature{\sigma}{\?B}$-structure"
obtained from $\?B$ by "interpreting@@predicate" each predicate $\unarypred{b}$ as the
singleton $\{b\}$.\sidenote{TODO: ASIA'S REMARK: this is called the ``idempotent core''.}

\begin{proposition}[Folklore]
	\!\sidenote{The non-easy part is to reduce $\HomFinDec{\marked{\?B}}$ to $\HomFinDec{\?B}$: only this reduction
	requires the assumption that $\?B$ is a core. This reduction is folklore, see "eg"
	\cite[Lemma 2.5]{LaroseTesson2009UniversalAlgebraCSP}.
	TODO: understand if the proof is indentical to us. We provide here a self-contained proof.}%
	\sidenote{The "marked structure" $\marked{\core{\?B}}$ of the "core" of $\?B$ is
	usually called the \AP""idempotent core""
	of $\?B$. By TODO:addref and this proposition, $\HomFinDec{\?B}$ and
	$\HomFinDec{\marked{\core{\?B}}}$ are equivalent under "first-order reductions".
	This reduction is a central tool
	in the algebraic approach to understand "constraint satisfaction
	problem" since the algebra associated to the "CSP" over an "idempotent core"
	only has idempotent operations, making it much easier to work with. See TODO:addref for
	more details.}%
	%%%
	\AP\label{prop:marking-preserves-csp-complexity}
	If $\?B$ is a finite "core", then the problems $\HomAllClass{\marked{\?B}}$ and 
	$\HomAllClass{\?B}$ are "first-order equivalent".
	Moreover, this equivalence preserves "finiteness@@structure".%
	\footnote{And hence, by restricting this equivalence, we obtain that
	$\HomFinClass{\marked{\?B}}$ and $\HomFinClass{\?B}$ are "first-order equivalent".}
\end{proposition}

\begin{proof}[Proof of \Cref{prop:marking-preserves-csp-complexity}]
	\proofcase{Reduction from $\,\HomAllClass{\?B}$ to $\,\HomAllClass{\marked{\?B}}$.}
	We reduce a "$\sigma$-structure" $\?A$ to the
	"$\extendedSignature{\sigma}{\?B}$-structure" $\?A'$ obtained
	from $\?A$ by "interpreting@@predicate" each predicate $\unarypred{b}$ as the empty set.
	Clearly, a function from $A$ to $B$ is a "homomorphism" from $\?A$ to $\?B$
	"iff" it is a "homomorphism" from $\?A'$ to $\marked{\?B}$, proving the correctness
	of the reduction. It is, by definition, "first-order@@reduction".

	\proofcase{Reduction from $\,\HomAllClass{\marked{\?B}}$ to $\,\HomAllClass{\?B}$.}
	We first define the reduction $\Phi$ and show its correctness, we will show later that it
	is a "first-order reduction". We reduce a "$\extendedSignature{\sigma}{\?B}$-structure" $\?A$ to the "$\sigma$-structure"
	$\Phi(\?A)$ defined as follows:
	\begin{itemize}
		\item its underlying universe is the disjoint union $A \dcup B$,
		\item given a "predicate" $\+R$ of arity $k$, its "hyperedges" are:
		\begin{itemize}
		\item all $\+R$-"hyperedges" of $\?A$,
		\item all $\+R$-"hyperedges" of $\?B$, and
		\item all $\+R$-"hyperedges" $\tup{b_1,\hdots,b_{i-1}, a_i, b_{i+1},\hdots,b_k}$
			"st" there exists $b_i$ for which the $\+R$-"hyperedge"
			$\tup{b_1,\hdots,b_{i-1}, b_i, b_{i+1},\hdots,b_k}$
			is in $\+R(\?B)$, and $a_i$ belongs to the "interpretation@@predicate" of 
			$\unarypred{b_i}$ in $\?A$.
		\end{itemize}
	\end{itemize}
	Note that by construction, the "neighbourhood" of $a \in A$ in $\Phi(\?A)$ is
	the union of its "neighbourhood" in $\?A$, and the union of the "neighbourhoods" of
	$b$ in $\?B$ for all $b$ "st" $a \in \unarypred{b}(\?A)$.

	We show that $\?A \in \HomAllClass{\marked{\?B}}$ "iff" $\Phi(\?A) \in \HomAllClass{\?B}$.
	So, assume that there exists a "homomorphism" $f\colon \?A \to \marked{\?B}$.
	Then we let $f'\colon A \dcup B \to B$ be defined by $f'(a) = f(a)$ for all $a\in A$ and
	$f'(b) = b$ for all $b \in B$. We claim that $f'$ is a "homomorphism" from
	$\Phi(\?A)$ to $\?B$. Indeed, consider a "hyperedge" of $\Phi(\?A)$:
	\begin{itemize}
		\item if it is a "hyperedge" of $\?A$, its image by $f'$ is still a "hyperedge"
			of $\?B$ since $f$ is a "homomorphism" from $\?A$ to $\marked{\?B}$;
		\item if it is a "hyperedge" of $\?B$, then its image by $f'$ is itself, and is hence
			a "hyperedge" of $\?B$;
		\item otherwise, it must be of the form 
			\[\tup{b_1,\hdots,b_{i-1}, a_i, b_{i+1},\hdots,b_k}\]
			"st" there exists $b_i$ for which
			$\tup{b_1,\hdots,b_{i-1}, b_i, b_{i+1},\hdots,b_k} \in \+R(\?B)$
			and $a_i \in \unarypred{b_i}(\?A)$:
			in this case, its image by $f'$ is 
			\[f'(\tup{b_1,\hdots,b_{i-1}, a_i, b_{i+1},\hdots,b_k})
			= \tup{b_1,\hdots,b_{i-1}, b_i, b_{i+1},\hdots,b_k} \in \+R(\?B)\]
			since $f'(b) = b$ for all $b\in B$ and $f'(a_i) = f(a_i) = b_i$ since
			$a_i \in \unarypred{b_i}(\?A)$ and $f$ is a "homomorphism" from
			$\?A$ to $\marked{\?B}$.
	\end{itemize}
	And hence, $\Phi(\?A) \homto \?B$.

	Dually, let $g\colon \Phi(\?A) \to \?B$ be a "homomorphism".
	Its restriction to $\?B$, namely $\restr{g}{B}$ is a "homomorphism" from $\?B$
	to itself, and since $\?B$ is a "core", it must be an "automorphism" over $\?B$
	by todo:addref.
	We then define a map $g' \colon A \to B$ by sending
	$a$ to $(\restr{g}{B})^{-1}\circ g(a)$, and claim that it is a "homomorphism"
	from $\?A$ to $\marked{\?B}$. As a matter of fact, it clearly preserves $\+R$-"hyperedges"
	for any $\+R$ in $\sigma$, since $g$ and $(\restr{g}{B})^{-1}$ are "homomorphisms".
	We must then show that it preserves all unary "predicates" $\unarypred{b}$, with $b\in B$:
	let $a \in A$ "st" $\unarypred{b}$ holds, "ie" $a \in \unarypred{b}(\?A)$.
	Now, by construction of $\Phi(\?A)$, the "neighbourhood" of $g(a)$ in $\?B$
	and the "neighbourhood" of $g(b)$ in $\?B$ are equal.
	Since $\?B$ is a "core", it follows by todo:addref that $g(a) = g(b)$.
	By definition of $g'$, this rewrites as $g'(a) = b$, "ie" $g'(a) = \unarypred{b}(\marked{\?B})$.
	Therefore, we have built a "homomorphism" from $\?A$ to $\marked{\?B}$.

	Overall, this proves that $\Phi$ is correct. It is straightforward to notice that
	it is a "first-order reduction" and that it preserves "finiteness@@structure".
\end{proof}

\subsection{Transitive Tournaments "vs" Paths}

\begin{figure}
	\centering
	\begin{tikzpicture}
		\node[vertex] (0) at (0,0) {};
\node[vertex, below right=of 0] (1) {};
\foreach \i in {1, 3, 5, 7, 9} {
	\pgfmathtruncatemacro{\next}{\i + 1}
	\pgfmathtruncatemacro{\nnext}{\i + 2}
	\node[vertex, below right=of \i] (\next) {};
	\node[vertex, above right=of \next] (\nnext) {};
};
\node[vertex, below right=of 11] (12) {};
\node[vertex, below right=of 12] (13) {};

% ---
% Transitions
% ---
\foreach \i in {0,1,3,5,7,9,11,12} {
	\pgfmathtruncatemacro{\next}{\i + 1}
	\draw[edge] (\i) to (\next);
};
\foreach \i in {2,4,6,8,10} {
	\pgfmathtruncatemacro{\next}{\i + 1}
	\draw[edge] (\next) to (\i);
}
		
\foreach \i in {1,3,5,7,9,11} {
	\pgfmathtruncatemacro{\half}{\i / 2}
	\node[draw=none, above right=-.6em of \i] {$a_{\half}$};
};
\foreach \i in {2,4,6, 8, 10, 12} {
	\pgfmathtruncatemacro{\half}{(\i / 2) - 1}
	\node[draw=none, below left=-.6em of \i] {$b_{\half}$};
};
\node[draw=none, below left=-.6em of 0] {$a'_0$};
\node[draw=none, above right=-.6em of 13] {$b'_5$};

\draw[decoration={brace}, decorate, transform canvas={yshift=1.5em}] (3.north west) to node[midway, above] {$\+O(n)$ nodes} (11.north east);
	\end{tikzpicture}
	\caption{\AP\label{fig:zigzag-graph}The "zigzag graph" $\zigzag{5}{2}$.}
\end{figure}
\begin{example}
	\AP\label{ex:zigzag-defn}
	Let $n\in\?N$.
	We define the \AP""zigzag graph"" $\intro*\zigzag{n}{2}$ of width $n$ and length 2
	to be "graph" whose vertices are $a_0, \hdots, a_n$, $b_0, \hdots, b_{n}$,
	$a'_0$ and $b'_n$, with edges from $a_i$ to $b_{i-1}$ and to $b_{i}$ (for $i \in \lBrack 0,n\rBrack$, whenever the nodes exist), and with an edge from $a'_0$ to $a_0$ and from $b_n$
	to $b'_n$. See \Cref{fig:zigzag-graph} for an illustration.
	
	Note that $\zigzag{n}{2}$ does not admit a "homomorphism" to the "$2$-path"---indeed, such a homomorphism should send $a'_0$, $a_0$ and $b_0$ onto $0$, $1$, and $2$, respectively, 
	and so all $a_i$'s (resp. $b_i$'s) must be sent onto $1$ (resp. $2$), but then $b'_n$ cannot be mapped anywhere.

	On the other hand, $\zigzag{n}{2}$ admits a "homomorphism" to the "$2$-transitive tournament", as witnessed by \Cref{fig:zigzag-graph-hom-T2}.
	In fact, this "homomorphism" is far from being unique:
	each vertex $a_1,\,a_2,\,\hdots,\,a_{n-1}$ can be sent on either $0$ or $1$
	(the red and purple vertices), 
	and similarly, each vertex $b_1,\,b_2,\,\hdots,\,b_{n-1}$ can be sent on either $1$ or $2$
	(the purple and blue vertices).\footnote{Note that it is straightforward
	to extend these results---namely $\zigzag{n}{2} \homto \transitiveTournament{2}$
	and $\zigzag{n}{2} \nothomto \pathGraph{2}$---to arbitrary values of $k\in \N$ with
	$k\geq 2$, namely $\zigzag{n}{k} \homto \transitiveTournament{k}$
	and $\zigzag{n}{k} \nothomto \pathGraph{k}$, by letting
	$\zigzag{n}{k}$ be the graph obtained from $\zigzag{n}{2}$ by
	replacing the path leading to $a_0$ by a path of length $k-1$.
	}
\end{example}
\begin{figure}
	\centering 
	\begin{tikzpicture}
		\node[vertex, draw=c0, fill=c0, fill opacity=.4] (0) at (0,0) {};
\node[vertex, below right=of 0, draw=c1, fill=c1, fill opacity=.4] (1) {};
\foreach \i in {1, 3, 5, 7, 9} {
	\pgfmathtruncatemacro{\next}{\i + 1}
	\pgfmathtruncatemacro{\nnext}{\i + 2}
	\node[vertex, below right=of \i, draw=c2, fill=c2, fill opacity=.4] (\next) {};
	\node[vertex, above right=of \next, draw=c0, fill=c0, fill opacity=.4] (\nnext) {};
};
\node[vertex, below right=of 11, draw=c1, fill=c1, fill opacity=.4] (12) {};
\node[vertex, below right=of 12, draw=c2, fill=c2, fill opacity=.4] (13) {};

% ---
% Transitions
% ---
\foreach \i in {0,1,3,5,7,9,11,12} {
	\pgfmathtruncatemacro{\next}{\i + 1}
	\draw[edge] (\i) to (\next);
};
\foreach \i in {2,4,6,8,10} {
	\pgfmathtruncatemacro{\next}{\i + 1}
	\draw[edge] (\next) to (\i);
}

% ---
% 2-transitive tournament
% ---
\node[vertex, above right=-.25em and 5em of 13, draw=c2, fill=c2, fill opacity=.4] (t2-2) {};
\node[vertex, above=of t2-2, draw=c1, fill=c1, fill opacity=.4] (t2-1) {};
\node[vertex, above=of t2-1, draw=c0, fill=c0, fill opacity=.4] (t2-0) {};

\draw[edge] (t2-0) to (t2-1) to (t2-2);
\draw[edge] (t2-0) to[bend left=60] (t2-2);
	\end{tikzpicture}
	\caption{\AP\label{fig:zigzag-graph-hom-T2}A "homomorphism" from the "zigzag graph" (left-hand side) to the "$2$-transitive tournament" (right-hand side).}
\end{figure}