\section{Introduction}
\label{sec:dichotomy-introduction}


\subsection{Classes of Relations}
\label{sec:dichotomy-introduction-relations}

The complex landscape of rationality for finite-word relations 
described in \Cref{sec:preliminaries-automatic-structures-relations}
yields a natural question of \emph{effectivity}: given a relation in some class,
can we decide if it belongs to a fixed subclass? 
% Recall that \AP"rational relations" $\RAT$ are those definable by "multitape automata",
% with heads possibly moving asynchronously;
% "automatic relations"
% are those that are accepted by "synchronous relations"; and 
% \AP"recognizable relations" $\REC$ correspond to finite unions of products of regular languages---or, equivalently, to languages recognized via finite monoids, by "Mezei theorem".
% By definition, all of these classes coincide with the 
% class of "regular languages" when restricted to unary relations.

% \AP ""Automatic relations"" are those definable by multi-head automata, with heads possibly moving aautomaticly; "automatic relations" are "automatic relations" that are accepted by multi-head automata whose heads are forced to move automaticly; and 
% "recognizable relations" correspond to finite unions of products of regular languages (or, equivalently, to languages recognized via finite monoids, by Mezei Theorem). By definition, all of these classes coincide with the 
% class of regular languages when restricted to unary relations.

Prior work has focused on the "$\REC$-membership problem", 
which takes as input an $n$-ary "rational relation"
$\+R$ and asks whether it is equivalent to a "recognizable relation"
$\bigcup_i L_{i,1} \times \cdots \times L_{i,n}$, where each $L_{i,j}$ is a "regular language". 
Intuitively, the problem asks whether the different components of  
the "rational relation" $\+R$ are almost independent of one another.
The study of "$\REC$-membership@@problem"
is relevant since relations enjoying this property are often amenable to some analysis including,
"eg", abstract interpretations in program verification, variable elimination in constraint logic 
programming, and query processing over constraint databases---see \cite[Introduction]{BarceloHongLeLinNiskanen2019MonadicDecomposability} for a more thorough discussion on this topic.

As mentioned in \Cref{sec:prelim-automatic-drat}, in general, "$\REC$-membership@@problem" of "rational relations" is undecidable
\cite[\S~III, Theorem~8.4]{Berstel1979Transductions},
but it becomes decidable  for the subclass of 
"deterministic rational relations", which extends the class of "automatic relations".
For "automatic relations", the decidability of "$\REC$-membership@@problem"
can be obtained by a  simple reduction to the problem of checking whether a finite automaton 
recognizes an infinite language \cite{LodingSpinrath2019DecisionProblems}---which is decidable via 
a standard reachability argument. 
The precise complexity of the problem, however, was only recently pinned down.
By applying techniques based on Ramsey Theorem over infinite graphs, it was shown that 
"$\REC$-membership@@problem" of "automatic relations" is 
"PSpace"-complete when relations are specified by non-deterministic automata
\cite[Theorem 1]{BarceloHongLeLinNiskanen2019MonadicDecomposability} \cite[Corollary 2.9]{BergstrasserGanardiLinZetzsche2022RamseyQuantifiers}.

\AP A natural generalization of this question is the
""$\RAT$/$\REC$-separability problem"", which takes two $n$-ary "rational relations" 
$\+R,\+R' \subseteq \Sigma^* \times \Sigma^*$ and checks whether there is a "recognizable relation" 
$S = \bigcup_i L_{i,1} \times \cdots \times L_{i,n}$ that "separates@@rel" $\+R$ from $\+R'$, meaning that $\+R \subseteq \+S$ and $\+R' \cap \+S = \emptyset$, in which case we say that $\+R$
and $\+R'$ are \AP""separable@@rel"" by a "recognizable relation".
In other words, this problem asks whether we can \emph{overapproximate} $\+R$ with a recognizable 
relation $\+S$ that is constrained not to intersect with $\+R'$. Separability problems of this kind abound in theoretical computer science, in particular
in formal language theory where they have gained a lot of attention over the last few years---see "eg" \cite{PlaceZeitoun2016SeparatingRegularLanguages,Kopczynski2016InvisiblePushdownLanguages,CzerwinskiMartensRooijenZeitounZetzsche2017DecidableSeparabilityPiecewiseTestable,ClementeCzerwinskiLasotaPaperman2017RegularSeparabilityParikhAutomata}. 

In this chapter, we focus on the following problem:
\decisionproblem{
	""$\AUT$/$\REC$-separability problem""
}{
	Two "automatic relations" $\+R$ and $\+R'$.
}{
	Does there exist a "recognizable relation" $\+S$ "st" $\+R \subseteq \+S$
	and $\+R' \cap \+S = \emptyset$?
}
Notice that when $\+R'$ is the complement of $\+R$ this problem boils down to "$\REC$-membership@@problem" on "automatic relations"---which is decidable. 
However, "$\AUT$/$\REC$-separability@@problem" is more general than this problem,
and to this day it is unknown whether it is decidable. 


\subsection{Constraint Satisfaction Problems}
\label{sec:dichotomy-intro-csp}

Our work on the "$\AUT$/$\REC$-separability problem" will eventually lead up to
the study of "homomorphism problems" where the "target structure" is fixed.
These problems are the central topic in the domain of "constraint satisfaction problems" ("CSPs").%
\footnote{A typical example of such a problem is to work on the "graph signature", and to fix the
target structure to be the "$k$-clique" ($k \in \Np$): it exactly corresponds to
the "$k$-colourability problem".}
When the target structure is finite, the "homomorphism problem" is known to be decidable
in "NP". The precise complexity of the problem, however, depends on the "structure" of the target,
and is tightly connected to the algebraic properties of the algebra associated to the
"structure".

A central result in the domain, originally conjectured by Feder and Vardi,
and known as the ``dichotomy theorem'' (formerly ``dichotomy conjecture''),
states that any "CSP" whose "target structure" is 
finite is either in "P" or "NP"-complete
\cite[\S~2, ``Dichotomy question'']{FederVardi1998ComputationalStructure},%
\footnote{In other words, there is no "NP"-intermediate 
problem amongst these problems.} and the characterization of "structures" for which the problem
was in "P" is precisely formulated in algebraic terms.%
\footnote{Structures that admit a "CSP" problem in "P" are exactly those whose algebra admits a
``weak near-unanimity operation''---see "eg" \cite[\S~1, p.~3]{Zhuk2020CSPDichotomy}.}
This conjecture extended a result of Schaefer, who proved two decades earlier
that "CSPs" over the Boolean domain\footnote{Meaning that the "target structure" can have only two elements---but recall that the "signature" $\sigma$ can be arbitrary complex.} was either "P" or "NP"-complete \cite[Theorem~2.1]{Schaefer1978ComplexitySatisfiability}.
Twenty-four years later, this conjecture was proven independently by
Bulatov \cite[Theorem 1]{Bulatov2017DichotomyCSPs}
and Zhuk \cite[Theorem 1.4]{Zhuk2020CSPDichotomy}.
This algebraic approach is particularly fruitful but lies beyond
the scope of this thesis: we refer the reader to \cite{BartoKrokhinWillard2017Polymorphisms,Larose2017DigraphCSP} for surveys on the topic.

\begin{marginfigure}
	\centering
	\begin{tikzpicture}
		% ---
% 2-transitive tournament
% ---
\node[vertex, draw=c2, fill=c2, fill opacity=.4] at (0,0) (t2-2) {};
\node[vertex, above=of t2-2, draw=c1, fill=c1, fill opacity=.4] (t2-1) {};
\node[vertex, above=of t2-1, draw=c0, fill=c0, fill opacity=.4] (t2-0) {};

\draw[edge] (t2-0) to (t2-1);
\draw[edge] (t2-1) to (t2-2);
\draw[edge] (t2-0) to[bend left=60] (t2-2);
	\end{tikzpicture}
	\hspace{4em}
	\begin{tikzpicture}
		% ---
% 2-path graph
% ---
\node[vertex, draw=c2, fill=c2, fill opacity=.4] at (0,0) (t2-2) {};
\node[vertex, above=of t2-2, draw=c1, fill=c1, fill opacity=.4] (t2-1) {};
\node[vertex, above=of t2-1, draw=c0, fill=c0, fill opacity=.4] (t2-0) {};

\draw[edge] (t2-0) to (t2-1);
\draw[edge] (t2-1) to (t2-2);
	\end{tikzpicture}
	\caption{
		\AP\label{fig:T2-P2}
		The "$2$-transitive tournament" $\transitiveTournament{2}$
		(left-hand side) and the "$2$-path" $\pathGraph{2}$ (right-hand side).
	}
\end{marginfigure}
Beyond "P" and "NP", the complexity of the "homomorphism" problem can reach some
surprisingly low complexities.
Even for Boolean CSPs\footnote{Meaning again that the "target structure" has only two elements},
Allender, Bauland, Immerman, Schnoor and Vollmer extended Schaefer's theorem to
prove that every not-so-easy problem---meaning that it is not solvable in "coNLogTime"---is
complete for one class among "NP", "P", "modL", "NL" or "L" under "AC0"-reductions
\cite[Theorem~3.1]{AllenderBaulandImmermanSchnoorVollmer2009Schaefer}.

For non-Boolean CSPs, the landscape becomes even more complex.
For instance, given $k\in\N$, we define the \AP""$k$-transitive tournament"", denoted
by \AP$\intro*\transitiveTournament{k}$ and illustrated on \Cref{fig:T2-P2},
to be the "directed graph" with vertices $0$, $1$, $\dotsc$, $k$,
and with an edge from $i$ to $j$ "iff" $i < j$.
Similarly, the \AP""$k$-path"" \AP$\intro*\pathGraph{k}$ has $\intInt{0,k}$ as
its set of vertices, with an edge from $i$ to $j$ "iff" $j = i + 1$---see \Cref{fig:T2-P2}.

Then, a "graph@@dir" $\?G$ admits a "homomorphism" to $\transitiveTournament{k}$ "iff"
it has no directed path of size $k+1$, in the sense that this is no "homomorphism"
from $\pathGraph{k+1}$ to $\?G$. Symbolically:
\[
	\forall \?G,\quad
	\pathGraph{k+1} \nothomto \?G
	\quad\text{"iff"}\quad
	\?G \homto \transitiveTournament{k}.
\]
As a result, to decide if, for a "finite graph" $\?G$, we have
$\?G \homto \transitiveTournament{k}$, it suffices
to test if $\pathGraph{k+1} \nothomto \?G$, which can be done in "FOfin"!

Structures sharing the same property as $\transitiveTournament{k}$, meaning that
the existence of a "homomorphism" to them amounts to not containing any "obstructions"
among a fixed finite set of "obstructions" are said to have "finite duality".
Like $\transitiveTournament{k}$, their "constraint satisfaction problem"
is in "FOfin".\footnote{Recall that "FOfin" $\subseteq$ "L", see \Cref{prop:FO-in-L}.}
Atserias proved the converse implication to this result \cite[Corollary 4]{Atserias2008DigraphColoring}.%
\footnote{This result was followed in the same year
by Rossman's theorem, that subsumes it \cite[Theorem~1.3]{Rossman2008Homomorphism}.}

\begin{proposition}[""Atserias' theorem""]
	\AP Let $\?B$ be a "finite $\sigma$-structure". Then $\?B$ has "finite duality"
	if, and only if, $\HomAll{\?B}$ is "first-order definable".
\end{proposition}

Moreover, Larose and Tesson proved a dichotomy theorem for "CSPs" with small complexity.
\begin{proposition}[""Larose-Tesson theorem""]
	Let $\?B$ be a "finite $\sigma$-structure". If $\?B$ does not have "finite duality",
	then $\HomFin{\?B}$ is "L"-hard under "first-order reductions".
\end{proposition}

Observe that this last proposition only allow for "finite structures" on the
"input side@@struct" of the problem.
This chapter will generalize some results---including the latter---on the "homomorphism problem"
from "finite structures" to "automatic structures". The study of "CSPs" beyond finite set
is far from a new topic, and has been the subject of many works in the past decade:
see for instance \cite{KlinKopczynskiOchremiakTorunczyk2015LocallyFiniteCSP} for "CSPs"
over ``nominal structures'', "ie" "structures" with ``atoms'',
or \cite{KlinLasotaOchremiakTorunczyk2016HomomorphismProblems} for "CSPs" over
"$\sigma$-structures" which are "first-order definable" over $\tup{\N,{=}}$.


\subsection{Contributions \& Organization}

In \Cref{sec:dichotomy-colouring}, we start
by rephrasing the "$\AUT$/$\REC$-separability problem" in graph-theoretic terms.

\decisionproblem{
	""Finite Regular Colourability of Automatic Graphs""
}{
	A "presentation" $\•G$ of an "automatic graph" $\?G$.
}{
	Does $\•G$ admit a "regular colouring" with finitely many colours?
}
A "regular colouring" with finitely many colours (resp. a "regular $k$-colouring")
is a partition of the vertices of $\•G$ into finitely many (resp. $k$) regular languages
"st" adjacent vertices cannot belong to the same set.
Similarly, the \AP""regular $k$-colourability problem"" asks instead if
$\•G$ admits a "regular $k$-colouring", in which case we say that $\•G$ is
\AP""regularly $k$-colourable"".

\begin{restatable*}{theorem}{regcolourabilityequivseparability}
    \AP\label{thm:reg-colourability-equiv-separability}
    There are polynomial-time reductions: 
    \begin{enumerate}
        \itemAP\label{item:reg-colourability-equiv-separability-1} from "$\AUT$/$\REC$-separability@@problem" to "finite regular colourability@@problem", 
        \itemAP\label{item:reg-colourability-equiv-separability-2} from "finite regular colourability@@problem" to "$\AUT$/$\REC$-separability@@problem", and
        \itemAP\label{item:reg-colourability-equiv-separability-3} from "regular $k$-colourability@@problem" to "$\AUT$/$\kREC$-separability@@problem", for every $k > 0$.
    \end{enumerate}
    Further, the last two reductions are so that the second relation in the instance of the separability problem is the identity $\Id$.
\end{restatable*}

While we do not know how to solve the "finite regular colourability@@problem",
we prove the problem to be undecidable if the number of colours is fixed.

\begin{restatable*}{theorem}{kregcolundec}
    \AP\label{thm:k-reg-col-undec}
    The "regular $k$-colourability problem" on "automatic graphs" is undecidable for every $k\geq 2$. More precisely, the problem is "RE"-complete. This holds also for "connected" "automatic graphs".
\end{restatable*}

Then, in \Cref{sec:dichotomy-bounded}, we build on this result to prove the undecidability of
some separability problems for "automatic relations" (\Cref{coro:krec-sep-undec,prop:kprod-undecidable}).

\begin{marginfigure}[-10em]
	\centering
	\begin{tikzpicture}
		% ---
% 3-clique
% ---
\node[vertex, draw=c0, fill=c0, fill opacity=.4] at (0,0) (k3-0) {};
\node[vertex, above right=1em and 2em of k3-0, draw=c1, fill=c1, fill opacity=.4] (k3-1) {};
\node[vertex, below right=1em and 2em of k3-0, draw=c2, fill=c2, fill opacity=.4] (k3-2) {};

\draw[edge, bend right=15] (k3-0) to (k3-1);
\draw[edge, bend right=15] (k3-0) to (k3-2);
\draw[edge, bend right=15] (k3-1) to (k3-0);
\draw[edge, bend right=15] (k3-1) to (k3-2);
\draw[edge, bend right=15] (k3-2) to (k3-0);
\draw[edge, bend right=15] (k3-2) to (k3-1);
	\end{tikzpicture}
	\caption{
		\AP\label{fig:dichotomy-3-clique}
		The "$3$-clique" $\clique{3}$.
		(Replica of \Cref{fig:intro-3-clique}.)
	}
\end{marginfigure}
\begin{marginfigure}
	\centering
	\begin{tikzpicture}
		% ---
% Example 3-colouring
% ---
\tikzset{every node/.style={fill opacity=.4}}
\node[vertex, color=c0, fill=c0] at (0,0) (0) {};
\node[vertex, above left=.75em of 0, color=c1, fill=c1] (0al) {};
\node[vertex, above right=.75em of 0, color=c2, fill=c2] (0ar) {};
\node[vertex, below left=2em of 0al, color=c0, fill=c0] (0l) {};
\node[vertex, below right=2em of 0ar, color=c0, fill=c0] (0r) {};
\node[vertex, above left=of 0l, color=c1, fill=c1] (0ll) {};
\node[vertex, above right=of 0r, color=c2, fill=c2] (0rr) {};
\node[vertex, below left=of 0, color=c2, fill=c2] (0bl) {};
\node[vertex, below right=of 0, color=c1, fill=c1] (0br) {};

\node[vertex, below=2em of 0, color=c0, fill=c0] (1) {};
\node[vertex, below left=1em of 1, color=c1, fill=c1] (1bl) {};
\node[vertex, below right=1em of 1, color=c2, fill=c2] (1br) {};
\node[vertex, above left=0em and 1.5em of 1bl, color=c2, fill=c2] (1bll) {};
\node[vertex, above right=0em and 1.5em of 1br, color=c1, fill=c1] (1brr) {};

\node[vertex, below=4em of 1, color=c0, fill=c0] (2) {};
\node[vertex, below left=1.5em and .75em of 1bl, color=c2, fill=c2] (2al) {};
\node[vertex, below right=1.5em and .75em of 1br, color=c1, fill=c1] (2ar) {};
\node[vertex, below left=.5em and 1.5em of 2, color=c1, fill=c1] (2bl) {};
\node[vertex, below right=.5em and 1.5em of 2, color=c2, fill=c2] (2br) {};
\node[vertex, below left=of 2al, color=c0, fill=c0] (2ll) {};
\node[vertex, below right=of 2ar, color=c0, fill=c0] (2rr) {};

\draw[edge] (0) to (0al);
\draw[edge] (0) to (0ar);

\draw[edge] (0al) to (0l);
\draw[edge] (0l) to (0bl);
\draw[edge] (0bl) to (0);
\draw[edge] (0l) to (0ll);

\draw[edge] (0ar) to (0r);
\draw[edge] (0r) to (0br);
\draw[edge] (0br) to (0);
\draw[edge] (0r) to (0rr);

\draw[edge] (0bl) to (1);
\draw[edge] (1) to (1bl);
\draw[edge] (1bl) to (0l);

\draw[edge] (0br) to (1);
\draw[edge] (1) to (1br);
\draw[edge] (1br) to (0r);

\draw[edge] (1bl) to (2al);
\draw[edge] (2al) to (2ll);
\draw[edge] (2) to (2al);
\draw[edge] (2) to (2bl);
\draw[edge] (2bl) to (2al);

\draw[edge] (1br) to (2ar);
\draw[edge] (2ar) to (2rr);
\draw[edge] (2) to (2ar);
\draw[edge] (2) to (2br);
\draw[edge] (2br) to (2ar);

\draw[edge] (1bl) to (1bll);
\draw[edge] (1br) to (1brr);

\draw[edge, bend right=20] (2) to (1bl);
\draw[edge, bend left=20] (2) to (1br);


	\end{tikzpicture}
	\caption{
		\AP\label{fig:dichotomy-ex-3-colouring}
		A "$3$-colouring" of some beetle-shaped "graph@@dir".
		(Replica of \Cref{fig:dichotomy-ex-3-clique}.)
	}
\end{marginfigure}

Observing that "regular $k$-colourability" can be rephrased as the existence of a 
"regular homomorphism"---meaning, in this context, a "homomorphism" whose preimages are all
"regular languages"---to the "$k$-clique",%
\footnote{See \Cref{fig:dichotomy-3-clique,fig:dichotomy-ex-3-colouring}.}
this motivates us to study the "homomorphism problem"
and "regular homomorphism problem" for "automatic structures" in \Cref{sec:dichotomy-undecidability,sec:dichotomy-decidability}. Our main result is a dichotomy theorem for "automatic structures".\footnote{Note that while for "finite structures" the dichotomy theorems either separate
"FOfin" from "L"---in the case of Larose and Tesson---, or "P" from "NP"---for Bulatov and Zhuk---, in our case the dichotomy is between "NL" and… undecidability!}

\begin{restatable*}[\introinrestatable{Dichotomy Theorem for Automatic Structures}]{theorem}{DichotomyThmDichotomyAutomatic}
	\AP\label{thm:dichotomy-theorem-automatic-structures}
	Let $\?B$ be an arbitrary "$\sigma$-structure". The following are equivalent:
	\begin{description}
		\itemAP[\introinrestatable\itemDTFinDual.] $\?B$ has "finite duality";
		\itemAP[\introinrestatable\itemDTHomDec.] $\HomAut{\?B}$ is decidable;
		\itemAP[\introinrestatable\itemDTHomRegDec.] $\HomRegAut{\?B}$ is decidable;
		\itemAP[\introinrestatable\itemDTEqual.] $\HomAut{\?B} = \HomRegAut{\?B}$, "ie" for any "automatic presentation" $\•A$ of a 
		"$\sigma$-structure" $\?A$, there is a "homomorphism" from $\?A$ to $\?B$ "iff" 
		there is a "regular homomorphism" from $\•A$ to $\?B$;
		\itemAP[\introinrestatable\itemDTFirstOrder.] $\HomAll{\?B}$ has "uniformly first-order definable homomorphisms".\footnote{The notion of "uniformly first-order definable homomorphisms" is defined in \Cref{sec:uniformly-first-order-definable-hom}.}
	\end{description}
	Moreover, when $\HomAut{\?B}$ and $\HomRegAut{\?B}$ are undecidable, they are "coRE"-complete
	and "RE"-complete, respectively. When they are decidable, they are "NL".
\end{restatable*}

The easy implications of this theorem are proven in \Cref{sec:dichotomy-overview},
and the undecidability results---one for "homomorphisms", and another one for "regular homomorphisms" in the rest of \Cref{sec:dichotomy-undecidability}:
both reductions for undecidability generalize Larose and Tesson's proof.
In the case of "regular homomorphisms", the full sequence of reductions
to prove undecidability goes as follows:
\begin{itemize}
	\item from the "halting problem" on "deterministic" "reversible Turing machines"
		to the "regular reachability problem" for "linear Turing machines";
	\item which is reduced in turn to "regular unconnectivity in automatic graphs";
	\item and lastly, this latter problem is reduced
		to $\HomRegAut{\?B}$ when $\?B$ does not have "finite duality"---this
		reduction is the adaptation of Larose and Tesson's proof.
\end{itemize}
To our knowledge, none of these problems---except the "halting problem"---have been studied before.
Decidability results are given in \Cref{sec:dichotomy-decidability}:
in fact, the decidability of $\HomAut{\?B}$ when $\?B$ has "finite duality"
is trivial, so this section is dedicated to the problem $\HomRegAut{\?B}$.
In \Cref{sec:uniformly-first-order-definable-hom} we give a succinct
logic-based proof of the decidability of $\HomAut{\?B}$, however the proof is
somewhat \emph{abstract}. In \Cref{sec:hyperedge-consistency-finite,sec:hyperedge-consistency-automatic} we give a more visual solution,
known as "hyperegde consistency algorithm@@automatic". It generalizes
the eponymous algorithm for "finite structures", that captures
the "homomorphism problem" when the "target structure" has "tree duality"---which is
a supclass of "finite duality".
However, the proof of correctness of our algorithm is non-trivial
and relies on providing a fine understanding of the behaviour of the algorithm
for "finite structures" in the special case of "target structures"
that have "finite duality".
We conclude in \Cref{sec:dichotomy-discussion}, by discussing on
conjectures and related problems.

