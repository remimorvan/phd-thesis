\section{Introduction}
\AP\label{sec:dichotomy-introduction}


\subsection{Classes of Relations}

The study of classes of relations on words\footnote{In this context, ``relations on words''
are often called ``transductions''.} has always been a central topic in language theory
\cite{ElgotMezi1965RelationsGeneralizedAutomata,Nivat1968TransductionChomsky,Berstel1979Transductions,FrougnySakarovitch1993SynchronizedRationalRelations,Choffrut2006Survey}. 
More recently, their study has been motivated by database theory and verification,
where they are used to build expressive languages. For instance, classes of relations of this kind are relevant for querying strings over relational 
databases \cite{BenediktLibkinSchwentickSegoufin2003DefinableRelations}, comparing paths in graph databases \cite{BarceloLibkinLinWood2012ExpressiveLanguages}, or defining  
string constraints for model checking \cite{LinBarcelo2016StringSolvingWordEquationsTransducers}. 
The most studied such classes include "recognizable@@rel", "automatic@@rel" "aka" "synchronous@@rel", and "rational relations", each one of the latter two strictly extending the previous one. 

% \AP ""Rational relations"" are those definable by multi-head automata, with heads possibly moving asynchronously; "automatic relations" are "rational relations" that are accepted by multi-head automata whose heads are forced to move synchronously; and 
% "recognizable relations" correspond to finite unions of products of regular languages (or, equivalently, to languages recognized via finite monoids, by Mezei's Theorem). By definition, all of these classes coincide with the 
% class of regular languages when restricted to unary relations.

\subsection{Constraint Satisfaction Problems}



\subsection{Contributions}



\subsection*{Old}

\begin{restatable*}{theorem}{DichotomyThmDichotomyAutomatic}[Dichotomy Theorem for Automatic Structures]
	\!\footnote{The equivalence between (1) and (2) still holds
	when the "homomorphism problem" takes as input ``higher-order automatic
	structures'' as defined in \cite[last remark of \S~XII.3]{Blumensath2024MSOModelTheory},
	since such structures have a decidable first-order theory.}%
	%%%
	\AP\label{thm:dichotomy-theorem-automatic-structures}
	Let $\?B$ be an arbitrary "$\sigma$-structure". The following are equivalent:
	\begin{description}
		\itemAP[\intro*\itemDTFinDual.] $\?B$ has "finite duality";
		\itemAP[\intro*\itemDTHomDec.] $\HomAutDec{\?B}$ is decidable;
		\itemAP[\intro*\itemDTHomRegDec.] $\HomRegAutDec{\?B}$ is decidable;
		\itemAP[\intro*\itemDTEqual.] $\HomAutDec{\?B} = \HomRegAutDec{\?B}$ "ie" for any "rational presentation" $\•A$ of a 
		"$\sigma$-structure" $\?A$, there is a "homomorphism" from $\?A$ to $\?B$ "iff" 
		there is a "regular homomorphism" from $\•A$ to $\?B$;
		\itemAP[\intro*\itemDTFirstOrder.] $\HomAllClass{\?B}$ has "uniformly first-order definable homomorphisms".
	\end{description}
	Moreover, when $\HomAutDec{\?B}$ and $\HomRegAutDec{\?B}$ undecidable, they are "coRE"-complete
	and "RE"-complete, respectively. When they are decidable, they are "NL".
\end{restatable*}
	  
TODO: say that to have "finite duality" can be decided.