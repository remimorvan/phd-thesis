\section{\AP\label{sec:dichotomy-introduction}%
	Introduction}

\begin{theorem}[Dichotomy Theorem for Automatic Structures]
	\!\footnote{The equivalence between (1) and (2) still holds
	when the "homomorphism problem" takes as input ``higher-order automatic
	structures'' as defined in \cite[last remark of \S~XII.3]{Blumensath2024MSOModelTheory},
	since such structures have a decidable first-order theory.}%
	%%%
	\AP\label{thm:dichotomy-theorem-automatic-structures}
	Let $\?B$ be a "finite $\sigma$-structure". The following are equivalent:
	\begin{description}
		\itemAP[\intro*\itemDTFinDual.] $\?B$ has "finite duality";
		\itemAP[\intro*\itemDTHomDec.] $\HomAutDec{\?B}$ is decidable;
		\itemAP[\intro*\itemDTHomRegDec.] $\HomRegAutDec{\?B}$ is decidable;
		\itemAP[\intro*\itemDTEqual.] $\HomAutDec{\?B} = \HomRegAutDec{\?B}$ "ie" for any "automatic presentation" $\•A$ of a 
		"$\sigma$-structure" $\?A$, there is a "homomorphism" from $\?A$ to $\?B$ "iff" 
		there is a "regular homomorphism" from $\•A$ to $\?B$;
		\itemAP[\intro*\itemDTFirstOrder.] $\HomAllClass{\?B}$ has "uniformly first-order definable homomorphisms".
	\end{description}
	Moreover, when $\HomAutDec{\?B}$ and $\HomRegAutDec{\?B}$ undecidable, they are "coRE"-complete
	and "RE"-complete, respectively. When they are decidable, they are TODO.
\end{theorem}
	  
TODO: say that to have "finite duality" can be decided.