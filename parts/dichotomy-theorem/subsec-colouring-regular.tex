\subsection{
    \AP\label{sec:dichotomy-k-regular-colourability}%
    $k$-Regular Colourability Problem
}

We show that the "$k$-regular colourability problem" is undecidable for $k\geq 2$.%
\footnote{
    Using this, we obtain in the next subsection 
    the undecidability for the separability problem on two natural classes of 
    "recognizable relations".
    % On the other hand, the undecidability of "regular colourability problem" is more involved and will only come later, in todo:addref.
}
This is proven by a reduction from a suitable problem on reversible 
Turing machines with certain restrictions, which we call ``well-founded''.

\paragraph*{Regularity of Reachability for Turing Machines.}
Consider a Turing machine $\+T = \tup{Q,\Gamma,\delta,q_0,\Acc}$, where $Q$ is the set of states, $\Gamma$ is tape alphabet,
\[
    \delta\colon (Q \setminus \Acc) \times \Gamma_{\smallpad} \to \pset{Q \times \Gamma \times \set{\leftarrow, \downarrow, \rightarrow}}
\]
is the transition relation, $\Gamma_{\smallpad} = \Gamma \dcup \set{\pad}$, and $q_0$ and $\Acc$ are the initial and set of final states, respectively.
%
We represent a "configuration@@TM" $\tup{u, q, v}$ by the word $uqv \in \Gamma^* Q \Gamma^*$:
in light of this, we will henceforth denote by ``configuration'' any string from the set  \AP$\intro*\configs \defeq  \Gamma^* Q \Gamma^*$.\footnote{We will often write
$uqv$ as the concatenation $u\cdot q \cdot v$ to emphasize
the separation between all three words.}
The \AP""configuration graph"" of $\+T$ is the infinite graph $\intro*\confGraph$ having $\configs$ as set of vertices and an edge from $\gamma$ to $\gamma'$, denoted $\gamma \rightarrow \gamma'$ if there is a one-step transition from $\gamma$ to $\gamma'$ in $\+T$. Observe that the "configuration graph" $\confGraph$ of any Turing machine $\+T$ is an effective "rational graph" (see, e.g., \cite{KuskeLohrey2010AutomaticGraphs} TODO:update pointer to prelims).

\AP We say that a Turing machine $\+T$ is ""reversible"", if every node of $\confGraph$ has in-degree at most 1, in other words if the machine is co-deterministic.%
\footnote{
    Note that a modern proof of undecidability of the isomorphism problem for automatic structures 
    by Blumensath \cite[\S VIII. Theorem 4.3, p. 396 \& second claim, p. 398]{Blumensath2023MSO} 
    also relies on the use of "reversible" Turing machines.
}%
\footnote{
    For more details and pointers on "reversible" Turing machines,
    see \cite[Chapter 5]{Morita2017Reversible}.
}
We say that a Turing machine $\+T$ is ""well-founded"" if its "configuration graph" is such that:
\begin{enumerate}
    \item the "initial configuration" has in-degree zero, and
    \item there are no infinite backward paths $\gamma_0 \leftarrow \gamma_1 \leftarrow \cdots$ in $\confGraph$. 
\end{enumerate}

We say that a Turing machine is ""linear@@TM"" if it is "well-founded", deterministic and "reversible".
By construction, a Turing machine is "linear@@TM" "iff" (1) its "configuration graph" consists of a possibly infinite disjoint union of directed paths, which are all finite, or isomorphic to $\tup{\N, +1}$ and (2) the "initial configuration" has in-degree zero.
Such a configuration graph is depicted on
\Cref{fig:reduction-wf-RTM-to-colouring-config-graph-wf-RTM}.\footnote{Note
that it is decidable whether a "Turing machine" is "linear@@TM". In fact, condition (1) can be expressed in "first-order logic" over $\univStructSynchronous{\Sigma}$.}

\AP The ""reachable regularity problem"" is the problem of, given a "Turing machine" $\+T$, to decide whether its set of "reachable configurations" is a "regular language". To show that is it undecidable, we exhibit a reduction from the halting problem on deterministic "reversible" Turing machines.

\begin{proposition}[{\cite[Theorem 1]{Lecerf1963MachinesReversibles}}]
    \AP\label{prop:halting-problem-detrevTM}
    The \DPfont{halting problem} on deterministic "reversible" Turing machines is undecidable.
\end{proposition}

\begin{lemma}
    \AP\label{lem:reachable-regularity}
    The "reachable regularity problem" is undecidable, even if restricted
    to "linear Turing machines".
\end{lemma}

\begin{proof}[Proof sketch]
    We reduce the \DPfont{halting problem} on deterministic "reversible" Turing machines,
    in such a way that the "reachable configurations" whose
    state $q$ coincide with the state of the original machine are
    of the form $u \cdot q \cdot v \triangleright^n \triangleleft^n$ where $u \cdot q \cdot v$ is a configuration of the 
    original machine, $\triangleright$ and $\triangleleft$ are new symbols,
    and $n\in\N$. Transitions are defined in such a way that the new machine is
    "linear@@TM": this is implemented by having, for every transition $u\cdot q \cdot v \to u' \cdot q' \cdot v'$ of the original machine and every $n,m\in \N$, a multistep transition
    \[ 
        u \cdot q\cdot v \triangleright^{n} \triangleleft^{m} \to^* u' \cdot q' \cdot v' \triangleright^{n+1} \triangleleft^{m+1}.
    \]
    The construction is illustrated in \Cref{fig:reachable-regularity}.
	\begin{figure}[htb]
		\centering
        \begin{tikzpicture}
		    % ---
% First tape
% ---
\node (0) at (0,0) {0};
\node[right = 0cm of 0] (1) {0};
\node[right = 0cm of 1] (2) {1};
\node[right = 0cm of 2] (3) {0};
\node[right = 0cm of 3] (4) {1};
\node[right = 0cm of 4] (5) {$\triangleright$};
\node[right = 0cm of 5] (6) {$\triangleright$};
\node[right = 0cm of 6] (7) {$\triangleright$};
\node[right = 0cm of 7] (8) {$\triangleleft$};
\node[right = 0cm of 8] (9) {$\triangleleft$};
\node[right = 0cm of 9] (10) {$\triangleleft$};

\draw[rounded corners=4pt] (0.south west) rectangle (10.north east);
\draw[->, thick] ($(5.north)+(0,.4)$) -- ($(5.north)+(0,.1)$);
\node[above=.05cm, circle, draw=cBlue, fill=cBlue, opacity=.5, text opacity=1, inner sep=1.5pt] at ($(5.north)+(0,.4)$) {$p$};

% ---
% Second tape
% ---
\node[below = 1.2cm of 0] (0') {0};
\node[right = 0cm of 0'] (1') {0};
\node[right = 0cm of 1'] (2') {1};
\node[right = 0cm of 2'] (3') {0};
\node[right = 0cm of 3'] (4') {1};
\node[right = 0cm of 4'] (5') {1};
\node[right = 0cm of 5'] (6') {$\triangleright$};
\node[right = 0cm of 6'] (7') {$\triangleright$};
\node[right = 0cm of 7'] (8') {$\triangleleft$};
\node[right = 0cm of 8'] (9') {$\triangleleft$};
\node[right = 0cm of 9'] (10') {$\triangleleft$};

\draw[rounded corners=4pt] (0'.south west) rectangle (10'.north east);
\draw[->, thick] ($(6'.north)+(0,.4)$) -- ($(6'.north)+(0,.1)$);
\node[above=.05cm, circle, draw=cRed, fill=cRed, opacity=.5, text opacity=1, inner sep=1.5pt] at ($(6'.north)+(0,.4)$) {$\phantom{q}$};

% ---
% Third tape
% ---
\node[below = 1.2cm of 0'] (0'') {0};
\node[right = 0cm of 0''] (1'') {0};
\node[right = 0cm of 1''] (2'') {1};
\node[right = 0cm of 2''] (3'') {0};
\node[right = 0cm of 3''] (4'') {1};
\node[right = 0cm of 4''] (5'') {1};
\node[right = 0cm of 5''] (6'') {$\triangleright$};
\node[right = 0cm of 6''] (7'') {$\triangleright$};
\node[right = 0cm of 7''] (8'') {$\triangleright$};
\node[right = 0cm of 8''] (9'') {$\triangleright$};
\node[right = 0cm of 9''] (10'') {$\triangleleft$};

\draw[rounded corners=4pt] (0''.south west) rectangle (10''.north east);
\draw[->, thick] ($(10''.north)+(0,.4)$) -- ($(10''.north)+(0,.1)$);
\node[above=.05cm, circle, draw=cRed, fill=cRed, opacity=.5, text opacity=1, inner sep=1.5pt] at ($(10''.north)+(0,.4)$) {$\phantom{q}$};

% ---
% Fourth tape
% ---
\node[below = 1.2cm of 0''] (0''') {0};
\node[right = 0cm of 0'''] (1''') {0};
\node[right = 0cm of 1'''] (2''') {1};
\node[right = 0cm of 2'''] (3''') {0};
\node[right = 0cm of 3'''] (4''') {1};
\node[right = 0cm of 4'''] (5''') {1};
\node[right = 0cm of 5'''] (6''') {$\triangleright$};
\node[right = 0cm of 6'''] (7''') {$\triangleright$};
\node[right = 0cm of 7'''] (8''') {$\triangleright$};
\node[right = 0cm of 8'''] (9''') {$\triangleright$};
\node[right = 0cm of 9'''] (10''') {$\triangleleft$};
\node[right = 0cm of 10'''] (11''') {$\triangleleft$};
\node[right = 0cm of 11'''] (12''') {$\triangleleft$};
\node[right = 0cm of 12'''] (13''') {$\triangleleft$};

\draw[rounded corners=4pt] (0'''.south west) rectangle (13'''.north east);
\draw[->, thick] ($(13'''.north)+(0,.4)$) -- ($(13'''.north)+(0,.1)$);
\node[above=.05cm, circle, draw=cRed, fill=cRed, opacity=.5, text opacity=1, inner sep=1.5pt] at ($(13'''.north)+(0,.4)$) {$\phantom{q}$};

% ---
% Fifth tape
% ---
\node[below = 1.2cm of 0'''] (0'''') {0};
\node[right = 0cm of 0''''] (1'''') {0};
\node[right = 0cm of 1''''] (2'''') {1};
\node[right = 0cm of 2''''] (3'''') {0};
\node[right = 0cm of 3''''] (4'''') {1};
\node[right = 0cm of 4''''] (5'''') {1};
\node[right = 0cm of 5''''] (6'''') {$\triangleright$};
\node[right = 0cm of 6''''] (7'''') {$\triangleright$};
\node[right = 0cm of 7''''] (8'''') {$\triangleright$};
\node[right = 0cm of 8''''] (9'''') {$\triangleright$};
\node[right = 0cm of 9''''] (10'''') {$\triangleleft$};
\node[right = 0cm of 10''''] (11'''') {$\triangleleft$};
\node[right = 0cm of 11''''] (12'''') {$\triangleleft$};
\node[right = 0cm of 12''''] (13'''') {$\triangleleft$};

\draw[rounded corners=4pt] (0''''.south west) rectangle (13''''.north east);
\draw[->, thick] ($(6''''.north)+(0,.4)$) -- ($(6''''.north)+(0,.1)$);
\node[above=.05cm, circle, draw=cBlue, fill=cBlue, opacity=.5, text opacity=1, inner sep=1.5pt] at ($(6''''.north)+(0,.4)$) {$q$};

% ---
% Transitions
% ---
\draw[->, dashed] ($(10.east)+(.3,-.2)$) to[bend left=50]
	node[midway, right=.15cm, align=left, text width=3.5cm, font=\footnotesize] {simulate $T$}
	($(10'.east)+(.3,.2)$);

\draw[->, dashed] ($(10'.east)+(.3,-.2)$) to[bend left=50]
	node[midway, right=.15cm, align=left, text width=3.5cm, font=\footnotesize]
		{overwrite the first two $\triangleleft$'s with $a$'s}
	($(10''.east)+(.3,.2)$);

% \draw[-, dashed] ($(10''.east)+(.3,-.1)$) to ($(10''.east)+(1.5,-.1)$);
\draw[->, dashed] ($(10''.east)+(1.5,-.2)$) to[bend left=50]
	node[midway, right=.15cm, align=left, text width=3.5cm, font=\footnotesize]
		{append three $\triangleleft$'s}
	($(13'''.east)+(.3,.2)$);


\draw[->, dashed] ($(13'''.east)+(.3,-.2)$) to[bend left=50]
	node[midway, right=.15cm, align=left, text width=3.5cm, font=\footnotesize]
		{go back to the new position, in the new state}
	($(13''''.east)+(.3,.2)$);
        \end{tikzpicture}
		\caption{
			\AP\label{fig:reachable-regularity}
			Encoding of a single transition of the form
			``when reading a blank in state $\color{cBlue} p$, write a
			$1$, go in state $\color{cBlue} q$ and move right''
			of the machine $\+T$ in the machine $\+T'$
			in the proof of \Cref{lem:reachable-regularity}.
			Red unlabelled states represent states of $\+T'$
			that are not originally present in $\+T$.
		}
	\end{figure}
	
    Moreover:
    \begin{itemize}
        \item if the original machine was halting, then the "reachable configurations"
            of the new one are finite and hence regular;
        \item otherwise, the set of "reachable configurations" is not regular,
            which follows from the non-regularity of any infinite subset of $\{\triangleright^n \triangleleft^n \mid n \in \N\}$.\qedhere
    \end{itemize}
\end{proof}

\begin{proof}[Some proof details]
    Letting $\+T$ denote the instance of the \DPfont{halting problem}, which runs on the empty word,
    we denote by $\+T'$ the instance of the "reachable regularity problem" to which it
    is reduced.

    $\+T'$ is defined as follows: every time there is a transition
    $u\cdot p \cdot v \to u' \cdot q \cdot v'$ in $\+T$,
	we simulate this transition in $T'$: to achieve this, `$\triangleright$' and `$\triangleleft$'s should be treated as blank symbols,
	and then we rewrite $\triangleright^n \triangleleft^m$ into $\triangleright^{n+1}\triangleleft^{m+1}$.
	When $\+T$ writes on a blank symbol that was actually an `$\triangleright$' in $\+T'$,
	we must also add an extra $\triangleright$ (to account for the one that was overwritten):
	this case is depicted in \Cref{fig:reachable-regularity}.
	Moreover, when $\+T$ deletes a symbol at the end of the tape,
	we must shift the $\triangleright^n \triangleleft^m$ suffix. This can be done by replacing the blank
	with an `$\triangleright$', the last `$\triangleright$' with a `$\triangleleft$', and deleting the last `$\triangleleft$'.
	
    We now prove that $\+T'$ is "linear@@TM":
    \begin{enumerate}
        \item it is deterministic and "reversible":
        \begin{itemize}
            \item every "configuration@@TM" inside a path
            \[u \cdot q\cdot v \triangleright^{n} \triangleleft^{m} \to^* u' \cdot q' \cdot v' \triangleright^{n+1} \triangleleft^{m+1}\]
            has, by definition, exactly in- and out-degree one;
            \item every "configuration@@TM" of the form $u \cdot q \cdot v a^n b^m$ has as many 
            predecessors (resp. successors) in $\+T'$ as $u \cdot q \cdot v$ in $\+T$, namely one since $\+T$ was assumed to be deterministic and "reversible";
        \end{itemize}
        \item the "initial configuration" $\pad \cdot q_0 \cdot \pad$ has no predecessor;
        \item it has no infinite backward path since $\N$ is well-founded,
    \end{enumerate}
    Moreover, $\+T'$ has no cycle,\footnote{Indeed, we encoded a strictly increasing counter inside the configurations of $\+T'$.} and so if $\+T$ is halting on an empty input, then the set of "reachable configurations" of $\+T'$ is finite, and thus regular. If $\+T$ is not halting, the set of "reachable configurations" of $\+T'$ is infinite and its projection onto $\set{\triangleright,\triangleleft}$ is an infinite set of words of the form $a^{n} b^{m}$ where $n-2 \leq m \leq n+2$. Hence, since regular languages are closed under homomorphic images, the "reachable configurations" of $\+T'$ cannot be regular.
\end{proof}


\paragraph*{Undecidability of the $k$-Regular Colourability Problem.}
We can now show undecidability for the "$k$-regular colourability problem" by reduction from the "reachable regularity problem" restricted to "linear Turing machines".

\begin{marginfigure}
    \centering
    \includegraphics[width=\linewidth]{fig/germinal.jpg}
    \caption{\emph{Allégorie pour le mois de Germinal}, Louis Lafitte.}
\end{marginfigure}

A configuration of a "Turing machine"---or more generally the node of a "rational graph"---is said 
to be \AP""germinal"" if it has in-degree 0.
% \sidenote{A natural terminology would be ``initial'' but it clashes with the well-established notion of "initial configuration".}

\begin{theorem}
    \AP\label{thm:k-reg-col-undec}
    The "$k$-regular colourability problem" on "rational graphs" is undecidable, for every $k\geq 2$. More precisely, the problem is "RE"-complete. This holds also for "connected" "rational graphs".
\end{theorem}

\begin{proof}%
	\proofcase{Lower bound.}
    By reduction from the "reachable regularity problem" for "linear Turing machines"
    (\Cref{lem:reachable-regularity}). We first show it for $k=2$.

    \AP Given a "linear Turing machine" $T$,
    observe that the set $\intro*\Germ$ of all "germinal configurations" of $\confGraph$.
    \begin{claim}
        $\Germ$ is effectively a "regular language". 
    \end{claim}
    
    Observe moreover that, by definition of "linear Turing machines",
    the "initial configuration" $\pad\cdot q_0 \cdot \pad$ is "germinal".
    Let $\bSymb$ and $\rSymb$ be fresh symbols. 
    Consider the "rational graph" $\AutGraph{V}{\+E}$ for $V \defeq \configs\cdot (\bSymb + \rSymb)$,
    having an edge from $\gamma \cdot c$ to $\gamma' \cdot c$ if either 
    \begin{enumerate}
        \item $\tup{c,c'} = \tup{\bSymb,\rSymb}$ and $\gamma=\gamma'$;
        \item $\tup{c,c'} = \tup{\rSymb,\bSymb}$ and there is an edge from $\gamma$ to $\gamma'$ in $\confGraph$; or
        \item $\tup{c,c'} = \tup{\bSymb,\bSymb}$, $\gamma$ is the "initial configuration",
        and $\gamma' \neq \gamma$ is "germinal".
    \end{enumerate}

    \begin{marginfigure}%
        \centering
        \begin{tikzpicture}
            % Reachable configuration
\fill[rounded corners, draw=cGreen, fill=cGreen, opacity=.3]
	(-.3,-.3) rectangle (2.75, .3);
% Initial configuration
\fill[rounded corners, draw=cYellow, fill=cYellow, opacity=.3]
	(-.3,.3) rectangle (.3, -1.95);

% First line
\node[vertex] (a0) at (0,0) {};
\foreach \i in {0,...,2} {
	\pgfmathtruncatemacro{\next}{\i + 1}
	\node[vertex, right = of a\i] (a\next) {};
	\draw[edge] (a\i) to (a\next);
}

% Second line
\node[vertex, below = of a0] (b0) {};
\foreach \i in {0,...,3} {
	\pgfmathtruncatemacro{\next}{\i + 1}
	\node[vertex, right = of b\i] (b\next) {};
	\draw[edge] (b\i) to (b\next);
}
\node[draw=none, fill=none, right = 0cm of b4] (binf) {$\cdots$};

% Third line
\node[vertex, below = of b0] (c0) {};
\foreach \i in {0,1} {
	\pgfmathtruncatemacro{\next}{\i + 1}
	\node[vertex, right = of c\i] (c\next) {};
	\draw[edge] (c\i) to (c\next);
}

% Labels
\node[draw=none, fill=none, font=\small] [below = 1em of c0] {$\GermC{cYellow}$};
\node[draw=none, fill=none, font=\small, align=center] [above = 1em of $(a1)!0.5!(a2)$]
	{$\ReachC{\+T}{cGreen}$};
        \end{tikzpicture}
        \caption{
            \AP\label{fig:reduction-wf-RTM-to-colouring-config-graph-wf-RTM}
            Configuration graph of a "linear Turing machine".
        }
    \end{marginfigure}%
    \begin{marginfigure}%
        \centering
        \begin{tikzpicture}
            
% Reachable configuration
\fill[rounded corners, draw=cGreen, fill=cGreen, opacity=.3]
	(-.3,-.9) rectangle (2.75, .3);
% Initial configuration
\fill[rounded corners, draw=cYellow, fill=cYellow, opacity=.3]
	(-.3,.3) rectangle (.3, -3.8);

% First line
\node[vertex, cBlue, fill=cBlue, fill opacity=.4] (a0) at (0,0) {};
\node[vertex, cRed, fill=cRed, fill opacity=.4] (a'0) [below = .3cm of a0] {};
\draw[edge, densely dotted] (a0) to (a'0);
\foreach \i in {0,...,2} {
	\pgfmathtruncatemacro{\next}{\i + 1}
	\node[vertex, cBlue, fill=cBlue, fill opacity=.4, right = of a\i] (a\next) {};
	\node[vertex, cRed, fill=cRed, fill opacity=.4, right = of a'\i] (a'\next) {};
	\draw[edge] (a'\i) to (a\next);
	\draw[edge, densely dotted] (a\next) to (a'\next);
}

% Second line
\node[vertex, cBlue, fill=cBlue, fill opacity=.4, below = of a'0] (b0) {};
\node[vertex, cRed, fill=cRed, fill opacity=.4, below =.3cm of b0] (b'0) {};
\draw[edge, densely dotted] (b0) to (b'0);
\foreach \i in {0,...,3} {
	\pgfmathtruncatemacro{\next}{\i + 1}
	\node[vertex, cBlue, fill=cBlue, fill opacity=.4, right = of b\i] (b\next) {};
	\node[vertex, cRed, fill=cRed, fill opacity=.4, right = of b'\i] (b'\next) {};
	\draw[edge] (b'\i) to (b\next);
	\draw[edge, densely dotted] (b\next) to (b'\next);
}
\node[draw=none, fill=none, right = .6em of $(b4)!0.5!(b'4)$] (binf) {$\cdots$};

% Third line
\node[vertex, cBlue, fill=cBlue, fill opacity=.4, below = of b'0] (c0) {};
\node[vertex, cRed, fill=cRed, fill opacity=.4, below = .3cm of c0] (c'0) {};
\draw[edge, densely dotted] (c0) to (c'0);
\foreach \i in {0,1} {
	\pgfmathtruncatemacro{\next}{\i + 1}
	\node[vertex, cBlue, fill=cBlue, fill opacity=.4, right = of c\i] (c\next) {};
	\node[vertex, cRed, fill=cRed, fill opacity=.4, right = of c'\i] (c'\next) {};
	\draw[edge] (c'\i) to (c\next);
	\draw[edge, densely dotted] (c\next) to (c'\next);
}

% Edges between Init nodes
\draw[edge, densely dashed] (a0) to[bend right] (b0);
\draw[edge, densely dashed] (a0) to[bend right] (c0);

% Labels
\node[draw=none, fill=none, cYellow, font=\footnotesize, align=left]
	[below right = 1em and -1em of $(c'0)$]
	{nodes originating from	$\GermC{\+T}{cYellow}$};
\node[draw=none, fill=none, cGreen, font=\footnotesize, align=left]
	[above right = 1em and -1em of $(a0)$]
	{nodes originating from $\ReachC{\+T}{cGreen}$};
        \end{tikzpicture}
        \caption{
            \AP\label{fig:reduction-wf-RTM-to-colouring}
            The "rational graph" to which the "configuration graph"
            of \Cref{fig:reduction-wf-RTM-to-colouring-config-graph-wf-RTM} is reduced.
        }
    \end{marginfigure}%

    Symbols $\bSymb$ and $\rSymb$ are utilized to represent two versions of each configuration.
    This graph is depicted in \Cref{fig:reduction-wf-RTM-to-colouring}.
    Note that $\AutGraph{V}{\+E}$ is "connected" and "2-colourable": in fact, it is a "directed tree" whose root is $\pad\cdot q_0\cdot \pad\cdot \bSymb$. 
    
    We claim that $\AutGraph{V}{\+E}$ is "$2$-regular colourable" if, and only if, the set of "reachable configurations" of $T$ is a "regular language". 
    In fact, up to permuting the two-colours, $\AutGraph{V}{\+E}$
    admits a unique 2-colouring $\tup{C_1,C_2}$, defined by:
    \[
        C_1 \defeq \Reach\cdot\bSymb \cup (\configs \setminus \Reach)\cdot\rSymb
    \]
    and $C_2$ is the complement of $C_1$.
    If $\Reach$ is regular, then so is $C_1$. Dually, if $C_1$ is regular, then
    $\Reach$ is exactly the set of configurations $\gamma$ such that
    $\gamma\cdot \bSymb \in C_1$ and hence is regular.
    It follows that $\AutGraph{V}{\+E}$ is "$2$-regular colourable" if and only if
    the "reachable configurations" of $\+T$ are regular, which concludes the proof for $k=2$.

    To prove the statement for any $k>2$, we define $\AutGraph{V}{\+E_k}$ as the result of adding a $(k-2)$-clique to $\AutGraph{V}{\+E}$ and adding an edge from every vertex of the clique to every vertex incident to an edge of $\+E$. This forces the clique to use $k-2$ colours that cannot be used in the remaining part of the graph and the proof is then analogous.

	\proofcase{Upper-bound.} We show that the problem is "RE". Let us define a \AP""$k$-coloured automaton"" like a regular (complete) DFA, except that instead of having
	a set of final states, it has a partition $\langle C_1,\hdots,C_k \rangle$ of its states.
	Such an automaton recognizes a "regular colouring" $\Sigma^* \to \lBrack 1, k\rBrack$.
	Given a "rational graph" $\+G = \AutGraph{V}{\+E}$---whose edge relations is given by
    a "synchronous automaton" $\relPres{\+E}{\+G}$---and a $k$-coloured automaton $\+B$,
	we can build, by a product construction, an automaton $\+A'$ which accepts
	all $u \otimes v \in \convolRel{R}$ such that the colour of $u$ is distinct from
    the colour of $v$.
	Then, $\+A'$ is equivalent to $\relPres{\+E}{\+G}$ if, and only if,
    $\+B$ describes a proper "$k$-colouring" of $\AutGraph{V}{\+E}$.
    The "RE" upper-bound of the "$k$-regular colourability problem" follows: 
    it suffices to enumerate all "$k$-coloured automata" and check for equivalence.
\end{proof}

Note that this reduction provides an easy way of building
graphs in the shape of \Cref{fig:reduction-wf-RTM-to-colouring} that are "2-colourable" (in fact, they are trees) but not "2-regular colourable". In fact, we can provide a slightly more
direct construction.

\begin{example}
    \AP\label{ex:tree-not-2-reg-colourable}
    On the alphabet $\Sigma = \{a,b\}$, the tree $\+T$ depicted in \Cref{fig:tree-not-2reg-colour} whose set of vertices is $V = a^*b^*$ and whose set 
    of edges is $\+E = \+E_{\mathrm{incr}} \cup \+E_{\mathrm{init}}$, with 
    \begin{align*}
        \+E_{\mathrm{incr}} & = \{(a^pb^q,\, a^{p+1}b^{q+1}) \mid p,q \in \N\} \\
        \+E_{\mathrm{init}} & = \{(\varepsilon,\, a^p) \mid p \in \N\} \cup \{(\varepsilon,\, b^q) \mid q \in \N\}, 
    \end{align*}    
    is "rational@rational graph" but not "2-regular colourable".
    Indeed, its only "2-colouring"
    consists in partitioning the vertices of $\+T$ into
    \begin{align*}
        C =\;& \{a^n b^n \mid n \in 2\N\} \\
            & \cup \{a^p b^q \mid p > q \text{ and $q$ is odd}\} \\
            & \cup \{a^p b^q \mid p < q \text{ and $p$ is odd}\}
    \end{align*}
    and its complement $V \setminus C$.
    Let $P = \{a^p b^q \mid p, q \in 2\N\} = (aa)^*(bb)^*$:
    $P$ is regular, yet $C \cap P = \{a^n b^n \mid n \in 2\N\}$ is not.
    Hence, $C$ is not regular, and thus $\+T$ is not "2-regular colourable".
\end{example}

\begin{figure}
    \centering
    \begin{tikzpicture}
        % Coloring
\fill[rounded corners, fill=cBlue, opacity=.5]
	(-.4,-0.25) rectangle (.4,0.23)
	(2.1,-0.25) rectangle (2.9,0.23)
	(0.85,-0.52) rectangle (1.65, -2.5)
	(3.35,-0.52) rectangle (4.15, -2.5);

\fill[rounded corners, fill=cRed, opacity=.5, xshift=1.25cm]
	(-.4,-0.25) rectangle (.4,0.23)
	(2.1,-0.25) rectangle (2.9,0.23);
\fill[rounded corners, fill=cRed, opacity=.5, xshift=-1.25cm]
	(0.85,-0.52) rectangle (1.65, -2.5)
	(3.35,-0.52) rectangle (4.15, -2.5);

% Tree
\node (eps) at (0,0) {$\varepsilon$};
\node (ab) at (1.25,0) {$ab$};
\node (aabb) at (2.5,0) {$a^2b^2$};
\node (aaabbb) at (3.75,0) {$a^3b^3$};

\node (a) at (0,-.75) {$a$};
\node (aab) at (1.25,-.75) {$aab$};
\node (aaabb) at (2.5,-.75) {$a^3b^2$};
\node (aaaabbb) at (3.75,-.75) {$a^4b^3$};

\node (b) at (0,-1.5) {$b$};
\node (abb) at (1.25,-1.5) {$abb$};
\node (aabbb) at (2.5,-1.5) {$a^2b^3$};
\node (aaabbbb) at (3.75,-1.5) {$a^3b^4$};

\node (aa) at (0,-2.25) {$aa$};
\node (aaab) at (1.25,-2.25) {$a^3b$};
\node (aaaabb) at (2.5,-2.25) {$a^4b^2$};
\node (aaaaabbb) at (3.75,-2.25) {$a^5b^3$};

\draw[->] (eps) to (ab);
\draw[->] (ab) to (aabb);
\draw[->] (aabb) to (aaabbb);
\draw[->, dashed] (aaabbb) to ($(aaabbb)+(1,0)$);

\draw[->] (a) to (aab);
\draw[->] (aab) to (aaabb);
\draw[->] (aaabb) to (aaaabbb);
\draw[->, dashed] (aaaabbb) to ($(aaaabbb)+(1,0)$);

\draw[->] (b) to (abb);
\draw[->] (abb) to (aabbb);
\draw[->] (aabbb) to (aaabbbb);
\draw[->, dashed] (aaabbbb) to ($(aaabbbb)+(1,0)$);

\draw[->] (aa) to (aaab);
\draw[->] (aaab) to (aaaabb);
\draw[->] (aaaabb) to (aaaaabbb);
\draw[->, dashed] (aaaaabbb) to ($(aaaaabbb)+(1,0)$);

\draw[->] (eps) edge[bend right=40] (a)
	edge[bend right=40] (b)
	edge[bend right=40] (aa)
	edge[dashed, bend right=40] ($(aa)+(0,-.75)$);

\node[below = .25cm of aaab, color=cBlue] {$C$}; 
\node[below = .25cm of aaaabb, color=cRed] {$V \setminus C$}; 
    \end{tikzpicture}
    \caption{
        \label{fig:tree-not-2reg-colour}
        The "rational tree@rational graph" $\+T$ of \Cref{ex:tree-not-2-reg-colourable},
        and its unique "2-colouring" $(C, V\setminus C)$, which is not "regular@@colouring".
    }
\end{figure}