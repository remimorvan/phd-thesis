\chapter{A Dichotomy Theorem for Automatic Structures}

Let \AP$\intro*\Fin$ denote the class of all "finite $\sigma$-structures",
\AP$\intro*\Aut$ the class of all "automatic $\sigma$-structures".
We let \AP$\intro*\AutPres$ be the class of all "automatic presentations of $\sigma$-structures",
and \AP$\intro*\FinPres$ be the subclass of "presentations" of "finite $\sigma$-structures".

Given a "$\sigma$-structure" $\?B$, and a class $\classStruct$ of "automatic presentations of $\sigma$-structures", we denote by \AP$\intro*\HomPb{\classStruct}{\?B}$ the class of all presentations $\•A \in \classStruct$ such that the "$\sigma$-structure" $\?A$
that they "represent" admits a "homomorphism" to $\?B$.
Similarly, we denote by \AP$\intro*\HomRegPb{\classStruct}{\?B}$ the class 
associated to "regular homomorphisms".%
\sidenote{Todo: set-theoretic bullshit. Not a set but it is if we fix an alphabet. Blabla.}

While the existence of a "homomorphism" from some "automatic presentation" of $\?A$ 
to a "finite $\sigma$-structure" $\?B$ only depends on $\?A$ and not the "presentation" itself,
this is not true for "regular homomorphism".
For a $\sigma$-structure $\?B$,
we use \AP$\intro*\HomFin{\?B}$, $\intro*\HomAut{\?B}$, $\intro*\HomRegFin{\?B}$
and $\intro*\HomRegAut{\?B}$ as shorthands for $\HomPb{\FinPres}{\?B}$, $\HomPb{\AutPres}{\?B}$,
$\HomRegPb{\FinPres}{\?B}$ and $\HomRegPb{\AutPres}{\?B}$, respectively.

Note that any "homomorphism" from an "automatic presentation" of a "finite structure"
to a "finite structure" is also "regular@@hom" and so $\HomFin{\?B} = \HomRegFin{\?B}$.\sidenote{Note also that membership in $\HomFin{-} = \HomRegFin{-}$ and in
$\HomAut{-}$ only depends on the "structure represented"
by the "automatic presentation", and not on the "presentation" itself.
This is not the case for $\HomRegAut{-}$. We use these definitions for two reasons: (1) for consistency with $\HomRegAut{-}$ and (2) for the associated decision problem to be defined unambiguously. Note moreover that in the case of "finite structures", the size of its representation, say using a adjacency lists, and of its "automatic presentation" are equal up to a TODO.}

\begin{theorem}[Dichotomy Theorem for Automatic Structures]
  \!\footnote{The equivalence between (1) and (2) still holds
  when the "homomorphism problem" takes as input ``higher-order automatic
  structures'' as defined in \cite[last remark of \S~XII.3]{Blumensath2024MSOModelTheory},
  since such structures have a decidable first-order theory.}%
  %%%
  \AP\label{thm:dichotomy-theorem-automatic-structures}
  Let $\?B$ be a "finite $\sigma$-structure". The following are equivalent:
  \begin{enumerate}
    \item $\?B$ has "finite duality";
    \item $\HomAut{\?B}$ is "decidable";
    \item $\HomRegAut{\?B}$ is "decidable";
    \item $\HomAut{\?B} = \HomRegAut{\?B}$ "ie" for any "automatic presentation" $\•A$ of a 
      "$\sigma$-structure" $\?A$, there is a "homomorphism" from $\?A$ to $\?B$ "iff" 
      there is a "regular homomorphism" from $\•A$ to $\?B$;
    \item $\HomRegAut{\?B}$ is "invariant under graph isomorphisms".
  \end{enumerate}
  Moreover, when $\HomAut{\?B}$ and $\HomRegAut{\?B}$ undecidable, they are "coRE"-complete
  and "RE"-complete, respectively. When they are decidable, they are "NL" and TODO, respectively.
\end{theorem}

TODO: say that to have "finite duality" can be decided.

\section{\AP\label{sec:dichotomy-preliminaries}%
	Preliminaries}

\subsection{Constructions on Structures}

Given two "structures" $\?A$ and $\?B$, we define the structure \AP$\intro*\powstruct{\?B}{\?A}$ as follows:
\begin{itemize}
  \item its domain are "homomorphisms" $\?A \to \?B$,
  \item for every "predicate" $\+R$ of arity $k$, for any homomorphism $f_1,\hdots,f_k$,
  we have $\langle f_1,\hdots,f_k\rangle \in \+R$ when for every
  $\langle a_1,\hdots,a_k\rangle \in \+R(\?A)$
  then $\langle f_1(a_1), \hdots, f_k(a_k) \rangle \in \+R(\?B)$.
\end{itemize}


\begin{proposition}[Folklore: Currying Homomorphisms]
	\AP\label{prop:currying-hom}
	Given "structures" $\?A$, $\?B$ and $\?C$, if $f\colon \?A\prodstruct \?B \to \?C$
	is a "homomorphism", then $F\colon \?A \to \powstruct{\?C}{\?B}$,
	defined by $a \mapsto (b \mapsto f(a,b))$, is a "homomorphism".
	In fact, this mapping $f \mapsto F$ is a bijection
	between "homomorphisms" $\?A\prodstruct \?B \to \?C$
	and "homomorphisms" $\?A \to \powstruct{\?C}{\?B}$.
\end{proposition}

\begin{proof}
Let $\+R$ be a predicate of arity $k$, and let
$\langle a_1,\hdots,a_k \rangle \in \+R(\?A)$.
We want to show that $\langle F(a_1),\hdots,F(a_k) \rangle \in \+R(\powstruct{\?C}{\?B})$:
for any $\langle b_1,\hdots,b_k \rangle \in \+R(\?B)$, we have
\[\langle F(a_1)(b_1), \hdots, F(a_k)(b_k)\rangle = \langle f(a_1,b_1),\hdots,f(a_k,b_k) \rangle \in \+R(\?C)\] since $f$ is a "homomorphism" from $\?A\prodstruct \?B$ to $\?C$.
Hence, $F$ is indeed a "homomorphism" from $\?A$ to $\powstruct{\?C}{\?B}$.

Dually, if $F$ is a "homomorphism" from $\?A$ to $\powstruct{\?C}{\?B}$,
we define $f\colon \?A\prodstruct \?B \to \?C$ by $\langle a,b \rangle \mapsto F(a)(b)$,
and claim that $f$ is a "homomorphism". Indeed, if $\+R$ be a predicate of arity $k$,
for any $\langle a_1, \hdots, a_k \rangle \in \+R(\?A)$
and $\langle b_1, \hdots, b_k \rangle \in \+R(\?B)$,
we have $\langle f(a_1,b_1), \hdots, f(a_k,b_k) \rangle
= \langle F(a_1)(b_1), \hdots, F(a_k)(b_k) \rangle$.
Since $\langle F(a_1), \hdots, F(a_k) \rangle \in \+R(\powstruct{\?C}{\?B})$
and $\langle b_1, \hdots,b_k \rangle \in \+R(\?B)$ 
it follows that $\langle F(a_1)(b_1), \hdots, F(a_k)(b_k) \rangle \in \+R(\?C)$.
Therefore, $f$ is a "homomorphism" from $\?A \prodstruct \?B$ to $\?C$.

It is then routine to check that the maps $f \mapsto F$ and $F \mapsto f$ defined
in the two previous paragraphs are mutually inverse bijections.
\end{proof}

\subsection{Constructions on rational presentations}
\label{sec:construction-automatic-presentations}

Let $\•A$ and $\•B$ be "rational presentations" of some "$\sigma$-structures"
$\?A$ and $\?B$. We define \AP$\•A \intro*\prodpres \•B$ to be the "presentation@@auto"
"st":
\begin{align*}
	\domainPres{\•A\prodpres \•B} & \defeq \{u\#v \mid u \in \domainPres{\•A} \land v \in \domainPres{\•B}\}\\
	\relPres{\•A\prodpres \•B}{\+R} & \defeq \{(u_1\#v_1) \convol \cdots \convol (u_k\#v_k) \mid
		u_1 \convol \cdots \convol u_k \in \relPres{\•A}{\+R} \land
		v_1 \convol \cdots \convol v_k \in \relPres{\•B}{\+R}
	\}
\end{align*}
for each "predicate" $\+R$ of arity $k$ in $\sigma$.
It is an "rational presentation" of $\?A \prodstruct \?B$. TODO:proof.

\begin{proposition}
	\label{prop:homreg-prod-finite}
	Let $\?A$, $\?B$ and $\?C$ be "automatic $\sigma$-structures", such that
	$\?B$ and $\?C$ are finite.
	Let $\•A$ (resp. $\•B$ and $\•B'$, resp. $\•C$ and $\•C'$) be an "rational presentation"
	of $\?A$ (resp. $\?B$, resp. $\?C$).
	Then $\•A \prodpres \•B \homregto \•C$ "iff" $\•A \prodpres \•B' \homregto \•C'$.
\end{proposition}

\begin{proof}
	The proof will follow from the following claim.
	\begin{claim}
		\label{claim:homreg-prod-finite}
		A function $f\colon\•A \prodpres \•B \homregto \•C$ is a "regular homomorphism"
		"iff" for every $b\in \domainPres{\•B}$, for every $c\in \domainPres{\•C}$,
		\(\{
			a\in \domainPres{\•A} \mid f(a,b) = c
		\}\)
		is a "regular language".
	\end{claim}
	TODO.
\end{proof}

In other words, the existence of a "regular homomorphism" does not depend on the
"rational presentation" of the \emph{finite} "structures" that are involved, but only
on the "structure" they represent.
As a consequence of \Cref{prop:homreg-prod-finite}, we write
\(\•A \prodpres \?B \homregto \?C\) as a synonym for \(\•A \prodpres \•B \homregto \•C\).

\begin{corollary}[Currying]
	\label{coro:homreg-currying}
	Let $\?A$, $\?B$ and $\?C$ be "automatic $\sigma$-structures",
	and let $\•A$ be an "rational presentation" of $\?A$.
	Then $\•A \prodpres \?B \homregto \?C$ "iff" $\•A \homregto \powstruct{\?C}{\?B}$.
\end{corollary}

\begin{proof}
	By \Cref{claim:homreg-prod-finite}… TODO.
\end{proof}

\subsection{Idempotent Core}

We fix a "purely relational signature" $\sigma$.
Given a "$\sigma$-structure" $\?B$,
we denote by \AP$\intro*\extendedSignature{\sigma}{\?B}$
the signature obtained from $\sigma$ by adding
a unary predicate \AP$\intro*\unarypred{b}$ for each $b\in B$.
The \AP""marked structure"" \AP$\intro*\marked{\?B}$ of $\?B$ is the
"$\extendedSignature{\sigma}{\?B}$-structure"
obtained from $\?B$ by "interpreting@@predicate" each predicate $\unarypred{b}$ as the
singleton $\{b\}$.\sidenote{TODO: ASIA'S REMARK: this is called the ``idempotent core''.}

\begin{proposition}[Folklore]
\!\sidenote{The non-easy part is to reduce $\HomFinDec{\marked{\?B}}$ to $\HomFinDec{\?B}$: only this reduction
requires the assumption that $\?B$ is a core. This reduction is folklore, see "eg"
\autocite[Lemma 2.5]{LaroseTesson2009UniversalAlgebraCSP}.
TODO: understand if the proof is indentical to us. We provide here a self-contained proof.}%
\sidenote{The "marked structure" $\marked{\core{\?B}}$ of the "core" of $\?B$ is
usually called the \AP""idempotent core""
of $\?B$. By TODO:addref and this proposition, $\HomFinDec{\?B}$ and
$\HomFinDec{\marked{\core{\?B}}}$ are equivalent under "first-order reductions".
This reduction is a central tool
in the algebraic approach to understand "constraint satisfaction
problem" since the algebra associated to the "CSP" over an "idempotent core"
only has idempotent operations, making it much easier to work with. See TODO:addref for
more details.}%
\AP\label{prop:marking-preserves-csp-complexity}
If $\?B$ is a finite "core", then the problems $\HomFinDec{\marked{\?B}}$ and 
$\HomFinDec{\?B}$ are equivalent under "first-order reductions".
\end{proposition}

\begin{proof}[Proof of \Cref{prop:marking-preserves-csp-complexity}]
\proofcase{Reduction from $\HomFinDec{\?B}$ to $\HomFinDec{\marked{\?B}}$.}
We reduce a "$\sigma$-structure" $\?A$ to the
"$\extendedSignature{\sigma}{\?B}$-structure" $\?A'$ obtained
from $\?A$ by "interpreting@@predicate" each predicate $\unarypred{b}$ as the emptyset.
Clearly, a function from $A$ to $B$ is a "homomorphism" from $\?A$ to $\?B$
"iff" it is a "homomorphism" from $\?A'$ to $\marked{\?B}$, proving the correctness
of the reduction. It is, by definition, "first-order@@reduction".

\proofcase{Reduction from $\HomFinDec{\marked{\?B}}$ to $\HomFinDec{\?B}$.}
We first define the reduction $\Phi$ and show its correctness, we will show later that it
is a "first-order reduction". We reduce a "$\extendedSignature{\sigma}{\?B}$-structure" $\?A$ to the "$\sigma$-structure"
$\Phi(\?A)$ defined as follows:
\begin{itemize}
	\item its underlying universe is the disjoint union $A \dcup B$,
	\item given a "predicate" $\+R$ of arity $k$, its "hyperedges" are:
	\begin{itemize}
	\item all $\+R$-"hyperedges" of $A$,
	\item all $\+R$-"hyperedges" of $B$, and
	\item all $\+R$-"hyperedges" $(b_1,\hdots,b_{i-1}, a_i, b_{i+1},\hdots,b_k)$
		"st" there exists $b_i$ for which the $\+R$-"hyperedge"
		$(b_1,\hdots,b_{i-1}, b_i, b_{i+1},\hdots,b_k)$
		is in $\+R(\?A)$, and $a_i$ belongs to the "interpretation@@predicate" of 
		$\unarypred{b_i}$ in $\?A$.
	\end{itemize}
\end{itemize}
Note that by construction, the "neighbourhood" of $a \in A$ in $\Phi(\?A)$ is
the union of its "neighbourhood" in $\?A$, and the union of the "neighbourhoods" of
$b$ in $\?B$ for all $b$ "st" $a \in \unarypred{b}(\?A)$.
TODO: To be concluded.  
\end{proof}

\subsection{Transitive Tournaments "vs" Paths}

\begin{figure}
	\centering
	\begin{tikzpicture}
		\node[vertex] (0) at (0,0) {};
\node[vertex, below right=of 0] (1) {};
\foreach \i in {1, 3, 5, 7, 9} {
	\pgfmathtruncatemacro{\next}{\i + 1}
	\pgfmathtruncatemacro{\nnext}{\i + 2}
	\node[vertex, below right=of \i] (\next) {};
	\node[vertex, above right=of \next] (\nnext) {};
};
\node[vertex, below right=of 11] (12) {};
\node[vertex, below right=of 12] (13) {};

% ---
% Transitions
% ---
\foreach \i in {0,1,3,5,7,9,11,12} {
	\pgfmathtruncatemacro{\next}{\i + 1}
	\draw[edge] (\i) to (\next);
};
\foreach \i in {2,4,6,8,10} {
	\pgfmathtruncatemacro{\next}{\i + 1}
	\draw[edge] (\next) to (\i);
}
		
\foreach \i in {1,3,5,7,9,11} {
	\pgfmathtruncatemacro{\half}{\i / 2}
	\node[draw=none, above right=-.6em of \i] {$a_{\half}$};
};
\foreach \i in {2,4,6, 8, 10, 12} {
	\pgfmathtruncatemacro{\half}{(\i / 2) - 1}
	\node[draw=none, below left=-.6em of \i] {$b_{\half}$};
};
\node[draw=none, below left=-.6em of 0] {$a'_0$};
\node[draw=none, above right=-.6em of 13] {$b'_5$};

\draw[decoration={brace}, decorate, transform canvas={yshift=1.5em}] (3.north west) to node[midway, above] {$\+O(n)$ nodes} (11.north east);
	\end{tikzpicture}
	\caption{\AP\label{fig:zigzag-graph}The "zigzag graph" $\zigzag{5}{2}$.}
\end{figure}
\begin{example}
	\AP\label{ex:zigzag-defn}
	Let $n\in\?N$.
	We define the \AP""zigzag graph"" $\intro*\zigzag{n}{2}$ of width $n$ and length 2
	to be "graph" whose vertices are $a_0, \hdots, a_n$, $b_0, \hdots, b_{n}$,
	$a'_0$ and $b'_n$, with edges from $a_i$ to $b_{i-1}$ and to $b_{i}$ (for $i \in \lBrack 0,n\rBrack$, whenever the nodes exist), and with an edge from $a'_0$ to $a_0$ and from $b_n$
	to $b'_n$. See \Cref{fig:zigzag-graph} for an illustration.
	
	Note that $\zigzag{n}{2}$ does not admit a "homomorphism" to the "$2$-path"---indeed, such a homomorphism should send $a'_0$, $a_0$ and $b_0$ onto $0$, $1$, and $2$, respectively, 
	and so all $a_i$'s (resp. $b_i$'s) must be sent onto $1$ (resp. $2$), but then $b'_n$ cannot be mapped anywhere.

	On the other hand, $\zigzag{n}{2}$ admits a "homomorphism" to the "$2$-transitive tournament", as witnessed by \Cref{fig:zigzag-graph-hom-T2}.
	In fact, this "homomorphism" is far from being unique:
	each vertex $a_1,\,a_2,\,\hdots,\,a_{n-1}$ can be sent on either $0$ or $1$
	(the red and purple vertices), 
	and similarly, each vertex $b_1,\,b_2,\,\hdots,\,b_{n-1}$ can be sent on either $1$ or $2$
	(the purple and blue vertices).\footnote{Note that it is straightforward
	to extend these results---namely $\zigzag{n}{2} \homto \transitiveTournament{2}$
	and $\zigzag{n}{2} \nothomto \pathGraph{2}$---to arbitrary values of $k\in \N$ with
	$k\geq 2$, namely $\zigzag{n}{k} \homto \transitiveTournament{k}$
	and $\zigzag{n}{k} \nothomto \pathGraph{k}$, by letting
	$\zigzag{n}{k}$ be the graph obtained from $\zigzag{n}{2}$ by
	replacing the path leading to $a_0$ by a path of length $k-1$.
	}
\end{example}
\begin{figure}
	\centering 
	\begin{tikzpicture}
		\node[vertex, draw=c0, fill=c0, fill opacity=.4] (0) at (0,0) {};
\node[vertex, below right=of 0, draw=c1, fill=c1, fill opacity=.4] (1) {};
\foreach \i in {1, 3, 5, 7, 9} {
	\pgfmathtruncatemacro{\next}{\i + 1}
	\pgfmathtruncatemacro{\nnext}{\i + 2}
	\node[vertex, below right=of \i, draw=c2, fill=c2, fill opacity=.4] (\next) {};
	\node[vertex, above right=of \next, draw=c0, fill=c0, fill opacity=.4] (\nnext) {};
};
\node[vertex, below right=of 11, draw=c1, fill=c1, fill opacity=.4] (12) {};
\node[vertex, below right=of 12, draw=c2, fill=c2, fill opacity=.4] (13) {};

% ---
% Transitions
% ---
\foreach \i in {0,1,3,5,7,9,11,12} {
	\pgfmathtruncatemacro{\next}{\i + 1}
	\draw[edge] (\i) to (\next);
};
\foreach \i in {2,4,6,8,10} {
	\pgfmathtruncatemacro{\next}{\i + 1}
	\draw[edge] (\next) to (\i);
}

% ---
% 2-transitive tournament
% ---
\node[vertex, above right=-.25em and 5em of 13, draw=c2, fill=c2, fill opacity=.4] (t2-2) {};
\node[vertex, above=of t2-2, draw=c1, fill=c1, fill opacity=.4] (t2-1) {};
\node[vertex, above=of t2-1, draw=c0, fill=c0, fill opacity=.4] (t2-0) {};

\draw[edge] (t2-0) to (t2-1) to (t2-2);
\draw[edge] (t2-0) to[bend left=60] (t2-2);
	\end{tikzpicture}
	\caption{\AP\label{fig:zigzag-graph-hom-T2}A "homomorphism" from the "zigzag graph" (left-hand side) to the "$2$-transitive tournament" (right-hand side).}
\end{figure}
\section{Undecidability of the Homomorphism Problems}
\label{sec:dichotomy-undecidability}

\subsection{Overview \& Easy Implications of the Dichotomy Theorem}
\label{sec:dichotomy-overview}

\begin{mainstatement}
	\DichotomyThmDichotomyAutomatic
\end{mainstatement}

\begin{remark}
	\todo{TODO GEneralization to "higher-order automatic structures".}
	\todo{Deterministic rational?}
	\todo{RHS is "automatic@@struct"?}
\end{remark}

\begin{marginfigure}
	\centering
	\begin{tikzpicture}
		\node (1) at (0,0) {\itemDTFinDual};
\node[below=2em of 1] (5) {\itemDTFirstOrder};
\node[below=2em of 5] (4) {\itemDTEqual};
\node[below left=2em and -1em of 4] (3) {\itemDTHomRegDec};
\node[below right=2em and -1em of 4] (2) {\itemDTHomDec};

% ---
% Transitions
% ---
\draw[implication] (1) to (5);
\draw[implication] (5) to (4);
\draw[implication] (4) to (3);
\draw[implication] (4) to (2);
\draw[implication, bend left=30] (3) to (1.west);
\draw[implication, bend right=30] (2) to (1.east);
	\end{tikzpicture}
	\caption{\AP\label{fig:dichotomy-overview}Implications shown in the chapter to prove
	\Cref{thm:dichotomy-theorem-automatic-structures}.}
\end{marginfigure}
We prove \Cref{thm:dichotomy-theorem-automatic-structures} by showing the
implications depicted in \Cref{fig:dichotomy-overview}.
The most difficult implications are $\itemDTFinDual \Rightarrow \itemDTFirstOrder$,
which we prove in \Cref{sec:dichotomy-decidability}, and the implications
$\itemDTHomDec \Rightarrow \itemDTFinDual$ and $\itemDTHomRegDec \Rightarrow \itemDTFinDual$,
which we prove by contraposition in \Cref{sec:undecidability-hom,sec:undecidability-homreg}.

On the other hand, the implications $\itemDTFirstOrder \Rightarrow \itemDTEqual$,
$\itemDTEqual \Rightarrow \itemDTHomRegDec$ and $\itemDTEqual \Rightarrow \itemDTHomDec$ are straightforward: we prove them in this section. 
Before showing these implications, we start by proving $\itemDTFinDual \Rightarrow \itemDTHomDec$.\footnote{While it is redundant with the implications of \Cref{fig:dichotomy-overview}, 
we prove this implication for two reasons: not only is it straightforward, it is also
the implication which, together with the fact that both $\HomAut{\clique{2}}$
and $\HomRegAut{\clique{2}}$ are undecidable by
\cite[Proposition 6.5]{Kocher2014AutomatischenGraphen}
and \cite[Theorem 4.4]{BarceloFigueiraMorvan2023SeparatingAutomatic}, that lead us to conjecture
\Cref{thm:dichotomy-theorem-automatic-structures}.}

\paragraph*{Decidability of the Homomorphism Problem.}

\begin{proposition}
	\!\footnote{This corresponds to the implication $\itemDTFinDual \Rightarrow \itemDTHomDec$
	of \Cref{thm:dichotomy-theorem-automatic-structures}.}
	\AP\label{prop:dichotomy-FinDual-implies-HomDec}
	Let $\?B$ be a "finite $\sigma$-structure".
	If $\?B$ has finite duality, then $\HomAut{\?B}$ is decidable in "NL".
\end{proposition}

\begin{proof}
	Given a "finite $\sigma$-structure" $\?D$ with domain $\{d_1,\hdots,d_n\}$,
	we build the "first-order sentence"
	\[
		\phi_{\?D} \defeq \exists x_1.\; \cdots \exists x_n.\;
		\bigwedge_{\+R_{(k)} \in \sigma} \bigwedge_{\substack{\tup{i_1,\hdots,i_k} \in \lBrack 1,n\rBrack^k\\\text{"st" }\tup{d_{i_1},\hdots,d_{i_k}} \in \+R(\?D)}}
		\+R(x_{i_1},\hdots,x_{i_k}).
	\]
	By construction, for any arbitrary "$\sigma$-structure" $\?A$, we have $\?A \FOmodels \phi_{\?D}$
	"iff" $\?D \homto \?A$.
	Then, since $\?B$ has "finite duality", it admits a finite "dual"
	$\?D_1,\hdots,\?D_m$.
	Then
	\[
		\?A \FOmodels \bigwedge_{i=1}^m \neg \phi_{\?D_i}
		\quad\text{"iff"}\quad 
		\?A \homto \?B.
	\]
	The conclusion follows from the fact that the "data complexity"
	of "first-order model checking of automatic structures" is "NL"
	by \Cref{prop:first-order-model-checking-automatic-structures}.
\end{proof}

Note that \Cref{prop:dichotomy-FinDual-implies-HomDec} still holds
when the "homomorphism problem" takes as input "higher-order automatic
structures" since such structures have a decidable first-order theory.


\paragraph*{Uniformly First-Order Definable Homomorphisms Imply Equality.}

\begin{proposition}
	\!\footnote{This corresponds to the implication $\itemDTFirstOrder \Rightarrow \itemDTEqual$
	of \Cref{thm:dichotomy-theorem-automatic-structures}.}
	\AP\label{prop:dichotomy-FirstOrder-implies-Equal}.
	Let $\?B$ be a "finite $\sigma$-structure".
	If $\HomAll{\?B}$ has "uniformly first-order definable homomorphisms", then
	$\HomAut{\?B} = \HomRegAut{\?B}$.
\end{proposition}

\begin{proof}
	By assumption,there exists "first-order formulas"
	$\langle \phi_b(x) \rangle_{b\in B}$ over $\sigma$ "st" for any
	arbitrary "$\sigma$-structure" $\?A$, for any $a\in A$,
	there is at most one $b \in B$ "st" $\langle \?A, a \rangle \FOmodels \phi_b(x)$, and moreover $\?A \homto \?B$
	"iff" for each $a\in A$, there is exactly one $b(a) \in B$ "st" $\langle \?A, a \rangle \FOmodels \phi_{b(a)}(x)$, and $a \mapsto b(a)$ is a "homomorphism".
	Now, when the right-hand side holds, given an "automatic structure" $\+A$,
	for each $b\in \?B$, we let $\tilde\phi_b(x)$ be the "first-order formula" over $\signatureSynchronous{\Sigma}$ obtained from $\phi_b(x)$
	by substituting $\+R(x_1,\hdots,x_k)$ for the "first-order formula"
	over $\signatureSynchronous{\Sigma}$ describing $\+R_{(k)}$ in $\+A$.
	It follows that if $\?A \homto \?B$, then the function
	mapping $u \in \Sigma^*$ to the unique $b \in B$ "st" $\Sigma^*, u \FOmodels \tilde\phi_b(x)$
	defines a "regular homomorphism" from $\+A$ to $\?B$---the regularity follows from
	todo:addref-first-order-iff-automatic.
	And hence $\HomAut{\?B} \subseteq \HomRegAut{\?B}$. The conclusion follows since
	the other implication is trivial.
\end{proof}

\paragraph*{Equality of the Homomorphism Problems Imply their Decidability.}

\begin{proposition}
	\!\footnote{This corresponds to the implication $\itemDTEqual \Rightarrow \itemDTHomDec$
	of \Cref{thm:dichotomy-theorem-automatic-structures}.}
	\AP\label{prop:dichotomy-Equal-implies-HomDec}.
	Let $\?B$ be a "finite $\sigma$-structure".
	If $\HomAut{\?B} = \HomRegAut{\?B}$ then $\HomAut{\?B}$ and $\HomRegAut{\?B}$
	are decidable.
\end{proposition}

To prove this, we first give an upper bound on the homomorphism problems independently
of any assumption on $\?B$.
\begin{proposition}
	\AP\label{prop:dichotomy-general-upper-bounds}
	Let $\?B$ be a "finite $\sigma$-structure".
	Then $\HomAut{\?B}$ is "coRE" and $\HomRegAut{\?B}$ is "RE".
\end{proposition}

\begin{proof}
	\proofcase{$\HomAut{\?B}$ is "coRE".}
	By the "De Bruijn–Erdős Theorem", for any arbitrary "$\sigma$-structure" $\?A$,
	we have $\?A \nothomto \?B$ "iff" there exists a finite "substructure" $\?A'$ of $\?A$
	"st" $\?A' \nothomto \?B$.
	Given a "finite $\sigma$-structure" $\?A'$ and an "automatic $\sigma$-structure",
	it is decidable whether $\?A'$ is a "substructure" of $\?A$: indeed, it suffices
	to check, using \Cref{prop:first-order-model-checking-automatic-structures} if
	\[
		\Bigl(
			\exists x_1,\; \cdots \exists x_n,\;
			\bigwedge_{\+R_{(k)} \in \sigma} \bigwedge_{\substack{\tup{i_1,\hdots,i_k} \in \lBrack 1,n\rBrack^k\\\text{"st" }\tup{a_{i_1},\hdots,a_{i_k}} \in \+R(\?A')}}
			\+R(x_{i_1},\hdots,x_{i_k})
		\Bigr)
		\FOmodels^{?} \?A,
	\]
	by letting $\{a_1,\hdots,a_n\} = A'$. 
	Moreover, whether $\?A' \nothomto \?B$ is also decidable---in "coNP"---by \todo{addref}.
	Overall, this provides a co-semi-algorithm for $\HomAut{\?B}$: we enumerate finite
	"$\sigma$-structure" $\?A'$, and test if (1) $\?A'$ is a "substructure" of $\?A$ and if (2)
	$\?A' \homto \?B$. And hence $\HomAut{\?B}$ is "coRE".

	\proofcase{$\HomRegAut{\?B}$ is "RE".} This is an easy generalization of
	\Cref{thm:k-reg-col-undec}: instead of having "$k$-coloured automaton", we define
	``$\?B$-automaton'', that admit a partition $\langle C_{b} \rangle_{b \in B}$.
	The semantics of such an automaton is a partial function $f\colon \Sigma^* \to B$. 
	Given an "automatic structure" $\+A$, we can then effectively test if $f$ is defined on $\domainPres{\+A}$, and if $f$ defines a "homomorphism" from $\?A$ to $\?B$---see
	the proof of \Cref{thm:k-reg-col-undec}. If so, $\+A \homregto \?B$. Dually, any "regular homomorphism" $\+A \homregto \?B$ can be described by such a $\?B$-automaton.
	Therefore, $\HomRegAut{\?B}$ is "RE".
\end{proof}

\begin{proof}[Proof of \Cref{prop:dichotomy-Equal-implies-HomDec}.]
	If $\HomAut{\?B} = \HomRegAut{\?B}$, then by \Cref{prop:dichotomy-general-upper-bounds},
	these problems are both "RE" and "coRE", and are hence decidable.
\end{proof}

\subsection{Undecidability of \,$\HomAut{\?B}$}
\label{sec:undecidability-hom}

We now prove the undecidability of $\HomAut{\?B}$ and $\HomRegAut{\?B}$
when $\?B$ does not have "finite duality". Both reductions are
direct adaptations of the proof that $\HomFin{\?B}$ is "L"-hard when $\?B$ does not
have finite duality by \textcite[Theorem 3.2]{LaroseTesson2009UniversalAlgebraCSP}.
However, proving the undecidability of the problem that is reduced
to $\HomRegAut{\?B}$ is not entirely trivial and requires some work.
\begin{itemize}
	\item For $\HomFin{\?B}$, we reduce the complement of "Connectivity in automatic graphs",
		providing a "coRE"-lowerbound.
	\item For $\HomRegAut{\?B}$, we reduce "regular unconnectivity in automatic graphs",
		which in turn is reduced from \todo{update}.
\end{itemize}

For $n\in\N$, we define the \AP""$n$-link"" $\intro*\link{n}$ be the "$\sigma$-structure"\sidenote{From \cite[\S~2]{LaroseLotenTardif2007CharacterisationFOCSP}.} 
whose domain is $\lBrack 0,n\rBrack$, and every "predicate" $\+R$
of arity $k$, is interpreted as the set of tuples $\langle a_1,\, \hdots,\, a_k \rangle$
"st" $|a_i-a_j| \leq 1$ for all $i,j \in \lBrack 0,n \rBrack$. See \Cref{fig:n-link}.
\begin{marginfigure}
	\centering
	\begin{tikzpicture}
		\node[vertex] (0) at (0,0) {};
\node[vertex, right=of 0] (1) {};
\node[vertex, right=of 1] (2) {};
\node[vertex, right=of 2, opacity=0] (3) {};
\node[vertex, right=of 3] (4) {};
\node[yshift=-.1em] at (3) (3-label)  {$\cdots$};

% ---
% Transitions
% ---
\foreach \i in {0,1,2,3} {
	\pgfmathtruncatemacro{\next}{\i + 1}
	\draw[edge, bend right=30] (\i) to (\next);
	\draw[edge, bend right=30] (\next) to (\i);
};
\foreach \i in {0,1,2,4} {
	\draw[edge, loop, out=60,in=120,looseness=6] (\i) to (\i);
}
	\end{tikzpicture}
	\caption{\AP\label{fig:n-link}The "$n$-link" $\link{n}$ over the "graph signature".}
\end{marginfigure}
Given a "$\sigma$-structure" $\?B$, say that $b \in \?B$ and $b'$ are
\AP""$n$-linked"" if there exists a "homomorphism" from $\link{n}$ to $\?B$
that sends $0$ to $b$ and $n$ to $b'$. We say that $b$ and $b'$ are \AP\reintro{linked} if
they are "$n$-linked" for some $n \in \N$.

Note that the fact that $k \mapsto n-k$
defines an "automorphism" of $\link{n}$ implies that the relation of being "$n$-linked"---%
and to a greater extent of being "linked"---is symmetric.
Moreover, being "linked" is transitive, but not necessarily reflexive.

% \begin{fact}[Link Composition]
% 	\AP\label{fact:link-composition}
% 	Let $n,m \in \N$, let $\?B$ be a "$\sigma$-structure", and let $b,b',b'' \in B$.
% 	If $b$ and $b'$ are "$n$-linked" and $b'$ and $b''$ are "$m$-linked", then
% 	$b$ and $b''$ are "$(n+m)$-linked".
% \end{fact}
% We define the binary relation $\sim_n$
% on $\link{n} \prodstruct \iterstruct{\?B}{2}$ as follows:\sidenote{From \cite[\S~4.3]{LaroseLotenTardif2007CharacterisationFOCSP}.}%
% \[
% 	\langle k,\, b_1,\, b_2\rangle \sim_n \langle k',\, b'_1,\, b'_2\rangle
% 	\quad\text{when}\quad
% 	\begin{cases*}
% 		\;\langle k,\,  b_1,\, b_2\rangle = \langle k',\, b'_1,\, b'_2\rangle\text{, or}\\
% 		\;k = k' = 0 \text{ and } b_1 = b'_1\text{, or}\\
% 		\;k = k' = n \text{ and } b_2 = b'_2.
% 	\end{cases*}
% \]

\begin{proposition}[{\cite[Theorem 4.7]{LaroseLotenTardif2007CharacterisationFOCSP}}]%
	\!\footnote{Actually \cite[Theorem 4.7]{LaroseLotenTardif2007CharacterisationFOCSP} assumes
	that $\HomFin{\?B}$ is "first-order definable", but this condition
	is equivalent to $\?B$ having "finite duality" by Atserias' result
	\cite[Corollary 4]{Atserias2008DigraphColoring}.}%
	%%%
	\AP\label{prop:characterization-finite-duality-path-projections}
	An arbitrary "$\sigma$-structure" $\?B$ has "finite duality" "iff"
	$\projHom{1}$ and $\projHom{2}$ are "linked" in $\powstruct{\?B}{(\iterstruct{\?B}{2})}$.
\end{proposition}

Equipped with the previous proposition, we can now show the undecidability 
of $\HomAut{\?B}$ by reduction from the following problem.

\decisionproblem{""Connectivity in Automatic Graphs""}{
	A "automatic presentation of a directed graph" $\•G$,
	and two elements $s,t \in \Sigma^*$.
}{
	Are $\•G(s)$ and $\•G(t)$ "connected" in $\?G$?
}
\begin{property}
	\AP\label{prop:undecidability-connectivity}
	For any "signature" $\sigma$ containing at least one "predicate" of
	arity at least 2, "Connectivity in automatic graphs" is "RE"-complete.
\end{property}

\begin{proof}
	This follows from the fact that the "configuration graph" of
	a "Turing machine" is always "automatic@@struct" by \todo{addref}.
	\todo{Give more precisions.}
\end{proof}

\begin{lemma}
	\AP\label{lem:reduction-hom}
	Assume that $\sigma$ contains at least one "predicate" of arity at least 2.
	If $\?B$ does not have "finite duality", then there is a "first-order reduction" 
	from the complement of "connectivity in automatic graphs" to $\HomAut{\marked{\?B}}$.
\end{lemma}

\begin{marginfigure}
	\centering
	\begin{tikzpicture}
		\node[vertex] (0) at (0,0) {};
\node[vertex, right = 2.5 em of 0] (1) {};
\node[vertex, above right = 1em and 2em of 1] (2) {};
\node[vertex, below right = 1em and 2em of 1] (3) {};

\draw[edge] (1) to (2);
\draw[edge, bend right=15] (2) to (3);
\draw[edge, bend right=15] (3) to (2);
\draw[edge] (1) to (3);
	\end{tikzpicture}
	\caption{
		\AP\label{fig:graph-to-struct-graph}
		A "graph@@dir" $\?G$.
	}
\end{marginfigure}
\begin{marginfigure}
	\centering
	\begin{tikzpicture}
		\node[vertex] (0) at (0,0) {};
\node[vertex, right = 2.5 em of 0] (1) {};
\node[vertex, above right = 1em and 2em of 1] (2) {};
\node[vertex, below right = 1em and 2em of 1] (3) {};

\draw[edge, bend right=15] (1) to (2);
\draw[edge, bend right=15] (2) to (1);
\draw[edge, bend right=15] (2) to (3);
\draw[edge, bend right=15] (3) to (2);
\draw[edge, bend right=15] (1) to (3);
\draw[edge, bend right=15] (3) to (1);

\draw[edge, loop, out=150,in=210,looseness=6] (1) to (1);
\draw[edge, loop, out=60,in=120,looseness=6] (2) to (2);
\draw[edge, loop, out=-120,in=-60,looseness=6] (3) to (3);
	\end{tikzpicture}
	\caption{
		\AP\label{fig:graph-to-struct-struct}
		The "structure" $\?A$ defined from $\?G$ (in \Cref{fig:graph-to-struct-graph}),
		using the construction done in the proof of \Cref{lem:reduction-hom},
		when $\sigma$ consists of a single binary relation.
	}
\end{marginfigure}

\begin{proof}
	Given an instance $\langle \•G, s, t \rangle$ of "connectivity in automatic graphs",
	we first define the $\sigma$-structure $\?A$ with "automatic presentation" $\•A$
	obtained by replacing every edge of $\•G$ by a "$1$-link".
	Formally, $\?A$ has the same domain as $\?G$, and for any
	"predicate" $\+R \in \sigma$ of arity $k$,
	$\langle g_1,\, \hdots,\, g_k \rangle \in \+R(\?A)$ "iff"
	$\{g_1, \hdots, g_k\} = \{g,g'\}$ for some $g,g' \in G$ "st"
	there is an edge from $g$ to $g'$ in $\?G$.
	See \Cref{fig:graph-to-struct-graph,fig:graph-to-struct-struct}.

	\begin{claim}
		\AP\label{claim:reduction-hom-from-graph-to-link}
		$\•G(s)$ and $\•G(t)$ are "connected" "iff"
		$\•A(s)$ and $\•A(t)$ are "linked".
	\end{claim}
	For the left-to-right implication: if there is an edge between two elements
	in $\?G$, then they are "$1$-linked" in $\?A$. Since being "linked" is
	reflexive and transitive, the conclusion follows.
	Conversely, if two elements $a$ and $a'$ of $\?A$ are "$1$-linked", 
	then pick a "predicate" $\+R \in \sigma$ of arity at least 2.
	Then $\langle a,\, \hdots,\, a,\, a' \rangle \in \+R(\?A)$,
	and so by definition of $\?A$ there is either an edge from $a$ to $a'$
	or from $a'$ to $a$ in $\?G$.\sidenote{Note that the proof of this claim
	is the only part of the proof of \Cref{lem:reduction-hom} that requires
	the assumption that $\sigma$ contains at least one "predicate" of arity at least 2.}

	We then consider the "automatic $\sigma$-structure" $\?A\prodstruct \iterstruct{\?B}{2}$,%
	\footnote{See \Cref{sec:construction-automatic-presentations} to have an explicit construction 
	of an "automatic presentation" for this structure.}
	and extend it to a
	"$\extendedSignature{\sigma}{\?B}$-structure" \AP\(\intro*\ConstrUndecHom{(\?A\prodstruct \iterstruct{\?B}{2})}\)
	in which for each $b_0 \in B$,
	we "interpret@@predicate" the unary "predicate" $\unarypred{b_0}$ as
	\[
		\big\{\;
			\langle a,\, b,\, b'\rangle \;\big\vert\;
			a = \•A(s) \text{ and } b = b_0 \text{, or }
			a = \•A(t) \text{ and } b' = b_0
		\;\big\}.
	\]
	To construct an "automatic presentation" for this structure, see \Cref{sec:construction-automatic-presentations}.
	\begin{claim}
		\AP\label{claim:reduction-hom-direct}
		If $\ConstrUndecHom{(\?A\prodstruct \iterstruct{\?B}{2})} \homto \marked{\?B}$,
		then $\•G(s)$ and $\•G(t)$ are not "connected" in $\?G$.
	\end{claim}
	Let $f\colon \ConstrUndecHom{(\?A\prodstruct \iterstruct{\?B}{2})} \homto \marked{\?B}$
	be a "homomorphism".\sidenote{Recall that both sides are
	"$\extendedSignature{\sigma}{\?B}$-structures".}
	It induces a "homomorphism"
	\[
		\overbar f\colon \?A\prodstruct \iterstruct{\?B}{2} \homto \?B
	\]
	between "$\sigma$-structures", and by currying (\Cref{prop:currying-hom}),
	$\overbar f$ can be seen as a "homomorphism"
	\[
		F\colon \?A \homto \powstruct{\?B}{(\iterstruct{\?B}{2})}.
	\]
	Note moreover that because $\overbar f$ comes from a "homomorphism" between
	$\extendedSignature{\sigma}{\?B}$ then we must have  
	$f(\•A(s),\, b,\, b') = b$
	and $f(\•A(t),\, b,\, b') = b'$ for all $b,b' \in B$.
	This implies that $F(\•A(s)) = \projHom{1}$ and $F(\•A(t)) = \projHom{2}$.
	
	We now assume by contradiction that there is some $n \in \N$
	"st" there is a "homomorphism" $g\colon \link{n} \to \?A$
	with $g(0) = \•A(s)$ and $g(n) = \•A(t)$.
	Then by composition, we obtain a "homomorphism"
	\[
		F \circ g\colon
		\link{n} \to \powstruct{\?B}{(\iterstruct{\?B}{2})},
 	\]	
	which sends $0$ to $F(g(0)) = F(\•A(s)) = \projHom{1}$
	and sends $n$ to $F(g(n)) = F(\•A(t)) = \projHom{2}$.
	So, by \Cref{prop:characterization-finite-duality-path-projections},
	$\?B$ would have "finite duality", which is a contradiction.
	Hence, $\•A(s)$ and $\•A(t)$ are not "linked",
	and so by \Cref{claim:reduction-hom-from-graph-to-link}, $\•G(s)$ and $\•G(t)$
	are not "connected".

	\begin{claim}
		\AP\label{claim:reduction-hom-converse}
		If $\•G(s)$ and $\•G(t)$ are not "connected" in $\?G$,
		then there is a "homomorphism" from
		$\ConstrUndecHom{(\?A\prodstruct \iterstruct{\?B}{2})}$ to $\marked{\?B}$.
	\end{claim}
	We define a "homomorphism" $f\colon \ConstrUndecHom{(\?A\prodstruct \iterstruct{\?B}{2})} \to \marked{\?B}$ by:
	\[
		f(a, b, b') \defeq \begin{cases*}
			\;b & \text{ if $\•A(s)$ and $a$ are "linked",} \\
			\;b' & \text{ otherwise.}
		\end{cases*}
	\]
	We show that this is indeed a "homomorphism": for any "predicate" $\+R$
	of arity $k$ in $\sigma$, if
	\[
		\langle a_1,\, b_1,\, b'_1 \rangle,\;
		\langle a_2,\, b_2,\, b'_2 \rangle,\;
		\hdots,\;
		\langle a_k,\, b_k,\, b'_k \rangle
	\]
	are all "$\+R$-hyperedges" of $\ConstrUndecHom{(\?A\prodstruct \iterstruct{\?B}{2})}$,
	then by definition of $\?A$, we have that either (1) all $a_i$'s are equal,
	or (2) $\{a_1,\, \hdots,\, a_k\} = {a,a'}$ for some $a \neq a' \in A$
	and there is an edge from $a$ to $a'$ or from $a'$ to $a$ in $\?G$.
	In both cases, it follows that $\•A(s)$ and $a_i$ are "linked"
	"iff" $\•A(s)$ and $a_j$ for all $i,j\in \lBrack 1,k\rBrack$.
	Hence, either $f(a_i,\, b_i,\, b'_i) = b_i$ for all $i\in \lBrack 1,k\rBrack$,
	or $f(a_i,\, b_i,\, b'_i) = b'_i$ for all $i\in \lBrack 1,k\rBrack$.
	In both cases, we get that
	\[
		\Big\langle
			f(a_1,\, b_1,\, b'_1),\;
			f(a_2,\, b_2,\, b'_2),\;
			\hdots,\;
			f(a_k,\, b_k,\, b'_k)
		\Big\rangle
		\in \+R(\?B).
	\]
	We also need to show that this map preserves the new unary "predicates" of
	$\extendedSignature{\sigma}{\?B}$: this follows from---and is in fact equivalent to---the
	fact that $\•A(s)$ and $\•A(t)$ are not "linked" by \Cref{claim:reduction-hom-from-graph-to-link}.
	Overall, this proves that $\ConstrUndecHom{(\?A\prodstruct \iterstruct{\?B}{2})} \homto \marked{\?B}$.

	Putting \Cref{claim:reduction-hom-direct,claim:reduction-hom-converse} together,
	we get that the reduction is correct.
	Lastly, note that it is a "first-order reduction" because \todo{TODO}.
\end{proof}

By \Cref{prop:undecidability-connectivity}, the complement of "Connectivity in automatic graphs"
is "coRE"-complete, and assuming that $\sigma$ contains at least one "predicate" of arity 2,
it reduces by \Cref{lem:reduction-hom} to any problem $\HomAut{\marked{\?B}}$ when $\?B$ has "finite duality". In turns, by \Cref{prop:idempotent-core-preserves-csp-complexity}, it reduces to
$\HomAut{\?B}$, which is thus "coRE"-hard. It remains to deal with "signatures" consisting of only
unary "predicates".\sidenote{It is not clear to us whether this case was properly handled in
\cite{LaroseLotenTardif2007CharacterisationFOCSP}.}

\begin{property}
	\AP\label{prop:finite-duality-unary-predicates}
	If $\sigma$ only consists of unary "predicates", then all "$\sigma$-structures"
	have "finite duality".
\end{property}	

\begin{proof}
	Fix a $\sigma$-structure $\?B$. We define the \AP""unary type""
	$\intro*\unaryType{b}{\?B}$ of $b \in \?B$
	to be the set of "predicates" $\+P$ "st" $b \in \+P(\?B)$.
	
	Given $\tau \subseteq \sigma$, define \AP$\intro*\structOfUnaryType{\tau}$
	to be the "$\sigma$-structure"
	consisting of a single element $*$, and "st" $* \in \+P(\?1_\tau)$ "iff"
	$\+P \in \tau$.
	We say that $\tau$ is \AP""obstructing@@unarytype"" if
	$\tau \not\subseteq \unaryType{b}{\?B}$ for all $b \in \?B$.

	\begin{claim}
		\AP\label{claim:finite-duality-unary-predicates-direct}
		If $\tau$ is "obstructing@@unarytype",
		then $\structOfUnaryType{\tau} \nothomto \?B$.
	\end{claim}
	We prove the result by contraposition.
	Any "homomorphism" from $\structOfUnaryType{\tau}$ to $\?B$
	should send $*$ on some element $b$ of $\?B$
	"st" $b \in \+P(\?B)$ for all $\+P \in \tau$, and
	hence $\tau \subseteq \unaryType{b}{\?B}$.

	\begin{claim}
		\AP\label{claim:finite-duality-unary-predicates-converse}
		If $\?A \nothomto \?B$ then there exists an "obstructing@@unarytype"
		$\tau \subseteq \sigma$ "st" $\structOfUnaryType{\tau} \homto \?A$.
	\end{claim}
	We define a partial homomorphism $f$ from $A$ to $B$,
	by sending $a \in A$ to any $b \in B$ "st" the "unary type" of $a$
	is included in the "unary type" of $b$. This is clearly a (partial) "homomorphism",
	and so since $\?A \nothomto \?B$, it follows that it must be partial,
	"ie" that some element $a \in \?A$ "st" $\unaryType{a}{\?A} \not\subseteq
	\unaryType{b}{\?B}$ for any $b \in B$. It follows that $\unaryType{a}{\?A}$
	is "obstructing@@unarytype". Since $\structOfUnaryType{\unaryType{a}{\?A}} \homto \?A$
	"via" $* \mapsto a$, the conclusion follows.

	Putting \Cref{claim:finite-duality-unary-predicates-direct,claim:finite-duality-unary-predicates-converse} together, we get that
	\[
		\big\{\;
			\structOfUnaryType{\tau}
			\;\big\vert\;
			\text{ $\tau \subseteq \sigma$ is "obstructing@@unarytype"} 
		\;\big\}
	\]
	is a finite "dual" for $\?B$.
\end{proof}

\begin{corollary}
	\!\sidenote[][-15em]{In the case of Larose and Tesson, they study the problem
	$\HomFin{-}$, and prove in \cite[Theorem 3.2]{LaroseTesson2009UniversalAlgebraCSP}
	that there is a "first-order reduction" from "Connectivity in Finite Graphs" to
	$\HomFin{\marked{\?B}}$ assuming that $\?B$ does not have finite duality.
	Moreover, "Connectivity in Finite Graphs" is "L"-hard under "first-order reductions" since
	Etessami \cite[Theorem 3.2]{Etessami1997CountingLogSpace} proved that the problem
	of given a directed path and two vertices $s$, $t$ to decide if there is a path from
	$s$ to $t$ is "L"-hard under "first-order reductions"-in fact even under quantifier-free reductions. It turn, this problem can be reduced in "first-order@@reduction"
	to "Connectivity in Finite Graphs" \cite{SamiD2015USTCONNLogspace}.
	Overall, and together with \Cref{prop:idempotent-core-preserves-csp-complexity},
	this shows that $\HomFin{\?B}$ is "L"-hard under "first-order reductions".}%
	%%%
	\AP\label{coro:lowerbound-hom}
	If $\?B$ does not have "finite duality", then $\HomAut{\?B}$
	is "coRE"-hard.
\end{corollary}

\begin{proof}
	By \Cref{prop:finite-duality-unary-predicates}, since $\?B$ does not have "finite duality",
	then $\sigma$ has at least one "predicate" of arity at least 2.
	The conclusion follows from \Cref{prop:idempotent-core-preserves-csp-complexity,prop:undecidability-connectivity,lem:reduction-hom}.
\end{proof}

\subsection{Undecidability of \,$\HomRegAut{\?B}$}
\label{sec:undecidability-homreg}

The reduction to show undecidability of is nearly identical to \Cref{lem:reduction-hom},
but the input problem differs quite a lot.
\decisionproblem{""Regular Unconnectivity in Automatic Graphs""}{
	An "automatic presentation" $\•G$ of a "directed graph" $\?G$,
	and two elements $s,t \in \Sigma^*$.
}{
	Is there a regular language $L \subseteq \Sigma^*$ 
	such that $s \in L$, $t\not\in L$ and $L$ is a union of "connected components"
	of $\•G$?\footnotemark{}
	In this case we say that $s$ and $t$ are \AP""regularly unconnected"".
}
\footnotetext{Formally, we mean that $L = \•G^{-1}[U]$ for some union $U$ of "connected components" of $\?G$.}

We will first reduce this problem to $\HomRegAut{\?B}$,
and will later settle its complexity.

\begin{lemma}
	\AP\label{lem:reduction-hom-reg}
	Assume that $\sigma$ contains at least one "predicate" of arity at least 2.
	If $\?B$ does not have "finite duality", then there is a "first-order reduction"
	from "regular unconnectivity in automatic graphs"
	to $\HomRegAut{\marked{\?B}}$.
\end{lemma}

\begin{proof}
	Given an instance $\langle \•G, s, t \rangle$ of "regular unconnectivity in automatic graphs",
	we first define the $\sigma$-structure $\?A$ with "automatic presentation" $\•A$
	obtained by replacing every edge by a "$1$-link", as in \Cref{lem:reduction-hom}.

	\begin{claim}
		\!\footnote{While ``being "linked"'' is not reflexive in general, it is over the
		structure $\?A$, by reflexivity of ``being "connected"'' in $\•G$.}%
		\AP\label{claim:reduction-homreg-from-graph-to-link}
		$\•G(s)$ and $\•G(t)$ are "regularly unconnected" "iff"
		there is no $L \subseteq \Sigma^*$ "st" $\•A(s)\in L$ and $t \not\in L$,
		and $L$ is a union of equivalences classes of $\domainPres{\•A}$
		under ``being "linked"''.
	\end{claim}
	The proof is similar to \Cref{claim:reduction-homreg-from-graph-to-link}.
	Then again, we reduce the instance $\langle \•G, s, t \rangle$
	to an "automatic presentation" of \(\ConstrUndecHom{(\?A\prodstruct \iterstruct{\?B}{2})}\),
	as in \Cref{lem:reduction-hom}.
	\begin{claim}
		\AP\label{claim:reduction-homreg-direct}
		If $\ConstrUndecHom{(\•A\prodpres \iterstruct{\?B}{2})} \homregto \marked{\?B}$,
		then $\•G(s)$ and $\•G(t)$ are "regularly unconnected" in $\?G$.
	\end{claim}
	
	Let \(f\colon \ConstrUndecHom{(\•A\prodpres \iterstruct{\?B}{2})} \to \marked{\?B}\)
	be a "regular homomorphism".
	By currying---see \Cref{coro:homreg-currying}---of the underlying "homomorphism"
	between "\(\sigma\)-structures", we obtain a "regular homomorphism"
	\[
		F\colon \•A \homto \powstruct{\?B}{(\iterstruct{\?B}{2})}.
	\]
	Moreover, using the "predicates" \(\unarypred{b}\), \(b \in B\),
	we get that $F(\•A(s)) = \projHom{1}$ and $F(\•A(s)) = \projHom{2}$.

	We then define \[\+X \defeq \{g \in \powstruct{\?B}{(\iterstruct{\?B}{2})} \mid \text{ $g$ and $\projHom{1}$ are "linked" or $g = \projHom{1}$}\}.\]
	We claim that ${F}^{-1}[\+X]$ witnesses the fact that
	$\•G(s)$ and $\•G(t)$ are "regularly unconnected".
	First, $\projHom{1} \in \+X$ so $\•A(s) \in {F}^{-1}[\+X]$.
	Since $\?B$ has "finite duality", by \Cref{prop:characterization-finite-duality-path-projections}, $\projHom{2} \not\in \+X$
	and so $\•A(t) \not\in {F}^{-1}[\+X]$.
	Then, ${F}^{-1}[\+X]$ is "regular@@lang" since $F$ is a "regular homomorphism". Finally, ${F}^{-1}[\+X]$ is a union of
	equivalences classes of $\domainPres{\•A}$ under ``being "linked"''.\footnote{Indeed,
	if $c_1, c_2 \in \?C$ are "linked" in some "structure" $\?C$ and if $f\colon \?C \to \?D$ is a "homomorphism", then $f(c_1)$ and $f(c_2)$ are "linked" in $\?D$.}
	Hence, by \Cref{claim:reduction-homreg-from-graph-to-link}, $\•G(s)$ and $\•G(t)$ are "regularly unconnected".

	\begin{claim}
		\AP\label{claim:reduction-homreg-converse}
		If $\•G(s)$ and $\•G(t)$ are "regularly unconnected" in $\?G$,
		then $\•A\prodpres \iterstruct{\?B}{2} \homregto \marked{\?B}$.
	\end{claim}

	Since $\•G(s)$ and $\•G(t)$ are "regularly unconnected" in $\?G$,
	by \Cref{claim:reduction-homreg-from-graph-to-link} there is a "regular language" $L \subseteq \Sigma^*$ "st" $\•A(s)\in L$ and $\•A(t) \not\in L$,
	and $L$ is a union of equivalences classes of $\domainPres{\•A}$
	under ``being "linked"''.
	We define a function $f\colon \domainPres{\•A}\times B^2 \to B$ by 
	\[
		f(a, b, b') \defeq \begin{cases*}
			\;b & \text{ if $\•A(s) \in L$,} \\
			\;b' & \text{ otherwise,}
		\end{cases*}
	\]
	and we claim that $f$ is a "regular homomorphism" from
	\(\•A\prodpres \iterstruct{\?B}{2}\) to \(\marked{\?B}\).
	The proof that it is a "homomorphism" is similar to \Cref{claim:reduction-hom-converse}:
	in particular, we use the fact that $\•G(s)$ and $\•G(t)$ are not "connected" in $\?G$,
	which is a consequence of the fact that they are "regularly unconnected".
	"Regularity@@hom" follows from the "regularity@@lang" of $L$. 
	Hence, $\•A\prodpres \iterstruct{\?B}{2} \homregto \marked{\?B}$.

	Putting \Cref{claim:reduction-homreg-direct,claim:reduction-homreg-converse} together,
	we get that the reduction is correct.
	Lastly, note that this is a first-order reduction because \todo{TODO}.
\end{proof}

We then prove a lower bound on the complexity of "regular unconnectivity in automatic graphs".

\begin{lemma}
	\AP\label{lemma:regular-unconnectivity-lowerbound}
	"Regular unconnectivity in automatic graphs" is "RE"-hard.
\end{lemma}

\begin{proof}
	\todo{BIG-TODO.}
	% We consider the following intermediate (promise) problem.
	% \decisionproblem{""Restricted Membership for Deterministic Reversible Turing Machines""}{
	% A "deterministic" "reversible Turing machine" $\+M$ with initial state $q_0$,
	% words $u_\top,u_\bot \in \Sigma^*$
	% "st"
	% \begin{enumerate}
	% 	\item $u_\bot \not\in \semTM{\+M}$,
	% 	\item $q_0\cdot u_\top$ does not have a predecessor in the
	% 		"configuration graph" of $\+M$, and
	% 	\item all "configurations" except those "connected" to $q_0\cdot u_\top$ or $q_0\cdot u_\bot$ belong to the same "connected component" of the "configuration graph" of $\+M$.
	% \end{enumerate}
	% }{
	% Does $u_\top \in \semTM{\+M}$?
	% }
	% We will prove that:
	% \begin{enumerate}
	% 	\item there is a reduction from "Restricted Membership for Deterministic Reversible Turing Machines" to "Regular Unconnectivity in automatic graphs", and 
	% 	\item the former problem is "RE"-hard.
	% \end{enumerate}

	% \proofcase{Reduction from "Restricted Membership for Deterministic Reversible Turing Machines" to "Regular Unconnectivity in automatic graphs".}
	% We map an instance
	% $\langle \+M,\, u_\bot,\, u_\top \rangle$ to $\langle \+M',\, q_0\cdot u_\top,\, q_0\cdot 
	% u_\bot \rangle$ where $\+M'$ is defined in such a way that for any "configuration transition"
	% $\gamma \to \gamma'$ in $\+M$, for any $n\in\N$, then there is a sequence of transitions from
	% $\gamma \triangleright^n \triangleleft^n$ to $\gamma' \triangleright^{n+1} \triangleleft^{n+1}$ 
	% in $\+M'$, where $\triangleright$ and $\triangleleft$ are new symbols.\footnote{This can be 
	% implemented by marking the current position, then using new states going to the right end of 
	% the tape, replacing the first $\triangleleft$ (or, if it doesn't exist a blank symbol) with a $\triangleright$, and adding the correct number of $\triangleleft$ symbols at the very end, and then moving back the head to the marked position.}
	% Moreover, if $\gamma$ is a final "configuration" of $\+M$ and whose state is rejecting,
	% then we add a sequence of transitions from 
	% $\gamma \triangleright^n \triangleleft^n$ to $\gamma \triangleright^{n+1} \triangleleft^{n+1}$ 
	% in $\+M'$.
	% Note that both operations can be done in such a way that all intermediate configurations
	% from $\gamma \triangleright^n \triangleleft^n$ to $\gamma' \triangleright^{n+1} \triangleleft^{n+1}$ are not in state of $\statesTM{\+M}$.
	% \begin{claim}
	% 	\label{claim:reduction-transitions-graphs}
	% 	For any transitions $\gamma$ and $\gamma'$ of $\+M$, the following are equivalent:
	% 	\begin{enumerate}
	% 		\item there exists $n \in \N$ "st" $n$ and is a path from $\gamma \triangleright^n \triangleleft^n$ to $\gamma' \triangleright^{n+1} \triangleleft^{n+1}$ in
	% 		the "configuration graph" of $\+M'$,
	% 		\item $\gamma \to \gamma'$ in the "configuration graph" of $\+M$.
	% 	\end{enumerate} 
	% \end{claim}

	% Note that by construction, for any $u \in \Sigma^*$---where $\Sigma$ denotes the alphabet of $\+M$---, $q_0\cdot u$ is a configuration of $\+M'$. The path leaving from it in the "configuration graph" of $\+M'$ is finite "iff" $u \in \semTM{\+M}$ since
	% our reductions maps (1) infinite paths to infinite paths, (2) finite paths ending on a rejecting state to infinite paths and (3) finite paths ending on an accepting state
	% to a finite path.
	% It follows that, by "irregularity@@lang" of $\{\triangleright^n \triangleleft^n \mid n\in \N\}$,
	% that the connected component of $q_0 \cdot u_\bot$ is "irregular@@lang".
	% Moreover, $q_0\cdot u_\top$ does not have any predecessor in the "configuration graph" of $\+M'$ since it didn't have any predecessor in the one of $\+M$. Hence, the "connected component" of $q_0 \cdot u_\top$ in the "configuration graph" of $\+M'$ is either:
	% \begin{itemize}
	% 	\item finite, and hence "regular@@lang", if $u_\top \in \semTM{\+M}$,
	% 	\item "irregular@@lang" otherwise.
	% \end{itemize}

	% So, if $u_\top \in \semTM{\+M}$, it means that $q_0\cdot u_\top$
	% and $q_0\cdot u_\bot$ are "regularly unconnected" by the connected component of $u_\top$:
	% Indeed $q_0\cdot u_\top$ does not belong to the connected component of $q_0\cdot u_\bot$
	% since we assumed that $u_\bot \not\in \semTM{\+M}$.
	% We claim that the converse property holds.

	% \begin{claim}
	% 	If $q_0\cdot u_\top$ and $q_0\cdot u_\bot$ are "regularly unconnected" in the "configuration graph" of $\+M'$, then $u_\top \in \semTM{\+M}$.
	% \end{claim}
	
	% Indeed, let $L' \subseteq \Gamma^*$ be a "regular language" "st" $q_0\cdot u_\top \in L'$ $q_0\cdot u_\bot \not\in L'$, "st" $L'$ is a union of "connected components" of the "configuration graph" of $\+M$.\footnote{Here, $\Gamma$ denotes the alphabet used in the "automatic presentation" of the "configuration graph" of $\+M'$. Namely, $\Gamma = \Sigma \dcup \statesTM{\+M} \dcup (\statesTM{\+M'} \smallsetminus \statesTM{\+M})$.} 
	% Then let \[L
	% 	\defeq \rightquotient{(L' \cap \Sigma^* \statesTM{\+M} \Sigma^*)}{(\triangleright^* \triangleleft^*)}.
	% \]
	% In other words, $L$ is obtained by only keeping configurations of $L'$ whose state is
	% a state of $\+M$, and removing $\triangleright$ and $\triangleleft$ symbols at their end.  
	% $L'$ is "regular@@lang" as an "intersection" and "right quotient" of regular languages.
	% Moreover, by \Cref{claim:reduction-transitions-graphs}, $L$ is a union of
	% "connected components" of the "configuration graph" of $\+M$.
	% Since $q_0\cdot u_\top \in L$ and $q_0\cdot u_\bot \not\in L$, and since we assumed that
	% the "configuration graph" of $\+M$ has at most three "connected components",
	% it follows that either:
	% \begin{itemize}
	% 	\item $L$ is the "connected component" of $q_0\cdot u_\top$,
	% 	\item $L$ is the complement of the "connected component" of $q_0\cdot u_\bot$.
	% \end{itemize}
	% TODO: PB: in this last case we could have $L' = q_0\cdot u_\bot\cdot \triangleright^* \triangleleft^*$.
\end{proof}


\begin{corollary}
	\AP\label{coro:lowerbound-homreg}
	If $\?B$ does not have "finite duality", then $\HomAut{\?B}$
	is "RE"-hard.
\end{corollary}

\begin{proof}
	Recall that $\HomAut{\?B} = \HomAut{\core{\?B}}$, so we assume "wlog"
	that $\?B$ is a "core".
	By \Cref{lemma:regular-unconnectivity-lowerbound},
	"regular unconnectivity in automatic graphs" is "RE"-hard.
	Then by \Cref{prop:finite-duality-unary-predicates}, since $\?B$ does not have "finite
	duality", $\sigma$ does not consist only of unary "predicates",
	and hence by \Cref{lem:reduction-hom-reg}, we get
	a reduction from "regular unconnectivity in automatic graphs" to
	$\HomRegAut{\marked{\?B}}$, which in turns
	reduces to $\HomRegAut{\?B}$ by \Cref{prop:idempotent-core-preserves-csp-complexity}
	since $\?B$ was assumed to be a "core".
	Indeed, "first-order reductions" preserves "regularity@@hom", by 
	\todo{addref-automatic-iff-first-order.}
\end{proof}
\section{\AP\label{sec:hyperedge-consistency}%
	Hyperedge Consistency}

In this section, we introduce the "hyperedge consistency algorithm for automatic structures",
which is a variation of the classical "hyperedge consistency algorithm" for "finite structures".
We start by explaining the later algorithm,
which solves $\HomFin{\?B}$ for some $\?B$'s.\footnote{We will
see later that the algorithm is correct for "$\sigma$-structures" with so-called "tree duality",
which is a superclass of the "structures" with "finite duality".}
Then, we will use the former algorithm to prove that assuming that $\?B$ has "finite duality",
then $\HomRegAut{\?B}$ is decidable.\footnote{Interestingly, this algorithm cannot solve
$\HomRegAut{\?B}$ when $\?B$ has only "tree duality" but not "finite duality". 
We discuss this in more details in TODO.}

\subsection{Hyperedge Consistency for Finite Structures}

Given a "homomorphism" $f\colon \?G \to \?H$ between "graphs",
note that if $g\in G$ has at least one successor in $\?G$, then must have $f(g)$ in $\?H$.
As a consequence, such an $g$ cannot be mapped by any "homomorphism" to a vertex of $\?H$ with no
successor. The idea behind "hyperedge consistency@@finite" is precisely to identify for each
$g\in G$ the set $\textrm{Im}_g$ of all elements of $\?H$ to which it can be mapped: initially this set is $H$,
and we try to find some ``obstructions''. These obstructions take the following form:
if $g \in G$ has a successor $g' \in G$, then any vertex of $\textrm{Im}_g$
must have a successor in $\?H$ that lives in $\textrm{Im}_{g'}$.

TODO:example $T_2$.

We formalize this algorithm as the greatest fixpoint of some operator.
Given a finite "$\sigma$-structure" $\?B$, and an arbitrary\sidenote{Note that in this part, while some results---mostly complexity/decidability ones---require the assumption that the "left structure" is finite, some results do not,
and are stated for arbitrary "structures".}%
"$\sigma$-structure" $\?A$, we say that a function $F\colon A \to \pset{B}$ is "subsumed"
by $G\colon A \to \pset{B}$, denoted by \AP\(F \intro*\subsumed G\),
if $F(a) \subseteq G(a)$ for each $a \in A$. We denote by
\AP\(\intro*\LatticeGuessFunctions{A}{B}\) the set of functions $A \to \pset{B}$ under this order.%
\footnote{Equivalently, $\LatticeGuessFunctions{A}{B}$ is the
set of binary relations between $A$ and $B$, ordered by inclusion.}

We then define an operator on this space, which corresponds to one step of the "hyperedge consistency algorithm@@finite":
\begin{center}
	\begin{tabular}{rccc}
		$\HCOperator_{\!\?A,\?B}\colon$ & $\LatticeGuessFunctions{A}{B}$ & $\to$ & $\LatticeGuessFunctions{A}{B}$ \\
		& $F$ & $\mapsto$ & $\HCOperator(F)$,
	\end{tabular}
\end{center}
where for each $a \in A$, \AP$\intro*\HCOperator_{\!\?A,\?B}(F)(a)$ is the set of $b \in F(a)$ "st"
for every $\+R_{(k)} \in \sigma$, for every $i \in \lBrack 1,k\rBrack$,
if $\langle a_1,\, \hdots,\, a_{k-1} \rangle \in \neighbourhood{a}{\?A}{\+R}{i}$,
then there exists $b_1 \in F(a_1)$, $\hdots$, $b_{k-1} \in F(a_{k-1})$ "st" 
$\langle b_1,\, \hdots,\, b_{k-1} \rangle \in \neighbourhood{b}{\?B}{\+R}{i}$.\footnote{%
We write $\HCOperator$ for $\HCOperator_{\!\?A,\?B}$ when there is no ambiguity on
the "structures" involved.}

\begin{fact}
	The ordered set $\LatticeGuessFunctions{A}{B}$ is a "complete lattice",
	and moreover $\HCOperator$ is monotonic.
\end{fact}

As a consequence of the "Knaster-Tarski theorem", $\HCOperator$ admits a greatest fixpoint, that
we denote by \AP$\intro*\HCFixpoint{\?A}{\?B} \in \LatticeGuessFunctions{A}{B}$.\footnote{Recall 
that this greatest fixpoint can be obtained by ordinal induction by starting from
the greatest element of $\LatticeGuessFunctions{A}{B}$, namely the map $a \mapsto B$,
and iterating $\HCOperator$.}

\begin{proposition}
	If $f\colon \?A \to \?B$ is a "homomorphism" then $f(a) \in \HCFixpoint{\?A}{\?B}(a)$
	for each $a \in A$.
\end{proposition}

\begin{proof}
	The property ``$f(a) \in F(a)$ for each $a\in A$'' holds for the greatest element
	of $\LatticeGuessFunctions{A}{B}$, is stable under application of $\HCOperator$ and
	under arbitrary meets.\footnote{Meaning that if all $F_i$ ($i \in I$ for some arbitrary set $I$)
	satisfy the property, then so does $a \mapsto \bigcap_{i \in I} F(a)$}
	Hence, by ordinal induction, it holds for $\HCFixpoint{\?A}{\?B}$.
\end{proof}

\begin{corollary}
	If $\HCFixpoint{\?A}{\?B}(a) = \emptyset$ for some $a\in A$, then
	$\?A \nothomto \?B$.
\end{corollary}

In general, the converse property does not hold. For instance, if $\sigma$ is the
"signature of graphs" and $\?B$ is the "$2$-clique", then $\HCFixpoint{\?A}{\?B}$
is always the map $a \mapsto B$, no matter whether there is a "homomorphism" from
$\?A$ to $\?B$.

Yet, TODO managed to identify a necessary and sufficient condition on $\?B$ for
the "hyperedge consistency algorithm@@finute" to decide whether $\?A \in \HomFin{\?B}$.
\begin{definition}
	A finite "$\sigma$-structure" $\?B$ has ""tree duality"" when TODO.
\end{definition}

\begin{proposition}[TODO:addref]
	\AP\label{prop:hyperedge-consistency-tree-duality}
	If $\?B$ has "tree duality" then $\?A \homto \?B$ "iff"
	$\HCFixpoint{\?A}{\?B}(a) \neq \emptyset$ for all $a\in A$.
\end{proposition}

When $\?A$ is moreover finite, this immediately gives an algorithm to decide
$\?A \homto \?B$ since $\HCFixpoint{\?A}{\?B}$ can be computed not only
by an ordinal induction but with a finite induction.

\begin{corollary}
	If $\?B$ has "tree duality", then $\HomFin{\?B}$ can be solved in
	polynomial time.\sidenote{TODO: Precise complexity.}
\end{corollary}


\subsection{Hyperedge Consistency for Automatic Structures}

When $\?A$ is "automatic", \Cref{prop:hyperedge-consistency-tree-duality} still applies,
however it is not clear how to compute $\HCFixpoint{\?A}{\?B}$ since this element cannot
necessarily be obtained by finite induction, namely as
$\HCOperator^{\,n}(\topLatticeGuessFunctions{B})$ for some $n\in\N$, where
\AP$\intro*\topLatticeGuessFunctions{B}$ is the maximum element
of $\LatticeGuessFunctions{A}{B}$, namely the constant map $a \mapsto B$.
Another issue is to have a finite representation of the functions of
$\LatticeGuessFunctions{A}{B}$ since $A$ can be infinite. This last point is easy to address.

Given an "automatic presentation" $\•A$ of $\?A$, we extend
$\HCOperator_{\!\•A,\?B}$ and $\HCFixpoint{\•A}{\?B}$ to be defined on $\domainPres{\•A}$
instead of $\?A$.

\begin{lemma}[$\HCOperator$ preserves "regularity@@fun"]
	\AP\label{lem:hyperedge-consistency-preserves-regularity}
	Let $\?A$ be an arbitrary "$\sigma$-structure" and $\?B$ a finite "$\sigma$-structure".
	For any $F\in \LatticeGuessFunctions{\•A}{B}$, if $F$ is "regular@@hom",
	then $\HCOperator(F)$ is "regular@@hom".
\end{lemma}

\begin{proof}
	Let $F\in \LatticeGuessFunctions{\•A}{B}$ be "regular@@hom",
	so for each $Y \in \pset{B}$, $F^{-1}[Y]$ is "regular@@lang", and so by \Cref{prop:synchronous-first-order}, there exists a "first-order formula" $\phi_Y(x)$ over
	\(\signatureSynchronous{\Sigma}\)
	"st" $F^{-1}[Y] = \semFO{\phi_Y(x)}{\univStructSynchronous{\Sigma}}$.
	Also, since $\•A$ is an "automatic presentation", for any $\+R \in \sigma$ of arity $k$,
	there exists by \Cref{prop:synchronous-first-order} a "first-order formula" $\psi_{\+R}(x_1,\ldots,x_k)$ over \(\signatureSynchronous{\Sigma}\) "st"
	$\relPres{\•A}{\•R} = \semFO{\psi_{\+R}(x_1,\hdots,x_k)}{\univStructSynchronous{\Sigma}}$.
	Similarly, $\domainPres{\•A} = \semFO{\psi_{\textrm{dom}}(x)}{\univStructSynchronous{\Sigma}}$
	for some "formula@@FO" $\psi_{\textrm{dom}}(x)$.

	It is then easy to prove that $\HCOperator(F)$ is "regular@@hom" by providing a "first-order 
	formula" $\widehat \phi_Y(x)$ for each $Y \in \pset{B}$, describing $\HCOperator(F)^{-1}[Y]$,
	using both the formulas above, and the definition of $\HCOperator$.
	Indeed, recall that an element $u \in \domainPres{\•A}$ should be sent via $\HCOperator(F)$ 
	onto $Y\in \pset{B}$ if $Y$ is exactly the set of elements $b \in B$ "st" for every $\+R_{(k)} \in \sigma$, for every $i \in \lBrack 1,k\rBrack$, if
	$\langle a_1,\, \hdots,\, a_{k-1} \rangle \in \neighbourhood{a}{\?A}{\+R}{i}$,
	then there exists $b_1 \in F(a_1)$, $\hdots$, $b_{k-1} \in F(a_{k-1})$ "st" 
	$\langle b_1,\, \hdots,\, b_{k-1} \rangle \in \neighbourhood{b}{\?B}{\+R}{i}$.
	Symbolically the set of such $u$'s can be written as
	$\semFO{\widehat \phi_Y(x)}{\univStructSynchronous{\Sigma}}$, where
	\[
		\widehat \phi_Y(x) \defeq 
			\psi_{\textrm{dom}}(x) \land \big(\bigwedge_{b\in Y} \chi_b(x) \land \bigwedge_{b\not\in Y} \neg\chi_b(x)\big)
	\]
	where $\chi_b(x)$ is the formula\footnote{Notice that since
	$\chi_b$ appear both positively and negatively in $\widehat \phi_Y$, going from
	the $\phi$'s to the $\widehat \phi's$ increases the "quantifier alternation" of the
	"formulas@@FO" by one. And so the "formulas@@FO" we build to describe $\HCOperator^{\,n}(\topLatticeGuessFunctions{B})$ are of "quantifier alternation" $n$.
	TODO:CONCLUDE SOMEWHERE THAT CONSTRUCTION IS NOT ELEMENTARY.}
	\begin{align*}
		\chi_b(x) \defeq\; &
			\bigwedge_{\+R_{(k)} \in \sigma} \bigwedge_{i\in \lBrack 1,k\rBrack}\;
			\forall x_1.\, \hdots\, \forall x_{i-1}.\, \forall x_{i+1}.\, \hdots \, \forall x_{k}.\,\\
			& \hspace{2em}\psi_{\+R}(x_1,\hdots, x_{i-1}, x, x_{i+1}, \hdots, x_k)
			\\ 
			& \hspace{2em} \Rightarrow \Big(\bigvee_{\substack{\langle b_1,\hdots,b_{i-1},b_{i+1},\hdots b_k\rangle \in \neighbourhood{b}{\?B}{\+R}{i}}}\;
			\bigwedge_{i \in \lBrack 1,k\rBrack \smallsetminus \{i\}}
			\underbrace{%
				\bigvee_{\substack{Y' \in \pset{B}\\ b_i\in Y'}} \phi_{Y'}(x_i)
			}_{%
				b_i \in F(x_i)
			}%
			\Big).\qedhere
	\end{align*}
\end{proof}

Notice that $\topLatticeGuessFunctions{B}$ is trivially "regular@@fun", and so
by immediate induction, each $\HCOperator^{\,n}(\topLatticeGuessFunctions{B})$ with $n\in \N$ is
also "regular@@fun". While this opens the door to solving $\HomRegAut{\?B}$ when $\?B$ has
"tree duality" using the "hyperedge consistency algorithm@@finite", the problem
of finite convergence remains.

First, we show that finite iterations are enough to detect the absence of "homomorphism".
\begin{proposition}
	\AP\label{prop:hyperedge-consistency-no-hom}
	Let $\?A$ be an arbitrary "$\sigma$-structure" and $\?B$ a "finite $\sigma$-structure"
	with "tree duality". If $\?A \nothomto \?B$, then there exists $n\in\N$ and $a\in A$ "st"
	$\HCOperator^{\,n}(\topLatticeGuessFunctions{B})(a) = \emptyset$. 
\end{proposition}

In order to prove this proposition, we rely on the following property.
\begin{property}[Monotonicity of $\HCOperator$]
	\!\footnote{Observe in particular that this property can be applied if $\?A'$ is a
	substructure of $\?A$.}%
	\AP\label{prop:hyperedge-consistency-antimonotonicity}
	Let $\?A$, $\?A'$ be arbitrary "$\sigma$-structures" "st"
	there is a "homomorphism" $h\colon \?A' \to \?A$. Let $\?B$ a "finite $\sigma$-structure".
	For any $F\in \LatticeGuessFunctions{A}{B}$ and $F' \in \LatticeGuessFunctions{\?A'}{\?B}$,
	if $F(h(a)) \subseteq F'(a)$ for all $a\in A'$, then $\HCOperator(F)(h(a)) \subseteq \HCOperator(F')(a)$ for all $a\in A'$.\footnote{In other words, if $F \circ h \subsumed F'$, then
	$\HCOperator(F) \circ h \subsumed \HCOperator(F')$.}
\end{property}

\begin{proof}
	Assume that $\restr{F}{A'} \subsumed F'$, and let us show that 
	$\restr{\HCOperator(F)}{A'} \subsumed \HCOperator(F')$.
	Let $a \in A'$, and let $b \in \HCOperator(F)(h(a))$.
	Then for every $\+R_{(k)} \in \sigma$,
	for every $i \in \lBrack 1,k\rBrack$,
	if $\langle a_1,\, \hdots,\, a_{k-1} \rangle \in \neighbourhood{h(a)}{\?A}{\+R}{i}$,
	then there exists $b_1 \in F(a_1)$, $\hdots$, $b_{k-1} \in F(a_{k-1})$ "st" 
	$\langle b_1,\, \hdots,\, b_{k-1} \rangle \in \neighbourhood{b}{\?B}{\+R}{i}$.
	Then let $\+R_{(k)} \in \sigma$ and $i \in \lBrack 1,k\rBrack$.
	Let $\langle a_1,\, \hdots,\, a_{k-1} \rangle \in \neighbourhood{a}{\?A'}{\+R}{i}$.
	Then $\langle h(a_1),\, \hdots,\, h(a_{k-1}) \rangle \in \neighbourhood{h(a)}{\?A}{\+R}{i}$
	since $h$ is a "homomorphism". And so there exists
	$b_1 \in F(h(a_1))$, $\hdots$, $b_{k-1} \in F(h(a_{k-1}))$ "st" 
	$\langle b_1,\, \hdots,\, b_{k-1} \rangle \in \neighbourhood{b}{\?B}{\+R}{i}$.
	Using the inclusions $F(h(a_i)) \subseteq F'(a_i)$ (for $a_i \in A'$), it follows that
	for every $\+R_{(k)} \in \sigma$, for every $i \in \lBrack 1,k\rBrack$,
	if $\langle a_1,\, \hdots,\, a_{k-1} \rangle \in \neighbourhood{a}{\?A'}{\+R}{i}$, then
	there exists $b_1 \in F'(a_1)$, $\hdots$, $b_{k-1} \in F'(a_{k-1})$
	"st" $\langle b_1,\, \hdots,\, b_{k-1} \rangle \in \neighbourhood{b}{\?B}{\+R}{i}$.
	And hence $b \in \HCOperator(F')(a)$, which concludes the proof.
\end{proof}

\begin{proof}[Proof of \Cref{prop:hyperedge-consistency-no-hom}]
	Let $\?A$ be an arbitrary "$\sigma$-structure" and $\?B$ a "finite $\sigma$-structure"
	with "tree duality". Assume that $\?A \nothomto \?B$.
	Then by \Cref{prop:de-bruijn-erdos} there exists a finite "substructure" $\?A'$
	of $\?A$ "st" $\?A' \nothomto \?B$,
	and so by \Cref{prop:hyperedge-consistency-tree-duality},
	there exists some $a\in A$ "st" $\HCFixpoint{\?A'}{\?B}(a) = \emptyset$.
	But since $\?A'$ is finite, $\HCFixpoint{\?A'}{\?B} = \HCOperator^{\,n}(\topLatticeGuessFunctions{B})$ for some $n\in\N$. Then using \Cref{prop:hyperedge-consistency-antimonotonicity},
	\[
		\HCOperator_{\!\?A,\?B}^{\,n}(\topLatticeGuessFunctions{B})(a)
		\subseteq \HCOperator_{\!\?A',\?B}^{\,n}(\topLatticeGuessFunctions{B})(a)
		= \HCFixpoint{\?A'}{\?B}(a)
		= \emptyset.\qedhere
	\]
\end{proof}

So, if $\?A \nothomto \?B$, then "hyperedge consistency" will detect it in finite time assuming 
that $\?B$ has "tree duality".
We will then show the dual implication, under the stronger assumption that $\?B$ has "finite 
duality": the reason we need this stronger assumption is that while the "hyperedge consistency algorithm@@finite" converges for finite "$\sigma$-structures" when $\?B$ has "tree duality",
the number of iterations needed to reach the fixpoint depends on $\?A$. For "structures"
with "finite duality", we show that this is not the case---and hence, converge generalizes to 
infinite "$\sigma$-structures".

\begin{lemma}[Uniform Convergence of Hyperedge Consistency for Structures with Finite Duality]
	\AP\label{lem:hyperedge-consistency-uniform-convergence}
	Let $\?B$ be a finite "$\sigma$-structure" with "tree duality".
	The following are equivalent:
	\begin{enumerate}
		\itemAP\label{item:hc-uniform-finite-duality}
			$\?B$ has finite duality;
		\itemAP\label{item:hc-uniform-finite-structures}
			there exists $n\in\N$ "st" for every \emph{finite} "$\sigma$-structure" $\?A$, $\HCOperator^{\,n}_{\!\?A,\?B}(\topLatticeGuessFunctions{B}) = \HCFixpoint{\?A}{\?B}$;
		\itemAP\label{item:hc-uniform-arbitrary-structures}
			there exists $n\in\N$ "st" for every \emph{arbitrary} "$\sigma$-structure" $\?A$, $\HCOperator^{\,n}_{\!\?A,\?B}(\topLatticeGuessFunctions{B}) = \HCFixpoint{\?A}{\?B}$;
		\itemAP\label{item:hc-uniform-first-order}
			for each $Y \subseteq B$, the class of "pointed $\sigma$-structures"
			$\{\langle \?A, a \rangle \mid \HCFixpoint_{\?A,\?B}(\topLatticeGuessFunctions{B})(a) = Y\}$ is "first-order definable".
	\end{enumerate}
\end{lemma}

\begin{proof}
	\proofcase{\eqref{item:hc-uniform-finite-duality} $\Rightarrow$
		\eqref{item:hc-uniform-finite-structures}.}

	\proofcase{\eqref{item:hc-uniform-finite-structures} $\Rightarrow$
		\eqref{item:hc-uniform-arbitrary-structures}.}
	Let $n\in \N$ "st" $\HCOperator^{\,n}_{\!\?A',\?B}(\topLatticeGuessFunctions{B}) = \HCFixpoint{\?A'}{\?B}$ for every finite "$\sigma$-structure" $\?A'$. Let $\?A$ be an arbitrary
	"$\sigma$-structure". Note that for any $F\in \LatticeGuessFunctions{A}{B}$, for any 
	"substructure" $\?A'$ of $\?A$ containing $a$, then by \Cref{prop:hyperedge-consistency-antimonotonicity}
	$\HCOperator_{\?A,\?B}(F)(a) \subseteq \HCOperator_{\?A',\?B}(F)(a)$.
	We show that equality is reached by a particular finite "substructure".
	\begin{claim}
		\label{claim:hyperedge-consistency-ball}
		For any $F\in \LatticeGuessFunctions{A}{B}$ and $m\in \N$,
		there exists a finite "substructure"\footnote{Of course, if $\?A$ is "locally finite",
		we can always take $\?A_{a,m} = \ball{\?A}{a}{m}$.}
		$\?A_{a,m}$ of $\ball{\?A}{a}{m}$ "st"
		\[\HCOperator^{\,m}_{\?A,\?B}(F)(a) = \HCOperator^{\,m}_{\?A_{a,m},\?B}(F)(a).\]
	\end{claim}
	We give a proof sketch of this claim. Note that by definition,
	$\HCOperator_{\?A,\?B}(F)(a)$ only depends on the values of $F(a')$ where $a'$ is at distance 1.
	More precisely, it only depends on whether for any "relation symbol" $\+R$ of arity $k$,
	for any $i\in\lBrack 1,k\rBrack$, for any $Y_1,\hdots,Y_{k-1} \subseteq B$, whether there
	are elements $\langle a_1,\,\hdots,\,a_{k-1}\rangle \in \neighbourhood{a}{\?A}{\+R}{i}$
	"st"  
	$\langle F(a_1),\,\hdots,\,F(a_{k-1}) \rangle = \langle Y_1,\,\hdots,\,Y_{k-1} \rangle$.
	Since $B$ is finite, there are finitely many such tuples, and for each of them it suffices
	to keep (for distance $m=1$) only one tuple $\langle a_1,\,\hdots,\,a_{k-1}\rangle \in \neighbourhood{a}{\?A}{\+R}{i}$. By induction on $m$, we
	obtain a finite "substructure" $\?A_{a,m}$ of $\ball{\?A}{a}{m}$ as in
	\Cref{claim:hyperedge-consistency-ball}.

	We now show that $\HCOperator^{\,n}_{\!\?A,\?B}(\topLatticeGuessFunctions{B}) =
	\HCFixpoint{\?A}{\?B}$. By \Cref{claim:hyperedge-consistency-ball,prop:hyperedge-consistency-antimonotonicity}
	\[\HCOperator^{\,n}_{\!\?A,\?B}(\topLatticeGuessFunctions{B})(a) =
	\HCOperator^{\,n}_{\?A_{a,n+1},\?B}(\topLatticeGuessFunctions{B})(a).\]
	Since $\?A_{a,n+1}$ is finite, by \eqref{item:hc-uniform-finite-structures}, the right-hand side
	of the equality above equals $\HCFixpoint{\?A_{a,m}}{\?B}(a)$.
	But then again by \Cref{claim:hyperedge-consistency-ball} and \eqref{item:hc-uniform-finite-structures}, 
	\[\HCOperator^{n+1}_{\!\?A,\?B}(\topLatticeGuessFunctions{B})(a) = \HCFixpoint{\?A_{a,n+1}}{\?B}(a).\]
	And hence $\HCOperator^{\,n}_{\!\?A,\?B}(\topLatticeGuessFunctions{B})(a) = \HCOperator^{n+1}_{\!\?A,\?B}(\topLatticeGuessFunctions{B})(a)$. Since this property holds
	for arbitrary values of $a\in A$, it follows that $\HCOperator^{\,n}_{\!\?A,\?B}(\topLatticeGuessFunctions{B}) = \HCFixpoint{\?A}{\?B}$. 

	\proofcase{\eqref{item:hc-uniform-arbitrary-structures} $\Rightarrow$
		\eqref{item:hc-uniform-first-order}.}
	Observe that if $F\colon \in \LatticeGuessFunctions{A}{B}$ is first-order
	definable---in the sense that for each $X \subseteq B$, the class of "pointed structures"
	$\langle \?A, a\rangle$ "st" $F(a) = X$ is "first-order definable"---, then so is
	$\HCOperator(F)$. By immediate induction and \eqref{item:hc-uniform-arbitrary-structures},
	it follows that $\HCFixpoint{\?A}{\?B}$ is first-order definable in this sense.\footnote{Note 
	that this proof crucially uses the fact that $n$ does not depend on $\?A$.}

	\proofcase{\eqref{item:hc-uniform-first-order} $\Rightarrow$
	\eqref{item:hc-uniform-finite-duality}.}
	For $Y = \emptyset$, we get that $\{\langle \?A, a \rangle \mid \HCFixpoint_{\?A,\?B}(\topLatticeGuessFunctions{B})(a) = \emptyset\}$ is "first-order definable", say by a formula
	$\phi(x)$. Then $\forall x.\, \neg \phi(x)$ is a "first-order formula" describing the class of
	$\{\?A \mid \?A \homto \?B\}$ by \Cref{prop:hyperedge-consistency-tree-duality}, since
	$\?B$ has "tree duality".\footnote{Note that this implication requires the assumption that
	$\?B$ has "tree duality": indeed, if $\?B$ is the "2-clique", then \eqref{item:hc-uniform-first-order} holds as $\HCFixpoint_{\?A,\?B}(\topLatticeGuessFunctions{B})(a) = B$ for each $a\in \?A$ and each $\?A$. Yet, \eqref{item:hc-uniform-finite-duality} does not hold since $\?B$ does not even have "tree duality" by TODO:addref.} 
\end{proof}

% \begin{claim}
% 	If $\?A \nothomto \?B$, then there
% 	exists $a\in A$ "st" $\HCOperator^{\,n}(\topLatticeGuessFunctions{B})(a) = \emptyset$. 
% \end{claim}

% Indeed, let $\?D_1, \hdots, \?D_p$ be a finite set of "obstructions" for $\?B$.
% For each $i\in\lBrack 1,p\rBrack$, since $\?D_i \nothomto \?B$, there exists
% $n_i\in\N$ "st" $\HCOperator^{n_i}(\topLatticeGuessFunctions{B})(d_i) = \emptyset$ for some $d_i\in D_i$. We let $n \defeq \max_{i\in\lBrack 1,p\rBrack} n_i$.

% Then let $\?A$ be an arbitrary "$\sigma$-structure" such that $\?A \nothomto \?B$.
% Then $\?D_i \homto \?A$ for some $i \in \lBrack 1,p\rBrack$. 
% Let $h\colon \?D_i \to \?A$ be such a "homomorphism". 
% Then by \Cref{prop:hyperedge-consistency-no-hom}, 
% $\HCOperator^{n_i}_{\!\?A,\?B}(h(d_i)) \subseteq \HCOperator^{n_i}_{\!\?D_i,\?B}(d_i) = \emptyset$.
% In particular, $\HCOperator^{\,n}_{\!\?A,\?B}(\topLatticeGuessFunctions{B})(a)$ for some
% $a \in \?A$.

TODO: Is it true that $\{\?A \mid \?A \homto \?B\}$ is Gaifman-local iff
$\?B$ has "tree duality"?
