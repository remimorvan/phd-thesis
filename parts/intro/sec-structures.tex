\section{The Two Sides of the Homomorphism Problem}

This thesis is devoted to studying variations of the "homomorphism problem".
This problem takes two "finite (directed) graphs", consisting of a finite set of vertices $V$
together with a set of "edges" $\+E \subseteq V \times V$, and asks if there is a "homomorphism" 
between them, which consists of a function between vertices that preserves
the "edges", in the sense that any edge must be mapped to another edge,
see \Cref{fig:example-graph-homomorphism}.

\begin{figure}
	\centering
	\begin{tikzpicture}
		\foreach \i in {0,1,...,5}{
			\node[vertex] (\i) at (\i*360/6: 1.2cm) {};
		}

		\draw[edge] (1) to (0);
		\draw[edge] (2) to (1);
		\draw[edge] (3) to (2);
		\draw[edge] (3) to (4);
		\draw[edge] (5) to (4);
		\draw[edge] (5) to (0);

		\node[vertex] at (4.33,.66) (a) {};
		\node[vertex] at (5.66,.66) (b) {};
		\node[vertex] at (5.66,-.66) (c) {};
		\node[vertex] at (4.33,-.66) (d) {};
		\node[vertex] at (6.8,0) (e) {};

		\draw[edge] (d) to (a);
		\draw[edge] (a) to (b);
		\draw[edge] (b) to (c);
		\draw[edge] (d) to (c);
		\draw[edge] (b) to (e);
		\draw[edge] (e) to (c);

		\draw[edge, draw=cBlue, dotted, out=-20, in=160] (3) to (d);
		\draw[edge, draw=cBlue, dotted, out=-25, in=170] (2) to (a);
		\draw[edge, draw=cBlue, dotted, out=5, in=130] (1) to (b);
		\draw[edge, draw=cBlue, dotted, out=10, in=145] (0) to (c);
		\draw[edge, draw=cBlue, dotted, out=-20, in=195] (5) to (d);
		\draw[edge, draw=cBlue, dotted, out=-30, in=210] (4) to (c);
	\end{tikzpicture}
	\caption{
		\AP\label{fig:example-graph-homomorphism}
		Two graphs (in black), and a homomorphism (blue dotted arrows) from the
		graph on the left-hand side to one right one.
	}
\end{figure}

To enrich the structure---but more importantly to add a splash of colour to this thesis---,
we will in fact consider more complex structures: 
we allow for multiple edge relations, or even relations of higher arity linking the vertices.
Which kind of relations (and how many of which kind) is allowed is known as the "signature" of
the "structure". 
These richer structures are known as "relational structures"---see
\Cref{fig:example-structure-homomorphism}---, and
"homomorphisms" between "relational structures" are asked to preserve \emph{all} relations.

\begin{figure}
	\centering
	\begin{tikzpicture}[scale=.9]
		\node (d0) at (-.6,.6) {};
		\node (d1) at (-.8,0) {};
		\node (d2) at (-.4,-1.2) {};
		\node (d3) at (-1.5,0) {};
		\node (d4) at (-0.3,1.3) {};

		\draw[use Hobby shortcut, closed=true, draw=c2, fill=c2, fill opacity=.4] 
			(d0) .. (d1) .. (d2) .. (d3) .. (d4) ;

		\node (c0) at (-.2,1) {};
		\node (c1) at (-0.5,1.5) {};
		\node (c2) at (-1,2.4) {};
		\node (c3) at (-2.7,2.2) {};
		\node (c4) at (-2.2,1.5) {};
		\node (c5) at (-1.2,1.5) {};
		\node (c6) at (-.6,.6) {};

		\draw[use Hobby shortcut, closed=true, draw=c1, fill=c1, fill opacity=.4] 
			(c0) .. (c1) .. (c2) .. (c3) .. (c4) .. (c5) .. (c6);

		\node (e0) at (1.5,0) {};
		\node (e1) at (1,.3) {};
		\node (e2) at (.95,-.3) {};
	
		\draw[use Hobby shortcut, closed=true, draw=c0, fill=c0, fill opacity=.4] 
			(e0) .. (e1) .. (e2) ;

		% grid
		% \foreach \x in {-4,-3,...,4} {
		% 	\draw[-,draw=cLightGrey] (\x, -4) to (\x, 4);
		% 	\draw[-,draw=cLightGrey] (-4, \x) to (4, \x);
		% }
		% \draw[-,draw=cGrey] (0, -4) to (0, 4);
		% \draw[-,draw=cGrey] (-4, 0) to (4, 0);

		\foreach \i in {0,1,...,5} {
			\node[vertex] (\i) at (\i*360/6: 1.2cm) {};
		}
		\foreach \i in {7,8,9,10} {
			\node[vertex] (\i) at ($(\i*360/6: 1.2cm)+(-1.8,1.04)$) {};
		}

		\draw[edge, bend left] (1) to (0);
		\draw[edge, double, bend right] (1) to (0);
		\draw[edge, <->] (2) to (1);
		\draw[edge, double] (3) to (2);
		\draw[edge] (3) to (4);
		\draw[edge, double] (5) to (4);
		\draw[edge, double] (5) to (0);
		\draw[edge, double] (2) to (7);
		\draw[edge, bend left] (7) to (8);
		\draw[edge, double, bend left] (8) to (7);
		\draw[edge] (8) to (9);
		\draw[edge, double] (10) to (9);
		\draw[edge, bend left] (10) to (3);
		\draw[edge, double, <->, bend left] (3) to (10);
		\draw[edge, loop, out=-90, in=0, looseness=8] (5) to (5);
	\end{tikzpicture}
	\caption{
		\AP\label{fig:example-structure-homomorphism}
		A "relational structure" with two kinds of binary relations (represented by
		simple and double arrows) and three kinds of unary relations (represented by
		red, yellow and blue potatoes).
	}
\end{figure}

More than a mere decision problem---which obviously lives in "NP"---,
the "homomorphism problem" should rather be seen as a \emph{framework} or
\emph{language} to formalize many common problems in computer science.

\begin{marginfigure}
	\centering
	\begin{tikzpicture}
		\node[vertex] (0) at (0,0) {};
		\node[vertex, right=of 0] (1) {};
		\node[vertex, right=of 1] (2) {};
		\node[vertex, right=of 2] (3) {};

		\draw[edge] (0) to node[above] {$a$} (1);
		\draw[edge] (1) to node[above] {$b$} (2);
		\draw[edge] (2) to node[above] {$a$} (3);

		\draw[edge] ($(0)+(-.4,0)$) to (0);
		\node[vertex, inner sep=3pt] (3b) at (3) {};

		\node[vertex, below=5em of 1] (q0) {};
		\node[vertex, right=of q0] (q1) {};
		
		\draw[edge] ($(q0)+(-.4,0)$) to (q0);
		\node[vertex, inner sep=3pt] (q1b) at (q1) {};

		\draw[edge, loop, out=-60, in=-120, looseness=8] (q0) to node[below] {$a,b$} (q0);
		\draw[edge] (q0) to node[above] {$a$} (q1);
		\draw[edge, loop, out=-60, in=-120, looseness=8] (q1) to node[below] {$a,b$} (q1);
		
		\draw[edge, draw=cBlue, dotted, out=-60, in=130] (0) to (q0);
		\draw[edge, draw=cBlue, dotted, out=-100, in=100] (1) to (q0);
		\draw[edge, draw=cBlue, dotted, out=-90, in=80] (2) to (q0);
		\draw[edge, draw=cBlue, dotted, out=-90, in=90] (3) to (q1);
	\end{tikzpicture}
	\caption{
		\AP\label{fig:example-automaton-as-rel}
		Automata acceptance as a "homomorphism problem":
		"relational structure" representing the finite word $aba$ (above),
		"structure" representing the minimal automaton of $(a+b)^* a (a+b)^*$
		(below) and a "homomorphism" from the former to the latter (blue dotted arrows).
		Vertices with a double circle (resp. incoming dangling arrow) represent
		final (resp. initial) states.
	}
\end{marginfigure}

\begin{example}[Non-deterministic automata]
	A (non-deterministic) automaton $\?A$ can be seen as a "relational structure",
	whose nodes are its states, on the "signature" with two unary predicates (one for
	describing initial states, one for final states), and one binary predicate
	for each letter of the alphabet $\Sigma$.
	Any finite word $u\in \Sigma^*$ can in turn by seen as a "relational structure"
	$\?W_u$ with $\lBrack 0,|u|\rBrack$ as its domain,
	where $0$ is initial, $|u|$ is final, and
	for each $i \in \lBrack 1,|u|\rBrack$, there is an edge from $i-1$ to $i$
	whose type is given by the $i$-th letter of $u$.
	Then, there is a "homomorphism" from $\?W_u$ to $\?A$ if, and only if, 
	the automaton $\?A$ accepts $u$. See \Cref{fig:example-automaton-as-rel}.
\end{example}

\begin{marginfigure}
	\centering
	\begin{tikzpicture}
		% ---
% 3-clique
% ---
\node[vertex, draw=c0, fill=c0, fill opacity=.4] at (0,0) (k3-0) {};
\node[vertex, above right=1em and 2em of k3-0, draw=c1, fill=c1, fill opacity=.4] (k3-1) {};
\node[vertex, below right=1em and 2em of k3-0, draw=c2, fill=c2, fill opacity=.4] (k3-2) {};

\draw[edge, bend right=15] (k3-0) to (k3-1);
\draw[edge, bend right=15] (k3-0) to (k3-2);
\draw[edge, bend right=15] (k3-1) to (k3-0);
\draw[edge, bend right=15] (k3-1) to (k3-2);
\draw[edge, bend right=15] (k3-2) to (k3-0);
\draw[edge, bend right=15] (k3-2) to (k3-1);
	\end{tikzpicture}
	\caption{
		\AP\label{fig:intro-3-clique}
		The "$3$-clique" $\clique{3}$.
	}
\end{marginfigure}
\begin{example}[Graph colouring]
	\label{ex:graph-colouring-as-hom}
	Let $k\in\Np$. We let the \AP""$k$-clique"", denoted by $\intro*\clique{k}$,
	to be the "graph" whose vertices are $\lBrack 1,k\rBrack$,
	and with an edge from $i$ to $j$ (with $i,j \in \lBrack 1,k\rBrack$)
	"iff" $i\neq j$, see \Cref{fig:intro-3-clique}.
	A finite "graph" $\?G$ is \AP""$k$-colourable""---"ie"
	we can map every vertex of $\?G$ to an element of $\lBrack 1,k\rBrack$
	"st" no two adjacent vertices are sent on the same colour/number---if,
	and only if, there is a "homomorphism" from $\?G$ to $\clique{k}$.
\end{example}

% \begin{figure}
	% \begin{tabular}{cccc}
	% 	\multicolumn{4}{c}{\textsc{Movies}} \toprule
	% 	Id & Title & Length (min) & Director \\ \midrule
	% 	197 & Eyes Wide Shut & 159 & Stanley Kubrick \\ 
	% 	205 & J'ai tué ma mère & 96 & Xavier Dolan \\
	% 	304 & Amadeus & 161 & Miloš Forman \\
	% 	321 & 120 Battements par minute & 143 & Robin Campillo \\ \bottomrule
	% \end{tabular}
	% \begin{tabular}{cc}
	% 	\multicolumn{2}{c}{\textsc{Rooms}} \toprule
	% 	Id & Capacity  \\ \midrule
	% 	1 & 
	% 	4 &  \\ \bottomrule
	% \end{tabular}
% 	\caption{A "relational databases" of a cinema consisting of two tables.}
% \end{figure}
\begin{example}[Database evaluation]
	A "relational database" such as the one depicted on\\
	\begin{tabular}{cccc}
		\multicolumn{4}{c}{\textsc{Movies}} \\ \toprule
		Id & Title & Length (min) & Director \\ \midrule
		197 & Eyes Wide Shut & 159 & Stanley Kubrick \\ 
		205 & J'ai tué ma mère & 96 & Xavier Dolan \\
		304 & Amadeus & 161 & Miloš Forman \\
		321 & 120 Battements par minute & 143 & Robin Campillo \\ \bottomrule
	\end{tabular}
\end{example}

\begin{marginfigure}
	\centering
	\begin{tikzpicture}[scale=1.2]
		\draw[step=.3333] (0, 0) grid (3, 3);
		\draw[very thick, step=1] (0, 0) grid (3, 3);

		\setcounter{row}{1}
		\setrow { }{2}{ }  {5}{ }{1}  { }{9}{ }
		\setrow {8}{ }{ }  {2}{ }{3}  { }{ }{6}
		\setrow { }{3}{ }  { }{6}{ }  { }{7}{ }

		\setrow { }{ }{1}  { }{ }{ }  {6}{ }{ }
		\setrow {5}{4}{ }  { }{ }{ }  { }{1}{9}
		\setrow { }{ }{2}  { }{ }{ }  {7}{ }{ }

		\setrow { }{9}{ }  { }{3}{ }  { }{8}{ }
		\setrow {2}{ }{ }  {8}{ }{4}  { }{ }{7}
		\setrow { }{1}{ }  {9}{ }{7}  { }{6}{ }
	\end{tikzpicture}
	\caption{
		\AP\label{fig:intro-sudoku}
		A prefilled sudoku grid.
	}
\end{marginfigure}
\begin{example}[Sudoku grids]
	We represent an empty sudoku grid as the "relational structure"
	whose domain is $\lBrack 1,9\rBrack \times \lBrack 1,9\rBrack$---corresponding to coordinates in the grid---with
	three kinds of binary "predicates": $\+R$, $\+C$ and $\+S$,
	that describe when two coordinates are on the same row, column or 3x3-square, respectively.
	For non-empty grids, we add nine unary "predicates" $P_k$ ($k \in \lBrack 1,9\rBrack$),
	and if coordinate $\tup{i,j}$ if prefilled with number $k$, then predicate $P_k$ must
	hold on the element $\tup{i,j}$.
	Given a prefilled grid $G$, we denote by $\?S_G$ the associated "relational structure".
	
	We then define the target structure $\?T$ to have $\lBrack 1,9\rBrack$ as its domain:
	with the binary relations $\+R$, $\+C$ and $\+S$ which all three consist of
	all tuples $\tup{\tup{i,j},\tup{i',j'}}$ "st" $\tup{i,j} \neq \tup{i',j'}$.
	Moreover, each unary relation $P_k$ ($k \in \lBrack 1,9\rBrack$) is defined
	to hold only on $\{i\}$.
	Then, a prefilled grid $G$ can be completed if, and only if,
	$\?S_G$ has a "homomorphism" to $\?T$.%
	\footnote{In fact, for this example we could use only one binary predicate instead of three.
	Note that this encoding is actually quite close
	to graph colouring (\Cref{ex:graph-colouring-as-hom}) with an extra trick
	to force some values. This trick---formally called "marked structure"---will
	actually prove crucial in \Cref{ch:dichotomy-theorem}.}
\end{example}

\begin{example}[SAT solving]
	\todo{todo}
\end{example}

\todo{homomorphism (not only existence of a homomorphism) have some meaning}
\todo{existential constraint vs universal constraint}