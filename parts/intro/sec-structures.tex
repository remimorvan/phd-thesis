\section{The Two Sides of the Homomorphism Problem}

This thesis is devoted to studying variations of the "homomorphism problem",
a central concept in theoretical computer science.
Homomorphisms capture the idea of maps preserving relational information between mathematical structures.
They arise naturally across a variety of domains: in logic, as the mechanism behind model-checking of "existential-positive formulas"; in database theory, as the semantics of "conjunctive query" evaluation; and in "constraint satisfaction", as the formalization of whether a structure
satisfies a given set of rules. This thesis approaches the "homomorphism problem" as a unifying framework, by emphasizing its complementary role in data querying and constraint solving.

To introduce this problem, we present a range of examples: automata acceptance, SQL query evaluation, and Sudoku solving that can all be encoded as "homomorphism problems".%
\footnote{We will see even further examples---3-colourability and SAT-solving---in the next sections.}
Interestingly, they reveal a dichotomy on how problems are encoded:
in some, the query appears on the left-hand side of the "homomorphism", 
while on others it appears on the right-hand side.
The first type of encoding correspond to what we call \emph{existential problems},
and are rooted in the field of \emph{database theory}:
we introduce this domain in \Cref{sec:intro-existential}.
On the other hand, the dual encoding deals with \emph{universal problems}
studied in the domain of \emph{constraint satisfaction}:
it is the focus of \Cref{sec:intro-universal}.
This dual perspective on the "homomorphism problem"
forms the basis for the two-part division of this thesis.

\paragraph*{Homomorphisms.}
The simplest mathematical structure is perhaps that of a (directed) finite graph:
it consists of a finite set $V$ of vertices (also called \emph{domain}),
together with a set of "edges" $\+E \subseteq V \times V$.
A "homomorphism" between two graphs is then a
function $f$ between vertices that preserves
the "edges", in the sense that if $\tup{u,v}$ is an edge, then
$\tup{f(u),f(v)}$ must also be an edge: we depict
an example of graph "homomorphism" in \Cref{fig:example-graph-homomorphism}.

\begin{figure}
	\centering
	\begin{tikzpicture}
		\foreach \i in {0,1,...,5}{
			\node[vertex] (\i) at (\i*360/6: 1.2cm) {};
		}

		\draw[edge] (1) to (0);
		\draw[edge] (2) to (1);
		\draw[edge] (3) to (2);
		\draw[edge] (3) to (4);
		\draw[edge] (5) to (4);
		\draw[edge] (5) to (0);

		\node[vertex] at (4.33,.66) (a) {};
		\node[vertex] at (5.66,.66) (b) {};
		\node[vertex] at (5.66,-.66) (c) {};
		\node[vertex] at (4.33,-.66) (d) {};
		\node[vertex] at (6.8,0) (e) {};

		\draw[edge] (d) to (a);
		\draw[edge] (a) to (b);
		\draw[edge] (b) to (c);
		\draw[edge] (d) to (c);
		\draw[edge] (b) to (e);
		\draw[edge] (e) to (c);

		\draw[edge, draw=cBlue, dotted, out=-20, in=160] (3) to (d);
		\draw[edge, draw=cBlue, dotted, out=-25, in=170] (2) to (a);
		\draw[edge, draw=cBlue, dotted, out=5, in=130] (1) to (b);
		\draw[edge, draw=cBlue, dotted, out=10, in=145] (0) to (c);
		\draw[edge, draw=cBlue, dotted, out=-20, in=195] (5) to (d);
		\draw[edge, draw=cBlue, dotted, out=-30, in=210] (4) to (c);
	\end{tikzpicture}
	\caption{
		\AP\label{fig:example-graph-homomorphism}
		Two graphs (in black) and a homomorphism (blue dotted arrows) from the
		graph on the left-hand side to the right one.
	}
\end{figure}

\begin{figure}
	\centering
	\begin{tikzpicture}[scale=.9]
		% \node (d0) at (-.6,.6) {};
		% \node (d1) at (-.8,0) {};
		% \node (d2) at (-1.15,-.3) {};
		% \node (d3) at (-1.4,.4) {};
		% \node (d4) at (-0.4,1.3) {};

		% \draw[use Hobby shortcut, closed=true, draw=c2, fill=c2, fill opacity=.4] 
		% 	(d0) .. (d1) .. (d2) .. (d3) .. (d4) ;

		% \node (f0) at (0.97,-1) {};
		% \node (f1) at (0.51,-.7) {};
		% \node (f2) at (.42,-1.3) {};
	
		% \draw[use Hobby shortcut, closed=true, draw=c2, fill=c2, fill opacity=.4] 
		% 	(f0) .. (f1) .. (f2) ;

		% \node (c0) at (-.2,1) {};
		% \node (c1) at (-0.5,1.5) {};
		% \node (c2) at (-1,2.4) {};
		% \node (c3) at (-2.7,2.2) {};
		% \node (c4) at (-2.2,1.5) {};
		% \node (c5) at (-1.2,1.5) {};
		% \node (c6) at (-.6,.6) {};

		% \draw[use Hobby shortcut, closed=true, draw=c1, fill=c1, fill opacity=.4] 
		% 	(c0) .. (c1) .. (c2) .. (c3) .. (c4) .. (c5) .. (c6);

		% \node (e0) at (1.5,0) {};
		% \node (e1) at (1,.3) {};
		% \node (e2) at (.95,-.3) {};
	
		% \draw[use Hobby shortcut, closed=true, draw=c0, fill=c0, fill opacity=.4] 
		% 	(e0) .. (e1) .. (e2) ;

		% grid
		% \foreach \x in {-4,-3,...,4} {
		% 	\draw[-,draw=cLightGrey] (\x, -4) to (\x, 4);
		% 	\draw[-,draw=cLightGrey] (-4, \x) to (4, \x);
		% }
		% \draw[-,draw=cGrey] (0, -4) to (0, 4);
		% \draw[-,draw=cGrey] (-4, 0) to (4, 0);

		\foreach \i in {0,1,...,5} {
			\node[vertex] (\i) at (\i*360/6: 1.2cm) {};
		}
		\foreach \i in {7,8,9,10} {
			\node[vertex] (\i) at ($(\i*360/6: 1.2cm)+(-1.8,1.04)$) {};
		}

		\node[tiny vertex, draw=cBlue, fill=cBlue, fill opacity=.4, left=-0em of 2] {};
		\node[tiny vertex, draw=cYellow, fill=cYellow, fill opacity=.4, left=.5em of 2] {};
		\node[tiny vertex, draw=cYellow, fill=cYellow, fill opacity=.4, right=0em of 3] {};
		\node[tiny vertex, draw=cYellow, fill=cYellow, fill opacity=.4, above=0em of 7] {};
		\node[tiny vertex, draw=cBlue, fill=cBlue, fill opacity=.4, left=0em of 9] {};
		\node[tiny vertex, draw=cRed, fill=cRed, fill opacity=.4, right=0em of 0] {};

		\draw[edge, bend left] (1) to (0);
		\draw[edge, double, bend right] (1) to (0);
		\draw[edge, <->] (2) to (1);
		\draw[edge, double] (3) to (2);
		\draw[edge] (3) to (4);
		\draw[edge, double] (5) to (4);
		\draw[edge, double] (5) to (0);
		\draw[edge, double] (2) to (7);
		\draw[edge, bend left] (7) to (8);
		\draw[edge, double, bend left] (8) to (7);
		\draw[edge] (8) to (9);
		\draw[edge, double] (10) to (9);
		\draw[edge, bend left] (10) to (3);
		\draw[edge, double, <->, bend left] (3) to (10);
		\draw[edge, loop, out=-90, in=0, looseness=8] (5) to (5);
	\end{tikzpicture}
	\caption{
		\AP\label{fig:example-structure-homomorphism}
		A "relational structure" with two kinds of binary relations (represented by
		simple and double arrows) and three kinds of unary relations (represented by
		small red, yellow and blue circles next to the vertices).
	}
\end{figure}
To enrich the structure---but also, perhaps more enjoyably,
to add a splash of colour to this thesis---we will consider more complex structures
by allowing for multiple edge relations, or even relations of higher arity linking the vertices.
The types and arities of relations allowed in a structure are specified by its "signature" $\sigma$.
These richer structures are known as "$\sigma$-structures" or "relational structures"---see
\Cref{fig:example-structure-homomorphism}---, and
"homomorphisms" between "$\sigma$-structures" are asked to preserve \emph{all} relations
in the "signature" $\sigma$.

\decisionproblem{The ""Homomorphism Problem"" over $\sigma$}{
	Two finite $\sigma$-structures $\?A$ and $\?B$.
}{
	Is there a "homomorphism" from $\?A$ to $\?B$?
}

In the problem above, we refer to $\?A$ as the \AP""source structure""
and to $\?B$ as the ""target structure"", and we denote by \AP$\?A \intro*\homto \?B$
the existence of a "homomorphism" from $\?A$ to $\?B$.

More than a mere decision problem---which is easily seen to lie in "NP"---,
the "homomorphism problem" should rather be understood as a \emph{framework} or
\emph{language} to formalize many problems arising in computer science.

\begin{marginfigure}
	\centering
	\begin{tikzpicture}
		\node[vertex] (0) at (0,0) {};
		\node[vertex, right=of 0] (1) {};
		\node[vertex, right=of 1] (2) {};
		\node[vertex, right=of 2] (3) {};
		\node[vertex, inner sep=3.5pt] (3b) at (3) {};

		\draw[edge] (0) to node[above] {$a$} (1);
		\draw[edge] (1) to node[above] {$b$} (2);
		\draw[edge] (2) to node[above] {$a$} (3b);

		\draw[edge] ($(0)+(-.4,0)$) to (0);

		\node[vertex, below=5em of 1] (q0) {};
		\node[vertex, right=of q0] (q1) {};
		\node[vertex, inner sep=3.5pt] (q1b) at (q1) {};
		
		\draw[edge] ($(q0)+(-.4,0)$) to (q0);

		\draw[edge, loop, out=-60, in=-120, looseness=8] (q0) to node[below] {$a,b$} (q0);
		\draw[edge] (q0) to node[above] {$a$} (q1b);
		\draw[edge, loop, out=-60, in=-120, looseness=8] (q1b) to node[below] {$a,b$} (q1b);
		
		\draw[edge, draw=cBlue, dotted, out=-60, in=130] (0) to (q0);
		\draw[edge, draw=cBlue, dotted, out=-100, in=100] (1) to (q0);
		\draw[edge, draw=cBlue, dotted, out=-90, in=80] (2) to (q0);
		\draw[edge, draw=cBlue, dotted, out=-90, in=90] (3b) to (q1b);
	\end{tikzpicture}
	\caption{
		\AP\label{fig:example-automaton-as-rel}
		Automata acceptance as a "homomorphism problem":
		"structure" representing the finite word $aba$ (above),
		"structure" representing the minimal automaton of $(a+b)^* a (a+b)^*$
		(below) and a "homomorphism" from the former to the latter (blue dotted arrows).
		Vertices with a double circle (resp. incoming dangling arrow) represent
		final (resp. initial) states.
	}
\end{marginfigure}

\begin{example}[Non-deterministic automata]
	\AP\label{ex:auto-as-rel}
	A non-deterministic automaton $\?A$ can be seen as a "relational structure"
	on the "signature" with two unary predicates (one for
	describing initial states, one for final states), and one binary predicate
	for each letter of the alphabet $\Sigma$ describing the transitions.
	As expected, its vertices are its states, 
	and each predicate is naturally interpreted.
	Any finite word $u\in \Sigma^*$ can in turn be seen as a "relational structure"
	$\?W_u$ with $\intInt{0,|u|}$ as its domain,
	where $0$ is initial, $|u|$ is final, and
	for each $i \in \intInt{1,|u|}$, there is an edge from $i-1$ to $i$
	whose type is given by the $i$-th letter of $u$, see \Cref{fig:example-automaton-as-rel}..
	Then, there is a "homomorphism" from $\?W_u$ to $\?A$ if, and only if, 
	the automaton $\?A$ accepts $u$.
\end{example}

Note that in \Cref{ex:auto-as-rel},
not only is the existence of a homomorphism equivalent to the existence of a solution, but the set 
of homomorphisms naturally corresponds to the set of solutions: "homomorphisms" from $\?W_u$
to $\?A$ exactly correspond to accepting runs of the automaton over $u$.

\begin{table}
	\centering%
	{%
	\footnotesize%
	\begin{tabular}{cccc}
		\multicolumn{4}{c}{\textsc{Movies}} \\ \toprule
		id & title & length & director \\ \midrule
		197 & Eyes Wide Shut & 159 & Stanley Kubrick \\ 
		205 & J'ai tué ma mère & 96 & Xavier Dolan \\
		304 & Amadeus & 161 & Miloš Forman \\
		321 & 120 Battements par minute & 143 & Robin Campillo \\ \bottomrule
	\end{tabular}
	\\\bigskip%
	\begin{tabular}{cc}
		\multicolumn{2}{c}{\textsc{Rooms}} \\ \toprule
		id & capacity  \\ \midrule
		1 & 108 \\
		2 & 124 \\
		3 & 96 \\
		4 & 102 \\ \bottomrule
	\end{tabular}%
	\hspace{1cm}%
	\begin{tabular}{ccc}
		\multicolumn{3}{c}{\textsc{Projections}} \\ \toprule
		movie\_id & room\_id & time \\ \midrule
		197 & 2 & 2025-03-28 14:00 \\
		205 & 3 & 2025-03-28 14:30 \\
		321 & 4 & 2025-03-28 14:30 \\
		197 & 1 & 2025-03-28 17:00 \\ \bottomrule
	\end{tabular}
}
	\caption{
		\AP\label{fig:example-db-as-rel}
		A "relational database" consisting of three tables, representing data
		stored by a cinema.
	}
\end{table}
\begin{marginfigure}[-12em]
	\centering
	\begin{tikzpicture}
		\node (a0) at (-.1,5.2) {};
		\node (a1) at (-.4,3.5) {};
		\node (a2) at (-.1,1.8) {};
		\node (a3) at (.1,1.8) {};
		\node (a4) at (.4,3.5) {};
		\node (a5) at (.1,5.2) {};

		\node (b0) at (-.2,5.2) {};
		\node (b1) at (-.8,3.1) {};
		\node (b2) at (-.5,1.5) {};
		\node (b3) at (-.4,.6) {};
		\node (b4) at (-.2,-.2) {};
		\node (b5) at (.2,-.2) {};
		\node (b6) at (.4,1) {};
		\node (b7) at (.2,1.4) {};
		\node (b8) at (-.5,2.6) {};
		\node (b9) at (-.4,3.8) {};
		\node (b10) at (.2,5.2) {};
	
		\draw[use Hobby shortcut, closed=true, draw=c2, fill=c2, fill opacity=.4] 
			(b0) .. (b1) .. (b2) .. (b3) .. (b4) .. (b5) .. (b6) .. (b7) .. (b8) .. (b9) .. (b10);
		\draw[use Hobby shortcut, closed=true, draw=c1, fill=c1, fill opacity=.4] 
			(a0) .. (a1) .. (a2) .. (a3) .. (a4) .. (a5);
		
		\node[vertex] at (0,5) (5) {};
		\node[vertex] at (0,4) (4) {};
		\node[vertex] at (0,3) (3) {};
		\node[vertex] at (0,2) (2) {};
		\node[vertex] at (0,1) (1) {};
		\node[vertex] at (0,0) (0) {};
		\node[font=\tiny, right=2em of 5] {\textsf{movie\_id}};
		\node[font=\tiny, right=2em of 4] {\textsf{title}};
		\node[font=\tiny, right=2em of 3] {\textsf{length}};
		\node[font=\tiny, right=2em of 2] {\textsf{director}};
		\node[font=\tiny, right=2em of 1] {\textsf{room\_id}};
		\node[font=\tiny, right=2em of 0] {\textsf{time}};
	\end{tikzpicture}
	\caption{
		\AP\label{fig:example-sql-as-rel}
		A SQL query seen as a "relational structure".
		The yellow potato represents the single tuple
		of the \textsc{Movies} relation, and the blue
		potato surrounds the only tuple that belongs to the
		\textsc{Projections} relation.
	}
\end{marginfigure}
\begin{example}[SQL queries]
	\AP\label{ex:sql-as-hom}
	A "relational database", such as the one depicted on
	\Cref{fig:example-db-as-rel}, can easily be seen as a "relational structure"
	whose domain is the set of elements occurring somewhere in a table,
	with one relation for each table.%
	\marginnote{~}%hand-made spacing!
	\footnote{In fact the only difference between "relational databases" and "relational structures"
	precisely lies in the fact that in the case of the former, the domain is implicit,
	while for the latter it is explicit.
	While this difference alters the theory, the difference is mostly negligible
	for the "query@@sem" languages we will study, see \Cref{ch:prelim-graph-databases}.}

	Consider the following SQL query, which outputs all pairs
	of movie titles together with their projection time.
	\lstinputlisting[language=mysql]{fig/intro/cinema.sql}

	This query $\gamma$ can in fact be seen itself as a "relational structure" $\?Q_\gamma$:
	its domain has six elements, corresponding to the attributes of
	the \textsc{Movies} and \textsc{Projections} table, merged on the attribute
	$\textsf{Projections.movie\_id} = \textsf{Movies.id}$.
	Both relations $\textsc{Movies}$ and $\textsc{Projections}$ consist of a single tuple,
	and the relation $\textsc{Rooms}$ is empty, as depicted in \Cref{fig:example-sql-as-rel}.

	Then, the set of pairs $\tup{x, y}$
	such that there is a "homomorphism" from $\?Q_\gamma$ to the relational database,
	sending $\textsf{title}$ to $x$ and $\textsf{time}$ to $y$
	is exactly the output set of the SQL query on the database.
\end{example}

We now provide a last example: Sudoku grids. While it is also encoded
as a "homomorphism problem", we will see in fact it
is of a different nature than the reductions of \Cref{ex:auto-as-rel,ex:sql-as-hom}.

\begin{marginfigure}[2em]
	\centering%
	\begin{tikzpicture}[scale=1.3]
		\draw[step=.3333] (0, 0) grid (3, 3);
		\draw[very thick, step=1] (0, 0) grid (3, 3);

		\setcounter{row}{1}
		\setrow { }{2}{ }  {5}{ }{1}  { }{9}{ }
		\setrow {8}{ }{ }  {2}{ }{3}  { }{ }{6}
		\setrow { }{3}{ }  { }{6}{ }  { }{7}{ }

		\setrow { }{ }{1}  { }{ }{ }  {6}{ }{ }
		\setrow {5}{4}{ }  { }{ }{ }  { }{1}{9}
		\setrow { }{ }{2}  { }{ }{ }  {7}{ }{ }

		\setrow { }{9}{ }  { }{3}{ }  { }{8}{ }
		\setrow {2}{ }{ }  {8}{ }{4}  { }{ }{7}
		\setrow { }{1}{ }  {9}{ }{7}  { }{6}{ }
	\end{tikzpicture}
	\caption{
		\AP\label{fig:intro-sudoku}
		A prefilled Sudoku grid.
	}
\end{marginfigure}
\begin{example}[Sudoku grids]
	\AP\label{ex:sudoku-as-hom}
	We represent an empty sudoku grid as the "relational structure"
	whose domain is $\intInt{1,9} \times \intInt{1,9}$---corresponding to coordinates in the grid---with
	three kinds of binary "predicates": $\+R$, $\+C$ and $\+S$,
	that describe when two coordinates are on the same row, column or 3$\ast$3-square, respectively.
	To deal with prefilled grids, we add nine unary "predicates" $P_k$ ($k \in \intInt{1,9}$),
	and if coordinate $\tup{i,j}$ is prefilled with number $k$, then predicate $P_k$ must
	hold on the element $\tup{i,j}$.
	Given a prefilled grid $G$, we denote by $\?S_G$ the associated "relational structure".
	
	We then define the target structure $\?T$ to have $\intInt{1,9}$ as its domain,
	with each vertex representing a possible value filling a cell.
	The binary relations $\+R$, $\+C$ and $\+S$ all three consist of
	all pairs $\tup{x,y}$ "st" $x \neq y$. This translates the fact that,
	if two cells are on the same row, column, or 3$\ast$3 grid, then they must
	have different values.
	Moreover, each unary relation $P_k$ ($k \in \intInt{1,9}$) is defined
	to hold only on $\{i\}$.

	Then, a prefilled grid $G$ can be completed if, and only if,
	$\?S_G$ has a "homomorphism" to $\?T$.%
	\footnote{In fact, for this example we could use only one binary predicate instead of three.
	Note that this encoding is actually quite close
	to graph colouring (\Cref{ex:graph-colouring-as-hom}) with an extra trick
	to force some values. This trick---formally called "marked structure"---will
	actually prove crucial in \Cref{ch:dichotomy-theorem}.}
	More precisely, "homomorphisms" $f\colon \?S_G \to \?T$ exactly correspond to
	complete Sudoku grid that extend $G$, with $f(\tup{i,j})$ giving
	the number contained in cell $\tup{i,j}$.
\end{example}

\begin{known}
	The "homomorphism problem" is a natural framework in which we can express many
	logical questions, ranging from database evaluation to Sudoku solving.
\end{known}

All three problems we presented can actually be seen as instances of model-checking:
part of the input represents some \emph{data}---or \emph{model}---and part of it represents a
\emph{"query@@sem"}---or \emph{logical question/specification}.
However, as depicted in \Cref{tab:examples-encodings-in-hom}, 
the encodings of these problems into "homomorphism problems" can either be of two types:
\begin{itemize}
	\item the \emph{"query@@sem"} is encoded in the "source structure",
		and the \emph{data} in the "target structure": we denote
		this family of problems by `$\textsf{query} \homto^? \textsf{data}$';
	\item in the other cases, the \emph{data} is encoded in the "source structure"
		and the \emph{"query@@sem"} is the "target structure": we denote
		these problems by `$\textsf{data} \homto^? \textsf{query}$'.
\end{itemize}
\begin{table}[h]
	\centering
	\begin{tabular}{cccc}
		\toprule 
		& data & query & hom problem \\ \midrule 
		Ex~\ref{ex:auto-as-rel} & automata & is $u$ accepted? & $\textsf{query} \homto^? \textsf{data}$ \\
		Ex~\ref{ex:sql-as-hom} & database & SQL query & $\textsf{query} \homto^? \textsf{data}$ \\
		Ex~\ref{ex:sudoku-as-hom} & Sudoku grid & is solvable? & $\textsf{data} \homto^? \textsf{query}$ \\
		Ex~\ref{ex:graph-colouring-as-hom} & graph &
		is "$k$-colourable"? & $\textsf{data} \homto^? \textsf{query}$ \\
		Ex~\ref{ex:sat-as-hom} & SAT formula &
		is satisfiable? & $\textsf{data} \homto^? \textsf{query}$ \\ \bottomrule
	\end{tabular}
	\caption{\AP\label{tab:examples-encodings-in-hom}
		Summary of the encodings of some
		natural model-checking problem into "homomorphism problems".
		The last column indicates whether the structure encoding the data
		(resp. the query) is on the left-hand side ("ie" acts as the "source structure")
		or the right-hand hand (the "target structure")
		of the "homomorphism problem".
		The last two examples will be described in \Cref{sec:intro-universal}.
	}
\end{table}

The two situations are far from symmetric: in model-checking, the size of the
data is usually much larger than query:
for instance, when analysing the documents in the \emph{Panama Papers} scandal,
the data represented 2.9 TB, while the queries were a few lines long \cite{Neo4jPanama}.
Hence, we often study problems where the query is fixed:
accordingly, not only does this fundamental asymmetry between data and query roles motivates the structure of this thesis in two independent parts, it also underlies distinct research domains.

\begin{known}
	Encodings of model-checking problems as "homomorphism problems" are dual by nature,
	leading to two schools that developed different techniques to tackle them:
	\emph{database theory} and \emph{constraint satisfaction problems}.
\end{known}
% Formally, we define:
% \begin{itemize}
% 	\itemAP the ""combined complexity"" of a model-checking problem to be the complexity of
% 		the problem when both the "query@@sem" and the model are part of the input of the decision 
% 		problem;
% 	\itemAP the ""data complexity"" is the complexity of the same problem when the
% 		"query@@sem" is fixed: as there are multiple such problems---one for each "query@@sem"---, we
% 		say that the "data complexity" is $\+C$-complete, for some complexity class $\+C$, when
% 		every such problem is in $\+C$, and when at least one problem is $\+C$-hard.%
% 		\footnote{Of course the dual notion of ``query complexity'', also called ``expression complexity'' exists.}
% \end{itemize}
% For instance, first-order logic has a "combined complexity" that is "PSPACE"-complete
% \cite[Theorem~3.1.6]{Gradel2007FiniteModelTheory},
% but its "data complexity" is "uniform-AC0"!\footnote{I failed to found a proper reference claiming
% this "uniform-AC0" bound: \cite[Corollary~3.1.8]{Gradel2007FiniteModelTheory} only claims "ALogTime", however the "uniform-AC0" is trivial to show, as the formula precisely gives the shape
% of the circuits that needs to be built.} What this means is that the complexity
% of the model-checking problems depends more on the size of the formula than the source structure.

% Similarly, the "homomorphism problem" is not symmetric.
% If the "target structure" is fixed, the problem remains "NP"-complete---for instance
% 3-colourability and 3-SAT (\Cref{ex:graph-colouring-as-hom,ex:sat-as-hom}) are both
% examples of "NP"-complete problems where the "target structure" is fixed.
% However, if the "source structure" is fixed, the number of possible "homomorphisms"
% from $\?A$ to $\?B$ is at most $|B|^{|A|}$ and hence polynomial. In fact, the naive algorithm
% enumerating all possible functions and checking whether they are "homomorphisms" works
% is "LogSpace", and even better we can build from $\?A$ a small circuit
% solving the problem, giving a "uniform-AC0" upper bound.

% Both these asymmetries imply that, when encoding
% a model-checking problem as a "homomorphism problem", whether this encoding
% of the form $\textsf{query} \homto^? \textsf{data}$ or
% $\textsf{data} \homto^? \textsf{query}$ has a non-trivial impact on
% the complexity, and more generally on the properties, of the problem.