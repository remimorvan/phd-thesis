\section{The Two Sides of the Homomorphism Problem}

This thesis is devoted to studying variations of the "homomorphism problem".
This problem takes two "finite (directed) graphs", consisting of a finite set of vertices (also called \emph{domain}) $V$ together with a set of
"edges" $\+E \subseteq V \times V$, and asks if there is a "homomorphism" 
between them, which consists of a function between vertices that preserves
the "edges", in the sense that any edge must be mapped to another edge,
see \Cref{fig:example-graph-homomorphism}.

\begin{figure}
	\centering
	\begin{tikzpicture}
		\foreach \i in {0,1,...,5}{
			\node[vertex] (\i) at (\i*360/6: 1.2cm) {};
		}

		\draw[edge] (1) to (0);
		\draw[edge] (2) to (1);
		\draw[edge] (3) to (2);
		\draw[edge] (3) to (4);
		\draw[edge] (5) to (4);
		\draw[edge] (5) to (0);

		\node[vertex] at (4.33,.66) (a) {};
		\node[vertex] at (5.66,.66) (b) {};
		\node[vertex] at (5.66,-.66) (c) {};
		\node[vertex] at (4.33,-.66) (d) {};
		\node[vertex] at (6.8,0) (e) {};

		\draw[edge] (d) to (a);
		\draw[edge] (a) to (b);
		\draw[edge] (b) to (c);
		\draw[edge] (d) to (c);
		\draw[edge] (b) to (e);
		\draw[edge] (e) to (c);

		\draw[edge, draw=cBlue, dotted, out=-20, in=160] (3) to (d);
		\draw[edge, draw=cBlue, dotted, out=-25, in=170] (2) to (a);
		\draw[edge, draw=cBlue, dotted, out=5, in=130] (1) to (b);
		\draw[edge, draw=cBlue, dotted, out=10, in=145] (0) to (c);
		\draw[edge, draw=cBlue, dotted, out=-20, in=195] (5) to (d);
		\draw[edge, draw=cBlue, dotted, out=-30, in=210] (4) to (c);
	\end{tikzpicture}
	\caption{
		\AP\label{fig:example-graph-homomorphism}
		Two graphs (in black), and a homomorphism (blue dotted arrows) from the
		graph on the left-hand side to one right one.
	}
\end{figure}

To enrich the structure---but more importantly to add a splash of colour to this thesis---,
we will in fact consider more complex structures: 
we allow for multiple edge relations, or even relations of higher arity linking the vertices.
Which kind of relations (and how many of which kind) is allowed is known as the "signature" $\sigma$ of the "structure". 
These richer structures are known as "-structures" or "relational structures"---see
\Cref{fig:example-structure-homomorphism}---, and
"homomorphisms" between "$\sigma$-structures" are asked to preserve \emph{all} relations
in the "signature" $\sigma$.

\decisionproblem{The ""Homomorphism Problem"" over $\sigma$}{
	Two finite $\sigma$-structures $\?A$ and $\?B$
}{
	Is there a "homomorphism" from $\?A$ to $\?B$?
}

In the problem above, we refer to $\?A$ as the \AP""input structure""
and to $\?B$ as the ""target structure"". We denote by \AP$\?A \intro*\homto \?B$
the existence of a "homomorphism" from $\?A$ to $\?B$.

\begin{figure}
	\centering
	\begin{tikzpicture}[scale=.9]
		\node (d0) at (-.6,.6) {};
		\node (d1) at (-.8,0) {};
		\node (d2) at (-.4,-1.2) {};
		\node (d3) at (-1.5,0) {};
		\node (d4) at (-0.3,1.3) {};

		\draw[use Hobby shortcut, closed=true, draw=c2, fill=c2, fill opacity=.4] 
			(d0) .. (d1) .. (d2) .. (d3) .. (d4) ;

		\node (c0) at (-.2,1) {};
		\node (c1) at (-0.5,1.5) {};
		\node (c2) at (-1,2.4) {};
		\node (c3) at (-2.7,2.2) {};
		\node (c4) at (-2.2,1.5) {};
		\node (c5) at (-1.2,1.5) {};
		\node (c6) at (-.6,.6) {};

		\draw[use Hobby shortcut, closed=true, draw=c1, fill=c1, fill opacity=.4] 
			(c0) .. (c1) .. (c2) .. (c3) .. (c4) .. (c5) .. (c6);

		\node (e0) at (1.5,0) {};
		\node (e1) at (1,.3) {};
		\node (e2) at (.95,-.3) {};
	
		\draw[use Hobby shortcut, closed=true, draw=c0, fill=c0, fill opacity=.4] 
			(e0) .. (e1) .. (e2) ;

		% grid
		% \foreach \x in {-4,-3,...,4} {
		% 	\draw[-,draw=cLightGrey] (\x, -4) to (\x, 4);
		% 	\draw[-,draw=cLightGrey] (-4, \x) to (4, \x);
		% }
		% \draw[-,draw=cGrey] (0, -4) to (0, 4);
		% \draw[-,draw=cGrey] (-4, 0) to (4, 0);

		\foreach \i in {0,1,...,5} {
			\node[vertex] (\i) at (\i*360/6: 1.2cm) {};
		}
		\foreach \i in {7,8,9,10} {
			\node[vertex] (\i) at ($(\i*360/6: 1.2cm)+(-1.8,1.04)$) {};
		}

		\draw[edge, bend left] (1) to (0);
		\draw[edge, double, bend right] (1) to (0);
		\draw[edge, <->] (2) to (1);
		\draw[edge, double] (3) to (2);
		\draw[edge] (3) to (4);
		\draw[edge, double] (5) to (4);
		\draw[edge, double] (5) to (0);
		\draw[edge, double] (2) to (7);
		\draw[edge, bend left] (7) to (8);
		\draw[edge, double, bend left] (8) to (7);
		\draw[edge] (8) to (9);
		\draw[edge, double] (10) to (9);
		\draw[edge, bend left] (10) to (3);
		\draw[edge, double, <->, bend left] (3) to (10);
		\draw[edge, loop, out=-90, in=0, looseness=8] (5) to (5);
	\end{tikzpicture}
	\caption{
		\AP\label{fig:example-structure-homomorphism}
		A "relational structure" with two kinds of binary relations (represented by
		simple and double arrows) and three kinds of unary relations (represented by
		red, yellow and blue potatoes).
	}
\end{figure}

More than a mere decision problem---which obviously lives in "NP"---,
the "homomorphism problem" should rather be seen as a \emph{framework} or
\emph{language} to formalize many common problems in computer science.

\begin{marginfigure}
	\centering
	\begin{tikzpicture}
		\node[vertex] (0) at (0,0) {};
		\node[vertex, right=of 0] (1) {};
		\node[vertex, right=of 1] (2) {};
		\node[vertex, right=of 2] (3) {};

		\draw[edge] (0) to node[above] {$a$} (1);
		\draw[edge] (1) to node[above] {$b$} (2);
		\draw[edge] (2) to node[above] {$a$} (3);

		\draw[edge] ($(0)+(-.4,0)$) to (0);
		\node[vertex, inner sep=3pt] (3b) at (3) {};

		\node[vertex, below=5em of 1] (q0) {};
		\node[vertex, right=of q0] (q1) {};
		
		\draw[edge] ($(q0)+(-.4,0)$) to (q0);
		\node[vertex, inner sep=3pt] (q1b) at (q1) {};

		\draw[edge, loop, out=-60, in=-120, looseness=8] (q0) to node[below] {$a,b$} (q0);
		\draw[edge] (q0) to node[above] {$a$} (q1);
		\draw[edge, loop, out=-60, in=-120, looseness=8] (q1) to node[below] {$a,b$} (q1);
		
		\draw[edge, draw=cBlue, dotted, out=-60, in=130] (0) to (q0);
		\draw[edge, draw=cBlue, dotted, out=-100, in=100] (1) to (q0);
		\draw[edge, draw=cBlue, dotted, out=-90, in=80] (2) to (q0);
		\draw[edge, draw=cBlue, dotted, out=-90, in=90] (3) to (q1);
	\end{tikzpicture}
	\caption{
		\AP\label{fig:example-automaton-as-rel}
		Automata acceptance as a "homomorphism problem":
		"relational structure" representing the finite word $aba$ (above),
		"structure" representing the minimal automaton of $(a+b)^* a (a+b)^*$
		(below) and a "homomorphism" from the former to the latter (blue dotted arrows).
		Vertices with a double circle (resp. incoming dangling arrow) represent
		final (resp. initial) states.
	}
\end{marginfigure}

\begin{example}[Non-deterministic automata]
	\AP\label{ex:auto-as-rel}
	A (non-deterministic) automaton $\?A$ can be seen as a "relational structure",
	whose nodes are its states, on the "signature" with two unary predicates (one for
	describing initial states, one for final states), and one binary predicate
	for each letter of the alphabet $\Sigma$.
	Any finite word $u\in \Sigma^*$ can in turn by seen as a "relational structure"
	$\?W_u$ with $\lBrack 0,|u|\rBrack$ as its domain,
	where $0$ is initial, $|u|$ is final, and
	for each $i \in \lBrack 1,|u|\rBrack$, there is an edge from $i-1$ to $i$
	whose type is given by the $i$-th letter of $u$.
	Then, there is a "homomorphism" from $\?W_u$ to $\?A$ if, and only if, 
	the automaton $\?A$ accepts $u$. See \Cref{fig:example-automaton-as-rel}.
\end{example}

\begin{marginfigure}
	\centering
	\begin{tikzpicture}
		% ---
% 3-clique
% ---
\node[vertex, draw=c0, fill=c0, fill opacity=.4] at (0,0) (k3-0) {};
\node[vertex, above right=1em and 2em of k3-0, draw=c1, fill=c1, fill opacity=.4] (k3-1) {};
\node[vertex, below right=1em and 2em of k3-0, draw=c2, fill=c2, fill opacity=.4] (k3-2) {};

\draw[edge, bend right=15] (k3-0) to (k3-1);
\draw[edge, bend right=15] (k3-0) to (k3-2);
\draw[edge, bend right=15] (k3-1) to (k3-0);
\draw[edge, bend right=15] (k3-1) to (k3-2);
\draw[edge, bend right=15] (k3-2) to (k3-0);
\draw[edge, bend right=15] (k3-2) to (k3-1);
	\end{tikzpicture}
	\caption{
		\AP\label{fig:intro-3-clique}
		The "$3$-clique" $\clique{3}$.
	}
\end{marginfigure}
\begin{example}[Graph colouring]
	\AP\label{ex:graph-colouring-as-hom}
	Let $k\in\Np$. We let the \AP""$k$-clique"", denoted by $\intro*\clique{k}$,
	to be the "graph@@dir" whose vertices are $\lBrack 1,k\rBrack$,
	and with an edge from $i$ to $j$ (with $i,j \in \lBrack 1,k\rBrack$)
	"iff" $i\neq j$, see \Cref{fig:intro-3-clique}.
	A finite "graph@@dir" $\?G$ is \AP""$k$-colourable""---"ie"
	we can map every vertex of $\?G$ to an element of $\lBrack 1,k\rBrack$
	"st" no two adjacent vertices are sent on the same colour/number---if,
	and only if, there is a "homomorphism" from $\?G$ to $\clique{k}$.
\end{example}

Note that in both \Cref{ex:auto-as-rel,ex:graph-colouring-as-hom},
not only does the existence of a "homomorphism" is equivalent to the existence
of a solution to the problem, but in fact the set of "homomorphism" is in fact
in natural bijection with the set of solutions:
\begin{itemize}
	\item in \Cref{ex:auto-as-rel}, "homomorphisms" from $\?W_u$ to $\?A$
		exactly correspond to accepting runs of the automaton over $u$;
	\item in \Cref{ex:graph-colouring-as-hom}, "homomorphisms" from $\?G$ to $\clique{k}$
		exactly correspond to $k$-colouring of $\?G$.
\end{itemize}

\begin{figure}
	\centering%
	{%
		\footnotesize%
		\begin{tabular}{cccc}
			\multicolumn{4}{c}{\textsc{Movies}} \\ \toprule
			id & title & length & director \\ \midrule
			197 & Eyes Wide Shut & 159 & Stanley Kubrick \\ 
			205 & J'ai tué ma mère & 96 & Xavier Dolan \\
			304 & Amadeus & 161 & Miloš Forman \\
			321 & 120 Battements par minute & 143 & Robin Campillo \\ \bottomrule
		\end{tabular}
		\\\bigskip%
		\begin{tabular}{cc}
			\multicolumn{2}{c}{\textsc{Rooms}} \\ \toprule
			id & capacity  \\ \midrule
			1 & 108 \\
			2 & 124 \\
			3 & 96 \\
			4 & 102 \\ \bottomrule
		\end{tabular}%
		\hspace{1cm}%
		\begin{tabular}{ccc}
			\multicolumn{3}{c}{\textsc{Projections}} \\ \toprule
			movie\_id & room\_id & time \\ \midrule
			197 & 2 & 2025-03-28 14:00 \\
			205 & 3 & 2025-03-28 14:30 \\
			321 & 4 & 2025-03-28 14:30 \\
			197 & 1 & 2025-03-28 17:00 \\ \bottomrule
		\end{tabular}
	}
	\caption{
		\AP\label{fig:example-db-as-rel}
		A "relational database" consisting of three tables, representing data
		stored by a cinema.
	}
\end{figure}
\begin{marginfigure}
	\centering
	\begin{tikzpicture}
		\node (a0) at (-.1,5.2) {};
		\node (a1) at (-.4,3.5) {};
		\node (a2) at (-.1,1.8) {};
		\node (a3) at (.1,1.8) {};
		\node (a4) at (.4,3.5) {};
		\node (a5) at (.1,5.2) {};

		\node (b0) at (-.2,5.2) {};
		\node (b1) at (-.8,3.1) {};
		\node (b2) at (-.5,1.5) {};
		\node (b3) at (-.4,.6) {};
		\node (b4) at (-.2,-.2) {};
		\node (b5) at (.2,-.2) {};
		\node (b6) at (.4,1) {};
		\node (b7) at (.2,1.4) {};
		\node (b8) at (-.5,2.6) {};
		\node (b9) at (-.4,3.8) {};
		\node (b10) at (.2,5.2) {};
	
		\draw[use Hobby shortcut, closed=true, draw=c2, fill=c2, fill opacity=.4] 
			(b0) .. (b1) .. (b2) .. (b3) .. (b4) .. (b5) .. (b6) .. (b7) .. (b8) .. (b9) .. (b10);
		\draw[use Hobby shortcut, closed=true, draw=c1, fill=c1, fill opacity=.4] 
			(a0) .. (a1) .. (a2) .. (a3) .. (a4) .. (a5);
		
		\node[vertex] at (0,5) (5) {};
		\node[vertex] at (0,4) (4) {};
		\node[vertex] at (0,3) (3) {};
		\node[vertex] at (0,2) (2) {};
		\node[vertex] at (0,1) (1) {};
		\node[vertex] at (0,0) (0) {};
		\node[font=\tiny, right=2em of 5] {\textsf{movie\_id}};
		\node[font=\tiny, right=2em of 4] {\textsf{title}};
		\node[font=\tiny, right=2em of 3] {\textsf{length}};
		\node[font=\tiny, right=2em of 2] {\textsf{director}};
		\node[font=\tiny, right=2em of 1] {\textsf{room\_id}};
		\node[font=\tiny, right=2em of 0] {\textsf{time}};
	\end{tikzpicture}
	\caption{
		\AP\label{fig:example-sql-as-rel}
		A SQL query seen as a "relational structure".
		The yellow potato represents the single tuple
		of the \textsc{Movies} relation, and the blue
		potato surrounds the only tuple that belongs to the
		\textsc{Projections} relation.
	}
\end{marginfigure}
\begin{example}[SQL queries]
	\AP\label{ex:sql-as-hom}
	A "relational database", such as the one depicted on
	\Cref{fig:example-db-as-rel}, can easily be seen as a "relational structure"
	whose domain is the set of elements occurring somewhere in a table,
	with one relation for each table.
	In fact, the only difference between "relational databases" and "relational structures"
	precisely lies in the fact that in the case of the former, the domain is implicit,
	while for the latter it is explicit.%
	\footnote{While this difference alters the theory, the difference is mostly negligible
	for the "query@@sem" languages we will study, see \todo{addref-rk-prelim}.}

	Consider the following SQL query, which outputs all pairs
	of movie titles together with their projection time.
	\lstinputlisting[language=SQL, deletekeywords={TIME,Time,time}]{fig/intro/cinema.sql}

	This query $\gamma$ can in fact be seen itself as a "relational structure" $\?Q_\gamma$	over
	the same "signature" as the "relational database".
	Its domain has six elements, corresponding to the attributes of
	the \textsc{Movies} and \textsc{Projections} table, merged on the attribute
	$\textsf{Projections.movie\_id} = \textsf{Movies.id}$.
	Both relations $\textsc{Movies}$ and $\textsc{Projections}$ consist of a single tuple,
	and the relation $\textsc{Rooms}$ is empty, see \Cref{fig:example-sql-as-rel}.

	Then, the set of pairs $\tup{\textsf{title}, \textsf{time}}$
	such that there is a "homomorphism" from $\?Q_\gamma$ to the relational database, seen as
	a "relational structure", is exactly the output set of the SQL query on the database.
\end{example}

\begin{marginfigure}
	\centering%
	\begin{tikzpicture}[scale=1.3]
		\draw[step=.3333] (0, 0) grid (3, 3);
		\draw[very thick, step=1] (0, 0) grid (3, 3);

		\setcounter{row}{1}
		\setrow { }{2}{ }  {5}{ }{1}  { }{9}{ }
		\setrow {8}{ }{ }  {2}{ }{3}  { }{ }{6}
		\setrow { }{3}{ }  { }{6}{ }  { }{7}{ }

		\setrow { }{ }{1}  { }{ }{ }  {6}{ }{ }
		\setrow {5}{4}{ }  { }{ }{ }  { }{1}{9}
		\setrow { }{ }{2}  { }{ }{ }  {7}{ }{ }

		\setrow { }{9}{ }  { }{3}{ }  { }{8}{ }
		\setrow {2}{ }{ }  {8}{ }{4}  { }{ }{7}
		\setrow { }{1}{ }  {9}{ }{7}  { }{6}{ }
	\end{tikzpicture}
	\caption{
		\AP\label{fig:intro-sudoku}
		A prefilled sudoku grid.
	}
\end{marginfigure}
\begin{example}[Sudoku grids]
	\AP\label{ex:sudoku-as-hom}
	We represent an empty sudoku grid as the "relational structure"
	whose domain is $\lBrack 1,9\rBrack \times \lBrack 1,9\rBrack$---corresponding to coordinates in the grid---with
	three kinds of binary "predicates": $\+R$, $\+C$ and $\+S$,
	that describe when two coordinates are on the same row, column or 3*3-square, respectively.
	For non-empty grids, we add nine unary "predicates" $P_k$ ($k \in \lBrack 1,9\rBrack$),
	and if coordinate $\tup{i,j}$ if prefilled with number $k$, then predicate $P_k$ must
	hold on the element $\tup{i,j}$.
	Given a prefilled grid $G$, we denote by $\?S_G$ the associated "relational structure".
	
	We then define the target structure $\?T$ to have $\lBrack 1,9\rBrack$ as its domain:
	with the binary relations $\+R$, $\+C$ and $\+S$ which all three consist of
	all tuples $\tup{\tup{i,j},\tup{i',j'}}$ "st" $\tup{i,j} \neq \tup{i',j'}$.
	Moreover, each unary relation $P_k$ ($k \in \lBrack 1,9\rBrack$) is defined
	to hold only on $\{i\}$.
	Then, a prefilled grid $G$ can be completed if, and only if,
	$\?S_G$ has a "homomorphism" to $\?T$.%
	\footnote{In fact, for this example we could use only one binary predicate instead of three.
	Note that this encoding is actually quite close
	to graph colouring (\Cref{ex:graph-colouring-as-hom}) with an extra trick
	to force some values. This trick---formally called "marked structure"---will
	actually prove crucial in \Cref{ch:dichotomy-theorem}.}
	In fact, "homomorphisms" $h\colon \?S_G \to \?T$ exactly correspond to
	complete Soduku grid that extend $G$, with $h(\tup{i,j})$ giving
	the number contained in cell $\tup{i,j}$.
\end{example}

\begin{example}[SAT solving]
	\AP\label{ex:sat-as-hom}
	We consider a 3-SAT instance, namely a finite conjunction of
	disjunctions of three litterals, say
	\[
		\phi \defeq \bigwedge_{i=1}^n \ell_{i,1} \lor \ell_{i,2} \lor \ell_{i,3},
	\]
	where each $\ell_{i,j}$ is either a variable, or the negation of a variable.
	We assume "wlog" that in each clause, positive variables appear before negative ones%
	\footnote{Meaning "eg" that $x \lor \neg y \lor z$ is not allowed, contrary to
	$x \lor z \lor \neg y$.}
	We let $\?B$ be the structure whose domain has two elements $\{0,1\}$,
	equipped with four ternary relations $\+R_0, \+R_1, \+R_2$ and $\+R_3$, defined by:
	\begin{align*}
		\+R_0 & \defeq \set{0,1}^3 \smallsetminus \set{\tup{0,0,0}}, &
		\+R_1 & \defeq \set{0,1}^3 \smallsetminus \set{\tup{1,0,0}}, \\
		\+R_2 & \defeq \set{0,1}^3 \smallsetminus \set{\tup{1,1,0}}, \text{ and} &
		\+R_3 & \defeq \set{0,1}^3 \smallsetminus \set{\tup{1,1,1}}.
	\end{align*}
	We then encode $\phi$ into the "relational structure" $\?F_\phi$
	whose domain is the set of variables of $\phi$,
	and for every $i \in \lBrack 1,n \rBrack$,
	$\+R_j$ ($j\in\lBrack 0,3\rBrack$) consists of all triplets of variables $\tup{x,y,z}$
	"st" there is a clause of $\phi$ containing exactly the variables $x$, $y$ and $z$ (with multiplicity), and exactly $j$ of these variables occur negatively.
	For instance, $\tup{x, y, \neg x} \in \+R_1$, and $\tup{\neg x, \neg y, \neg z} \in \+R_3$.
	A function $f$ from the domain of $\?F_\phi$ to the domain of $\?B$ amounts to picking
	a Boolean valuation of the variables occurring in $\phi$.
	Observe that by definition of the relations $\+R_j$'s,
	given a clause $\psi$ containing variables $x,y,z$, 
	$f$ is a "homomorphism" from $\?F_\psi$ to $\?B$ "iff"
	$f$, seen as a valuation, satisfies $\psi$.\footnote{For instance,
	if all variables are positive, then all valuations except the one putting
	all variables to false satisfy the formula. This is why $\+R_0$ is defined
	on $\?B$ as $\set{0,1}^3\smallsetminus \set{\tup{0,0,0}}$.}
	By taking conjunction, the conclusion follows: "homomorphisms" from
	$\?F_\phi$ to $\?B$ exactly correspond to valuations satisfying $\phi$.
	In particular, there is a "homomorphism" from $\?F_\phi$ to $\?B$ "iff"
	$\phi$ is satisfiable.%
	\footnote{Note that this example can be easily generalized to $k$-SAT for any $k\in\Np$.
	However, the "signature" depends on $k$.}
\end{example}

\begin{known}
	The "homomorphism problem" is a natural framework in which we can express many
	logical questions, ranging from database evaluation to SAT solving.
\end{known}

\begin{table}
	\centering
	\begin{tabular}{cccc}
		\toprule 
		& data & query & hom problem \\ \midrule 
		Ex~\ref{ex:auto-as-rel} & automata & is $u$ accepted? & $\textsf{query} \homto^? \textsf{data}$ \\
		Ex~\ref{ex:graph-colouring-as-hom} & graph & is "$k$-colourable"?& $\textsf{data} \homto^? \textsf{query}$ \\
		Ex~\ref{ex:sql-as-hom} & database & SQL query & $\textsf{query} \homto^? \textsf{data}$ \\
		Ex~\ref{ex:sudoku-as-hom} & Sudoku grid & is solvable? & $\textsf{data} \homto^? \textsf{query}$ \\
		Ex~\ref{ex:sat-as-hom} & formula & is satisfiable? & $\textsf{data} \homto^? \textsf{query}$ \\ \bottomrule
	\end{tabular}
	\caption{\AP\label{tab:examples-encodings-in-hom}
		Summary of the encodings of some
		natural model-checking problem into "homomorphism problems".
		The last column indicates whether the structure encoding the data
		(resp. the query) is on the left-hand side ("ie" acts as the "input structure")
		or the right-hand hand (the "target structure")
		of the homomorphism problem.
	}
\end{table}

All five problems we presented can actually be seen as model-checking problem:
part of the input represents some \emph{data}---or \emph{model}---and part of it represents a
\emph{"query@sem"}---or \emph{logical question/specification}.
However, as depicted in \Cref{tab:examples-encodings-in-hom}, 
the encodings of these problems into "homomorphism problems" can either be of two types:
\begin{itemize}
	\item the \emph{"query@sem"} is encoded in the "input structure",
		and the \emph{data} in the "target structure": we denote
		this family of problems by `$\textsf{query} \homto^? \textsf{data}$';
	\item in the other cases, the \emph{data} is encoded in the "input structure"
		and the \emph{"query@sem"} is the "target structure": we denote
		these problems by `$\textsf{data} \homto^? \textsf{query}$'.
\end{itemize}

The two situations are far from symmetric: in model-checking, the size of the
data is usually much larger than query,%
\footnote{For instance, when analysing the documents in the \emph{Panama Papers} scandal,
the data represented 2.9 terabytes, while the queries were a few lines long. \cite{Neo4jPanama}}
and we often study problems where
the query is fixed---for instance the 3-colourability problem.
Accordingly, not only do these two sides of the "homomorphism problem" split this thesis
in two parts: both sides form independent research domains, known as \emph{database theory}
and \emph{constraint satisfaction problems}, which is related logic programming...%
\footnote{... and to `artificial intelligence', as many finite model theorists
like to point out in grant applications.}
% Formally, we define:
% \begin{itemize}
% 	\itemAP the ""combined complexity"" of a model-checking problem to be the complexity of
% 		the problem when both the "query@sem" and the model are part of the input of the decision 
% 		problem;
% 	\itemAP the ""data complexity"" is the complexity of the same problem when the
% 		"query@sem" is fixed: as there are multiple such problems---one for each "query@sem"---, we
% 		say that the "data complexity" is $\+C$-complete, for some complexity class $\+C$, when
% 		every such problem is in $\+C$, and when at least one problem is $\+C$-hard.%
% 		\footnote{Of course the dual notion of ``query complexity'', also called ``expression complexity'' exists.}
% \end{itemize}
% For instance, first-order logic has a "combined complexity" that is "PSPACE"-complete
% \cite[Theorem~3.1.6]{Gradel2007FiniteModelTheory},
% but its "data complexity" is "uniform-AC0"!\footnote{I failed to found a proper reference claiming
% this "uniform-AC0" bound: \cite[Corollary~3.1.8]{Gradel2007FiniteModelTheory} only claims "ALogTime", however the "uniform-AC0" is trivial to show, as the formula precisely gives the shape
% of the circuits that needs to be built.} What this means is that the complexity
% of the model-checking problems depends more on the size of the formula than the input structure.

% Similarly, the "homomorphism problem" is not symmetric.
% If the "target structure" is fixed, the problem remains "NP"-complete---for instance
% 3-colourability and 3-SAT (\Cref{ex:graph-colouring-as-hom,ex:sat-as-hom}) are both
% examples of "NP"-complete problems where the "target structure" is fixed.
% However, if the "input structure" is fixed, the number of possible "homomorphisms"
% from $\?A$ to $\?B$ is at most $|B|^{|A|}$ and hence polynomial. In fact, the naive algorithm
% enumerating all possible functions and checking whether they are "homomorphisms" works
% is "LogSpace", and even better we can build from $\?A$ a small circuit
% solving the problem, giving a "uniform-AC0" upper bound.

% Both these asymmetries imply that, when encoding
% a model-checking problem as a "homomorphism problem", whether this encoding
% of the form $\textsf{query} \homto^? \textsf{data}$ or
% $\textsf{data} \homto^? \textsf{query}$ has a non-trivial impact on
% the complexity, and more generally on the properties, of the problem.