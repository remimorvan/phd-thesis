\section{\AP{}Preliminaries}
\label{sec:prelim}

\paragraph*{Some intuitions on maximal under-approximations}
Given a "conjunctive query" $\gamma$,
the union of all "conjunctive queries"
that are "contained" in $\gamma$ is "semantically equivalent" to the union
$\bigvee \{ \gamma' \mid \gamma \surj \gamma' \}$. Naturally, this statement borders on the trivial since $\gamma'$ belongs to this union. It becomes interesting when we add a restriction:
given a class $\class$ of "CQs" (to which $\gamma$ may not belong) closed under "subqueries", then $\Gamma' \defeq \bigvee \{ \gamma' \in \class \mid \gamma \surj \gamma' \}$ is the maximal under-approximations
of $\gamma$ by finite unions of "conjunctive queries" of $\class$, in the following sense:
\begin{enumerate}[label=\roman*.]
	\item (finite) $\Gamma'$ is a finite union of "CQs" of $\class$,
	\item (under-approximation) $\Gamma' \contained \gamma$, and
	\item (maximality) for any finite union $\Delta$ of "CQs" of $\class$, if $\Delta \contained \gamma$, then $\Delta \contained \Gamma'$.
\end{enumerate}

\begin{proof}
Only the last point is non-trivial, and follows from the fact that if
$\Delta \contained \gamma$, then for each $\delta \in \Delta$, $\delta \contained \gamma$,
so there is a "homomorphism" $f\colon \gamma \to \delta$. The image $\delta'$
of $f$ is a "subquery" of $\delta$, and $\+C$ is closed under "subqueries",
so it belongs to $\+C$, and hence to $\Gamma'$. Since there is a trivial homomorphism
from $\delta'$ to $\delta$, we moreover have that $\delta \contained \delta'$.
Hence, for each "CQ" $\delta \in \Delta$, there is a CQ $\delta' \in \Gamma'$ such
that $\delta \contained \delta'$, and hence $\Delta \contained \Gamma'$.
\end{proof}

As a consequence, we deduce that for each $k \geq 1$,
the "maximal under-approximation" of a "CQ" by
a finite union of "CQs" of "tree-width" at most $k$ is computable, and hence
we can effectively decide if some "CQ" is "equivalent" to a query of "tree-width" at
most $k$ by testing the equivalence with this maximal under-approximation.
For more details on approximations of "CQs", see \cite{BarceloLibkinRomero2014Efficient}.
Note that interestingly, changing $\Gamma'$ from
$\bigvee \{ \gamma' \in \class \mid \gamma \surj \gamma' \}$
to $\bigvee \{ \gamma' \in \class \mid \gamma' \contained \gamma \}$
preserves both under-approximation and maximality, but $\Gamma'$ is now an infinite
union of "CQs" of $\+C$.

Unfortunately, these results cannot be straightforwardly extended to "conjunctive regular
path queries" since the previous proof implicitly relied on two points:
\begin{enumerate}
	\item the equivalence between the
	"containment" $\gamma' \contained \gamma$ and the existence of a "homomorphism"
	$\gamma \homto \gamma'$, and
	\item the possibility to restrict $\gamma'$ to its image $\gamma \homto \gamma'$ while 
	obtaining a semantically bigger query.
\end{enumerate}
These two crucial ingredients is what allows us to build a finite set $\Gamma'$ from $\gamma$.
For "CRPQs", the second point still holds, but not the first one.
For instance, the "CQ" $\gamma(x,y) = x \atom{a} z \atom{b} y$ is
contained in (in fact "equivalent" to) the "CRPQ" $\gamma'(x,y) = x \atom{ab} y$,
but there is no "homomorphism" from $\gamma'(x,y)$ to $\gamma(x,y)$.
Our main result shows that to find "maximal under-approximations" of "C2RPQs",
it suffices to take "homomorphic images" of so-called ``"refinements"'' of $\gamma$,
instead of "homomorphic images" of $\gamma$ itself. The next paragraphs are devoted to
introducing "refinements" and tools related to them.

\paragraph*{Refinements and Tree-Width.}
Our approach to proving \Cref{thm:decidability-semtw,thm:closure-under-sublanguages}
and the "Key Lemma" heavily rely on "refinements". One crucial property
that these objects satisfy is that they preserve "tree-width" $k$, unless $k=1$,
as illustrated in \Cref{fig:tree-decompositon-expansion}.

\begin{restatable}{fact}{refinementtw}
    \AP\label{fact:refinement-tw}
    Let $k \geq 2$ and let $\gamma$ be a "C2RPQ" of "tree-width" at most $k$.
    Then any "refinement" of $\gamma$ has "tree-width" at most $k$.
\end{restatable}

\begin{figure}
    \centering
	\subfloat[A multigraph together with a "tree decomposition" of "width" $k$.]{%
		\AP\label{subfig:tree-decompositon-before-expansion}%
		\includegraphics*[width=.41\textwidth]{tree-decompositon-before-expansion.pdf}
	}
	\hfill
	\subfloat[A "refinement" of the multigraph of \Cref{subfig:tree-decompositon-before-expansion} together with a "tree decomposition" of "width" $\max(k,2)$.]{%
		\AP\label{subfig:tree-decompositon-after-expansion}
		\includegraphics*[width=.54\textwidth]{tree-decompositon-after-expansion.pdf}
	}
    \caption{\AP\label{fig:tree-decompositon-expansion} "Refinements" and "expansions" preserve "tree-width" at
	most $k\geq 2$.
    }
\end{figure}
\begin{proof}
    The underlying graph of a "refinement" of $\gamma$ is obtained from the underlying graph
    of $\gamma$ by either contracting some edges (when dealing with "equality atoms"),
	or by replacing
    a single edge by a path of edges (where the non-extremal nodes are new nodes).

	This first operation preserves "tree-width" at most $k$ (even if $k = 1$),
    see "eg" \cite[Lemma 16]{Bodlaender1998Arboretum}. The second operation
    preserves "tree-width" at most $k$, assuming $k > 1$: if a graph $G'$
    is obtained from a graph $G$ by replacing an edge $x_0 \atom{} x_n$
    by a path $x_0 \atom{} x_1 \atom{} \cdots \atom{} x_n$, then
    from a "tree decomposition" of $G$ it suffices to pick a "bag" containing
    both $x_0$ and $x_n$, and add a branch to the tree, rooted at this "bag",
    and containing bags with nodes
    \begin{align*}
        \{x_0,x_1,x_n\},\,
        \{x_1,x_2,x_n\},\,
		\hdots,\,
        \{x_i,x_{i+1},x_n\},\,
		\hdots,\,
        \{x_{n-2},x_{n-1},x_n\},
    \end{align*}
	as depicted in \Cref{fig:tree-decompositon-expansion}.
    All bags contain exactly three nodes, so we obtain "tree decomposition" of
    $G'$ whose "width" is the maximum between 2 and the "width" of the original
    "tree decomposition" of $G$.
\end{proof}

For $k=1$, the property fails: for instance the "CRPQ" $\gamma(x) = x \atom{a^*} x$
has "tree-width" at most 1 (in fact it has "tree-width" 0), but its "refinement"
$\rho(x) = x \atom{a^*} t_1 \atom{a^*} t_2 \atom{a^*} x$ has "tree-width" 2.

\paragraph*{Fine tree decompositions}
\AP For technical reasons---the proof of \Cref{lemma:shape-decomposition}---, we will use a restrictive class of "tree decompositions" which we call ``"fine@fine tree decomposition"''\footnote{This is similar---but orthogonal---to the classical notion of
``nice tree decomposition'', see "eg" \cite[Definition 13.1.4, page 149]{Kloks1994Treewidth}.}. A \AP""fine tree decomposition"" is a "tree decomposition" $\tup{\?T, \bagmap}$ in which:
\begin{equation}
	\AP\label{eq:fine-tree-dec}
	\parbox{.65\linewidth}{every non-root "bag" can be obtained from its parent "bag" by
	either adding or removing a non-empty set of vertices.}%\tag{\adfdownleafleft}
\end{equation}
In the context of a "fine tree decomposition" of "width" $k$, a \AP""full bag"" is any "bag" of size $k+1$.

A "C2RPQ" has "tree-width" $k$ if and only if it has a "fine tree decomposition" of "width" at most $k$. Indeed, from a "tree decomposition", it suffices to:
\begin{enumerate}
	\item first merge every consecutive pair of "bags" that contain exactly the same variables;
	\item between every pair of "bags"
	that does not satisfy \eqref{eq:fine-tree-dec}, add a "bag" whose set of vertices
	correspond to the intersection of the two adjacent "bags".
\end{enumerate}