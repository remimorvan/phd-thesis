\chapter{%
	\AP\label{ch:preliminaries-automatic-structures}
	Automatic Structures and Synchronous Relations%
}

Should either go to the introduction (general stuff of structures), or here.

Distinction between "automatic presentation" of a "structure" (or "automatic structure" for short) and an "automatic structure"---existential quantifier.

Given a "relational signature" $\sigma$, a \AP"automatic $\sigma$-presentation" $\+A$
over the alphabet $\Sigma$ consists of a "regular language"
$\domainPres{\•A} \subseteq \Sigma^*$ and for each $\+R_{(k)} \in \sigma$
a "synchronous relation" $\relPres{\+R}{\•A} \subseteq (\Sigma^*)^k$.

\AP A ""automatic graph"" is a "automatic $\sigma$-presentation" when $\sigma$ is the
"graph signature". It boils down a pair $\+G = \langle V, \+E \rangle$, where $V$
is a "regular language" over some alphabet $\Sigma$, and $\+E$ is a
"synchronous binary relation".

TODO:generalized presentation (non-injective) + effective construction to make them injective.
-> this explains our choice. We avoid the terminology ``regular structure'' because of
``regular graph''.

\section{Stuff Needed in Dichotomy Chapter}

\subsection{First-Order Reduction and First-Order Model Checking}

For the classical notion of \AP""first-order reductions"", see
\cite[Definition 2.11 \& Definition 1.26]{Immerman1998DescriptiveComplexity}.
Note that the complexity of a decision problem is traditionally measured
as a function of the size of the binary encoding of its input.
However, here, the notion of "first-order reductions" deals with
decision problems defined on "relational structures":
\begin{itemize}
	\item any ``classical'' decision problem, "ie" language $L \subseteq \{0,1\}^*$
		can be seen as a class of "structure" over the "signature of binary strings";
	\item for any "relational signature" $\sigma$,
		there is an encoding of "finite $\sigma$-structures" as "finite structures"
		over the "signature of binary strings", "i.e." as a ``classical''
		decision problem---see "eg" \cite[\S~2.2]{Immerman1998DescriptiveComplexity}.
\end{itemize}
Importantly, this last encoding is a "first-order reduction", proving that these reductions
make sense not only from a logical perspective, but also from a complexity-theoretic point of
view.

\begin{proposition}[Folklore]
	\label{prop:FO-in-L}
	"FOfin" $\subseteq$ "L".
\end{proposition}

\begin{proof}[Proof sketch]
	The naive recursive algorithm, recursing over the "first-order formula",
	works in logarithmic space: it suffices to keep one pointer to the "structure"
	for every variable of the "formula@@FO", but since this formula is fixed, we only
	require a constant number of pointers.
\end{proof}

Note that usually, "L"-hardness is defined using "first-order reductions".

What we said above work with the implicit assumptions that all structures at hand are 
"finite@@struct". However, part of this can be generalized to "automatic structures"---however, this comes to the cost of losing the inclusion in "L".

\begin{proposition}[Folklore]
	\AP\label{prop:first-order-reduction-preserve-automaticity}
	The image of an "automatic structure" by a "first-order reduction" is
	still an "automatic structure".
\end{proposition}
\begin{proof}
	TODO
\end{proof}

\decisionproblem{"First-order Model Checking of Automatic Structures"}{
	A "first-order formula" $\phi(x_1,\hdots,x_k)$ over $\sigma$,
	a "automatic presentation" $\•A$ of an "automatic $\sigma$-structure" $\?A$,
	and words $u_1,\hdots,u_k \in \domainPres{\+A}$.
}{
	Does $\?A, u_1,\hdots, u_k \models \phi$?
}

\begin{proposition}
	\!\footnote{See "eg" \cite[Theorem XII.1.7]{Blumensath2024MSOModelTheory}.}
	%
	\label{prop:first-order-model-checking-automatic-structures}
	"First-order model checking of automatic structures" is decidable in todo.
	Its "data complexity" is "NL"-complete.
\end{proposition}

\begin{proof}
	TODO.

	For the "NL" lower bound, reduction from NFA non-emptiness.
\end{proof}

We define \AP""FOaut"" to be the class of all problems over "automatic structures"
which are "first-order definable". By \Cref{prop:first-order-reduction-preserve-automaticity},
this class is closed under "first-order reductions". Moreover,
by \Cref{prop:first-order-model-checking-automatic-structures}, we obtain an upper bound.

\begin{proposition}
	"FOaut" $\subseteq$ "NL".
\end{proposition}


\subsection{A Model-Theoretic Perspective}

Given an alphabet $\Sigma$, we define on $\Sigma^*$:
\begin{itemize}
	\itemAP a predicate $\intro*\lastLetter{a}$ indicating that the last letter of a word is $a$,
	\itemAP a binary relation $\intro*\equalLength$ indicating that two words have the same length,
	\itemAP a binary relation $\intro*\prefix$ indicating that a word is a prefix of another.
\end{itemize} 
We denote by $\intro*\signatureSynchronous{\Sigma}$ the "signature" $\langle \langle\lastLetter{a}\rangle_{a \in \Sigma},\, \equalLength,\, \prefix \rangle$\footnote{We abusively use the same notations for
the "predicates" and their "interpretations@@predicate" in the "signature".} and
by \AP$\intro*\univStructSynchronous{\Sigma}$ the "$\signatureSynchronous{\Sigma}$-structure" over $\Sigma^*$ where
the "predicates" are "interpreted@@predicate" as above.

\begin{proposition}
	\label{prop:synchronous-first-order}
	A relation over $\Sigma^*$ is "synchronous" "iff" it is definable by a "first-order formula" over \(\univStructSynchronous{\Sigma}\).
\end{proposition}

\subsection{Neighbourhoods}

Given a "$\sigma$-structure" $\?A$ and $a \in A$, we define the \AP""neighbourhood"" of $a$
in $\?A$
to be the tuple of sets
\[
	\intro*\neighbourhood{a}{\?A}{\+R}{i} \defeq
	% \Big\langle
		\big\{
			\langle a_1, \hdots, a_{i-1}, a_{i+1}, \hdots, a_k \rangle \in A^{k-1} \mid
			\langle a_1, \hdots, a_{i-1}, a, a_{i+1}, \hdots, a_k \rangle \in \+R(\?A)
		\big\},
		% \;\big\vert\;
		% \+R_{(k)} \in \sigma \text{ and } i \in \lBrack 1,k\rBrack
	% \Big\rangle.
\]
when $\+R$ ranges over "predicate" of arity $k$ of $\sigma$ and $i \in \lBrack 1,k\rBrack$. 
\marginnote{TODO: introduce notation $\+R_{(k)} \in \sigma$}
For "graphs", the "neighbourhood" of a vertex corresponds to its set of predecessors and
its set of successors.

\begin{proposition}
	\AP\label{prop:neighbourhood-core}
	Given a "$\sigma$-structure" $\?B$, if $\?B$ is a "core", then
	two elements $b_1$ and $b_2$ of $\?B$ have the same "neighbourhood" "iff" $b_1 = b_2$.
\end{proposition}

\begin{proof}
	The right-to-left implication is trivial.
	For the converse one, consider the "homomorphism" from $\?B$ to itself
	which maps $b_2$ to $b_1$, and all elements of $B \smallsetminus \{b_2\}$
	to themselves. Since we assumed that $b_1$ and $b_2$ have the same "neighbourhood",
	this is indeed a "homomorphism", which is clearly not bijective, and
	hence by \Cref{prop:automorphism-core}, $\?B$ is not a "core".
\end{proof}

\subsection{Undirected Paths}

An \AP""undirected path"" in a "$\sigma$-structure" $\?A$ consists of a sequence
\[\big\langle a_0,\, e_0,\, a_1,\, \hdots,\, e_{n-1},\, a_n\big\rangle, \text{ with } n \in \N,\]
where $a_i \in \?A$ and each $e_i$ is a "hyperedge" of $\?A$ "st" both
$a_i$ and $a_{i+1}$ occur in $e_i$. When such an "undirected path" exists, we say that
there is an "undirected path" between $a_0$ and $a_n$, or equivalently
that $a_0$ and $a_n$ are \AP""connected"".\sidenote{Note that this relation is reflexive
and symmetric.} A \AP"connected component" of $\?A$ consists of an equivalence class of the 
transitive closure of this relation.

\subsection{De Bruijn–Erdős Theorem}

\AP\begin{proposition}[""De Bruijn–Erdős Theorem""]
	\!\footnote{It is straightforward to note that
	one can replace ``every finite "substructure"'' by
	``every finite "induced substructure"'' in the statement of the theorem.}%
	\AP\label{prop:de-bruijn-erdos}
	Let $\?A$ be an arbitrary "$\sigma$-structure" and $\?B$ a "finite $\sigma$-structure".
	There is a "homomorphism" from $\?A$ to $\?B$ "iff" for every finite "substructure" $\?A'$
	of $\?A$, there is a "homomorphism" from $\?A$ to $\?B$.
\end{proposition}

\begin{proof}
	The left-to-right implication is direct.
	We prove the converse by using the "Tychonoff's compactness theorem".\footnote{This is a direct adaptation from todo:addref-wiki, which proves this result in the particular case of graph colouring.}
	So, assume that for every finite "substructure" $\?A'$ of $\?A$,
	there is a "homomorphism" from $\?A$ to $\?B$. 
	Consider the topological space $B^A$, consisting of all functions from $A$ to $B$,
	together with the product topology.\footnote{We equip $B$ with the discrete topology,
	making it compact since $B$ is finite.} By "Tychonoff's compactness theorem",
	$B^A$ is compact. For each finite subset $X$ of $A$, let
	$H_X$ denote the set of all $f \in B^A$ "st" $\restr{f}{X}$ is a "homomorphism"
	from the "substructure" of $\?A$ "induced@@structure" by $X$ to $\?B$.
	Then, each $H_X$ is closed---indeed, whether $f\in B^A$ belongs to $H_X$ only depends
	on finitely many $f(x)$'s---, and moreover the intersection of finitely many
	$H_X$'s, say $H_{X_1} \cap \cdots \cap H_{X_n}$, is non-empty since
	$H_{X_1} \cap \cdots \cap H_{X_n} \supseteq H_{X_1\cup \hdots \cup X_n}$
	and by assumption $H_{X_1\cup \hdots \cup X_n}$ is non-empty since $X_1 \cup \cdots \cup X_n$ is finite. Hence, by compactness of $B^A$ and the "finite intersection property", it follows
	that $\bigcap_X H_X$ is non-empty, which means that there is a "homomorphism" from $\?A$ to $\?B$.
\end{proof}

\begin{corollary}
	\!\footnote{Another important consequence of the "De Bruijn–Erdős Theorem" is that,
	for instance, the notion of "dual" does not depend on whether we are considering finite or
	arbitrary "$\sigma$-structures".}
	\AP\label{coro:de-bruijn-erdos}
	Given arbitrary $\sigma$-structures $\?B_1$ and $\?B_2$, the following are equivalent:
	\begin{enumerate}
		\itemAP\label{item:de-bruijn-erdos-finite} for every finite "$\sigma$-structure" $\?A$, then $\?A \homto \?B_1$
		"iff" $\?A \homto \?B_2$;
		\itemAP\label{item:de-bruijn-erdos-arbitrary} for every arbitrary "$\sigma$-structure" $\?A$, then $\?A \homto \?B_1$
			"iff" $\?A \homto \?B_2$;
		\itemAP\label{item:de-bruijn-erdos-hom} $\?B_1$ and $\?B_2$ are "homomorphically equivalent".
	\end{enumerate}
\end{corollary}
\begin{proof}
	\eqref{item:de-bruijn-erdos-arbitrary} $\Rightarrow$ \eqref{item:de-bruijn-erdos-hom}
	and \eqref{item:de-bruijn-erdos-hom} $\Rightarrow$ \eqref{item:de-bruijn-erdos-finite}
	are trivial.
	For \eqref{item:de-bruijn-erdos-finite} $\Rightarrow$ \eqref{item:de-bruijn-erdos-arbitrary},
	we assume "wlog" by contradiction that there is an arbitrary "$\sigma$-structure" $\?A$ "st" $\?A \homto \?B_1$ but $\?A \nothomto \?B_2$. Then by \Cref{prop:de-bruijn-erdos},
	there exists a finite "substructure" $\?A_0$ of $\?A$ "st" $\?A_0 \nothomto \?B_2$.
	But then $\?A_0 \homto \?A \homto \?B_1$, which contradicts \eqref{item:de-bruijn-erdos-finite}.
\end{proof}