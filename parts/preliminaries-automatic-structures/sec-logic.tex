\section{A Logical Excursion}
\label{sec:preliminaries-automatic-structures-logic}

\subsection{First-Order Reduction and First-Order Model Checking}

For the classical notion of \AP""first-order reductions"", see
\cite[Definition 2.11 \& Definition 1.26]{Immerman1998DescriptiveComplexity}.
Note that the complexity of a decision problem is traditionally measured
as a function of the size of the binary encoding of its input.
However, here, the notion of "first-order reductions" deals with
decision problems defined on "relational structures":
\begin{itemize}
	\item any ``classical'' decision problem, "ie" language $L \subseteq \{0,1\}^*$
		can be seen as a class of "structure" over the "signature of binary strings";
	\item for any "relational signature" $\sigma$,
		there is an encoding of "finite $\sigma$-structures" as "finite structures"
		over the "signature of binary strings", "ie" as a ``classical''
		decision problem---see "eg" \cite[\S~2.2]{Immerman1998DescriptiveComplexity}.
\end{itemize}
Importantly, this last encoding is a "first-order reduction", proving that these reductions
make sense not only from a logical perspective, but also from a complexity-theoretic point of
view.

\begin{proposition}[Folklore]
	\label{prop:FO-in-L}
	"FOfin" $\subseteq$ "L".
\end{proposition}

\begin{proof}[Proof sketch]
	The naive recursive algorithm, recursing over the "first-order formula",
	works in logarithmic space: it suffices to keep one pointer to the "structure"
	for every variable of the "formula@@FO", but since this formula is fixed, we only
	require a constant number of pointers.
\end{proof}

Note that usually, "L"-hardness is defined using "first-order reductions".

What we said above work with the implicit assumptions that all structures at hand are 
"finite@@struct". However, part of this can be generalized to "automatic structures"---however, this comes to the cost of losing the inclusion in "L".

\begin{proposition}[Folklore]
	\AP\label{prop:first-order-reduction-preserve-automaticity}
	The image of an "automatic structure" by a "first-order reduction" is
	still an "automatic structure".
\end{proposition}
\begin{proof}
	TODO
\end{proof}

\decisionproblem{""First-Order Model Checking of Automatic Structures""}{
	A "first-order formula" $\phi(x_1,\hdots,x_k)$ over $\sigma$,
	an "automatic presentation" $\•A$ of an "automatic $\sigma$-structure" $\?A$,
	and words $u_1,\hdots,u_k \in \domainPres{\+A}$.
}{
	Does $\?A, u_1,\hdots, u_k \FOmodels \phi$?
}

\begin{proposition}
	\!\footnote{See "eg" \cite[Theorem XII.1.7]{Blumensath2024MSOModelTheory}.}
	%
	\label{prop:first-order-model-checking-automatic-structures}
	"First-order model checking of automatic structures" is decidable in todo.
	Its "data complexity" is "NL"-complete.
\end{proposition}

\begin{proof}
	TODO.

	For the "NL" lower bound, reduction from NFA non-emptiness.
\end{proof}

We define \AP""FOaut"" to be the class of all problems over "automatic structures"
which are "first-order definable". By \Cref{prop:first-order-reduction-preserve-automaticity},
this class is closed under "first-order reductions". Moreover,
by \Cref{prop:first-order-model-checking-automatic-structures}, we obtain an upper bound.

\begin{proposition}
	"FOaut" $\subseteq$ "NL".
\end{proposition}


\subsection{A Model-Theoretic Perspective}

Given an alphabet $\Sigma$, we define on $\Sigma^*$:
\begin{itemize}
	\itemAP a predicate $\intro*\lastLetter{a}$ indicating that the last letter of a word is $a$,
	\itemAP a binary relation $\intro*\equalLength$ indicating that two words have the same length,
	\itemAP a binary relation $\intro*\prefix$ indicating that a word is a prefix of another.
\end{itemize} 
We denote by $\intro*\signatureSynchronous{\Sigma}$ the "signature" $\langle \langle\lastLetter{a}\rangle_{a \in \Sigma},\, \equalLength,\, \prefix \rangle$%
\footnote{For the sake of simplicity, we abusively use the same notations for
the "predicates" and their "interpretations@@predicate" in the "signature".} and
by \AP$\intro*\univStructSynchronous{\Sigma}$ the "$\signatureSynchronous{\Sigma}$-structure" over $\Sigma^*$ where
the "predicates" are "interpreted@@predicate" as above.

\begin{proposition}
	\label{prop:automatic-first-order}
	A relation over $\Sigma^*$ is "automatic@@rel" "iff" it is definable by a "first-order formula" over \(\univStructSynchronous{\Sigma}\).
\end{proposition}