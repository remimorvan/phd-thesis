\section{A Logical Excursion}
\label{sec:preliminaries-automatic-structures-logic}

\subsection{First-Order Interpretation}

A major construction in logic is to restrict the scope of the class
of models---thereafter called ``universe''.
Formally, given a universe $\+U$ and a class of "queries@@sem" $\+Q$,%
\footnote{Recall that we see "queries@@sem" here in a purely semantical way: it is nothing but a subclass of $\+U$.}
given a subclass $\+V$ of $\+U$, we can consider the restriction
\[
	\restr{\+Q}{\+V} \defeq \{\phi \cap \+V \mid \phi \in \+Q\}.
\]
For instance, letting $\+U$ be the class of all "$\sigma$-structures" and $\+Q$ be the set of all 
"first-order sentences" over $\sigma$, by taking $\+V$ to be the set of all \emph{finite}
"$\sigma$-structures", then $\restr{\+Q}{\+V}$ precisely corresponds to
the "first-order sentences" over "finite structures". For instance, the class of all
\emph{finite} "$\sigma$-structures" belongs to $\restr{\+Q}{\+V}$ but not to
$\+Q$.

However, this construction does not preserve any reasonable property on $\+Q$,
since $\+V$ is an arbitrary subclass of $\+U$. The idea behind "first-order interpretation"
is precisely to circumvent this problem, by \emph{interpreting} a class of structures
inside another one by using "first-order logic", allowing this way to preserve some properties.

Let $\sigma$ and $\tau$ be "relational signatures".
A \AP ""$d$-dimensional first-order interpretation"" $\+I$ consists of the following tuple of
"first-order formulas" over $\sigma$:
\begin{itemize}
	\itemAP $\intro*\domainInter{\+I}(x^1,\dotsc, x^d)$ (specifying the domain)
	\itemAP $\intro*\relInter{=}{\+I}(\tup{x^1,\dotsc, x^d}, \tup{y^1,\dotsc, y^d})$ (specifying equality)
	\itemAP $\reintro*\relInter{\+R}{\+I}(\tup{x^1_1,\dotsc,x^d_1}, \dotsc,
		\tup{x^1_k,\dotsc,x^d_k})$ for any predicate $\+R \in \tau$ of arity $k$,
\end{itemize}
such that $\relInter{=}{\+I}$ defines an equivalence relation on the domain of $\+I$.

Given a "$\sigma$-structure" $\?A$, the \AP""$\+I$-interpretation of $\?A$"" is the "$\tau$-structure" denoted by \AP$\intro*\interpretation{\+I}{\?A}$ defined as follows:%
\footnote{For the sake of readability, we use $\bar a$ or $\bar x$ to
denote $d$-tuples of elements of $\?A$ and $d$-tuples of variables, respectively.}
\begin{itemize}
	\item its domain is the quotient of 
		$\{\bar a \in \?A^d \mid \?A, \bar a \FOmodels \domainInter{\+I}(\bar x)\}$
		by the equivalence relation
		$\set{\tup{\bar a, \bar b} \mid \?A, \bar a, \bar b \FOmodels \relInter{=}{\+I}(\bar x, \bar y)}$,
	\item for any predicate $\+R_{(k)} \in \tau$,
		we have
		\[
			\+R_{(k)}(\interpretation{\+I}{\?A}) =
			\Big\{\tup{\bar x_1, \dotsc, \bar x_k} \mid
				\exists \bar y_1.\, \dotsc.\, \exists \bar y_k.\,
				\bigwedge_{i=1}^k \bar x_i \mathrel{\relInter{=}{\+I}} \bar y_i
				\land \relInter{\+R}{\+I}(\bar y_1, \dotsc, \bar y_k)
			\Big\}.
		\]
\end{itemize}
For instance, letting $\sigma =\tau$ be the "graph signature", consider
the "one-dimensional interpretation" $\+I$ where:
\begin{itemize}
	\item the domain formula keeps all vertices,
	\item the equality formula is proper equality,
	\item the formula for the edge predicate puts two variables $x$ and $y$ in relation
		if either there is an edge from $x$ to $y$, or if $x$ has no successor and $y$ has no predecessor.
\end{itemize}
Then, the $\+I$-interpretation of a directed path is a cycle,
see \Cref{fig:interpretation-path-into-cycle}.
\begin{marginfigure}
	\centering
	\begin{tikzpicture}
		\node[vertex, draw=c2, fill=c2, fill opacity=.4] at (0,0) (2) {};
		\node[vertex, left=of 2, draw=c1, fill=c1, fill opacity=.4] (1) {};
		\node[vertex, left=of 1, draw=c0, fill=c0, fill opacity=.4] (0) {};

		\draw[edge] (0) to (1);
		\draw[edge] (1) to (2);
	\end{tikzpicture}
	\qquad
	\begin{tikzpicture}
		\node[vertex, draw=c2, fill=c2, fill opacity=.4] at (0,0) (2) {};
		\node[vertex, left=of 2, draw=c1, fill=c1, fill opacity=.4] (1) {};
		\node[vertex, left=of 1, draw=c0, fill=c0, fill opacity=.4] (0) {};

		\draw[edge] (0) to (1);
		\draw[edge] (1) to (2);
		\draw[edge, bend right=60] (2) to (0);
	\end{tikzpicture}
	\caption{
		\AP\label{fig:interpretation-path-into-cycle}
		A directed path (left) and its "interpretation" by $\+I$ (right),
		that adds an edge from any vertex with no successor to any vertex with no predecessor.
	}
\end{marginfigure}

The idea behind "first-order interpretation" is that
"first-order formulas" about the interpretation $\interpretation{\+I}{\?A}$
can be translated into "first-order formulas" over $\?A$.
Formally, given a "first-order formula" $\phi(x_1,\dotsc,x_k)$ over $\tau$,
we can define a "first-order formula" $\phi^{\+I}(\bar x_1,\dotsc,\bar x_k)$ over $\sigma$
defined inductively by:
\begin{itemize}
	\item replacing each variable $x$ by a $d$-tuple $\bar x$,
	\item replacing every occurrence of
		$\+R(x_1,\dotsc, x_k)$ (with $\+R_{(k)} \in \tau$)
		by the formula
		\[\exists \bar x'_1.\, \dotsc \exists \bar x'_k.\, 
		\Big( \bigwedge_{i=1}^k \bar x_i \mathrel{\relInter{=}{\+I}} \bar x'_i \Big)
		\land \relInter{\+R}{\+I}(\bar x'_1, \dotsc, \bar x'_k),\]
	\item relativizing quantification with respect to $\domainInter{\+I}$,
		meaning that
		\begin{align*}
			(\exists x.\, \phi(x))^{\+I} 
			& \defeq \exists \bar x.\, \domainInter{\+I}(\bar x) \land \phi^{\+I}(\bar x),
			\text{ and} \\ 
			(\forall x.\, \phi(x))^{\+I} 
			& \defeq \forall \bar x.\, \domainInter{\+I}(\bar x) \Rightarrow \phi^{\+I}(\bar x).
		\end{align*}
\end{itemize} 
\begin{proposition}
	\AP\label{prop:first-order-interpretation}
	For any "first-order formula" $\phi(x_1,\dotsc,x_k)$ over $\tau$,
	for any "pointed $\sigma$-structure" $\tup{\?A, \bar a_1,\dotsc, \bar a_k}$, we have:
	\[
		\tup{\interpretation{\+I}{\?A}, \equivclass{\bar a_1}, \dotsc, \equivclass{\bar a_k}}
		\FOmodels \phi(x_1, \dotsc, x_k)
		\quad\text{"iff"}\quad
		\tup{\?A, \bar a_1,\dotsc, \bar a_k} \FOmodels \phi^{\+I}(\bar x_1,\dotsc,\bar x_k).
	\]
\end{proposition}
And so, in particular, if $\+C$ is a class of structures, then
model checking (resp. satisfiability, resp. validity) over the class $\interpretation{\+I}{\+C}$
can be reduced to the same problem over $\+C$.

Typical examples of "one-dimensional interpretations" include
restricting a structure to a first-order definable subset, 
or taking the complement---that is, swapping "hyperedges" and "non-hyperedges".

\begin{marginfigure}
	\centering
	\begin{tikzpicture}
		\tikzset{every node/.style={fill opacity=.4}}

		\node[vertex] at (0,0) (0-0) {};
		\foreach \j in {1,2} {
			\pgfmathtruncatemacro{\prev}{\j - 1}
			\pgfmathtruncatemacro{\jc}{mod(\j,2) ? 2 : 1}
			\node[vertex, right=2em of 0-\prev] (0-\j) {};
		}
		\foreach \i in {1,2} {
			\pgfmathtruncatemacro{\prev}{\i - 1}
			\foreach \j in {0,1,2} {
				\pgfmathtruncatemacro{\jc}{mod(\i+\j,2) ? 2 : 1}
				\node[vertex, above=2em of \prev-\j] (\i-\j) {};
			}
		}

		\foreach \i in {0,1,2} {
			\foreach \j in {0,1} {
				\draw[edge] (\i-\j) to (\i-\the\numexpr\j+1\relax);
			}
		}
		\foreach \i in {0,1} {
			\foreach \j in {0,1,2} {
				\draw[edge] (\i-\j) to (\the\numexpr\i+1\relax-\j);
			}
		}
	\end{tikzpicture}
	\caption{
		\AP\label{fig:interpretation-path-into-grid}
		"Interpretation" of the directed path of
		\Cref{fig:interpretation-path-into-cycle} by the
		``box product interpretation''.
	}
\end{marginfigure}
\begin{example}
	We consider a 2-dimensional interpretation that we call ``box product''.
	Let $\sigma$ and $\tau$ be the "graph signature".
	We define a "two-dimensional interpretation" $\+I$ with trivial domain and equality%
	\footnote{Meaning formally that \[\domainInter{\+I}(x^1,x^2) \defeq \top,\text{ and}\]
	and $\relInter{=}{\+I}(\tup{x^1,x^2}, \tup{y^1,y^2})$ is
	the formula $x^1 = y^1 \land x^2 = y^2$.}
	and we put an edge from $\tup{x^1,x^2}$ to $\tup{y^1,y^2}$ if either
	there is an edge from $x^1$ to $y^1$ and $x^2 = y^2$, or if
	$x^1 = y^1$ and there is an edge from $x^2$ to $y^2$.
	Then the interpretation of a directed path is a directed grid,
	see \Cref{fig:interpretation-path-into-grid}.
\end{example}

An "interpretation" is said to be \AP""injective@@interpretation"" if the equality formula
is the proper tuple-equality.
Note that each "interpretation" $\+I$ can be transformed into
an "injective interpretation" $\+J$ "st" 
for any \emph{totally ordered structure} $\?A$---meaning that there is a binary predicate
in the "signature" that is "interpreted@@pred" as a total order in $\?A$,
then $\interpretation{\+I}{\?A}$ and $\interpretation{\+J}{\?A}$ are "isomorphic":
for this, it suffices to encode any equivalence class in $\?A$ by its minimal element---for more
details, see \Cref{prop:making-presentations-injective}.

We say that $\?A$ is \AP""first-order interpretable"" in $\?B$ if there exists
a "first-order interpretation" $\+I$ "st" $\interpretation{\+I}{\?A}$ is "isomorphic" to $\?B$.
We say that two structures are \AP""first-order equivalent"" if they are "first-order interpretable" in one another. Two structures that are "isomorphic" are clearly
"first-order equivalent". Moreover, if a $k$-ary relation $\+R$ is "first-order definable"
in a "$\sigma$-structure" $\?A$, then $\?A$ is "first-order equivalent" to the "structure"
$\?A$ to which we add a new $k$-ary "predicate interpreted" as $\+R$.
For instance, $\tup{\N, +}$ and $\tup{\N, +, <, 0, 1}$ are "first-order equivalent".

\subsection{First-Order Reduction and First-Order Model Checking}

As we have seen, "first-order interpretations" preserve logical properties of classes
of "structures". In this part, we show that most complexity classes
are closed under this construction.

We define a \AP""first-order reductions"" to be an "injective first-order interpretation".%
\footnote{The terminology ``"FO-interpretations"'' is usually employed in automata theory and graph theory, while ``"FO-reduction"'' is used in complexity theory---see "eg" \cite[Definition 2.11 \& Definition 1.26]{Immerman1998DescriptiveComplexity}---however there is no good reason to
distinguish these two notions.}
Complexity classes are defined in terms of decision problems, namely 
languages $L \subseteq \2^*$, rather than as classes of "structures". 
However, it is known that:
\begin{itemize}
	\item any language $L \subseteq \2^*$
		can be seen as a class of "structures" over the "signature of binary strings";
	\item for any "relational signature" $\sigma$,
		there is an encoding of "finite $\sigma$-structures" as a language
		$L \subseteq \2^*$, "ie" as a class of "structure" over the "signature of binary strings"---see "eg" \cite[\S~2.2]{Immerman1998DescriptiveComplexity}.
\end{itemize}
Importantly, these encodings between "$\sigma$-structures" and structures over
the "signature of binary strings" are both "first-order reductions".
It follows that there is a "first-order reduction" between two classes of "structures"
"iff" there is a "first-order reduction" between their encodings as languages.
Overall, it implies that "first-order reductions" are proper reductions in the complexity-theoretic 
sense. We will see that first-order logic actually lives in the lowest levels of the hierarchy of 
complexity classes.

We denote by \AP""FOfin"" the class of all problems over "finite structures" that are "first-order definable".\footnote{We can either assume the "signature" to be fixed, or to be part of the input.}

\begin{proposition}[Folklore]
	\label{prop:FO-in-L}
	"FOfin" $\subseteq$ "L".
\end{proposition}

\begin{proof}[Proof sketch]
	The naive recursive algorithm, recursing over the "first-order formula",
	works in logarithmic space: it suffices to keep one pointer to the "structure"
	for every variable of the "formula@@FO", but since this formula is fixed, we only
	require a constant number of pointers.
\end{proof}

Moreover, "L"-hardness is usually defined using "first-order reductions".
When the "signature" is sufficiently expressive, namely when it is able to
express basic arithmetic \cite[Proviso~1.14]{Immerman1998DescriptiveComplexity},
Immerman showed that "FOfin" corresponds to the circuit-class "AC0",
and also to the whole logarithmic-time hierarchy
\cite[Corollary~5.32]{Immerman1998DescriptiveComplexity}.


\subsection{A Model-Theoretic Perspective on Automatic Relations}

Given an alphabet $\Sigma$, we define on $\Sigma^*$:
\begin{itemize}
	\itemAP a unary relation $\intro*\lastLetter{a}$ indicating that the last letter of a word is $a$,
	\itemAP a binary relation $\reintro*\equalLength$ indicating that two words have the same length,
	\itemAP a binary relation $\reintro*\prefix$ indicating that a word is a prefix of another.
\end{itemize} 
We denote by $\intro*\signatureSynchronous{\Sigma}$ the "signature" $\langle \langle\lastLetter{a}\rangle_{a \in \Sigma},\, \equalLength,\, \prefix \rangle$%
\footnote{For the sake of simplicity, we abusively use the same notations for
the "predicates" and their "interpretations@@predicate" in the "signature".} and
by \AP$\intro*\univStructSynchronous{\Sigma}$ the "$\signatureSynchronous{\Sigma}$-structure"
over $\Sigma^*$ where the "predicates" are "interpreted@@predicate" as above.%
\footnote{This model is also denoted $\symbf{S}_{\mathrm{len}}$ in \cite{BenediktLibkinSchwentickSegoufin2003DefinableRelations}. In this model $\lastLetter{a}$ is
not a unary predicate but a function that adds an `$a$' to the end of the word: of course this
makes no difference whatsoever since both models are "first-order equivalent".}

\begin{proposition}[{\cite[Theorems~1 \& 2]{EilenbergElgotShepherdson1969TapeAutomata}}]
	\AP\label{prop:automatic-first-order}
	If $\Sigma$ has at least two letters, then a relation over $\Sigma^*$
	is "automatic@@rel" "iff" it is "first-order definable" in
	$\univStructSynchronous{\Sigma}$.\footnote{When $\Sigma$ has a single letter,
	then the right-to-left implication holds, but not the converse one.
	For instance, it can be shown that $(aa)^*$ is not "first-order definable"
	in $\univStructSynchronous{\{a\}}$ \cite[\S~9]{EilenbergElgotShepherdson1969TapeAutomata}.}
\end{proposition}

\begin{proof}
	\proofcase{From logic to automata.}
	This implication is easy: it suffices to observe first that each
	relation $\lastLetter{a}$, $\equalLength$ and $\prefix$ is "automatic@@rel",
	and then to use the fact that "automatic relations" are closed under
	Boolean operations and quantification. This last point can be proven
	using a powerset construction on the automaton.

	\proofcase{From automata to logic.}
	The converse implication is less straightforward. It was originally proven by
	Eilenberg, Elgot and Shepherdson starting from a rational expression, making
	the proof somewhat tedious.
	In \cite[\S~``1969'']{Choffrut2006Survey}, Choffrut notes that
	``mimicking the automaton yields a much
	more intuitive proof and can be reconstructed by a good PhD student'':
	we present here this proof.
	
	We start with an example: we want to build a "first-order formula" $\phi$ over $\signatureSynchronous{\2}$ for the language $(00)^*$, in the sense
	that for all $u\in \2^*$,
	\[
		\tup{\univStructSynchronous{\2}, u} \FOmodels \phi(x) \quad\text{iff}\quad u \in (00)^*.
	\]
	For this, it suffices to guess a word $v$ of the same length as $u$, which alternates
	zeroes and ones, and then to check that the first letter $v$ is distinct from
	its last letter.

	Given a "synchronous automaton" $\+A$ recognizing $\+R \subseteq (\Sigma^*)^k$,
	we want to build a "first-order formula" $\phi_{\+A}$ over $\signatureSynchronous{\Sigma}$
	"st" for all $\tup{u_1,\dotsc, u_k} \in (\Sigma^*)^k$:
	\[
		\tup{u_1,\dotsc, u_k} \in \+R
		\quad\text{iff}\quad
		\tup{\univStructSynchronous{\Sigma}, u_1,\dotsc, u_k} \FOmodels \phi_{\+A}(x_1,\dotsc,x_k).
	\]
	The formula $\phi_{\+A}$ is built as follows:
	\begin{itemize}
		\item for each state $q$ of $\+A$, we guess a word $v_q \in \2^*$ "st"
			the length of $v_q$ is the maximal length of a word $u_i$ ($i \in \intInt{1,k}$),
			"ie" $|u_1 \convol \cdots \convol u_k|$,
		\item we check that for each position $i$, there is exactly one state $q$
			"st" the $i$-th letter of $(v_q)$ is 1,
		\item this way, the tuple of words $\tup{v_q}_{q}$ represents a potential run of $\+A$
			over $\+A$: we then check that this indeed defines an accepting run of the automaton. 
	\end{itemize}
	It is routine to check that all these properties can be written in "first-order logic"
	using the predicates $\lastLetter{a}$, $\equalLength$ and $\prefix$.
	Note in particular that this proof crucially relies on the fact that we can guess words
	in $\2^*$, which is allowed by the assumption that $\Sigma$ has size at least 2.%
	\footnote{Indeed, we can then assume "wlog" that $\2 \subseteq \Sigma$, and then
	check in "first-order logic" that a word $v_q \in \Sigma^*$ actually
	belongs to $\2^*$.}
	Also, the formula $\phi_{\+A}$ is of polynomial size in the size of $\+A$.
\end{proof}

We want to highlight that \Cref{prop:automatic-first-order} can be extended for $k=0$,
since the $0$-ary relations are the subsets of $\emptyset^* = \{\varepsilon\}$,
and so there is exactly two 
relations: $\{\textbullet\}$ (identified with ``true'') and $\emptyset$ (identified with ``false'').
Both are "first-order definable", and "automatic@@rel": observe that "synchronous automata" over
$\emptyset$ either return ``true'' if they have an initial state that is accepting, and otherwise they return ``false''.
$0$-ary relations naturally arise because they correspond to Boolean queries.
For instance, assume that $\phi(x_1,x_2,x_3)$ is a "first-order formula" over
$\signatureSynchronous{\Sigma}$: then
$\forall x_1.\, \exists x_2.\, \forall x_3.\, \phi(x_1,x_2,x_3)$ is a "first-order sentence"
over $\signatureSynchronous{\Sigma}$. The proof of \Cref{prop:automatic-first-order}
builds from a "synchronous ($3$-ary) automaton@synchronous automaton" for
$\phi(x_1,x_2,x_3)$ a triple-exponential-sized "synchronous ($0$-ary) automaton@synchronous automaton" for $\forall x_1.\, \exists x_2.\, \forall x_3.\, \phi(x_1,x_2,x_3)$.
% Moreover, the structure $\univStructSynchronous{\Sigma}$ is sometimes described by replacing
% the unary "predicate" $\lastLetter{a}$ by the function $w \mapsto w\cdot a$.
% The two models are clearly "first-order equivalent".

Let us point out that the formula built from the automaton is of the form
$\exists^* \forall^* \exists^*$ followed by a quantifier-free formula.
It shows that "first-order logic" over $\univStructSynchronous{\Sigma}$
collapses to the $\Sigma^3$ level. Figueira, Ramanathan and Weil
proved that this hierarchy does not collapse to a lower level 
\cite[Theorem~3]{FigueiraRamanathanWeil2019QuantifierAlternation},
and provide effective characterization of these lower levels
\cite[Theorem~9]{FigueiraRamanathanWeil2019QuantifierAlternation}.

\begin{example}
	\AP\label{ex:lexicographic-is-automatic}
	Given an ordered alphabet $\Sigma = \{a_1,\dotsc,a_n\}$ with $a_1 < \dotsc < a_n$,
	we define the \AP""lexicographic order"" by $u \intro*\lex v$ "iff" when there is 
	a prefix $w$ both of $u$ and of $v$, "st" either $w$ has the same length as $u$,
	or the letter following $w$ in $u$ is strictly smaller than the letter following $w$ in $v$.
	
	Note that this can be defined by a "first-order formula" over $\signatureSynchronous{\Sigma}$
	since saying that the letter following $w$ in $u$ is $a$ amounts to
	guessing the smallest prefix $w'$ of $u$ that must strictly contain $w$ as a prefix,
	and checking that $w'$ ends with letter $a$.
	From \Cref{prop:automatic-first-order}, it follows that the "lexicographic order"
	is an "automatic relation".
\end{example}

\begin{remark}
	\AP\label{rk:internal-vs-external-logic}
	We want to highlight that, in the case of unary relations,
	\Cref{prop:automatic-first-order} shows that "regular languages" are exactly
	the "first-order definable" sets of $\univStructSynchronous{\Sigma}$
	(if $\Sigma$ has at least two letters).
	This result contradicts in no way the fact that "regular languages" exactly those definable
	in "monadic second-order logic", nor does it imply that "monadic second-order logic"
	collapses to "first-order logic". Indeed, in this last characterization,
	the models are finite words, and its first-class citizens%
	\footnote{By ``first-class citizens'' we mean the objects over which
	we can do first-order quantification.} are the positions in these words.
	In the case of \Cref{prop:automatic-first-order}, the model is fixed (namely $\univStructSynchronous{\Sigma}$), and its first-class citizens are the finite words.
	In some sense, this logic is \emph{external} to finite words---the logic does not have access to the
	``inner workings'' of the words---, while monadic second-logic is \emph{internal}
	as traditionally used.
	Overall, \Cref{prop:automatic-first-order} can be rephrased as the equivalence between
	\emph{external} first-order logic and \emph{internal} "monadic second-order logic".%
	\footnote{We thank Géraud Sénizergues for suggesting this terminology to us.}
	Said otherwise, when going from "FO" over the structure $\univStructSynchronous{\Sigma}$ to "MSO" 
	over the structure of a finite word, we gain one level of quantification "via" the logic,
	but we lose one level of quantification "via" the change of model.
\end{remark}

\subsection{Logical Characterization of Other Classes of Relations}

Benedikt, Libkin, Schwentick and Segoufin studied submodels of
\[\univStructSynchronous{\Sigma} =
\tup{\Sigma^*, \langle\lastLetter{a}\rangle_{a \in \Sigma},\, \equalLength,\, \prefix}\]
in \cite{BenediktLibkinSchwentickSegoufin2003DefinableRelations}.
For instance, they show that by removing $\equalLength$, the definable sets are exactly
the star-free regular languages
\cite[Corollary 3.7]{BenediktLibkinSchwentickSegoufin2003DefinableRelations}.

Adding a binary predicate for each "regular language" $L$, "interpreted@@pred"
as $\{\tup{u,uv} \mid u,v \in \Sigma^*, v \in L \}$, yields
a logic called $\symbf{S}_{\mathrm{reg}}$, whose "definable relations"
are exactly the ``regular prefix relations'', also called ``special relations''
\cite[Corollary 3.22]{BenediktLibkinSchwentickSegoufin2003DefinableRelations}.
This class lives between "recognizable relations" and "automatic relations",
see \cite[\S~``1984'']{Choffrut2006Survey}.