\section{Automatic Structures}
\label{sec:preliminaries-automatic-structures-automatic-structures}

todo:blabla

We fix a finite "relational signature" $\sigma$.
An \AP""automatic presentation $\+A$ of a $\sigma$-structure"" consists of:
\begin{itemize}
	\item an alphabet $\Sigma$,
	\itemAP a "regular language" $\intro*\domainPres{\+A} \subseteq \Sigma^*$,
	\itemAP for every relation $\+R_{(k)} \in \sigma$, an
		"automatic relation" $\intro*\relPres{\+R}{\+A} \subseteq (\Sigma^*)^k$, and
	\itemAP a binary "automatic relation" $\reintro*\relPres{=}{\+A} \subseteq
		\Sigma^* \times \Sigma^*$. 
\end{itemize} 
The \AP""structure represented"" $\?A$ by ""an automatic presentation"" $\+A$ has
$\domainPres{\+A}/{\relPres{=}{\+A}}$ as its domain, and the predicate $\+R_{(k)} \in \sigma$
is "interpreted@@pred" in such a way that a tuple $\tup{X_1,\hdots,X_k}$ of equivalence classes
belongs to $\+R(\?A)$ if, and only if, there exists $\tup{u_i}_{1 \leq i \leq k} \in\tup{X_i}_{1 \leq i \leq k}$ "st" $\tup{u_1,\hdots,u_k} \in \relPres{\+R}{\+A}$.
We say that a "$\sigma$-structure" is ""automatic@@struct"" if it is
"represented@@struct" by ""an automatic presentation"".%
\footnote{Note that we can always assume "wlog" that $\relPres{\+R}{\+A} \subseteq
(\domainPres{\+A})^k$ since $(\domainPres{\+A})^k$ is "automatic@@rel" by
\Cref{prop:rec-implies-aut} and "automatic relations" are closed under intersection.}

For instance, the infinite binary tree admits can be represented by the "automatic presentation"
$\+B$ with $\domainPres{\+B} = \2^*$,
\[
	\relPres{\+E}{\+B} =
	\{\tup{u,u0} \mid u \in \2^*\} \cup \{\tup{u,u1} \mid u \in \2^*\},
\]
and $\relPres{=}{\+B}$ is equality.
todo: figure.
Presentations such as this one, where $\relPres{=}{\+B}$ is the equality relation,
are called ""injective presentations"".

\begin{example}
	The structure $\tup{\N, +}$ is "automatic".%
	\footnote{We see $+$ as a ternary relation, given by $\{\tup{x,y,x+y} \mid x, y \in \N^2\}$.}
	We build an "automatic presentation" $\+N$ by using
	a binary encoding $\domainPres{\+N} = \2^*$. A word $w \in \2^*$ will
	represent the number $\sum_{i=1}^{|w|} w_i\cdot 2^{i-1}$. 
	Naturally, it follows that $\relPres{=}{\+N}$ puts two words $u$ and $v$ in relation
	if they are equal after removing the trailing zeroes.
	We then need to describe $\relPres{+}{\+N}$: the idea is to simulate addition, reading words
	from left to right, by using two states ($0$ and $1$) to remember the carry.
	\begin{marginfigure}
		\centering
		\begin{tikzpicture}[shorten >= 1pt, node distance = 1.8cm, on grid, baseline]
			\node[state, initial left, accepting] (q0) {}; 
			\node[state, right=of q0] (q1) {}; 
			\path[->]
				(q0) edge[loop above] node[font=\scriptsize, above left = 0em and -1.1em] {$\triple{0}{0}{0},
					\triple{0}{1}{1},\triple{1}{0}{1}$} (q0)
				(q0) edge[bend left] node[font=\scriptsize, above] {$\triple{1}{1}{0}$} (q1)
				(q1) edge[loop below] node[font=\scriptsize, below right = 0em and -1.1em] {$\triple{0}{1}{0},\triple{1}{0}{0},\triple{1}{1}{1}$} (q1)
				(q1) edge[bend left] node[font=\scriptsize, below] {$\triple{0}{0}{1}$} (q0);
		\end{tikzpicture}
		\caption{
			\AP\label{fig:automaton-presburger}
			"Synchronous automaton" describing the addition of
			natural numbers. For the sake of readability, transitions involving
			the "padding symbol" are not represented: $\pad$ is treated as a zero.
		}
	\end{marginfigure}
	For instance, the transition
	$\textcolor{c0}{0}
	\transition{\tup{\textcolor{c1}{0},\textcolor{c1}{1},\textcolor{c3}{1}}}
	\textcolor{c2}{0}$ can be understood as
	``when adding $\textcolor{c1}{0}$ and $\textcolor{c1}{1}$, with current carry $\textcolor{c0}{0}$, the results equals $\textcolor{c3}{1}$, with a carry of $\textcolor{c2}{0}$ for the next bit''.
	In general, we have a transition
	\[
		\textcolor{c0}{p}
		\transition{\tup{\textcolor{c1}{x},\textcolor{c1}{y},\textcolor{c3}{z}}}
		\textcolor{c2}{q}
		\quad\text{"iff"}\quad
		\textcolor{c1}{x} + \textcolor{c1}{y} + \textcolor{c0}{p}
		= \textcolor{c3}{z} + 2\textcolor{c2}{q}
	\]
	for all $p,q \in \2$ and
	$\tup{x,y,z} \in \2 \convolAlpha \2 \convolAlpha \2$.%
	\footnote{Of course, we identify $\pad$ with $0$.}
	This gives the automaton of figure \Cref{fig:automaton-presburger}.
	By construction, $\+N$ "represents@@struct" $\tup{\N, +}$.
\end{example}

Recall that, using \Cref{prop:automatic-first-order}, a "relation" is "automatic@@rel" 
if and only if they are "first-order definable" over $\signatureSynchronous{\Sigma}$.
It means that we can alternatively see "automatic presentations" as a
collection of first-order formulas---one for the domain, one for equality, and
one for each predicate. In turns, this view helps us prove the following result.
For instance, $\langle \univStructSynchronous{\Sigma}, \langle\lastLetter{a}\rangle_{a \in \Sigma},\, \equalLength,\, \prefix \rangle$ is an "automatic structure".


\begin{proposition}
	\label{prop:making-presentations-injective}
	Any "automatic structures" admits an "injective presentation".
\end{proposition}

\begin{proof}
	We prove this property using logic. Let $\+A$ be an arbitrary "automatic presentation"
	that "represents@@struct" $\?A$. The idea to build an "injective automatic presentation" $\+A'$
	of $\?A$ is to represent an equivalence class $\equivclass[\relPres{=}{\+A}]{u} \subseteq
	\domainPres{\+A}$ by its "lexicographic"-minimal element.%
	\footnote{This makes sense as the "lexicographic order" is a well-founded total order.}

	Concretely, $\+A'$ has the same alphabet as $\+A$.
	Then, we define $\domainPres{\+A'}$ as the set of lexicographic-minimal elements of
	$\equivclass[\relPres{=}{\+A}]{u}$ ($u \in \domainPres{\+A}$),
	which can be described by the "first-order formula"%
	\footnote{For the sake of readability, we identify first-order formulas with the relations they define.}
	\[
		\domainPres{\+A'}(x) \defeq
		\forall y.\; (\domainPres{\+A}(y) \land x \mathrel{\relPres{=}{\+A}} y) \Rightarrow 
		x \lex y.
	\]
	In is indeed a "first-order formula" since $\+A$ is assumed to be "automatic@@pres",
	and since $\lex$ is "automatic@@rel" by \Cref{ex:lexicographic-is-automatic}.
	Equality is interpreted as proper equality,
	and then
	\[
		\relPres{\+A'}{\+R}(x_1,\hdots, x_k) \defeq
		\exists y_1.\,\hdots \exists y_k.\,
		\Big( \bigwedge_{i=1}^k x_i \mathrel{\relPres{=}{\+A}} y_i \Big)
		\land \relPres{\+A}{\+R}(y_1,\hdots, y_k),
	\]
	for any predicate $\+R_{(k)} \in \sigma$.
	Overall, $\+A'$ is an "injective presentation" that "represents@@struct" 
	exactly the same "structure" as $\+A$.
\end{proof}

\begin{proposition}
	\label{prop:making-presentations-binary}
	Any "automatic structures" admits a "presentation" with an alphabet of size 2.
\end{proposition}

\begin{proof}
	The idea is to encode each letter of $\Sigma = \{a_1,\hdots,a_n\}$
	over $\2$ in such a way that each letter $a_i$
	is encoded over a word $\hat a_i \in \2^*$ "st"
	all $\hat a_i$ have the same length---this is necessary to preserve "automaticity".
	For instance, take $\hat a_i$ to be the binary encoding of $i$
	over $k$ bits, for some fixed value of $k \gtrsim \log_2(n)$.
	Then, by \Cref{prop:automatic-closure-length-multiplying-morphism},
	the relations we obtain are still "automatic@@rel".
\end{proof}

Putting bla and bla together, we get the following result, that says that bla.
todo: introduce notation for this structure.
\begin{proposition}
	\label{prop:universal-automatic-structure}
	Let $\?A$ be a $\sigma$-structure. The following are equivalent:
	\begin{enumerate}
		\item $\?A$ is "automatic@@struct",
		\item $\?A$ is an injective one-dimensional "first-order interpretation" of
			$\univStructSynchronous{2}$,
		\item $\?A$ is a "first-order interpretation" of 
			$\univStructSynchronous{2}$.
	\end{enumerate}
\end{proposition}

todo:proof

Observe that $\univStructSynchronous{2}$ is the structure $\2^*$ equipped
with $\lastLetter{0}$, $\lastLetter{1}$, $\equalLength$ and $\prefix$: it can be
seen as the infinite binary tree, with unary relation saying if a node is a
left or right child, and two binary relations saying if two nodes are
at the same depth, and if one is an ancestor of the other.
In the statement of \label{prop:universal-automatic-structure} the
$\univStructSynchronous{2}$ can be replaced by any "automatic structure" $\?U$
"st" there is an injective one-dimensional "first-order interpretation" of
$\univStructSynchronous{2}$ in $\?U$. Such a structure is ``universal'', in the
sense that it is not only automatic, but it ``contains'' all "automatic structures".
Other examples include TODO.

\begin{itemize}
	\itemAP ""automatic graph""
	\item see https://www.logic.rwth-aachen.de/pub/graedel/lics2020.pdf
	\item \cite[\S~6]{KhoussainovNerode1995AutomaticPresentations}
	\item no infinite integral domain \cite{KhoussainovNiesRubinStephan2007Automatic}
	\item Olivier Finkel's papers
\end{itemize}

TODO:generalized presentation (non-injective) + effective construction to make them injective.
-> this explains our choice. We avoid the terminology ``regular structure'' because of
``regular graph''.
