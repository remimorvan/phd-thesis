\section{Automatic Structures}
\label{sec:preliminaries-automatic-structures-automatic-structures}

\subsection{Definitions}
\todo{blabla}

We fix a finite "relational signature" $\sigma$.
An \AP""automatic presentation $\+A$ of a $\sigma$-structure"" consists of:
\begin{itemize}
	\item an alphabet $\Sigma$,
	\itemAP a "regular language" $\intro*\domainPres{\+A} \subseteq \Sigma^*$,
	\itemAP for every relation $\+R_{(k)} \in \sigma$, an
		"automatic relation" $\intro*\relPres{\+R}{\+A} \subseteq (\Sigma^*)^k$, and
	\itemAP a binary "automatic relation" $\reintro*\relPres{=}{\+A} \subseteq
		\Sigma^* \times \Sigma^*$. 
\end{itemize} 
The \AP\reintro{structure represented} $\?A$ by an "automatic presentation" $\+A$ has
$\domainPres{\+A}/{\relPres{=}{\+A}}$ as its domain, and the predicate $\+R_{(k)} \in \sigma$
is "interpreted@@pred" in such a way that a tuple $\tup{X_1,\hdots,X_k}$ of equivalence classes
belongs to $\+R(\?A)$ if, and only if, there exists $\tup{u_i}_{1 \leq i \leq k} \in\tup{X_i}_{1 \leq i \leq k}$ "st" $\tup{u_1,\hdots,u_k} \in \relPres{\+R}{\+A}$.
We say that a "$\sigma$-structure" is ""automatic@@struct"" if it is
"represented@@struct" by an "automatic presentation".%
\footnote{Note that we can always assume "wlog" that $\relPres{\+R}{\+A} \subseteq
(\domainPres{\+A})^k$ since $(\domainPres{\+A})^k$ is "automatic@@rel" by
\Cref{prop:rec-implies-aut} and "automatic relations" are closed under intersection.}

For instance, the infinite binary tree can be represented by the "automatic presentation"
$\+B$ with $\domainPres{\+B} = \2^*$,
\[
	\relPres{\+E}{\+B} =
	\{\tup{u,u0} \mid u \in \2^*\} \cup \{\tup{u,u1} \mid u \in \2^*\},
\]
and $\relPres{=}{\+B}$ is equality.
\todo{ figure.}
Presentations such as this one, where $\relPres{=}{\+B}$ is the equality relation,
are called ""injective presentations"".

\begin{example}
	\AP\label{ex:presburger}
	The structure $\tup{\N, +}$ is "automatic@@struct".%
	\footnote{We see $+$ as a ternary relation, given by $\{\tup{x,y,x+y} \mid x, y \in \N^2\}$.}
	We build an "automatic presentation" $\+N$ by using
	a binary encoding $\domainPres{\+N} = \2^*$. A word $w \in \2^*$ will
	represent the number $\sum_{i=0}^{|w|-1} w_i\cdot 2^{i-1}$. 
	Naturally, it follows that $\relPres{=}{\+N}$ puts two words $u$ and $v$ in relation
	if they are equal after removing the trailing zeroes.
	We then need to describe $\relPres{+}{\+N}$: the idea is to simulate addition, reading words
	from left to right, by using two states ($0$ and $1$) to remember the carry.
	\begin{marginfigure}
		\centering
		\begin{tikzpicture}[shorten >= 1pt, node distance = 1.8cm, on grid, baseline]
			\node[state, initial left, accepting] (q0) {}; 
			\node[state, right=of q0] (q1) {}; 
			\path[->]
				(q0) edge[loop above] node[font=\scriptsize, above left = 0em and -1.1em] {$\triple{0}{0}{0},
					\triple{0}{1}{1},\triple{1}{0}{1}$} (q0)
				(q0) edge[bend left] node[font=\scriptsize, above] {$\triple{1}{1}{0}$} (q1)
				(q1) edge[loop below] node[font=\scriptsize, below right = 0em and -1.1em] {$\triple{0}{1}{0},\triple{1}{0}{0},\triple{1}{1}{1}$} (q1)
				(q1) edge[bend left] node[font=\scriptsize, below] {$\triple{0}{0}{1}$} (q0);
		\end{tikzpicture}
		\caption{
			\AP\label{fig:automaton-presburger}
			"Synchronous automaton" describing the addition of
			natural numbers. For the sake of readability, transitions involving
			the "padding symbol" are not represented: $\pad$ is treated as a zero.
		}
	\end{marginfigure}
	For instance, the transition
	$\textcolor{c0}{0}
	\transition{\tup{\textcolor{c1}{0},\textcolor{c1}{1},\textcolor{c3}{1}}}
	\textcolor{c2}{0}$ can be understood as
	``when adding $\textcolor{c1}{0}$ and $\textcolor{c1}{1}$, with current carry $\textcolor{c0}{0}$, the results equals $\textcolor{c3}{1}$, with a carry of $\textcolor{c2}{0}$ for the next bit''.
	In general, we have a transition
	\[
		\textcolor{c0}{p}
		\transition{\tup{\textcolor{c1}{x},\textcolor{c1}{y},\textcolor{c3}{z}}}
		\textcolor{c2}{q}
		\quad\text{"iff"}\quad
		\textcolor{c1}{x} + \textcolor{c1}{y} + \textcolor{c0}{p}
		= \textcolor{c3}{z} + 2\textcolor{c2}{q}
	\]
	for all $p,q \in \2$ and
	$\tup{x,y,z} \in \2 \convolAlpha \2 \convolAlpha \2$.%
	\footnote{Of course, we identify $\pad$ with $0$.}
	This gives the automaton of figure \Cref{fig:automaton-presburger}.
	By construction, $\+N$ "represents@@struct" $\tup{\N, +}$.

	\todo{model theory rk after model-checking}
\end{example}

Recall that, using \Cref{prop:automatic-first-order}, a relation is "automatic@@rel" 
if and only if they are "first-order definable" over $\signatureSynchronous{\Sigma}$.
It means that we can alternatively see "automatic presentations" as a
collection of first-order formulas---one for the domain, one for equality, and
one for each predicate. In turns, this view helps us prove the following result.
For instance, $\langle \univStructSynchronous{\Sigma}, \langle\lastLetter{a}\rangle_{a \in \Sigma},\, \equalLength,\, \prefix \rangle$ is an "automatic structure".


\begin{proposition}
	\AP\label{prop:making-presentations-injective}
	Any "automatic structures" admits an "injective presentation".
\end{proposition}

\begin{proof}
	We prove this property using logic. Let $\+A$ be an arbitrary "automatic presentation"
	that "represents@@struct" $\?A$. The idea to build an "injective automatic presentation" $\+A'$
	of $\?A$ is to represent an equivalence class $\equivclass[\relPres{=}{\+A}]{u} \subseteq
	\domainPres{\+A}$ by its "lexicographic"-minimal element.%
	\footnote{This makes sense as the "lexicographic order" is a well-founded total order.}

	Concretely, $\+A'$ has the same alphabet as $\+A$.
	Then, we define $\domainPres{\+A'}$ as the set of lexicographic-minimal elements of
	$\equivclass[\relPres{=}{\+A}]{u}$ ($u \in \domainPres{\+A}$),
	which can be described by the "first-order formula"%
	\footnote{For the sake of readability, we identify first-order formulas with the relations they define.}
	\[
		\domainPres{\+A'}(x) \defeq
		\forall y.\; (\domainPres{\+A}(y) \land x \mathrel{\relPres{=}{\+A}} y) \Rightarrow 
		x \lex y.
	\]
	In is indeed a "first-order formula" since $\+A$ is assumed to be "automatic@@pres",
	and since $\lex$ is "automatic@@rel" by \Cref{ex:lexicographic-is-automatic}.
	Equality is interpreted as proper equality,
	and then
	\[
		\relPres{\+R}{\+A'}(x_1,\hdots, x_k) \defeq
		\exists y_1.\,\hdots \exists y_k.\,
		\Big( \bigwedge_{i=1}^k x_i \mathrel{\relPres{=}{\+A}} y_i \Big)
		\land \relPres{\+R}{\+A}(y_1,\hdots, y_k),
	\]
	for any predicate $\+R_{(k)} \in \sigma$.
	Overall, $\+A'$ is an "injective presentation" that "represents@@struct" 
	exactly the same "structure" as $\+A$.
\end{proof}

\begin{proposition}
	\AP\label{prop:making-presentations-binary}
	Any "automatic structures" admits a "presentation" with an alphabet of size 2.
\end{proposition}

\begin{proof}
	The idea is to encode each letter of $\Sigma = \{a_1,\hdots,a_n\}$
	over $\2$ in such a way that each letter $a_i$
	is encoded over a word $\hat a_i \in \2^*$ "st"
	all $\hat a_i$ have the same length---this is necessary to preserve "automaticity@@struct".
	For instance, take $\hat a_i$ to be the binary encoding of $i$
	over $k$ bits, for some fixed value of $k \gtrsim \log_2(n)$.
	Then, by \Cref{prop:automatic-closure-length-multiplying-morphism},
	the relations we obtain are still "automatic@@rel".
\end{proof}

Putting \Cref{prop:making-presentations-injective,prop:making-presentations-binary}
together, we get the following result, that says that $\univStructSynchronous{2}$ is ``universal'',
in the sense that it is not only automatic, but it ``contains'' all "automatic structures".
\begin{proposition}
	\AP\label{prop:universal-automatic-structure}
	Let $\?A$ be a $\sigma$-structure. The following are equivalent:
	\begin{enumerate}
		\itemAP\label{item:universal-automatic-structure-auto}
			$\?A$ is "automatic@@struct",
		\itemAP\label{item:universal-automatic-structure-injective-FO}
			$\?A$ is an injective one-dimensional "first-order interpretation" of
			$\univStructSynchronous{2}$,
		\itemAP\label{item:universal-automatic-structure-FO}
			$\?A$ is a "first-order interpretation" of $\univStructSynchronous{2}$.
	\end{enumerate}
\end{proposition}

Recall that $\univStructSynchronous{2}$ is the structure $\2^*$ equipped
with $\lastLetter{0}$, $\lastLetter{1}$, $\equalLength$ and $\prefix$: it can be
seen as the infinite binary tree, with unary relation saying if a node is a
left or right child, and two binary relations saying if two nodes are
at the same depth, and if one is an ancestor of the other.

\begin{proof}
	% \eqref{item:universal-automatic-structure-FO} $\Rightarrow$
	% \eqref{item:universal-automatic-structure-injective-FO} and
	% \eqref{item:universal-automatic-structure-injective-FO} $\Rightarrow$
	% \eqref{item:universal-automatic-structure-auto} are trivial.
	\eqref{item:universal-automatic-structure-auto} $\Rightarrow$
	\eqref{item:universal-automatic-structure-injective-FO} follows from
	\Cref{prop:making-presentations-injective,prop:making-presentations-binary}.
	\eqref{item:universal-automatic-structure-injective-FO} $\Rightarrow$
	\eqref{item:universal-automatic-structure-FO} is trivial.
	\eqref{item:universal-automatic-structure-FO} $\Rightarrow$
	\eqref{item:universal-automatic-structure-auto} can be proven by,
	given a "$d$-dimensional interpretation", defining an "automatic presentation"
	over the alphabet
	\[
		\underbrace{\2 \convolAlpha \cdots \convolAlpha \2}_{
			\text{$k$ times}
		},
	\]
	and changing any "first-order formula"
	\[
		\phi\big((x_1^1, \hdots, x_1^d), \hdots, (x_n^1, \hdots, x_n^d)\big)
		\quad\text{into}\quad
		\phi^*(x_1^1 \convol \cdots \convol x_1^d, \hdots, x_n^1 \convol \cdots \convol x_n^d).\qedhere
	\]
\end{proof}

In the statement of \label{prop:universal-automatic-structure} the
$\univStructSynchronous{2}$ can be replaced by any "automatic structure" $\?U$
"st" there is an injective one-dimensional "first-order interpretation" of
$\univStructSynchronous{2}$ in $\?U$. 
Another example of ``universal'' structure consists of the finite subsets of $\N$ equipped
with inclusion and the preorder $\preceq$ defined by $X \preceq Y$ "iff" $X$ and $Y$ are 
singletons, say $\{x\}$ and $\{y\}$, respectively, and $x \leq y$,
see \cite[Theorem~XII.2.3]{Blumensath2024MSOModelTheory}.

The notion of "automatic structures" can be generalized:
\begin{itemize}
	\itemAP to ""$\omega$-automatic structures"" by replacing finite words with $\omega$-words;
	\itemAP to ""tree-automatic structures"" by replacing finite words with finite trees;
	\itemAP to ""$\omega$-tree-automatic structures"" by replacing finite words with $\omega$-trees;
\end{itemize}
we refer the reader to "eg" \cite[\S~XII]{Blumensath2024MSOModelTheory} for more details.

\hypothesis{In light of \Cref{prop:making-presentations-injective},
we will always assume, unless specified otherwise, the "automatic presentations"
to be "injective@@pres".}

\subsection{Model-Checking}

One of the key interest of "automatic structures" is that, while they are infinite,
their model checking problem remains decidable.

\decisionproblem{""First-Order Model Checking of Automatic  $\sigma$-Structures""}{
	A "first-order sentence" $\phi$ over $\sigma$ and
	an "automatic presentation" $\•A$ of an "automatic $\sigma$-structure" $\?A$.
}{
	Does $\?A \FOmodels \phi$?
}

\begin{proposition}[{""Hodgson's theorem""}]
	\!\footnote{Decidability was originally proven in \cite[Théorème~3.5]{Hodgson1983Decidabilite}.
	The lower bound is sometimes claimed to be true in the literature, but we never found a proper, or even sketch of proof.}
	%
	\AP\label{prop:first-order-model-checking-automatic-structures}
	"First-order model checking of automatic structures" is decidable,
	and in fact is "TOWER"-complete under polynomial-time reductions.
\end{proposition}

\Cref{prop:first-order-interpretation,prop:automatic-first-order}
prove this problem to be polynomial-time equivalent to its restriction
to the structure $\univStructSynchronous{2}$.%
\footnote{However, this does not work
for "data complexity" since, when reducing $\?A \FOmodels^? \phi$ to
$\univStructSynchronous{2} \FOmodels^? \phi^{\•A}$, the formula $\phi^{\•A}$ depends on $\•A$.}

\begin{proof}[Proof sketch]
	% Putting \Cref{prop:first-order-interpretation,prop:automatic-first-order},
	% from any "first-order sentence" $\phi$ over $\sigma$ and any "automatic presentation" $\•A$,
	% we can build a "first-order formula" $\phi^{\•A}$ over
	% $\signatureSynchronous{\2}$ "st"
	% \[
	% 	\univStructSynchronous{2} \FOmodels \phi^{\•A}
	% 	\quad\text{"iff"}\quad
	% 	\?A \FOmodels \phi.
	% \]
	% The size of $\phi^{\•A}$ is polynomial in the size of "first-order formulas"
	% for $\+A$---which are polynomial in the size of the automata for $\+A$ by \Cref{prop:automatic-first-order}---and in the size of $\phi$.
	% Then, we use the other implication of \Cref{prop:automatic-first-order}
	% to build a "synchronous ($0$-ary) automaton" for $\phi$. Checking whether
	% this automaton accepts or rejects is trivial---see the paragraph
	% after \Cref{prop:automatic-first-order}---say in logarithmic space.
	% Note however that the size of this automaton is
	% non-elementary---it is a tower of exponentials in the
	% number of quantifier alternation in $\phi^{\•A}$---and so we obtain a
	% "TOWER" algorithm.

	% For the "data complexity", we now assume that $\phi$ is fixed.
	\proofcase{"Tower" upper bound.}
	The construction is similar to the easy implication of \Cref{prop:automatic-first-order}:
	for any "automatic presentation" $\•A$, we build by induction on $\phi(x_1,\hdots,x_k)$ a "synchronous ($k$-ary) automaton@synchronous automaton" $\+B^\phi$ "st"
	for all $u_1,\hdots,u_k \in \Sigma^*$
	\[
		\+B^\phi \text{ accepts } u_1\convol\hdots\convol u_k
		\quad\text{"iff"}\quad
		\tup{\?A, u_1, \hdots, u_k} \FOmodels \phi(x_1,\hdots,x_k).
	\]
	In the end, we get a "synchronous ($0$-ary) automaton@synchronous automaton" $\+B^\phi$ that either accepts
	or rejects depending on whether $\?A \FOmodels \phi$ or not.
	Notice that each quantifier alternation implies a powerset construction, and so
	the size of $\+B^\phi$ is a tower of exponentials in the number of quantifier alternations in $\phi$. Hence, we get a "Tower" algorithm.

	\proofcase{"Tower" lower bound.}%
	\footnote{We often found this result to be incorrectly credited to various papers in the literature.}
	In 1990, proved that there exists a constant $c>0$ "st" "first-order model-checking"
	restricted to the structure $\univStructSynchronous{2}$ admits $\tower(cn)$ as
	a lower bound on the running time of a non-deterministic "Turing machine" solving the problem
	\cite[Example~8.3]{ComptonWardHenson1990UniformMethod}.
	In fact, applying \cite[Theorem~6.1.(iv)]{ComptonWardHenson1990UniformMethod}
	proves the problem to be hard under polynomial-time reductions for the class 
	of problems which can be solved in non-deterministic time at most $\tower(n^c)$ for some $c>0$.
	This corresponds to the class "Tower".
\end{proof}

\begin{proposition}
	\AP\label{prop:data-complexity-model-checking}
	The "data complexity" of its restriction to "existential-positive sentences" is "NL"-complete.
\end{proposition}

\begin{proof}
	We now assume that $\phi$ is of the form
	\[
		\exists x_1.\,\hdots \exists x_k.\, \psi(x_1,\hdots,x_k),
	\]
	where $\psi$ is a "positive quantifier-free formula",
	the size of the automaton for $\psi$ is of the order $|\•A|^{|\psi|}$:
	conjunctions and disjunctions only require products but no powerset construction.
	Now instead of explicitly building an automaton for
	$\exists x_1.\,\hdots \exists x_k.\, \psi(x_1,\hdots,x_k)$,
	we test if the automaton for $\psi$ accepts any word.
	The answer is ``yes'' "iff" $\?A \FOmodels \phi$.
	Testing non-emptiness of the automaton amounts to checking if an accepting state can be
	reached from an initial state, which is "NL".
	Since $\phi$ is fixed, the automaton in question is of polynomial size.
	To argue that we can effectively obtain an "NL" algorithm, it suffices to notice that
	we do not have to explicitly build the automaton for $\psi$, but it suffices to
	work with pointers to the automaton of $\•A$.

	Lastly, the "NL" lower bound, can be proven by a reduction from NFA non-emptiness, which is
	itself "NL"-hard by reduction from the "finite graph reachability problem".
\end{proof}

Recall that since "NL" is closed under complementation---see "eg" \cite[Corollary~9.23]{Immerman1998DescriptiveComplexity}---, we obtain the same complexity for
negations of "existential-positive sentences".

We refer the reader to \cite[\S~3]{BlumensathGradel2000AutomaticStructures} for the detailed 
complexity of other variations of "Hodgson's theorem". For instance,
the "data complexity" of its restriction to "existential sentences" (here negation is allowed)
in "NP"-complete \cite[Theorem~3.7]{BlumensathGradel2000AutomaticStructures}.

What we said above work with the implicit assumptions that all structures at hand are 
"finite@@struct". However, part of this can be generalized to "automatic structures"---however, this comes to the cost of losing the inclusion in "L".

\begin{proposition}[Folklore]
	\AP\label{prop:first-order-reduction-preserve-automaticity}
	The image of an "automatic structure" by a "first-order reduction" is
	still an "automatic structure".
\end{proposition}
\begin{proof}
	This follows "eg" from \Cref{prop:automatic-first-order}.
\end{proof}

We define \AP""FOaut"" to be the class of all problems over "automatic structures"
which are "first-order definable". By \Cref{prop:first-order-reduction-preserve-automaticity},
this class is closed under "first-order reductions". Moreover,
by \Cref{prop:first-order-model-checking-automatic-structures}, we obtain an upper bound.

\begin{corollary}
	"FOaut" $\subseteq$ "Tower".
\end{corollary}

The goal of the corollary above is to highlight the difference with \Cref{prop:FO-in-L}:
"first-order definable" problems on "automatic structures", while decidable, are not necessarily
in "L".

\begin{remark}[Presburger arithmetic]
	\AP""Presburger arithmetic"" is the first-order theory (over the signature $\tup{+, 0, 1}$)
	derived from the following axioms:
	\begin{itemize}
		\item $\forall x.\, 0 \neq x + 1$;
		\item $\forall x.\,\forall y.\, x + 1 = y + 1 \Rightarrow x = y$;
		\item $\forall x.\, x + 0 = x$;
		\item $\forall x.\,\forall y.\,\forall z.\, x + (y + z) = (x + y) + z$;
		\item $(\phi(0) \land (\forall x.\; \phi(x) \Rightarrow \phi(x+1))) \Rightarrow
			(\forall x.\; \phi(x))$, where $\phi(x)$ ranges over all "first-order formulas".
	\end{itemize}
	Note that $\tup{\N,+,0,1}$ is a model of this theory, which is moreover complete
	(\todo{addref}) and so, for a first-order sentence $\phi$, the following 
	are equivalent:
	\begin{itemize}
		\item $\phi$ belongs to the theory, "ie" it is a logical consequence of the axioms above,
		\item $\phi$ holds in all models satisfying the axioms above, and 
		\item $\tup{\N, +, 0, 1} \FOmodels \phi$.
	\end{itemize}
	Since $\tup{\N, +, 0, 1}$ is "first-order equivalent" to $\tup{\N, +}$, which
	is "automatic@@struct" by \Cref{ex:presburger},
	it follows that we can decide "Presburger arithmetic".
\end{remark}

Of course, \Cref{prop:first-order-model-checking-automatic-structures} can be extended
to ``$\omega$-tree automatic structures'' defined analogously to "automatic structures" by
replacing ``languages of finite words'' by ``languages of $\omega$-trees''
(see \cite[\S~XII]{Blumensath2024MSOModelTheory} for more details).
In fact this property can even be extended to so-called
\AP""higher-order automatic structures"", see
\cite[last remark of \S~XII.2]{Blumensath2024MSOModelTheory}.
Note that while "automatic structures" are always countable,
"higher-order automatic structures" have at most the cardinality of the continuum.%
From this it follows that there exists "structures" with a decidable "first-order model-checking" that are not "automatic@@struct", and not even "higher-order automatic".
It suffices for instance to take any non-standard model of "Presburger arithmetic"
of cardinality strictly bigger than the continuum, which must exist by
"upward Löwenheim–Skolem theorem". The same argument works to show that ``elementary equivalence''
does not preserve the notion of "automaticity@@struct".

\paragraph*{Order-Invariant First-Order Logic}
An \AP""order-invariant first-order formula""%
\footnote{This is a standard notion in model theory, see "eg"
\cite[Exercise~3.1.12]{Gradel2007FiniteModelTheory}.}
over $\sigma$
is any "first-order formula"\footnote{For the sake of simplicity we give all definitions
for "sentences@@FO", but they easily be generalized to handle free varirables.} $\phi$
over the signature $\sigma \dcup \{\leq\}$ "st" for any
"$\sigma$-structure" $\?A$, for any \emph{total orders} $\leq_1$, $\leq_2$ over $A$,
we have $\tup{\?A, \leq_1} \FOmodels \phi$ "iff" $\tup{\?A, \leq_2} \FOmodels \phi$.
In this case, we say that $\?A$ models $\phi$.

For instance, the formula
\[
	\forall x.\; \exists y.\; x \leq y
\]
is clearly "order-invariant@@FO": it expresses the fact that the model is infinite.
\emph{A priori}, it is non-trivial to check if an "order-invariant first-order formula" holds in an
"automatic $\sigma$-structure": how should this order be "interpreted@@pred"?
By "order-invariance@@FO", any total order will do, and moreover by
\Cref{ex:lexicographic-is-automatic}, the "lexicographic order" is always "automatic@@rel"!

\begin{proposition}[{\cite[\S~3.2]{Rubin2008AutomataPresentingStructures}}]
	\label{prop:order-invariance}
	Model-checking of "order-invariant first-order formulas" over "automatic structures"
	is decidable.
\end{proposition}

A restricted form of this result was originally proved by Blumensath and Grädel
\cite[Corollary~5.4]{BlumensathGradel2004FinitePresentations} for the extension of
"first-order logic" with the quantifier ``there are infinitely many''.
Similarly, one can add counting quantifiers of the form ``the number of $x$'s "st" $\phi(x)$ holds
is congruent to $k$ mod $n$'' \cite[\S~3.2]{Rubin2008AutomataPresentingStructures}
while preserving decidability.

\subsection{Problems on Automatic Structures}

From the notion of "automaticity@@struct" two questions naturally arise:
\begin{enumerate}
	\item What are the \emph{structural properties} of "automatic structures"?
		For instance, we have seen that all "automatic structures" are countable.
		This question has been somewhat extensively studied for algebraic structures.
	\item Given a \emph{decidable decision problem} on finite structures, is its generalization
		to "automatic structures" still decidable?
\end{enumerate}

\decisionproblem{""Isomorphism Problem for Automatic Structures""}{
	Two "automatic presentations" $\•A$ and $\•B$.
}{
	Is $\?A$ "isomorphic@@struct" to $\?B$?	
}
Blumensath and Grädel proved this problem
to be undecidable \cite[Theorem~5.15]{BlumensathGradel2004FinitePresentations}.
The problem was later shown to be complete for the first level
of the "analytical hierarchy" \cite[Theorem~5.9]{KhoussainovNiesRubinStephan2007Automatic}.

\paragraph*{Automatic Ordinals.}
"Automatic ordinals" are quite simple: using "eg" Cantor's normal form, it is straightforward
to prove that any ordinal strictly smaller than $\omega^\omega$ is "automatic@@struct".%
\footnote{We see an ordinal as a structure with a binary relation describing its order.}
Delhommé proved the converse property to be true:
an ordinal is "automatic@@struct" "iff" it is strictly smaller than $\omega^\omega$
\cite[Corollaire~2.2]{Delhommé2004AutomaticitéOrdinaux}.
Using again Cantor's normal form,
Khoussainov, Rubin \& Stephan proved that the isomorphism problem is decidable for
"automatic ordinals" \cite[Theorem~5.3]{KhoussainovRubinStephan2005AutomaticLinearOrders}.
These results have been generalized first to "$\omega$-tree-automatic ordinals"
by Finkel and Todor\v{c}ević \cite{FinkelTodorcevic2013AutomaticOrdinals}
and then to "tree-automatic ordinals" which
are moreover equipped with addition, by Jain, Khoussainov, Schlicht and Stephan
\cite{JainKhoussainovSchlichtStephan2019IsomorphismTreeAutomaticOrdinals}.

\paragraph*{Automatic Boolean Algebras.}
Recall that a Boolean algebra can be either seen as a partially ordered set
following a set of axioms ensuring among other the existence of a join, a meet
and a negation, or a set equipped these three operations, together with a minimal and maximal 
element. Both definitions are actually "first-order equivalent" and so considering one
or the other does not change the notion of "automaticity@@struct".

The "isomorphism problem" for "automatic Boolean algebras" is decidable
\cite[Corollary~3.5]{KhoussainovNiesRubinStephan2007Automatic} essentially
because there are very few "automatic Boolean algebras" \cite[Theorem~3.4]{KhoussainovNiesRubinStephan2007Automatic}.
However, when going to "$\omega$-tree-automaticity", not only is the "isomorphism problem"
undecidable, but Finkel and Todor\v{c}ević exhibited two somewhat simple-looking
"$\omega$-tree-automatic Boolean algebras" for which whether they were isomorphic
is independent from "ZFC" \cite[Theorem~6.1]{FinkelTodorcevic2010Isomorphism}.

\paragraph*{Automatic Groups.}
The literature on "automatic group" is remarkably extensive.
Their study was introduced in the late 1980s,
by showing that they have a solvable word problem \cite[Theorem~2.1.9]{Epstein1992Word}.%
\footnote{Actually this last result is proved on the slightly bigger class of ``regularly generated groups''.}
Typical examples of infinite "automatic group" are the braid group
\cite[Theorem~9.3.1]{Epstein1992Word} and the finitely-generated free groups.
We refer the reader to \cite{Rees2022AutomaticGroups} for a recent and detailed account on
the history of the development of the theory of "automatic groups".
Remarkably, despite being a very active research area, the decidability of
the "isomorphism problem" remains open.

\begin{openproblem}
	Is the "isomorphism problem" decidable for "automatic groups"?
\end{openproblem}

\paragraph*{Automatic Rings and Fields.}
Richer algebraic structures, like rings or fields actually often fail to be captured by
the notion of "automaticity@@struct". For instance, $\tup{\N,{+},{\cdot},0,1}$ (Peano's arithmetic)
is not "automatic@@struct".
This essentially follows from \Cref{prop:bound-automatic-structures}, applied to multiplication,
together with a counting argument, see \cite[Corollary~XII.8.11]{Blumensath2024MSOModelTheory}.
Khoussainov, Nies, Rubin \& Stephan moreover proved that no infinite field\footnote{In fact their result also holds for integral domains.} can be "automatic@@struct"
\cite[Theorem~3.10 \& Corollary~3.11]{JainKhoussainovSchlichtStephan2019IsomorphismTreeAutomaticOrdinals}.%
\footnote{Note that this contrasts with Tarski's result that the first-order theory
of $\tup{\R,+,\cdot,0,1}$, and more generally of any real closed fields is 
decidable since it admits quantifier elimination \cite[Theorem~8.4.4]{Hodges1993ModelTheory}.}


\subsection{Automatic Graphs}
\Cref{ch:dichotomy-theorem} will mostly focus on "automatic graphs", in which we will study
the question of colourability and the homomorphism problem.
One interesting source of undecidability for "automatic graphs" comes from the following construction.

\begin{example}
	Consider a Turing machine $\+T = \tup{Q,\Gamma,\delta,q_0,\Acc}$, where $Q$ is the set of states, $\Gamma$ is tape alphabet,
	\[
		\delta\colon (Q \setminus \Acc) \times \Gamma_{\smallpad} \to \pset{Q \times \Gamma \times \set{\leftarrow, \downarrow, \rightarrow}}
	\]
	is the transition relation, $\Gamma_{\smallpad} \defeq \Gamma \dcup \set{\pad}$, and $q_0$ and $\Acc$ are the initial and set of final states, respectively.
	%
	We represent a "configuration@@TM" $\tup{u, q, v}$ by the word $uqv \in \Gamma^* Q \Gamma^*$:
	in light of this, we will henceforth denote by ``configuration'' any string from the set  \AP$\intro*\configs \defeq  \Gamma^* Q \Gamma^*$.\footnote{We will often write
	$uqv$ as the concatenation $u\cdot q \cdot v$ to emphasize
	the separation between all three words.}
	The \AP""configuration graph"" of $\+T$ is the infinite graph $\intro*\confGraph$ having $\configs$ as set of vertices and an edge from $\gamma$ to $\gamma'$, denoted $\gamma \rightarrow \gamma'$ if there is a one-step transition from $\gamma$ to $\gamma'$ in $\+T$. The "configuration graph" $\confGraph$ of any Turing machine $\+T$ is an effective "automatic graph".
\end{example}

As a consequence, by reduction from the halting problem of a universal Turing machine, 
we obtain the following result.
\begin{proposition}
	There exists a fixed "automatic graph" $\?G$ over the alphabet $\2$ "st"
	the problem of whether, given two words $u$ and $v \in \2^*$, there is a path from
	$u$ to $v$ in $\?G$ is undecidable.
\end{proposition}

Problems on automatic graphs have been mainly studied by Kuske and Lohrey, who showed that over "automatic graphs":
\begin{description}
	\item[highly undecidable problems:] the existence of a Hamiltonian path
		is undecidable, in fact it is complete for the first level of the "analytical hierarchy"
		\cite[Theorem~3.2]{KuskeLohrey2010AutomaticGraphs};
		the existence of an infinite path in "directed trees" shares the same complexity
		\cite[Theorem~3.6]{KuskeLohrey2010AutomaticGraphs};
	\item[moderately undecidable problems:]
		the existence of an Eulerian path is undecidable, and is only complete for
		the second level of the "arithmetical hierarchy" "Pi0-2" \cite[Theorem~4.13]{KuskeLohrey2010AutomaticGraphs}.
\end{description}

Moreover, as a consequence of \Cref{prop:order-invariance}, the existence of an infinite clique,
being an "order-invariant@@FO" property, is decidable over "automatic graphs".

\begin{proposition}[{\cite[Proposition~6.5]{Kocher2014AutomatischenGraphen}}]
	Whether an "automatic graph" is 2-colourable, or equivalently bipartite, is undecidable.
	More precisely it is "coRE"-complete.
\end{proposition}

For another survey on "automatic structures", see \cite{Gradel2020AutomaticStructures}.
Lastly, we want to note that \emph{recursive structures} have been extensively studied
since the late XXth century: they are defined analogously to "automatic structures",
but "synchronous automata" are replaced Turing machines.
Unsurprisingly, all non-trivial problems are undecidable: hence, from a computability perspective,
the main question is to characterize the Turing degree of these problems.
The extent of the literature on this topic is too vast to be summarized here:
we refer the reader to the two-volume handbook \cite{ErshovGoncharovNerodeRemmelMarek1998Handbook1,ErshovGoncharovNerodeRemmelMarek1998Handbook2}.

\todo{ colcombet-loding paper}