\section{Preliminaries}
\label{sec-prelims}

\paragraph{Graph databases.}
\AP ""Graph databases"" are abstracted as edge-labelled directed graphs
$G = \langle \vertex{G}, \edges{G} \rangle$, 
where nodes of $\intro*\vertex{G}$ represent entities and labelled edges $\intro*\edges{G} \subseteq \vertex G \times \A \times \vertex G$
represent relations between these entities, with $\A$ being a fixed finite alphabet.

\paragraph{Conjunctive regular path queries (CRPQs) and unions of CRPQs (UCRPQs).}
\AP A ""CRPQ"" $\gamma$ is defined as a tuple $\bar z = (z_1,\hdots,z_n)$
of ""output variables""\footnote{For technical reasons (see the definition of "equality atoms") we allow for a variable to appear multiple times.},
together with a conjunction of ""atoms"" of the form
\AP$\bigwedge_{j=1}^m x_j \intro*\atom{L_j} y_j$, where each $L_j$ is a regular language
and where $m \geq 0$.
The set of all variables occurring in $\gamma$, namely%
\footnote{We neither assume 
disjointness nor inclusion between $\{z_1,\hdots,z_n\}$ and $\{x_1,y_1,\hdots,x_m,y_m\}$.}
$\{z_1,\hdots,z_n\}\cup\{x_1,y_1,\hdots,x_m,y_m\}$, is denoted by
$\intro*\vars(\gamma)$. Variables in $\vars(\gamma)\setminus \{z_1,\hdots,z_n\}$ are existentially quantified. 
We denote by $\intro*\atoms(\gamma)$ the set of "atoms" of $\gamma$.
Given a "database@@graph" $G$, we say that a tuple of nodes $\bar u = (u_1,\hdots,u_n)$
\AP""satisfies@@db"" $\gamma$ 
on $G$ if there is a mapping
$\fun\colon \vars(\gamma) \to \vertex{G}$ such that $u_i = \fun(z_i)$ for all
$1 \leq i \leq n$, and for each $1 \leq j \leq m$,
there exists a (directed) path from $\fun(x_j)$ to $\fun(y_j)$ in $G$, labelled by
a word from $L_j$ (if the path is empty, the label is $\varepsilon$). The \AP""evaluation"" of $\gamma$ on $G$ is then the set of all tuples that "satisfy@@db" $\gamma$ on $G$.

A ""union of CRPQs"" (\reintro{UCRPQs})
is defined as a finite set of "CRPQs", called \AP""disjuncts"", whose tuples of "output variables" have all the same arity.
The "evaluation" of a union is defined as the union of its "evaluations". 
If a query has no "output variables" we call it ""Boolean"", and
its "evaluation" can either be the set $\set{()}$, in which case we say that $G$
\reintro(db){satisfies} the query, or the empty set $\set{}$.

\AP Given two "UCRPQs" $\Gamma$
and $\Gamma'$ whose "output variables" have the same arity,
we say that $\Gamma$ is \AP""contained"" in $\Gamma'$,
denoted by $\Gamma \intro*\contained \Gamma'$, if
for every "graph database" $G$, for every tuple $\bar u$ of $\vertex{G}$,
if $\bar u$ "satisfies@@db" $\Gamma$ on $G$, then so does $\Gamma'$. We will hence reserve the symbol `$\subseteq$' for set inclusion.
The \AP""containment problem"" for "UCRPQs" is the problem of, given
two "UCRPQs" $\Gamma$ and $\Gamma'$, to decide if $\Gamma \contained \Gamma'$.
When $\Gamma \contained \Gamma'$ and $\Gamma' \contained \Gamma$  we say that
$\Gamma$ and $\Gamma'$ are \AP""equivalent"", denoted by
$\Gamma \intro*\semequiv \Gamma'$. 

\AP A ""conjunctive query"" (\reintro{CQ}) is in this context a "CRPQ" whose every atom is of the form $x \atom{a} y$ for $a \in \A$ ("ie", every language is a singleton $\set{a}$).
\AP A ""union of CQs"" (\reintro{UCQs}) is defined as a "UCRPQ" with the same property.

A \AP""canonical database"" $G$ of a "CRPQ" $\gamma$ is any "canonical database" associated
to an "expansion" of $\gamma$, see \cite[Definition 3.1]{FlorescuLevySuciu1998Containment}
for a formal definition. We denote it by \AP$G \intro*\cdb \gamma$.
A \reintro{canonical database} of a "UCRPQ" is a "canonical database" of one
of its "disjuncts".

An \AP""evaluation map"" from a "CRPQ" $\gamma$ to a "graph database" $G$
in a function $f$ from variables of $\gamma$ to $G$ "st"
for any atom $x \atom{L} y$ in $\gamma$, there is path from $f(x)$ to $f(y)$ in $G$
labelled by a word of $L$.

The "containment" between "UCRPQs" $\Gamma_1 \contained \Gamma_2$ is exactly characterized
by the fact that for all "canonical database" $G_1 \cdb \Gamma_1$,
there exists a "disjunct" $\gamma_2$ of $\Gamma_2$ "st" there is an "evaluation map"
from $\gamma_2$ to $G_1$.

\paragraph{Homomorphisms.}
\AP A ""homomorphism"" $\fun$ from a "CRPQ" $\gamma(x_1, \dotsc, x_m)$ to a "CRPQ" $\gamma'(y_1, \dotsc, y_m)$ is a mapping from $\vars(\gamma)$ to $\vars(\gamma')$ such that $\fun(x) \atom{L} \fun(y)$ is an "atom" of $\gamma'$ for every "atom" $x \atom{L} y$ of $\gamma$, and further $\fun(x_i)=y_i$ for every $i$.
Such a "homomorphism" $\fun$ is \AP""strong onto"" if for every "atom" $x' \atom{L} y'$ of $\gamma'$ there is an "atom" $x \atom{L} y$ of $\gamma$ such that $\fun(x)=x'$ and $\fun(y)=y'$.
% An example of "homomorphism" is provided in \Cref{fig:basic-hom}.
We write $\gamma \intro*\homto \gamma'$ if there is a "homomorphism" from $\gamma$ to $\gamma'$, and $\gamma \intro*\surjto \gamma'$ if there is a "strong onto homomorphism".
In the latter case, we say that $\gamma'$ is a \AP""homomorphic image"" of $\gamma$.
A \reintro{homomorphism} $\fun$ from a graph database $G$ to a graph database $G'$ is a mapping from $\vertex{G}$ to $\vertex{G'}$ such that  for every edge $u \atom{a} v$ of $G$, it holds that $\fun(u) \atom{a} \fun(v)$ is an edge in $G'$. A "homomorphism" from a "CQ" to a graph database is defined analogously.  

\AP It is easy to see that if $\gamma \homto \delta$ then $\delta \contained \gamma$, and in the case where $\gamma,\delta$ are "CQs" this is an ``if and only if'' \cite[Lemma 13]{ChandraMerlin1977Implementation}. 
\AP Two "CQs" $\gamma,\delta$ are ""hom-equivalent"" if there are "homomorphisms" $\gamma \homto \delta$ and $\delta \homto \gamma$.
Hence, for any two "CQs" $\gamma, \delta$, we have $\gamma \semequiv \delta$ if, and only if, they are "hom-equivalent".
\AP The ""core"" of a "CQ" $\gamma$, denoted by $\intro*\core(\gamma)$
is the result of repeatedly removing any atom which results in an equivalent query. It is unique up to isomorphism (see, "eg", \cite{ChandraMerlin1977Implementation}). We say that a "CQ" is `a core' if it is isomorphic to its "core". If $\gamma$ and $\delta$ are "hom-equivalent" then they have
the same "core". Moreover, there is always an \AP""embedding""---"ie", a "homomorphism" which is injective both on variables and "atoms"---of $\core(\gamma)$ into $\gamma$.

\paragraph*{Refinements and expansions of (U)CRPQs.}
\AP For an NFA $\+A$ and two states $q,q'$ thereof, we denote by $\intro*\subaut{\+A}{q}{q'}$ the ""sublanguage"" of $\+A$ recognized  when considering $\set{q}$ as the set of initial states and $\set{q'}$ as the set of final states.
\AP An ""atom $m$-refinement"" of a "CRPQ" "atom" $\gamma(x,y) = x \atom{L} y$ where $m\geq 1$ and $L$ is given by the NFA $\+A_L$ is any "CRPQ" of the form 
\begin{equation}
	\AP\label{eq:refinement}
	\rho(x,y) = x \atom{L_1} t_1 \atom{L_2} \hdots \atom{L_{n-1}} t_{n-1} \atom{L_n} y
\end{equation}
where $1 \leq n \leq m$, $t_1,\hdots,t_{n-1}$ are fresh (existentially quantified) variables,
and $L_1,\hdots,L_n$ are such that there exists a sequence $(q_0,\dotsc,q_n)$ of states of $\+A_L$
such that $q_0$ is initial, $q_n$ is final, and for each $i$, $L_i$ is either of the form
\begin{enumerate}[label=\roman*.]
	\item $\subaut{\+A_L}{q_{i-1}}{q_{i}}$, or 
	\item $\{a\}$ if the letter $a\in \A$ belongs to $\subaut{\+A_L}{q_{i-1}}{q_{i}}$.
\end{enumerate}
Additionally, if $\varepsilon \in L$, the \AP""equality atom"" ``$x = y$'' is also an \reintro{atom $m$-refinement}. Thus, an \reintro{atom $m$-refinement} can be either of the form \eqref{eq:refinement} or ``$x=y$''.
By definition, note that the concatenation
$L_1\cdots L_n$ is a subset of  $L$ and hence $\rho \contained \gamma$ for any "atom $m$-refinement" $\rho$ of $\gamma$.
An \AP""atom refinement"" is an "atom $m$-refinement" for some $m$.

Given a natural number $m$, an \AP""$m$-refinement"" of a "CRPQ" $\gamma(\bar x) = \bigwedge_{i} x_i \atom{L_i} y_i$ is any query resulting from: (1) replacing every "atom" by one of its "$m$-refinements@@atom", and (2)
should some "$m$-refinements@@atom" have "equality atoms",
collapsing the variables (and removing the identity atoms `$x=x$').
\AP A ""refinement"" is an "$m$-refinement" for some $m$.
Note that in a "refinement" of a "CRPQ"
the "atom refinements" need not have the same length.
For instance, both $\rho(x,x) = x \atom{c} x$ and $\rho'(x,y) = x \atom{a} t_1 \atom{a} y \coatom{c} y$ are "refinements" of $\gamma(x,y) = x \atom{a^*} y \coatom{c} x$.
\AP
We write $\intro*\Refin(\gamma(\bar x))$ to denote the set of all "refinements" of $\gamma(\bar x)$ and $\reintro*\Refin[\leq m](\gamma(\bar x))$ to the $m$-refinements. 

The set of \AP""expansions"" of a "CRPQ" $\gamma$ is the set $\intro*\Exp(\gamma)$ of all "CQs" which are "refinements" of $\gamma$.
In other words, an "expansion" of $\gamma$ is any "CQ" obtained from $\gamma$
by replacing each "atom" $x \atom{L} y$ by a path $x \atom{w} y$ for some
word $w \in L$. The "expansions" (resp.\ "refinements") of a "UCRPQ" are the "expansions" (resp.\ "refinements") of the "CRPQs" it contains.
We define \AP""atom expansions"" analogously to "atom refinements". For "UCRPQs" we use  $\Exp(\Gamma)$, $\Refin(\Gamma)$ and $\Refin[\leq m](\Gamma)$ as for "CRPQs".

Any "UCRPQ" is equivalent to the infinitary union of its "expansions". In light of this, 
the semantics for "UCRPQs" can be rephrased as follows. 
Given a "UCRPQ" $\Gamma(\bar x)$ and a graph database $G$, 
the "evaluation" of $\Gamma(\bar x)$ on $G$, denoted by $\Gamma(G)$, is the set of tuples 
$\bar{v}$ of nodes for which there is $\anexpansion \in \Exp(\Gamma)$ such that there is a "homomorphism" $\anexpansion \homto G$ that sends $\bar x$ onto $\bar v$. 

"Containment" of "UCRPQs" can also be characterized in terms of "expansions".
\begin{proposition}[Folklore, see e.g. {\cite[Proposition 3.2]{FlorescuLevySuciu1998Containment}} or
	{\cite[Theorem 2]{CalvaneseDeGiacomoLenzeriniVardi2000Containment}}]
	\AP\label{prop:cont-char-exp-st} 
	Let $\Gamma_1$ and $\Gamma_2$ be "UCRPQs". Then the following are equivalent:
	% \begin{itemize}
		% \item 
		(i) $\Gamma_1 \contained \Gamma_2$;
		% \item 
		(ii) for every $\anexpansion_1\in \Exp(\Gamma_1)$, $\anexpansion_1 \contained \Gamma_2$;
		% \item 
		(iii) for every $\anexpansion_1\in \Exp(\Gamma_1)$ there is $\anexpansion_2\in \Exp(\Gamma_2)$ such that $\anexpansion_2\homto \anexpansion_1$. 
	% \end{itemize}
\end{proposition}


\paragraph*{General assumptions.}
To simplify proofs, we often assume that the regular languages are described via non-deterministic finite automata (NFA) instead of regular expressions,
which does not affect any of our complexity bounds.
However, for readability all our examples will be given in terms of regular expressions.

\AP We denote by $\intro*\nbatoms{\gamma}$ the number of "atoms" of a "CRPQ" $\gamma$ and by $\intro*\nbvar{\gamma}$ the number of variables.
We extend these notations to a "UCRPQ" $\Gamma$ by letting
$\nbatoms{\Gamma} = \max_{\gamma \in \Gamma} \nbatoms{\gamma}$
and $\nbvar{\Gamma} = \max_{\gamma \in \Gamma} \nbvar{\gamma}$. 
We denote by $\size{\Gamma}$ the size (of a reasonable encoding) of a  "UCRPQ". 
For a CRPQ $\gamma$, we define its  \AP""underlying graph"" $\intro*\underlying{\gamma}$ of $\gamma$ as the directed multigraph obtained from $\gamma$ by ignoring the regular languages labelling the atoms of $\gamma$. 

We assume familiarity with basic concepts of directed multigraphs.
For simplicity, thorough the paper, by `graph' we mean a directed multigraph. We also adapt implicitly in the natural way, concepts defined for "CRPQs" to (directed multi)graphs (such a "homomorphisms", "embeddings", etc.).
A \AP""minor"" of a graph is any graph that can be obtained by removing
edges, removing vertices---and their adjacent edges---, and \AP""contracting edges""---meaning that we identify the two endpoints of the edge and remove the edge from the graph.%
\footnote{This definition is a trivial generalization of the notion of minors for undirected graphs.}
