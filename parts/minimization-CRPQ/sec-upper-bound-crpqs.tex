\section{An Upper Bound for Minimization of CRPQs}
\AP\label{sec:upperCRPQ}
In this section we show that the "minimization problem" for "CRPQs" is decidable, in particular, it belongs to the class "2ExpSpace".

\begin{theorem}
	\AP\label{thm:2expspace-min-crpqs}
	The "minimization problem" for "CRPQs" is in "2ExpSpace".
\end{theorem}

The proof of \Cref{thm:2expspace-min-crpqs} is based on a key lemma stated below. Intuitively, this lemma tells us that if a "CRPQ" $\gamma$ is "equivalent" to another "CRPQ" $\alpha$, then $\gamma$ is also equivalent to a "CRPQ" $\beta$, where $\beta$ has the same “shape” than $\alpha$ but the sizes of the NFAs appearing in $\beta$ are bounded by the size of $\gamma$. In particular, if $\gamma$  is equivalent to a  "CRPQ" with at most $k$ atoms, then it is equivalent to a "CRPQ" with at most $k$ atoms and NFAs of bounded sizes, and hence the search space in the minimization problem becomes finite. A careful analysis of this idea yields our "2ExpSpace" upper bound.

\begin{lemma}
	\AP\label{lemma:crpq-size-bound}
	Let $\gamma$ and $\alpha$ be two "CRPQs" such that $\alpha \contained \gamma$. Then there exists a "CRPQ" $\beta$ satisfying $\alpha \contained \beta\contained \gamma$ such that:
	\begin{enumerate}
		\item The "underlying graphs" of $\alpha$ and $\beta$ coincide.
		\item The size of each NFA appearing in $\beta$ is bounded by $f(\size{\gamma})$,
			where $f$ is a double-exponential function.
	\end{enumerate}
\end{lemma}

We star by giving a proof sketch of this lemma before giving the full
detailed proof.

\todo{check that proof sketch/proof don't overlap}
\begin{proof}[Proof sketch]
	The idea is to define the "CRPQ" $\beta$ as the "CRPQ" obtained from $\alpha$ by replacing each regular language $L$ by a suitable regular language $\tilde{L}$, which depends also on $\gamma$. Consider the following equivalence relation on $\A^*$, where $\A$ is the underlying alphabet. Given $u,v\in \A^*$, we write $u \intro*\weakequiv{\gamma} v$ if for every NFA $\+A$ appearing on $\gamma$, and pair of states $p,q$ of $\+A$:
	$$\text{$u$ is accepted by $\subaut{\+A}{p}{q}$ $\iff$  $v$ is accepted by $\subaut{\+A}{p}{q}$}.$$
	Recall that  $\subaut{\+A}{p}{q}$ denote the "sublanguage" of $\+A$ recognized  when considering $\set{p}$ as the set of initial states and $\set{q}$ as the set of final states. %So in a sense, whenever $u \weakequiv{\gamma} v$ holds, then $u$ and 
	For $u\in \A^*$, we define its \AP""$\gamma$-type"" to be:
	\[
		\AP\intro*\typeMin{\gamma}{u} \defeq \{
		([u_1]_{\weakequiv{\gamma}},\dots,[u_\ell]_{\weakequiv{\gamma}}) \mid \text{$\ell\leq (\nbvar{\gamma} +1)$, and $u_1,\dots,u_\ell\in \A^*$ satisfy that $u=u_1\cdots u_\ell$} \}.
	\]
	The idea is that $\typeMin{\gamma}{u}$ encodes all the possible ways $u$ can be broken into $\ell\leq (\nbvar{\gamma}+1)$ subwords: we are not interested in the particular subwords $u_i$, but in their equivalence classes $[u_i]_{\weakequiv{\gamma}}$ with respect to $\weakequiv{\gamma}$.

	We define the sought "CRPQ" $\beta$ to be the "CRPQ" obtained from $\alpha$ by replacing each regular language $L$ by $\tilde{L}$, where:
	\[
	\tilde{L} \defeq \bigcup_{u\in L} \{z\in \A^*\mid \typeMin{\gamma}{u}\subseteq \typeMin{\gamma}{z}\}.
	\]
  
  
	By definition, the "underlying graphs" of $\alpha$ and $\beta$ are the same, and hence condition (1) holds. It remains to verify  $\alpha \contained \beta\contained \gamma$  and condition (2). Note that $u\in \{z\in \A^*\mid \typeMin{\gamma}{u}\subseteq \typeMin{\gamma}{z}\}$ always holds, and hence $L\subseteq \tilde{L}$. It follows that $\alpha \contained \beta$. 

	Showing $\beta\contained \gamma$ is more involved. By Proposition~\ref{prop:cont-char-exp-st}, we need to prove that for every "expansion" $\anexpansion_\beta\in \Exp(\beta)$, there exists an "expansion" $\eta_\beta\in \Exp(\gamma)$, such that $\eta_\beta\homto \anexpansion_\beta$. Assume that $\beta$ and $\gamma$ are of the form $\bigwedge_{j=1}^m x_j \atom{\tilde{L}_j} y_j$ and $\bigwedge_{i=1}^r t_i \atom{M_i} s_i$, respectively. In particular, $\alpha$ must be of the form $\bigwedge_{j=1}^m x_j \atom{L_j} y_j$. Suppose $\anexpansion_\beta$ is of the form  $\bigwedge_{j=1}^m x_j \atom{z_j} y_j$, where $z_j\in \tilde{L}_j$. By construction, $\anexpansion_\beta$ has a corresponding expansion $\anexpansion_\alpha$ of $\alpha$: since each $z_j\in \tilde{L}_j$, there must be a $u_j\in L_j$ such that $\typeMin{\gamma}{u_j}\subseteq \typeMin{\gamma}{z_j}$, and then we can take $\anexpansion_\alpha$ as $\bigwedge_{j=1}^m x_j \atom{u_j} y_j$. In turn, since $\alpha\contained \gamma$, there exists an "expansion" $\eta_\alpha\in \Exp(\gamma)$ such that $\eta_\alpha\homto \anexpansion_\alpha$ via a homomorphism $f$. 

	The idea is to modify $\eta_\alpha$ into another expansion $\eta_\beta$ with  $\eta_\beta\to \anexpansion_\beta$, as desired. Note that $f$ maps the external variables of $\eta_\alpha$ to external or internal variables in $\anexpansion_\alpha$. This determines a subdivision for each path $x_j \atom{u_j} y_j$ of $\anexpansion_\alpha$ into smaller or `basic paths’, whose endpoints correspond to external variables of $\xi_\alpha$ or images of the external variables  of $\eta_\alpha$ via $f$. The number of these paths is hence bounded by $\nbvar{\gamma} +1$. Since $\typeMin{\gamma}{u_j}\subseteq \typeMin{\gamma}{z_j}$, then each path $x_j \atom{z_j} y_j$ in $\anexpansion_\beta$ can also be subdivided in an equivalent way than  $x_j \atom{u_j} y_j$. Overall, the decomposition of $\anexpansion_\alpha$  into basic paths can be `simulated’ in $\anexpansion_\beta$. This gives us a homomorphism from $\eta_\beta$ to $\anexpansion_\beta$, where $\eta_\beta$ is obtained from $\eta_\alpha$ in the following way: for each path $t_i \atom{w_i} s_i$ in $\eta_\alpha$, where $w_i\in M_i$, the image of the path via $f$ induces a subdivision of $t_i \atom{w_i} s_i$ into basic paths  of $\anexpansion_\alpha$. We can replace each of these basic paths by its equivalent basic path in $\anexpansion_\beta$. As the label of these paths are equivalent w.r.t to the relation $\weakequiv{\gamma}$, membership in $M_i$ is maintained after this transformation. Hence $\eta_\beta$ is indeed an expansion of $\gamma$.

	For condition (2), it is easy to see that every equivalence class $C$ of the relation $\weakequiv{\gamma}$ can be accepted by an NFA $\+A_C$ of single-exponential size on $\size{\gamma}$. Also, for each word $u\in \A^*$, and  each tuple $\bar{c} =(C_1,\dots,C_\ell)\in \typeMin{\gamma}{u}$, there is a single-exponential size NFA $\+A_{\bar{c}}$  accepting the language $\{z\in \A^*\mid \bar{c} \in \typeMin{\gamma}{z}\}$. By intersecting these NFAs, we obtain an NFA $\+A_{u}$ accepting the language $\{z\in \A^*\mid \typeMin{\gamma}{u}\subseteq \typeMin{\gamma}{z}\}$. It is easy to see that the number of tuples of the form $(C_1,\dots,C_\ell)$ is at most single-exponential, and then the size of $\+A_{u}$ is at most double-exponential. It follows that  the number of possible  $\typeMin{\gamma}{u}$ is at most double-exponential, and hence the languages  $\tilde{L}$ in $\beta$ can be described by a union of double-exponential many NFAs, each of size at most double-exponential. Overall, each $\tilde{L}$ can be described by an NFA of at most double-exponential size on $\size{\gamma}$.
\end{proof}

\begin{proof}[Proof details]
	\proofcase{Construction of $\beta$.}
	% The idea is to define the "CRPQ" $\beta$ as the "CRPQ" obtained from $\alpha$ by replacing each regular language $L$ by a suitable regular language $\tilde{L}$, which depends also on $\gamma$. 
	% Consider the following equivalence relation on $\A^*$, where $\A$ is the underlying alphabet. Given $u,v\in \A^*$, we write $u \intro*\weakequiv{\gamma} v$ if for every NFA $\+A$ appearing on $\gamma$, and pair of states $p,q$ of $\+A$:
	% $$\text{$u$ is accepted by $\subaut{\+A}{p}{q}$ $\iff$  $v$ is accepted by $\subaut{\+A}{p}{q}$}.$$
	% For $u\in \A^*$, we define its ""$\gamma$-type"" to be:
	% \[
	% 	\typeMin{\gamma}{u} \defeq \{
	% 	([u_1]_{\weakequiv{\gamma}},\dots,[u_\ell]_{\weakequiv{\gamma}}) \mid \text{$\ell\leq (\nbvar{\gamma} +1)$, and $u_1,\dots,u_\ell\in \A^*$ satisfy that $u=u_1\cdots u_\ell$} \}.
	% \]
	% The idea is that $\typeMin{\gamma}{u}$ encodes all the possible ways $u$ can be broken into $\ell\leq (\nbvar{\gamma}+1)$ subwords: we are not interested in the particular subwords $u_i$, but in their equivalence classes $[u_i]_{\weakequiv{\gamma}}$ with respect to $\weakequiv{\gamma}$.
	Recall that $\beta$ is defined as the "CRPQ" obtained from $\alpha$ by replacing each regular language $L$ by $\tilde{L}$, where:
	\[
		\tilde{L} \defeq \bigcup_{u\in L}
			\{z\in \A^*\mid \typeMin{\gamma}{u}\subseteq \typeMin{\gamma}{z}\}.
	\]

	
	\proofcase{Correctness of the construction.} By definition, the "underlying graphs" of $\alpha$ and $\beta$ are the same, and hence condition (1) holds. It remains to verify  $\alpha \contained \beta\contained \gamma$  and condition (2).

	\proofsubcase{$\alpha \contained \beta\contained \gamma$.} 
	Note that $u\in \{z\in \A^*\mid \typeMin{\gamma}{u}\subseteq \typeMin{\gamma}{z}\}$ always holds, and hence $L\subseteq \tilde{L}$. It follows that $\alpha \contained \beta$. 

	We check $\beta\contained \gamma$ using Proposition~\ref{prop:cont-char-exp-st}. Assume $\beta$ is of the form $\bigwedge_{j=1}^m x_j \atom{\tilde{L}_j} y_j$. Consider an "expansion" $\anexpansion_\beta\in \Exp(\beta)$ defined by replacing each atom $x_j \atom{\tilde{L}_j} y_j$ by the path $x_j \atom{z_j} y_j$, for  $z_j\in \tilde{L}_j$. There must exist $u_j\in L_j$ such that $\typeMin{\gamma}{u_j}\subseteq \typeMin{\gamma}{z_j}$. Let $\anexpansion_{\alpha}\in \Exp(\alpha)$ be the expansion of $\alpha=\bigwedge_{j=1}^m x_j \atom{L_j} y_j$ obtained from replacing each atom $x_j \atom{L_j} y_j$ by the path $x_j \atom{u_j} y_j$. As $\alpha \contained \gamma$, then there exists an expansion $\eta_\alpha\in \Exp(\gamma)$ such that $\eta_\alpha\homto \anexpansion_\alpha$. Assume $\gamma$ is of the form $\bigwedge_{i=1}^r t_i \atom{M_i} s_i$ and $\eta_\alpha$ is of the form $\bigwedge_{i=1}^r t_i \atom{w_i} s_i$, where $w_i\in M_i$. Let $f$ be a homomorphism from $\eta_\alpha$ to $\anexpansion_\alpha$. We can decompose each path $x_j \atom{u_j} y_j$ in $\anexpansion_\alpha$ into
	\[
		x_j \atom{u_{j,1}} h_{j,1} \atom{u_{j,2}} \cdots \atom{u_{j,\ell-1}} h_{j,\ell-1}\atom{u_{j,\ell}} y_j
	\]
	where each $h_{j,p}$ is the image via $f$ of some variable in $\vars{\gamma}=\{s_1,t_1,\dots,s_r,t_r\}$ and each path $x \atom{u_{j,p}} x’$ satisfies that all of its internal variables are not  images via $f$ of variables in $\vars{\gamma}$. Note that $\ell\leq  (\nbvar{\gamma} +1)$. We say that the paths of the form $x \atom{u_{j,p}} x’$ are the ""basic paths"" of  $\anexpansion_\alpha$ with respect to $f$. Since $\typeMin{\gamma}{u_j}\subseteq \typeMin{\gamma}{z_j}$, each path $x_j \atom{z_j} y_j$ in $\anexpansion_\beta$ can be decomposed into
	\[
		x_j \atom{z_{j,1}} g_{j,1} \atom{z_{j,2}} \cdots \atom{z_{j,\ell-1}} g_{j,\ell-1}\atom{z_{j,\ell}} y_j
	\]
	where $z_{j,p}\weakequiv{\gamma} u_{j,p}$. Again, we say that the paths $x \atom{z_{j,p}} x’$ are the "basic paths" of  $\anexpansion_\beta$ with respect to $f$. We conclude by showing that there is $\eta_\beta\in \Exp(\gamma)$ such that $\eta_\beta\homto \anexpansion_\beta$. Intuitively, $\anexpansion_\alpha$ and $\anexpansion_\beta$ have ``equivalent’’ decompositions in terms of "basic paths". Hence, it is possible to turn $\eta_\alpha$ into another expansion $\eta_\beta\in \Exp(\gamma)$ by replacing "basic paths" of $\anexpansion_\alpha$  by their corresponding "basic paths" in $\anexpansion_\beta$. By doing so, we indeed obtain an expansion $\eta_\beta$ of $\gamma$ such that $\eta_\beta\homto \anexpansion_\beta$. 
	

	Formally, observe that each path $t_i \atom{w_i} s_i$ of expansion $\eta_\alpha$ is mapped via $f$ to path in $\anexpansion_\alpha$ that can be decomposed into
		\[
		k_0 \atom{w_{i,1}} k_1 \atom{w_{i,2}} \cdots \atom{w_{i,n-1}} k_{n-1}\atom{w_{i,n}} k_n
	\]
	where $w_i=w_{i,1}\cdots w_{i,n}$, $k_0=f(t_i)$, $k_n=f(s_i)$ and each path $k_{q-1} \atom{w_{i,q}} k_q$ is a "basic path" of $\anexpansion_\alpha$ w.r.t.\ $f$. Each path $k_{q-1} \atom{w_{i,q}} k_q$ has a corresponding "basic path" $k’_{q-1} \atom{w_{i,q}’} k_q’$ of $\anexpansion_\beta$ w.r.t.\ $f$ defined in the natural way: if $k_o\in \vars{\alpha}=\{x_1,y_1,\dots,x_m,y_m\}$, then $k_o’=k_o$; if $k_o=h_{j,p}$, then $k_o’=g_{j,p}$; and if $w_{i,q}= u_{j,p}$, then $w_{i,q}’= z_{j,p}$. In particular, we have that $w_{i,q} \weakequiv{\gamma} w_{i,q}’$ and that the following path belongs to $\anexpansion_\beta$:
	\[
		k_0’ \atom{w_{i,1}’} k_1’ \atom{w_{i,2}’} \cdots \atom{w_{i,n-1}’} k_{n-1}’\atom{w_{i,n}’} k_n’
	\]
	It follows that the CQ $\eta_{\beta}$ obtained from $\gamma$ by replacing each atom $t_i \atom{M_i} s_i$ by the path  $t_i \atom{w_i’} s_i$, where $w’_{i}=w_{i,1}’\cdots w_{i,n}’$ satisfies that $\eta_{\beta}\homto \anexpansion_{\beta}$. It remains to show that  $\eta_{\beta}$ is actually an expansion of  $\gamma$, that is, each $w’_i\in M_i$. Suppose $M_i$ is represented by the NFA $\+A_i$. Since $w_i\in M_i$ and $w_i=w_{i,1}\cdots w_{i,n}$, there is a sequence $e_0,\dots,e_n$ of states of $\+A_i$ such that $e_0$ is initial, $e_n$ is final, and $w_{i,q}$ is accepted by $\subaut{\+A_i}{e_{q-1}}{e_q}$. As $w_{i,q}’\weakequiv{\gamma} w_{i,q}$, we have that $w_{i,q}’$ is also accepted by $\subaut{\+A_i}{e_{q-1}}{e_q}$, and hence $w’_{i}=w_{i,1}’\cdots w_{i,n}’$ is accepted by $\+A_i$.
		
	\proofsubcase{Bounding the size of NFAs in $\beta$.} 
	First note that for any equivalence class $C$ of $\weakequiv{\gamma}$ the language $\{z\in \A^*\mid [z]_{\weakequiv{\gamma}} = C\}$ is accepted by an NFA $\+A_{C}$ of single-exponential size. Indeed, the class $C$ can be described by a set of triples
		\[
			\+T_C\subseteq \+T=\{(\+A,p,q)\mid \text{$\+A$ is an NFA in $\gamma$, and $p,q$ are states of $\+A$}\}
		\]
	in such a way that $v\in C$ iff $v$ is accepted by $\subaut{\+A}{p}{q}$, for every $(\+A,p,q)\in \+T_C$, and $v$ is accepted by $\subaut{\+A}{p}{q}^{\complement}$, for every $(\+A,p,q)\not\in \+T_C$, where $\+A^{\complement}$ denotes the complement NFA of $\+A$. Hence $\+A_{C}$ can be written as the following intersection of NFAs:
	\[
			\+A_{C} = \bigcap_{(\+A,p,q)\in \+T_C} \subaut{\+A}{p}{q} \cap  \bigcap_{(\+A,p,q)\not\in \+T_C} \subaut{\+A}{p}{q}^{\complement} 
		\]
		
	The number of states in $\subaut{\+A}{p}{q}^{\complement}$ is at most $2^{r}$ and $|\+T|\leq r^2 \nbatoms{\gamma}$, where $r$ is the maximum number of states in an NFA of $\gamma$. Hence the number of states of $\+A_{C}$ is at most $2^{r^3\nbatoms{\gamma}}$, that is, single-exponential on $\gamma$.

	Fix $u\in\A^*$. We claim that $\{z\in \A^*\mid \typeMin{\gamma}{u}\subseteq \typeMin{\gamma}{z}\}$ can be accepted by an NFA $\+A_u$ of at most double-exponential size on $\gamma$. It is easy to see that for every tuple $\bar{c} =(C_1,\dots,C_\ell)\in \typeMin{\gamma}{u}$, the language $\{z\in \A^*\mid \bar{c} \subseteq \typeMin{\gamma}{z}\}$ can be described by an NFA $\+A_{\bar{c}}$ of single-exponential size: we guess a decomposition $z=z_1\cdots z_\ell$ of the input word $z$ and check that each $z_i$ is accepted by $\+A_{C_i}$. The number of states of $\+A_{\bar{c}}$ is $O(\sum_{i=1}^{\ell} s_i)$, where $s_i$ is the number of states of $\+A_{C_i}$. Since $\ell\leq (\nbvar{\gamma} +1)$, this is $O(\nbvar{\gamma}2^{r^3\nbatoms{\gamma}})$. Now, the NFA $\+A_u$ is simply the intersection of all the NFAs in $\{\+A_{\bar{c}}\mid \bar{c}\in \typeMin{\gamma}{u}\}$. The number of possible equivalence classes of $\weakequiv{\gamma}$ is at most $2^{|\+T|}\leq 2^{r^2 \nbatoms{\gamma}}$, and then the number of possible tuples of the form $\bar{c}=(C_1,\dots,C_\ell)$ is $N=O(2^{r^2 \nbatoms{\gamma}(\nbvar{\gamma} +1)})$. It follows that the number of states of $\+A_u$ is at most $O(\nbvar{\gamma}2^{r^3\nbatoms{\gamma} N})$, i.e., double-exponential on $\gamma$.

	Finally, note that the number of possible $\typeMin{\gamma}{u}$ is at most double-exponential on $\gamma$, more precisely, $2^N$. We conclude that every language $\tilde{L}$ in $\beta$ can be represented by an NFA $\+A_{\tilde{L}}$ which is the union of at most double-exponential many NFAs of the form $\+A_u$, each of these having at most double-exponential many states. We conclude that the size of $\+A_{\tilde{L}}$ can be bounded by a double-exponential function $f(\size{\gamma})$. 
\end{proof}

Using this lemma, we can then easily deduce the desired "2ExpSpace" upper bound.

\begin{proof}[Proof of \Cref{thm:2expspace-min-crpqs}]
	The "2ExpSpace" algorithm proceeds as follows. Let $\gamma$ be a "CRPQ"  over $\A$ and $k\in\N$. Let $f$ be the double-exponential function from \Cref{lemma:crpq-size-bound}. We enumerate all possible directed multigraphs with at most $k$ edges. For each of these graphs, we enumerate all the possible "CRPQs" $\beta$ obtained by labelling each edge with an NFA of size bounded by $f(\size{\gamma})$. If for some of these $\beta$, we have that $\beta\semequiv \gamma$ then we accept the instance, otherwise we reject it. \Cref{lemma:crpq-size-bound} ensures that this algorithm is correct.

	It remains to show that  $\beta\semequiv \gamma$ can be carried out in "2ExpSpace". We use a proposition from \cite[Proposition 3.11, p. 17]{FigueiraMorvan2025SemanticTreeWidthLMCS} stating that containment $\Gamma\contained \Delta$, for "UCRPQs" $\Gamma$ and  $\Delta$, can be solved in non-deterministic space $O(\size{\Gamma} + \size{\Delta}^{c\cdot \nbatoms{\Delta}})$, where $c$ is a constant. This implies that $\beta\contained \gamma$ can be checked in space $O(\size{\beta} + \size{\gamma}^{c\cdot \nbatoms{\gamma}})$, and then within "2ExpSpace". The containment $\gamma\contained \beta$ can be solved in space $O(\size{\gamma} + \size{\beta}^{c\cdot k})$, and hence also in "2ExpSpace".
\end{proof}


