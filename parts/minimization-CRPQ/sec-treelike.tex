\section{Tree-Like Queries}

\diego{Show that for pseudo-tree queries the minimization pb is in PSpace? And that if langauges are ``simple'', then it matches the complexity for "tree pattern" minimization, namely "SigmaP2"?}

\textcolor{Dark Ruby Red}{WARNING: In this section, all CRPQs are assumed to be positive, meaning
that no language can contain the empty word.}

\subsection{Forest-Shaped and DAG-Shaped Queries}

Say that a "CRPQ" is \AP""forest-shaped"" if its underlying directed multigraph 
is a directed simple forest---without self loops---, with the arrows going from roots to leaves.

We say that a CRPQ is \AP""semantically forest-shaped"" if it is
"semantically equivalent" to a CRPQ which is "forest-shaped".
A \AP""maximal canonical database"" of $\gamma$ is a "canonical database"
$G$ of $\gamma$ "st" for all "canonical database" $H$ of $\gamma$,
if $G \cohomto H$, then $H \cohomto G$.

Say that a CRPQ is \AP""DAG-shaped"" if its underlying directed multigraph
is a DAG---parallel edges are allowed, not self loops. If $\delta$ is "DAG-shaped",
define its \AP""unfolding"", denoted by \AP$\intro*\Unfold(\delta)$, as the following "CRPQ":
\begin{itemize}
	\item its variables are exactly labelled path of $\delta$
	of the form $x_0 \atom{L_1} \cdots \atom{L_n} x_n$ with $n\in \N$
	and $x_0$ is a vertex of $\delta$ with no predecessor;
	\item the atoms exactly go from $x_0 \atom{L_1} \cdots \atom{L_n} x_n$
	to $x_0 \atom{L_1} \cdots \atom{L_n} x_n \atom{L_{n+1}} x_{n+1}$,
	with label~$L_{n+1}$.
\end{itemize}

\begin{fact}
	\AP\label{fact:unfolding-is-forest}
	If $\delta$ is a "DAG-shaped CRPQ", then $\Unfold(\delta)$ is a "forest-shaped CRPQ",
	and moreover $\delta \contained \Unfold(\delta)$.
\end{fact}

The rest of this section is devoted to proving the following result.

\begin{theorem}
	\AP\label{thm:charac-semantically-forest-shaped}
	Let $\delta$ be a "CRPQ". The following are equivalent:
	\begin{enumerate}
		\item $\delta$ is "semantically forest-shaped",
		\item $\delta$ is "DAG-shaped" and
			for all "maximal canonical database" $D$ of $\delta$, the "core" of $D$ is a forest,
		\item $\delta$ is "DAG-shaped" and $\delta \semequiv \Unfold(\delta)$. 
	\end{enumerate}
\end{theorem}

Note that since "semantical equivalence" is decidable in "ExpSpace" and
since $\Unfold(\delta)$ has exponential size, it follows that one can test if
a "CRPQ" is "semantically forest-shaped" in "2ExpSpace".

\subsection{Semantically DAG-Shaped Queries}

\begin{proposition}
	A "CRPQ" is "semantically DAG-shaped" "iff" it is "DAG-shaped".
\end{proposition}

\begin{proof}
todo
\end{proof}

\begin{corollary}
	\AP\label{coro:sem-forest-implies-DAG}
	If a "CRPQ" is "semantically forest-shaped", then it is "DAG-shaped".
\end{corollary}

\subsection{Semantically Forest-Shaped}

From the fact that a "CRPQ" $\delta$ is equivalent to a "forest-shaped" query $\phi$
we know that for all "canonical database" $D_0$ of $\delta$, since $\delta \contained \phi$,
there exists a "canonical database" $F_0$ of $\phi$ "st" $D_0 \cohomto F_0$.
But dually since $\phi \contained \delta$, there exists $D_1 \cdb \delta$ "st" $F_0 \cohomto D_1$.
By induction---and the axiom of choice---we obtain an infinite co-chain of "homomorphisms"
\[
	D_0 \cohomto F_0 \cohomto D_1 \cohomto F_1 \cohomto \cdots \cohomto D_n \cohomto F_n \cohomto \cdots.
\]
We show that co-chains of forests are actually quite simple.

\begin{fact}
	If $F_0 \cohomto F_1 \cohomto \cdots \cohomto F_n \cohomto \cdots$
	is an infinite co-chain of "homomorphisms" betweens "forests", 
	then there exists $n \in \N$ "st" all $F_m$ with $m\geq n$
	are "hom-equivalent" to one another.
\end{fact}

\begin{proof}
	For all $n \in \N$, from $F_n \cohomto F_{n+1}$ it follows that
	the maximal depth of a tree in $F_{n+1}$ is smaller or equal to the
	maximal depth of a tree in $F_{n}$. So, at some point this parameter
	must become stationary, say $d$. Then observe that there
	are finitely many forests with depth at most $d$, up to "hom-equivalence",
	and hence, one of these must occur infinitely often in the co-chain.
\end{proof}

\begin{corollary}
	\AP\label{coro:all-cdb-are-dominated-by-forests}
	If $\delta$ is "semantically forest-shaped", then for any
	$D \cdb \delta$, there exists $D' \cdb \delta$ such that
	$D \cohomto D'$, $D'$ is "maximal" and the core of $D'$ is a forest.
\end{corollary}

We can now proceed with the proof of \Cref{thm:charac-semantically-forest-shaped}, after giving a
proposition that will prove useful.

\begin{proposition}
	\AP\label{prop:hom-from-forest}
	Let $F$ be a forest and $D$ a graph. If $F \homto D$ then $F \homto \Unfold(D)$.
\end{proposition}

\begin{proof}
	The "homomorphism" $F \homto \Unfold(D)$ can be defined by induction on $F$, from roots
	to leaves.
\end{proof}

\begin{proof}[Proof of \Cref{thm:charac-semantically-forest-shaped}.]
	\proofcase{$(1) \Rightarrow (2)$.} This follows from
	\Cref{coro:sem-forest-implies-DAG,coro:all-cdb-are-dominated-by-forests}.

	\proofcase{$(2) \Rightarrow (3)$.} By \Cref{fact:unfolding-is-forest} we have $\delta \contained \Unfold(\delta)$ so it suffices to prove the converse "containment".
	Let $U$ be a "canonical database" of $\Unfold(\delta)$. Then there exists $D \cdb \delta$
	"st" $U = \Unfold(D)$. By \Cref{coro:all-cdb-are-dominated-by-forests} there exists
	$D' \cdb \delta$ "st" $D \cohomto D'$ and the core of $D'$ is a forest. So $D \cohomto \core(D')$, and so by \Cref{prop:hom-from-forest}, since $\core(D')$ is a "forest",
	then $\Unfold(D) \cohomto \core(D')$ "ie" $\Unfold(D) \cohomto D'$, which proves
	that $\Unfold(D) \models \delta$. Therefore, $\Unfold(\delta) \contained \delta$.

	\proofcase{$(3) \Rightarrow (1)$.} This is because $\Unfold(\delta)$ is "forest-shaped"
		by \Cref{fact:unfolding-is-forest}.
\end{proof}