\section{Minimization of UCRPQs via Approximations}
\label{sec:ucrpqs}

\AP\label{sec:upperUCRPQ}
We now focus on minimizations by \emph{finite unions} of "CRPQs". This is a different problem than the one seen in the previous section, "ie", there is no obvious reduction in either direction between the "minimization problem" for "CRPQs" and the "minimization problem" for "UCRPQs", and indeed we will solve this problem using an altogether different approach. 
%In particular, having unions may help in minimizing the (maximum) number of atoms of a query as the next subsection shows.


\subsection{Unions Allow Further Minimization}

As it turns out, having unions may help in minimizing the (maximum) number of atoms of a query as the next proposition shows.
\begin{proposition}
	\AP\label{prop:unionsmatter}
	There exist "CRPQs" which are "minimal" among "CRPQs" but not among "UCRPQs".
\end{proposition}
\begin{proof}
	The following example is inspired from \cite[Example 1.2]{FigueiraMorvan2025SemanticTreeWidthLMCS}.
	Consider the following "CRPQs":\leavevmode
	\begin{center}
		\small
		\begin{tikzcd}
		                    &[2em] x  
		                    \dar["a"]
		                    \ar[dd, "a(bb)^+" below left, bend right=80]
		                    \ar[dd, "{ab(bb)^*,}" below right, bend left=80]
		                                & \\
		    \gamma() \defeq &       y   
		                    \ar["\marking" above right=.4em and 0em, loop left]
		                    \dar["b^+" description]
		                                & \\
		                    &       z   &
		\end{tikzcd}
		\hspace{.4cm}
		\begin{tikzcd}
		                    & x  
		                    \dar["a"]
		                    \ar[dd, "ab(bb)^*"  below right, bend left=80]
		                                & \\
		    \delta_0() \defeq &       y   
		                    \ar["\marking" above right=.4em and 0em, loop left]
		                    \dar["(bb)^+" left]
		                                & \\
		                    &       z   &
		\end{tikzcd}
		\hspace{.2cm}and\hspace{.2cm}
		\begin{tikzcd}
		                    &[2em] x  
		                    \dar["a"]
		                    \ar[dd, "a(bb)^+"  below left, bend right=80]
		                                & \\
		    \delta_1() \defeq &       y.   
		                    \ar["\marking" above right=.4em and 0em, loop left]
		                    \dar["b(bb)^*"]
		                                & \\
		                    &       z   &
		\end{tikzcd}
	\end{center}
	It is easy to see that $\gamma \semequiv \delta_0 \lor \delta_1$ 
	by doing a case disjunction on the parity of the path between $y$ and $z$---the even case
	is handled by $\delta_0$, the odd case by $\delta_1$.
	Hence, $\gamma$ is not "minimal" among "UCRPQs".
	
	We want to show that $\gamma$ is "minimal" among "CRPQs".
	Let $\zeta()$ be a "CRPQ" that is "equivalent" to $\gamma$, and assume by contradiction
	that it has at most four "atoms".
	Given natural numbers $l,m,r \in \Np$ where $l$ is even and $r$ is odd,
	consider the "expansion"
	\begin{center}
		\small
		\begin{tikzcd}
		                    &[2em] x  
		                    \dar["a"]
		                    \ar[dd, "ab^{l}" below left, squiggly, bend right=80]
		                    \ar[dd, "{ab^{r},}" below right, squiggly, bend left=80]
		                                & \\
		    \anexpansion_{l,m,r}() \defeq &       y   
		                    \ar["\marking" above right=.4em and 0em, loop left]
		                    \dar["b^m" description, squiggly]
		                                & \\
		                    &       z   &
		\end{tikzcd}
	\end{center}
	where squiggly arrows represent paths of atoms.
	\begin{claim}
		\AP\label{claim:example-union-matter-1}
		The "expansion" $\anexpansion_{l,m,r}$ is "hom-minimal" "iff" $l = m$ or $m=r$.
	\end{claim}
	Indeed, if $l \neq m$ and $m \neq r$, then
	if $m$ is even then $\anexpansion_{m,m,r} \homto \anexpansion_{l,m,r}$
	but $\anexpansion_{l,m,r} \nothomto \anexpansion_{m,m,r}$ (since $l \neq m$)
	and dually if $m$ is odd then $\anexpansion_{l,m,m} \homto \anexpansion_{l,m,r}$
	but $\anexpansion_{l,m,r} \nothomto \anexpansion_{l,m,m}$ (since $m \neq r$).
	In both cases, $\anexpansion_{l,m,r}$ is not "hom-minimal".
	
	Conversely, if $l = m$ or $m=r$, assume "wlog", by symmetry, that $l=m$.
	Let $l',m',r' \in \Np$ "st" $l'$ is even and $r'$ is odd,
	and assume that there is a "homomorphism" $f\colon \anexpansion_{l',m',r'} \to \anexpansion_{l,m,r}$. Because of the $\marking$-self-loop, $f(y) = y$ and hence $f(x) = x$. It then follows that
	$f(z) = z$ because we must have both an $a(bb)^+$- and an $ab(bb)^*$-path
	from $f(x)$ and $f(z)$. 
	Moreover, we must have $m=m'$,
	and since $l'$ is even, we must have $l' = l = m$,
	and dually, since $r'$ is odd, we must have $r' = r$.
	It follows that $\tup{l',m',r'} = \tup{l,m,r}$, and hence $\anexpansion_{l,m,r}$ is "hom-minimal".

	\begin{claim}
		\AP\label{claim:example-union-matter-2}
		If $m=r$, then $\core(\anexpansion_{l,m,r})$ equals
		\begin{center}
			\small
			\begin{tikzcd}
			    x  
			    \dar["a"]
			    \ar[dd, "ab^{l}" below left, squiggly, bend right=80]
			                & \\
			    y.   
			    \ar["\marking" above right=.4em and 0em, loop left]
			    \dar["b^m" description, squiggly]
			                & \\
			    z   &
			\end{tikzcd}
		\end{center}
	\end{claim}
	Putting \Cref{claim:example-union-matter-1,claim:example-union-matter-2} and
	\Cref{thm:structure-theorem} together, we get that 
	$\seggraph(\core(\anexpansion_{2,1,1}))$ is a "minor" of $\zeta$.
	Since $\zeta$ was assumed to have at most four "atoms",
	it follows that $\zeta()$ must be exactly of the form
	\begin{center}
		\begin{tikzcd}
		              &[.5em] x  
		              \dar["U"]
		              \ar[dd, "L" below left, squiggly, bend right=80]
		                          & \\
		    \zeta() = & y.   
		          \ar["S" above right=.4em and 0em, loop left]
		          \dar["D" description, squiggly]
		                      & \\
		              & z   &
		\end{tikzcd}
	\end{center}
	Using the containment $\zeta \contained \gamma$,
	since the only directed simple cycle in an "expansion" of $\gamma$ is labelled
	by $\marking$, we get $S = \{\marking\}$.
	Similarly, using again that $\zeta \contained \gamma$ and that any "expansion" of $\gamma$
	must have both an $a(bb)^+$- and an $ab(bb)^*$-path, it can be shown that $U = \{a\}$.

	Now let $l,m,r \in \Np$ with $l$ even and $r$ odd.
	From $\gamma \contained \zeta$, we get that each $\anexpansion_{l,m,r}$ "satisfies@@db"
	$\zeta$ and so there is an "evaluation map" $f\colon \zeta \to \anexpansion_{l,m,r}$.
	Clearly $f(y) = y$ and then using the
	$a(bb)^+$- and $ab(bb)^*$-paths, we get that $f(x) = x$ and $f(z) = z$. 
	It then follows that
	\begin{itemize}
		\item $b^m \in D$, and
		\item either $ab^l \in L$, or $ab^r \in L$, or $a \marking^k b^m$ for some $k\in\N$
	\end{itemize}
	From the first point we get $b^+ \subseteq D$, and in fact using $\zeta\contained \gamma$,
	$D = b^+$.

	Now let $w \in L$ and $m \in \N$, and consider the "expansion"
	\begin{center}
		\small
		\begin{tikzcd}
			&[.5em] x  
			\dar["a"]
			\ar[dd, "w" below left, squiggly, bend right=80]
						& \\
			\anexpansion_\zeta() \defeq & y.   
			\ar["\marking" above right=.4em and 0em, loop left]
			\dar["b^m" description, squiggly]
						& \\
			& z   &
		\end{tikzcd}
	\end{center}
	of $\zeta$. Since $\zeta \contained \gamma$, there is
	an "evaluation map" $f\colon \gamma \to \anexpansion_\zeta$.
	Taking $m$ to be any even value, we get that $w$ must be in $ab(bb)^*$,
	say $w = ab^r$ for some odd $r \in \Np$.
	Taking $m$ to be any odd value, we get dually that $w$ must be in $a(bb)^+$,
	say $w = ab^l$ for some even $l \in \Np$.
	Hence, we get $\{ab^l, ab^r\} \subseteq L$.

	We can now pick $m$ to be even and
	$w = ab^l$, and then $\anexpansion_\zeta$ should be a model of $\gamma$.
	But $\anexpansion_\zeta$ does cannot satisfy the atom $x \atom{ab(bb)^*} z$.
	Contradiction. Hence, $\gamma$ cannot be equivalent to a single "CRPQ"
	with at most four "atoms".
\end{proof}

\subsection{Maximal Under-Approximations}

\AP\label{sec:maximal-under-approxiamtions}
Our approach exploits having unions at our disposal, enabling the possibility of defining and computing "maximal under-approximations" for "UCRPQs" having some given underlying shape. This will lead to an "ExpSpace" upper bound for "UCRPQ" minimization.

For a graph class $\+C$ (remember that these are directed multigraphs) we denote by \AP$\intro*\CQ[\+C]$, $\intro*\CRPQ[\+C]$, $\intro*\UCRPQ[\+C]$, $\intro*\infUCRPQ[\+C]$, the class of "CQs", of "CRPQs", of finite unions of "CRPQs" 
and of infinite unions of "CRPQs" $\gamma$ such that $\underlying{\gamma} \in \+C$.
Given a graph class $\+C$ and a "UCRPQ" $\Gamma$, we define $\AppInf{\Gamma}{\+C}$ to be the (infinite) set of all $\CRPQ[\+C]$ queries which are "contractions" of "CQs" "contained" in $\Gamma$. More formally:\footnote{Note that the "contraction" of a "CQ" is a "CRPQ": this is why $\AppInf{\Gamma}{\+C}$ is an infinite union of "CRPQs" and not of "CQs".}
\[
	\intro*\AppInf{\Gamma}{\+C} \defeq \{
		\alpha \in \CRPQ[\+C] \mid \exists \xi \in \Exp(\Gamma), \exists \eta \in \CQ \text{ s.t. }
		\xi \homto \eta \text{ and } \alpha \text{ is a "contraction" of } \eta
	\}.
\]
\begin{proposition}
	\AP\label{prop:max-under-approx}
	If $\+C$ is a graph class closed under taking "minors", then
	$\AppInf{\Gamma}{\+C}$ is a \AP""maximal under-approximation"" of $\Gamma$ by $\infUCRPQ[\+C]$ queries, in the sense that:
	\begin{enumerate}
		\item $\AppInf{\Gamma}{\+C} \in \infUCRPQ[\+C]$,
		\item $\AppInf{\Gamma}{\+C} \contained \Gamma$, and
		\item for any $\Delta \in \infUCRPQ[\+C]$, if $\Delta \contained \Gamma$, then $\Delta \contained \AppInf{\Gamma}{\+C}$.
	\end{enumerate}
\end{proposition}

\begin{proof}
	%TODO (Diego) Make it short
	The first point is trivial. For the second one, observe that
	if $\alpha \in \CRPQ[\+C]$ is "st" there exist $\xi \in \Exp(\Gamma)$ and
	$\eta \in \CQ$ "st" $\xi \homto \eta$ and $\alpha$ is a "contraction" of $\eta$,
	then $\eta \contained \xi$ and $\eta$ is "semantically equivalent" to $\alpha$ by \Cref{fact:produce-fully-contracted}, and hence $\alpha \contained \xi$, from which it follows that $\AppInf{\Gamma}{\+C} \contained \Gamma$.

	For the third point, let $\Delta \in \infUCRPQ[\+C]$ "st" $\Delta \contained \Gamma$. Let $\delta \in \Delta$ and $\xi_{\delta}$ be an "expansion" of $\delta$. Since $\Delta \contained \Gamma$, there exist $\gamma \in \Gamma$ and $\xi_{\gamma} \in \Exp(\gamma)$ "st" $\xi_{\gamma} \homto \xi_{\delta}$. From the fact that $\underlying{\delta} \in \+C$ and that $\xi_{\delta} \in \Exp(\delta)$, it follows that there exists a "contraction" $\alpha_{\delta}$ of $\xi_{\delta}$ "st" $\underlying{\alpha_{\delta}}$ is a "minor" of $\underlying{\delta}$, and thus belongs to $\+C$.
	% $\underlying{\alpha_{\delta}} \in \+C$.
	% \miguel{I don’t see why this is true. The only thing we can ensure is that $G_{\alpha_\delta}$ is a minor of $G_\delta$ (because of the $\varepsilon$’s). If this is the case, we need to assume that $\+C$ is minor-closed. }
	Hence, it follows that $\alpha_{\delta} \in \AppInf{\Gamma}{\+C}$.
	So $\xi_{\delta} \contained \AppInf{\Gamma}{\+C}$ and thus
	$\Delta \contained \AppInf{\Gamma}{\+C}$.
\end{proof}

\begin{remark}
	If $\+C$ is closed under taking "subgraphs" and "expansions", then
	\[\AppInf{\gamma}{\+C} \semequiv \{
		\alpha \in \CQ[\+C] \mid \exists \xi \in \Exp(\gamma), 
		\xi \surjto \alpha
	\}.\]
\end{remark}

Our "ExpSpace" upper bound relies on the following technical lemma.

\begin{lemma}
	\AP\label{lemma:approximation-for-finclass}
	If $\+C$ is finite and closed under taking "minors", then $\AppInf{\Gamma}{\+C}$ is equivalent to a query $\Delta\in \UCRPQ[\+C]$ with exponentially many "CRPQs", each "CRPQ" of $\Delta$ being of size polynomial in $\size{\Gamma} + \max{\set{\nbatoms{G} \mid G \in \+C}}$.
	Further, the membership $\delta \in^? \Delta$ can be tested "NP". 
	In particular, this "UCRPQ" $\Delta$ can be computed in exponential time from $\Gamma$.
\end{lemma}

\begin{proof}
	We start by observing that $\AppInf{\Gamma}{\+C}$ admits an equivalent and more flexible definition in terms of "refinements". This definition will allow us to effectively compute our desired equivalent query $\Delta$, by considering "refinements" of bounded lengths.

	For each $m \in \N\cup\{+\infty\}$, we define the $\infUCRPQ[\+C]$
	\[\AP\intro*\App{\Gamma}{\+C}{m} \defeq \{
		\alpha \in \CRPQ[\+C] \mid \exists \rho \in \Refin[\leq m](\Gamma), 
		\exists \eta \in \CRPQ \text{ s.t. }
		\rho \homto \eta \text{ and } \alpha \text{ is a "contraction" of } \eta
	\}.\]

	We have that $\AppInf{\Gamma}{\+C}\semequiv \App{\Gamma}{\+C}{+\infty}$. Indeed, $\AppInf{\Gamma}{\+C}\contained \App{\Gamma}{\+C}{+\infty}$ as the former is a subset of the latter. On the other hand, $\App{\Gamma}{\+C}{+\infty}\contained \AppInf{\Gamma}{\+C}$ follows from Proposition \ref{prop:max-under-approx} and the fact that $\App{\Gamma}{\+C}{+\infty}\contained \Gamma$ by construction.
	
	Let $\nbatoms{\+C}\defeq \max{\set{\nbatoms{G} \mid G \in \+C}}$ and $r_{\Gamma}$ be the maximum number of states over the NFAs describing the languages appearing in $\Gamma$. 
	We claim that $\App{\Gamma}{\+C}{+\infty} \semequiv \App{\Gamma}{\+C}{O(\nbatoms{\Gamma}\cdot r_{\Gamma}\cdot \nbatoms{\+C})}$.
	The right-to-left containment is trivial, so it suffices to show
	$\App{\Gamma}{\+C}{+\infty} \contained \App{\Gamma}{\+C}{O(\nbatoms{\Gamma}\cdot r_{\Gamma}\cdot \nbatoms{\+C})}$.

	We define an \AP""explicit approximation"" of $\Gamma$ over $\+C$ as a tuple
	$\vec{\alpha} = \tup{\rho, \eta, \alpha, h, \orig, \cont}$ consisting of the following:
	\begin{itemize}
		\item queries $\rho \in \Refin(\Gamma)$, $\eta \in \CRPQ$ and
			$\alpha \in \CRPQ[\+C]$,
		\item a "homomorphism" $h\colon \rho \homto \eta$,
		\item witnesses that $\alpha$ is a "contraction" of $\eta$, in the form of
			a function $\orig \colon \vars(\alpha) \to \vars(\eta)$,
			saying from which variable of $\eta$ a variable of $\alpha$ originates,
			and a function $\cont \colon \atoms(\eta) \to \atoms(\alpha)$
			saying onto which atom of $\alpha$ an atom of $\eta$ is contracted ("ie", the functions $\orig,\cont$ must meet the expected properties).
	\end{itemize}
	We say that an "explicit approximation" $\vec{\alpha_1}$ is \AP""contained@@approx"" in
	an "explicit approximation" $\vec{\alpha_2}$ if $\alpha_1 \contained \alpha_2$.
	For any $m$, to prove that $\App{\Gamma}{\+C}{+\infty} \contained \App{\Gamma}{\+C}{m}$,
	it suffices to prove that any "explicit approximation" $\vec{\alpha_1}=\tup{\rho_1, \eta_1, \alpha_1, h_1, \orig_1, \cont_1}$ is "contained@@approx"
	in an "explicit approximation" $\vec{\alpha_2}=\tup{\rho_2, \eta_2, \alpha_2, h_2, \orig_2, \cont_2}$ such that $\rho_2 \in \Refin[\leq m](\Gamma)$.

	\proofcase{1\textsuperscript{st} step: Bounding the size of $\eta$.}
	% We assume "wlog" that $\gamma$ has no isolated vertices.
	We define the \AP""contraction length"" of an "explicit approximation" $\vec{\alpha}$
	as the size of the longest path in $\eta$ whose atoms are all sent on the same
	atom of $\alpha$ via $\cont$. In symbols, this is $\max{\set{|\cont^{-1}(\mu)| : \mu \in \atoms(\alpha)}}$.
	We show that any "explicit approximation" $\vec{\alpha_1}$ is "contained@@approx"
	in an "explicit approximation" $\vec{\alpha_2}$ of bounded "contraction length".

	%Indeed, let $\vec{\alpha_1}$ be an "explicit approximation".
	Let $\vec{\alpha_1}$ be an "explicit approximation".
	Consider a path $x_0 \atom{} x_1 \atom{} \cdots \atom{} x_k$ of $\eta_1$ whose
	atoms are all sent on the same atom of $\alpha_1$ via $\cont_1$. 
	If this path is very long, in particular greater than $\nbvar{\Gamma}$, it must contain an "internal variable" $x_i$ such that all of its $h_1$-preimages are "internal variables" of $\rho_1$. 
	Then we will be able to "contract" $x_i$ as well as the "internal variables" of the preimage, obtaining an "explicit approximation" which contains $\vec{\alpha_1}$.

	We now formalize the previous claim: assume that for some $x_i$, with $i\in \lBrack 1,k-1\rBrack$, all variables in $h_1^{-1}(x_i)$ are "internal" in $\rho_1$. If $x_{i-1}\atom{L} x_i$ and $x_i\atom{L’} x_{i+1}$ are the only atoms containing $x_i$ in $\eta_1$, then for a variable $z\in h_1^{-1}(x_i)$, the only atoms in  containing $z$ in $\rho_1$ must have the form $w\atom{L} z$ and $z\atom{L’} w’$. Let $\eta_2$ be the query resulting from $\eta_1$ by contracting the internal variable $x_i$ and replacing $L\cdot L’$ by $K$, where $K$ is defined as follows. Since $L$ and $L’$ appear consecutively in internal paths of the refinement $\rho_1$, there must be an  NFA $\+A$ in $\gamma$ and three states $p,q,r$ such that $L=\subaut{\+A}{p}{q}$ or $L=\{a\}$ with $a\in \subaut{\+A}{p}{q}$, and   $L’=\subaut{\+A}{q}{r}$ or $L’=\{a\}$ with $a\in \subaut{\+A}{q}{r}$. We define $K=\subaut{\+A}{p}{r}$. Note that in any case, $L\cdot L’\subseteq K$. Similarly, define $\rho_2$ to be the query resulting from $\rho_1$ by contracting each internal variable $z\in h_1^{-1}(x_i)$ and replacing $L\cdot L’$ by $K$. Note that $\rho_2$ is still a refinement of $\gamma$ and that the homomorphism  $h_1\colon \rho_1 \to \eta_1$ induces a "homomorphism" $f_2\colon \rho_2 \to \eta_2$. Define $\alpha_2$ be the contraction of $\eta_2$ obtained by contracting all the remaining internal variables as  the contraction $\alpha_1$ is obtained from  $\rho_1$. Since $\alpha_1\contained \alpha_2$ as $L\cdot L’\subseteq K$, this defines an  "explicit approximation" $\vec{\alpha_2}$ that contains $\vec{\alpha_1}$. Note that in the case $h_1^{-1}(x_i)=\emptyset$, we can take $K=L\cdot L’$, and $\alpha_1\equiv \alpha_2$.

	If $k-1>\nbvar{\Gamma}$, then path $x_0 \atom{} x_1 \atom{} \cdots \atom{} x_k$ contains a variable satisfying the condition above, and hence we can apply the simplification. Overall, this shows that any "explicit approximation" is "contained@@approx"
	in an "explicit approximation" of "contraction length" at most
	$O(\nbvar{\Gamma})\leq O(\nbatoms{\Gamma})$.

	\proofcase{2\textsuperscript{nd} step: bounding the size of $\rho$.}
	We now show that we can bound the \AP""refinement length"" of an "explicit approximation",
	namely the maximal length of an "atom refinement" in $\rho$.
	Let $\vec{\alpha_1}$ be an "explicit approximation" of "contraction length" at most
	$O(\nbatoms{\Gamma})$. Then $\eta_1$ has at most $O(\nbatoms{\Gamma}\cdot\nbatoms{\+C})$
	"atoms". It follows then, by the pigeonhole principle, that we can bound the
	"refinement length" of $\rho_1$ by $O(\nbatoms{\Gamma}\cdot r_{\Gamma}\cdot\nbatoms{\alpha})$. Indeed, if the length of an "atom refinement" of $\rho_1$  is greater than this bound, there are two  atoms in the refinement $x\atom{L_i} y$ and  $x’\atom{L_j} y’$, with $i<j$, mapped to the same atom via $h_1$ and whose corresponding NFA states $q_i$ and $q_j$ in the definition of refinements are the same. We can then remove the path between $y$ and $y’$.
	In conclusion, this shows that $\App{\Gamma}{\+C}{+\infty} \contained \App{\Gamma}{\+C}{O(\nbatoms{\Gamma}\cdot r_{\Gamma}\cdot \nbatoms{\+C})}$.

	\proofcase{Conclusion: Expressing \& computing $\App{\Gamma}{\+C}{O(\nbatoms{\Gamma}\cdot r_{\Gamma} \cdot \nbatoms{\+C})}$ as a "UCRPQ".} 
	In order to compute $\App{\Gamma}{\+C}{O(\nbatoms{\Gamma}\cdot r_{\Gamma} \cdot \nbatoms{\+C})}$ we can enumerate the finitely many $m$-refinements $\rho$ of $\Gamma$, where $m=O(\nbatoms{\Gamma}\cdot r_{\Gamma} \cdot \nbatoms{\+C})$, and the finitely many "CRPQs"  $\eta$ with at most $O(\nbatoms{\Gamma} \cdot \nbatoms{\+C})$ atoms such that $\rho\homto \eta$. The only issue here is that we have infinitely many possibilities to choose languages labelling the atoms that are not in the "homomorphic image" of $\rho \homto \eta$. However, we can  choose the most general language $\A^*$ obtaining a query equivalent to $\App{\Gamma}{\+C}{m}$. 
	Note that each "CRPQ" has at most $O(\nbatoms{\Gamma} \cdot \nbatoms{\+C})$ atoms and its languages are concatenations of
	 $O(\nbatoms{\Gamma} \cdot \nbatoms{\+C})$ "sublanguages" of $\Gamma$ or $\A^*$, and
	so they can be described by NFAs of polynomial size on $\size{\Gamma}$ and $\nbatoms{\+C}$.
\end{proof}

\begin{corollary}
	\AP\label{coro:upperbound-ucrpqs}
	Testing whether a "UCRPQ" is "equivalent" to a "UCRPQ" of at most $k$ atoms is "ExpSpace"-complete.
\end{corollary}

\begin{proof}
	It suffices to test if the "UCRPQ" $\Gamma$ is "equivalent"
	to $\App{\Gamma}{\+C}{m}$ where $\+C$ is the class of all graphs
	with at most $k$ edges and $m=O(\nbatoms{\Gamma}\cdot r_{\Gamma} \cdot \nbatoms{\+C})$ as in the proof of \Cref{lemma:approximation-for-finclass}. 
	The correctness follows from \Cref{prop:max-under-approx} since $\+C$ is
	trivially closed under "minors".
	Each $\alpha \in \App{\Gamma}{\+C}{m}$ has at most $k$ edges,
	and $\App{\Gamma}{\+C}{m}$ contains exponentially many queries,
	so by \cite[Proposition 3.11]{FigueiraMorvan2025SemanticTreeWidthLMCS} (see also proof of \Cref{thm:2expspace-min-crpqs}), it can be solved
	in "ExpSpace".
	Finally, "ExpSpace"-hardness will follow from \Cref{thm:minimization-lowerbound}.
\end{proof}

\subsection{CRPQs over Simple Regular Expressions}

\AP Let ""UCRPQ(SRE)"" ("resp" ""CRPQ(SRE)"") be the set of all "UCRPQs" ("resp" "CRPQs") whose languages are expressed via "SREs@@positive" (as defined in \Cref{sec:intro}). We show that if we restrict the regular expressions we obtain a much better complexity for the "minimization problem" for "UCRPQs".

\begin{theorem}\AP\label{thm:minimization-SRE}
	The "minimization problem" for "UCRPQ(SRE)" is "PiP2"-complete.%
	\footnote{By this we mean the "minimization problem" for "UCRPQs" whose input instances are "UCRPQ(SRE)".}
\end{theorem}

\begin{proof}
	We first begin with an easy small counterexample property.
	\begin{claim}
		\AP\label{cl:small-counterexample-SRE}
		Let $\Gamma, \Delta \in \text{"UCRPQ"}$ containing only atoms with expressions of the form \AP""(i)@@sre"" $a^+$, or ""(ii)@@sre"" $a_1 + \dotsb + a_k$. Additionally, $\Delta$ may also have expressions of the form ""(iii)@@sre"" $\A^*$. If $\Gamma \centernot{\contained} \Delta$, then there exists $\anexpansion \in \Exp(\Gamma)$ such that (a) $\anexpansion \centernot{\contained} \Delta$ and (b) $\nbatoms{\anexpansion} \leq O(\max_{\gamma \in \Gamma}\nbatoms{\gamma} \cdot \max_{\delta \in \Delta}\nbatoms{\delta})$.
	\end{claim}

	The intuition is that if a counterexample includes an atom expansion $x \atom{a^n} y$ of some atom $x \atom{a^+} y$, where $n$ is greater than the maximum number of atoms in $\Delta$ (plus one), then the expansion obtained by replacing $x \atom{a^n} y$ with $x \atom{a^{n-1}} y$ must also be a counterexample. Hence, a minimal counterexample must have all atom expansions bounded by the maximum number of atoms in $\Delta$.

	\begin{proof}
		This fact follows from a standard technique as used in, "eg", \cite{FigueiraEtal2020Containment}.
		Take any counterexample $\anexpansion \in \Exp(\Gamma)$ as in the statement, and suppose it is of minimal size. By means of contradiction, assume $\nbatoms{\anexpansion} > \max_{\gamma \in \Gamma}\nbatoms{\gamma} \cdot (\max_{\delta \in \Delta}\nbatoms{\delta}+1)$. 
		Then, it contains an "atom expansion" $x \atom{a^m} y$ of size $m>\max_{\delta \in \Delta}\nbatoms{\delta}+1$. 
		Consider removing one atom from such expansion ("ie", replacing $x \atom{a^m} y$ with $x \atom{a^{m-1}} y$), obtaining some expansion $\anexpansion' \in \Exp(\Gamma)$ of smaller size. By minimality $\anexpansion'$ is not a counterexample: in other words there is  $\anexpansion'' \in \Exp(\delta)$ such that $h:\anexpansion'' \homto \anexpansion'$ for some $\delta \in \Delta$ and $h$. Since $x \atom{a^{m-1}} y$ contains more than $\nbatoms{\delta}$ "atoms", there must be some $a$-"atom" of $x \atom{a^{m-1}} y$ which either (1) has no $h$-preimage or (2) every $h$-preimage is in an "atom expansion" of a $\A^*$-"atom" or a $a^+$-"atom" of $\delta$. We can then replace the $a$-atom with two $a$-atoms in $\anexpansion'$ and do similarly in the atom expansions of $\anexpansion''$ in the $h$-preimage, obtaining that $\anexpansion' \contained \delta$. But this is in contradiction with our hypothesis, hence any minimal counterexample is of size smaller or equal to $\max_{\gamma \in \Gamma}\nbatoms{\gamma} \cdot (\max_{\delta \in \Delta}\nbatoms{\delta}+1)$.
	\end{proof}
	
	Given a "UCRPQ(SRE)" $\Gamma$, the construction of \Cref{lemma:approximation-for-finclass} yields its "maximal under-approximation" by "UCRPQs" of at most $k$ atoms as a "UCRPQ" $\Delta_{\textnormal{App}}$ whose every regular expression is a concatenation of expressions of the form "(i)@@sre", "(ii)@@sre" and "(iii)@@sre" above.
	It suffices then to test $\gamma \contained \Delta_{\textnormal{App}}$ for every "CRPQ(SRE)" $\gamma$ in $\Gamma$. Due to \Cref{cl:small-counterexample-SRE}
	(and observing that equivalent queries without concatenations can be obtained in polynomial time)
	its negation $\gamma \centernot{\contained} \Delta_{\textnormal{App}}$ can be tested by guessing a polynomial sized "expansion" $\anexpansion$ of $\gamma$ and then testing $\anexpansion \centernot{\contained} \Delta_{\textnormal{App}}$.
	In turn, $\anexpansion \contained \Delta_{\textnormal{App}}$ can be tested in "NP" by 
	\cite[Theorem 4.2]{FigueiraEtal2020Containment}.\footnote{Simply by (1) guessing a polynomial size "CRPQ" $\delta$, (2) testing $\delta \in \Delta_{\textnormal{App}}$ in "NP" by \Cref{lemma:approximation-for-finclass}, and (3) guessing a (small) expansion $\anexpansion'$ of $\delta$ and testing $\anexpansion' \homto \anexpansion$.}

	This yields a "PiP2" algorithm for testing $\gamma \contained \Delta_{\textnormal{App}}$, and thus also for $\Gamma \contained \Delta_{\textnormal{App}}$.
	"PiP2"-hardness follows from \Cref{coro:lowerbounds}.
\end{proof}

