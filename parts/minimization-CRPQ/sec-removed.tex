
% TREE PATTERNS (SUB?)SECTION HERE:
\section{Not included: Tree-like Queries}

\diego{Tree patterns as CRPQs: define "standard encoding" of letters by self-loops}
The minimization of ""tree patterns"" has been thoroughly studied \cite{CzerwinskiMartensNiewerthParys2018Minimization}. 
A "tree pattern" can be seen in our context as a special kind of tree-like "CRPQ". Let us call a \AP""pseudo-tree"" to any directed multigraph such that (i) it has no parallel edges (in particular no parallel self-loops) and (ii) once the self-loops are removed, it is a directed `root-to-leaves' tree.
Let us fix an alphabet letter $e \in \A$ which will be used to model the child relation.
A \AP""tree pattern CRPQ"" is then any "CRPQ" such that
\begin{itemize}
  \item the underlying directed multigraph of the query is a "pseudo-tree"; and
  \item its "self-loops" are of the form $x \atom{a} x$ for some $a \in \A \setminus \set{e}$
  \item its remaining "atoms" are form (a) $x \atom{e^+} y$ or (b) $x \atom{e} y$.
\end{itemize}
A \AP""pseudo-tree database"" is a "graph database" such that
\begin{itemize}
  \item the underlying directed multigraph of the query is a "pseudo-tree"; and
  \item every "vertex" $v$ there is an edge $(v,a,v)$ for some $a \in \A \setminus \set{e}$
  \item its remaining "edges" are of the form $(v,e,v')$.
\end{itemize}

\knowledgenewrobustcmd{\treePatEnc}[1]{\cmdkl{\textup{\bf t}_{#1}}}
We can then define a bijection between "tree pattern CRPQs" and "tree patterns". Given a "tree pattern CRPQ" $q$, we define $\intro*\treePatEnc{q}$ to be the "tree pattern" consisting of treating self-loops as node labels, $x \atom{e^+} y$ as the descendant relation, and $x \atom{e} y$ as the child relation. It is easy to see that $\set{q \mapsto \treePatEnc{q}}$ is a bijection. 

\begin{lemma}\AP\label{lem:treepat-crpq}
  A "tree pattern CRPQ" $q$ is "atom-minimal" if, and only if, $\treePatEnc{q}$ is minimal (among tree patterns over trees).
\end{lemma}
\begin{proof}
  \proofcase{Left-to-right.}
  Observe the following:
  \begin{claim}\AP\label{cl:atommin-over-trees=treepatmin}
    The following are equivalent for each "tree pattern CRPQ" $q$:
    \begin{enumerate}
      \item $q$ is "atom-minimal" among "tree pattern CRPQs" over "pseudo-tree databases";
      \item $\treePatEnc{q}$ is minimal among "tree patterns" over trees.
    \end{enumerate}  
  \end{claim}
  \begin{claim}\AP\label{cl:equiv-overtrees-overgraphs}
    Two "tree pattern CRPQs" are "equivalent" over "pseudo-tree databases" if, and only if, they are "equivalent" over arbitrary "graph databases".
  \end{claim}
  In view of \Cref{cl:atommin-over-trees=treepatmin,cl:equiv-overtrees-overgraphs} we have that
  % \begin{remark}
    if a "tree pattern CRPQ" $q$ is "atom-minimal" (among all "CRPQs" over arbitrary "graph databases"), then $\treePatEnc{q}$ is "minimal" (among "tree patterns" over trees).
  % \end{remark}

  \proofcase{Right-to-left.}
  By \Cref{cl:atommin-over-trees=treepatmin,cl:equiv-overtrees-overgraphs}, it suffices to show that if a "tree pattern CRPQ" is "atom-minimal" among "tree pattern CRPQs", then it is also "atom-minimal" among all "CRPQs". In other words, if a "tree pattern CRPQs" $q$ is "equivalent" to a "CRPQ" $q'$ with fewer "atoms", then it is also equivalent to a "tree pattern CRPQ" $q''$ with fewer "atoms".
  \diego{TBC... Rémi, the floor is yours :)}
\end{proof}

As a consequence of \cite[Theorem~3.1]{CzerwinskiMartensNiewerthParys2018Minimization} and the previous \Cref{lem:treepat-crpq} we then have:
\begin{corollary}
  The "atom-minimization problem" for "tree pattern CRPQs" is "SigmaP2"-complete.
\end{corollary}  
\section{other things removed}
\paragraph{Strong Redundancy}
\diego{Remove an "atom" $\alpha = x \atom{L} y$ of a "CRPQ" $q$ if $L$ contains the language induced by the concatenation of the languages on a path from $x$ to $y$ on $q \setminus \set{\alpha}$. We obtain $q \semequiv q \setminus \set\alpha$. It is therefore in "PSpace" to test if there is a "strongly redundant" "atom".}
\paragraph{Completion}
\diego{adding an atom between two existing variables with the induced langauge. Granted, this actually increases the number of atoms! But in conjunction with "redundancy" or even "weak-redundancy" it may lead to smaller queries. In the example of \Cref{fig:ex-equiv-queries} we can see the passage from left to right as the result of performing: first completion, then redundancy elimination and last contraction.}
\paragraph{Weak Redundancy}
\diego{We can apply Completion, Contraction and (strong) Redundancy. Is there a limit on how many times these have to be applied?}

A ""weak-homomorphism"" from a "CRPQ" $q$ to another "CRPQ" $q'$ is a mapping $h: \vars(q) \to \vars(q')$ such that for every atom $x \atom{L} y$ of $q$ there is a path $h(x) \atom{L_1} \dotsb \atom{L_n} h(y)$ from $h(x)$ to $h(y)$  in $q'$ where $L_1 \dotsb L_n \subseteq L$.
\begin{lemma}
  The "weak-homomorphism" relation is reflexive and transitive.
  If there is a "weak-homomorphism" from $q$ to $q'$, then $q' \contained q$.
\end{lemma}
An "atom" $\alpha$ of $q$ is ""weakly-redundant"" if there is a "weak-homomorphism" from $q$ to $q \setminus \set\alpha$.
\diego{Question: is there a notion of ``core''?}

=====================

Say that a class $\classCRPQ$ of CRPQs satisfies the \AP""uniform small model property""
if for every $\gamma \in \classCRPQ$, there exists a function
$\intro*\shrink{\gamma}\colon \Exp(\gamma) \times \N \to \Exp(\gamma)$ "st":
\begin{enumerate}
  \item for each $\gamma \in \classCRPQ$, for each $n \in \N$ and $\delta \in \classCRPQ$,
  if $\nbatoms{\delta} \leq n$ then for any $G \in \Exp(\gamma)$, $G \FOmodels \delta$ "iff"
  $\shrink{\gamma}(G, n) \FOmodels \delta$, and
  \item the size of $\shrink{\gamma}(G, n)$ only depends on $\gamma$ and $n$---but not $G$.
\end{enumerate} 
Note that $\shrink{\gamma}(G, n) \defeq G$ trivially satisfies the first point but not the second one.
In the name "uniform small model property" ``uniform'' refers to the fact
that the way we shrink $G$ onto $\shrink{\gamma}(G,n)$ is the same for all $\delta$---provided that $\nbatoms{\delta} \leq n$. The ``small model property'' part refers to the fact that to check if
$\gamma \contained \delta$, it suffices to check that all canonical databases of $\gamma$
of size at most $\nbatoms{\shrink{\gamma}(-, \nbatoms{\delta})}$ are models of $\delta$.

\begin{proposition}
  \AP\label{prop:containment-uniform-smp}
  If $\classCRPQ$ has the "uniform small model property", if $\gamma \in \classCRPQ$ and $\Delta$ 
  is a finite union of $\classCRPQ$-queries, then
  $\gamma \contained \Delta$ "iff" for every "canonical database" $G$ of $\gamma$,
  $\shrink{\gamma}(G, m) \FOmodels \Delta$
  where $m \defeq \max{\set{\nbatoms{\delta} \mid \delta \in \Delta}}$.
\end{proposition}

\begin{proof}
  The left-to-right implication is trivial. For the right-to-left implication, 
  let $G$ by any "canonical database" of $\gamma$,
  and assume by contradiction that $G \notFOmodels \Delta$.
  Then for all $\delta \in \Delta$, $G \notFOmodels \delta$,
  and so $\shrink{\gamma}(G, m) \notFOmodels \delta$.
  It follows that $\shrink{\gamma}(G, m) \notFOmodels \Delta$.
  Contradiction.
\end{proof}

\begin{proposition}
  \AP\label{prop:tree-patterns-uniform-smp}
  The class of tree patterns---and more generally CRPQs over $(a, a^+)$ or even over $(a, a^*)$---
  has the "uniform small model property", where $\shrink{\gamma}(G, n)$ consists
  in shrinking every "atom refinement" of size at least $n+1$ to an "atom refinement" of
  size $n$.
\end{proposition}

\begin{proof}
  TODO. Addref?
\end{proof}

\begin{corollary}
  \AP\label{coro:containment-union-tree-patterns-better}
  The containment problem of deciding if $\gamma \contained \Delta$, given:
  \begin{itemize}
    \item an integer $n$,
    \item a "tree pattern" $\gamma$ of size at most $n$,
    \item a union of "tree patterns" $\Delta$ "st" each $\delta \in \Delta$ has size at most $n$
    and deciding if a "tree pattern" $\delta$ belongs to $\Delta$ can be done in "NP" ("wrt" $n$),
  \end{itemize} 
  can be solved in "PiP2".
\end{corollary}

\begin{proof}
  By \Cref{prop:containment-uniform-smp,prop:tree-patterns-uniform-smp},
  $\gamma \contained \Delta$ "iff" for every "canonical database" $G$ of $\gamma$,
  there exists $\delta \in \Delta$ "st"
  $\shrink{\gamma}(G, n) \FOmodels \delta$.
  In turn, this is equivalent to asking that for every "canonical database" $G$ of $\gamma$
  of size at most $|\shrink{\gamma}(-, n)|$, then there exists $\delta\in \Delta$
  "st" $G \FOmodels \delta$.
  This concludes the proof that the problem lies in "PiP2" since by \Cref{prop:tree-patterns-uniform-smp}, $\nbatoms{\shrink{\gamma}(-, n)} = n^2$ so this
  upper bound is polynomial.
\end{proof}

\begin{proof}[Proof of \Cref{proof:union-tree-patterns}]
  \proofcase{Upper bound.}
  A "tree pattern" $\gamma$ is equivalent to a finite union of smaller "tree patterns"
  "iff" $\gamma \contained \App{\gamma}{\+C}{\textit{poly}}$ by \Cref{prop:max-under-approx},
  where $\+C$ is the class of all trees which are strictly smaller as "tree patterns"
  than $\gamma$.
  \remi{TODO: well not exactly…
  Also: size of tree pattern $\neq$ size of its CRPQ encoding (because of node labels). Solution: consider classes of queries, not of graphs.}
  But then \remi{also argue this point}
  \[
    \App{\gamma}{\+C}{\textit{poly}} \semequiv 
    \{
      \alpha \in \CRPQ[\+C] \mid \exists \rho \in \Refin[\leq m](\gamma), 
      \rho \surjto \alpha
    \}.
  \]
  Each $\alpha$ has polynomial (in fact linear) size,
  and testing if $\alpha$ belongs to this latter union can be done in "NP" by definition,
  so \Cref{coro:containment-union-tree-patterns-better} provides an algorithm in "PiP2"
  for the problem.

  \proofcase{Lower bound.} \remi{todo}
\end{proof}

\remi{Check if it also works for "CRPQs" over simple regular expressions.}

\remi{What about minimization of variables?}
\section{Tree-Like Queries}

\begin{hypothesis}
	 In this section, all "CRPQs" are assumed to be positive, meaning
	that no language can contain the empty word.
\end{hypothesis}

\subsection{Forest-Shaped and DAG-Shaped Queries}

We say that a CRPQ is \AP""semantically forest-shaped"" if it is
"semantically equivalent" to a CRPQ which is "forest-shaped".

Say that a CRPQ is \AP""DAG-shaped"" if its underlying directed multigraph
is a DAG---parallel edges are allowed, not self loops. If $\delta$ is "DAG-shaped",
define its \AP""unfolding"", denoted by \AP$\intro*\Unfold(\delta)$, as the following "CRPQ":
\begin{itemize}
	\item its variables are exactly labelled path of $\delta$
	of the form $x_0 \atom{L_1} \cdots \atom{L_n} x_n$ with $n\in \N$
	and $x_0$ is a vertex of $\delta$ with no predecessor;
	\item the atoms exactly go from $x_0 \atom{L_1} \cdots \atom{L_n} x_n$
	to $x_0 \atom{L_1} \cdots \atom{L_n} x_n \atom{L_{n+1}} x_{n+1}$,
	with label~$L_{n+1}$.
\end{itemize}

\begin{fact}
	\AP\label{fact:unfolding-is-forest}
	If $\delta$ is a "DAG-shaped CRPQ", then $\Unfold(\delta)$ is a "forest-shaped CRPQ",
	and moreover $\delta \contained \Unfold(\delta)$.
\end{fact}

The rest of this section is devoted to proving the following result.

\begin{theorem}
	\AP\label{thm:charac-semantically-forest-shaped}
	Let $\delta$ be a "CRPQ". The following are equivalent:
	\begin{enumerate}
		\item $\delta$ is "semantically forest-shaped",
		\item $\delta$ is "DAG-shaped" and
			for every "hom-minimal canonical database" $\?D$ of $\delta$, the "core" of $\?D$ is a forest,
		\item $\delta$ is "DAG-shaped" and $\delta \semequiv \Unfold(\delta)$. 
	\end{enumerate}
\end{theorem}

Note that since "semantical equivalence" is decidable in "ExpSpace" and
since $\Unfold(\delta)$ has exponential size, it follows that one can test if
a "CRPQ" is "semantically forest-shaped" in "2ExpSpace".

\subsection{Semantically DAG-Shaped Queries}

\begin{fact}
	A "CRPQ" is "semantically DAG-shaped" "iff" it is "DAG-shaped".
\end{fact}

\begin{corollary}
	\AP\label{coro:sem-forest-implies-DAG}
	If a "CRPQ" is "semantically forest-shaped", then it is "DAG-shaped".
\end{corollary}

\subsection{Semantically Forest-Shaped}

From the fact that a "CRPQ" $\delta$ is equivalent to a "forest-shaped" query $\phi$
we know that for all "canonical database" $\?D_0$ of $\delta$, since $\delta \contained \phi$,
there exists a "canonical database" $\?F_0$ of $\phi$ "st" $\?D_0 \cohomto \?F_0$.
But dually since $\phi \contained \delta$, there exists $\?D_1 \cdb \delta$ "st" $\?F_0 \cohomto \?D_1$.
By induction---and the axiom of choice---we obtain an infinite co-chain of "homomorphisms"
\[
	\?D_0 \cohomto \?F_0 \cohomto \?D_1 \cohomto \?F_1 \cohomto \cdots \cohomto \?D_n \cohomto \?F_n \cohomto \cdots.
\]
We show that co-chains of forests are actually quite simple.

\begin{fact}
	If $\?F_0 \cohomto \?F_1 \cohomto \cdots \cohomto \?F_n \cohomto \cdots$
	is an infinite co-chain of "homomorphisms" betweens "forests", 
	then there exists $n \in \N$ "st" all $F_m$ with $m\geq n$
	are "hom-equivalent" to one another.
\end{fact}

\begin{proof}
	For all $n \in \N$, from $F_n \cohomto F_{n+1}$ it follows that
	the maximal depth of a tree in $F_{n+1}$ is smaller or equal to the
	maximal depth of a tree in $F_{n}$. So, at some point this parameter
	must become stationary, say $d$. Then observe that there
	are finitely many forests with depth at most $d$, up to "hom-equivalence",
	and hence, one of these must occur infinitely often in the co-chain.
\end{proof}

\begin{corollary}
	\AP\label{coro:all-cdb-are-dominated-by-forests}
	If $\delta$ is "semantically forest-shaped", then for any
	$\?D \cdb \delta$, there exists $\?D' \cdb \delta$ such that
	$\?D \cohomto \?D'$, $\?D'$ is "hom-minimal" and the "core" of $\?D'$ is a forest.
\end{corollary}

We can now proceed with the proof of \Cref{thm:charac-semantically-forest-shaped}, after giving a
proposition that will prove useful.

\begin{proposition}
	\AP\label{prop:hom-from-forest}
	Let $\?F$ be a forest and $\?D$ a graph. If $\?F \homto \?D$ then $F \homto \Unfold(D)$.
\end{proposition}

\begin{proof}
	The "homomorphism" $\?F \homto \Unfold(\?D)$ can be defined by induction on $F$, from roots
	to leaves.
\end{proof}

\begin{proof}[Proof of \Cref{thm:charac-semantically-forest-shaped}.]
	\proofcase{$(1) \Rightarrow (2)$.} This follows from
	\Cref{coro:sem-forest-implies-DAG,coro:all-cdb-are-dominated-by-forests}.

	\proofcase{$(2) \Rightarrow (3)$.} By \Cref{fact:unfolding-is-forest} we have $\delta \contained \Unfold(\delta)$ so it suffices to prove the converse "containment".
	Let $U$ be a "canonical database" of $\Unfold(\delta)$. Then there exists $\?D \cdb \delta$
	"st" $\?U = \Unfold(\?D)$. By \Cref{coro:all-cdb-are-dominated-by-forests} there exists
	$\?D' \cdb \delta$ "st" $\?D \cohomto \?D'$ and the core of $\?D'$ is a forest. So $\?D \cohomto \core\?D'$, and so by \Cref{prop:hom-from-forest}, since $\core\?D'$ is a "forest",
	then $\Unfold(\?D) \cohomto \core\?D'$ "ie" $\Unfold(\?D) \cohomto \?D'$, which proves
	that $\Unfold(\?D) \FOmodels \delta$. Therefore, $\Unfold(\delta) \contained \delta$.

	\proofcase{$(3) \Rightarrow (1)$.} This is because $\Unfold(\delta)$ is "forest-shaped"
		by \Cref{fact:unfolding-is-forest}.
\end{proof}

\section{Removed from Discussion}
\subsection{Tree patterns} Tree pattern minimization on the class of tree patterns is "PiP2". Tree pattern minimization on the class of unions of tree patterns may be better. Possibly the example of \cite[Fig.~9]{CzerwinskiMartensNiewerthParys2018Minimization} is minimal, and yet it is equivalent to a union of small conjunctive queries.
Our results on "SRE@@positive" yield that unions of tree pattern minimization is still "PiP2", and hence that by going to a more expressive formalism we don't lose in complexity.

\begin{conjecture}
If a tree-like CRPQ is equivalent to a CRPQ of size $k$, then it is equivalent to a tree-like CRPQ of size $k$.
\end{conjecture}



\begin{openproblem}
    What's the exact complexity of CRPQ (atom or variable) minimization?
\end{openproblem}

\paragraph{Graphs with transitive edges} A very simple query language is the case of graphs with edges and transitive edges, which naturally corresponds to CRPQs having only two possible languages $a$ and $a^+$. In this simple case the upper bound for CRPQ minimality can be improved to XXX.

\begin{openproblem}
    Does minimization of CRPQ on SRE have a better complexity than "2ExpSpace"?
\end{openproblem}