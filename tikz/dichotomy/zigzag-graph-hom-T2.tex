\node[vertex, draw=c0, fill=c0, fill opacity=.4] (0) at (0,0) {};
\node[vertex, below right=of 0, draw=c1, fill=c1, fill opacity=.4] (1) {};
\foreach \i in {1, 3, 5, 7, 9} {
	\pgfmathtruncatemacro{\next}{\i + 1}
	\pgfmathtruncatemacro{\nnext}{\i + 2}
	\node[vertex, below right=of \i, draw=c2, fill=c2, fill opacity=.4] (\next) {};
	\node[vertex, above right=of \next, draw=c0, fill=c0, fill opacity=.4] (\nnext) {};
};
\node[vertex, below right=of 11, draw=c1, fill=c1, fill opacity=.4] (12) {};
\node[vertex, below right=of 12, draw=c2, fill=c2, fill opacity=.4] (13) {};

% ---
% Transitions
% ---
\foreach \i in {0,1,3,5,7,9,11,12} {
	\pgfmathtruncatemacro{\next}{\i + 1}
	\draw[edge] (\i) to (\next);
};
\foreach \i in {2,4,6,8,10} {
	\pgfmathtruncatemacro{\next}{\i + 1}
	\draw[edge] (\next) to (\i);
}

% ---
% 2-transitive tournament
% ---
\node[vertex, above right=-.25em and 5em of 13, draw=c2, fill=c2, fill opacity=.4] (t2-2) {};
\node[vertex, above=of t2-2, draw=c1, fill=c1, fill opacity=.4] (t2-1) {};
\node[vertex, above=of t2-1, draw=c0, fill=c0, fill opacity=.4] (t2-0) {};

\draw[edge] (t2-0) to (t2-1) to (t2-2);
\draw[edge] (t2-0) to[bend left=60] (t2-2);