\begin{luacode}
-- parameters to tweak
-- nb of nodes in the lattice
width = 17 -- half-width
height = 17 -- half-height
-- height and width (in cm) of the tikzpicture
display_width = 11
display_height = 11
-- decay rates 
x_opacity_decay = 0.5
y_opacity_decay = 1.5
pos_decay = 1.6

cdb_x = 0
cdb_y = 0

dx = display_width / (2 * width)
dy = display_height / (2 * height)

function grid_to_picture_x(x)
	-- returns the position in the picture from the coordinates in the grid
	return x * dx * (math.abs(x)/width)^(pos_decay-1)
end 
function grid_to_picture_y(y)
	-- returns the position in the picture from the coordinates in the grid
	return y * dy * (math.abs(y)/height)^(pos_decay-1)
end

function is_part_of_grid(x,y)
	in_square = x >= -width and x <= width and y >= -height and y <= height
	in_check = (x+y)%2 ~= 0
	in_diamond = math.abs(x) <= height - math.abs(y)
	return in_square and in_check and in_diamond
end

function is_above(x1,y1,x2,y2) -- check if (x1,y1) is above (x2,y2)
	if y1 < y2 then
		return false
	end
	delta_y = y1 - y2 -- >= 0
	delta_x = math.abs(x2 - x1)
	return (delta_x <= delta_y)
end

function get_color(x,y)
	if is_above(x, y, cdb_x, cdb_y) then
		return "c1"
	else
		return "black" 
	end
end

function process_opacity(x)
	return 1/(1+math.exp(2.5-20*x))
end

function get_opacity(x,y)
	if y == 0 then
		return 0
	elseif math.abs(y) == height then
		return process_opacity(1)
	elseif math.abs(y) == 1 and math.abs(x) < width-1 then
		return process_opacity(0)
	elseif math.abs(x) == 1 and math.abs(y) < height-1 then
		y_opac = (math.abs(y/3)/height)^y_opacity_decay
		return process_opacity(y_opac)
	elseif math.abs(x) == 0 and math.abs(y) < height-1 then
		y_opac = (math.abs(y/4)/height)^y_opacity_decay
		return process_opacity(y_opac)
	else
		x_opac = (math.abs(x)/width)^x_opacity_decay
		y_opac = (math.abs(y)/height)^y_opacity_decay
		return process_opacity(x_opac * y_opac)
	end
end

function get_scale(x,y)
	return 2*get_opacity(x,y)^0.3
end

function get_edge_color(x1,y1,x2,y2)
	c1 = get_color(x1,y1)
	c2 = get_color(x2,y2)
	if c1 == c2 then
		return c1
	elseif c1 ~= "black" and c2 ~= "black" then
		return c1
	else
		return "black"
	end
end

function get_edge_opacity(x1,y1,x2,y2)
	o1 = get_opacity(x1,y1)
	o2 = get_opacity(x2,y2)
	return math.min(o1,o2)
end

-- define coordinates
for x = -width,width do
	for y = -height,height do
		if is_part_of_grid(x,y) then
			tex.print(string.format("\\coordinate (%i %i) at (%.4f, %.4f) {};", x, y, grid_to_picture_x(x), grid_to_picture_y(y)))
		end
	end 
end 
-- draw edges
for x = -width,width do
	for y = -height,height do
		if is_part_of_grid(x,y) then
			above = {{x-1,y+1}, {x+1, y+1}}
			for _, coord in ipairs(above) do 
				x2, y2 = coord[1], coord[2]
				if is_part_of_grid(x2,y2) then
					tex.print(string.format("\\draw[color=%s, opacity=%.4f] (%i %i) to[-] (%i %i);", get_edge_color(x,y,x2,y2), get_edge_opacity(x,y,x2,y2), x, y, x2, y2))
				end
			end
		end
	end 
end 
-- draw grid
for x = -width,width do
	for y = -height,height do
		if is_part_of_grid(x,y) then
			-- white background
			tex.print(string.format("\\node[circle,fill=white, minimum size=%.4f pt] at (%i %i) {};", get_scale(x,y), x, y))
			-- proper colour
			tex.print(string.format("\\node[circle,fill=%s, minimum size=%.4f pt, opacity=%.4f] at (%i %i) {};", get_color(x,y), get_scale(x,y), get_opacity(x,y), x, y))
		end
	end 
end 

\end{luacode}