\RequirePackage{luatex85}
\documentclass[a4paper,sfsidenotes,twoside,justified,nobib]{tufte-book-custom}
% \usepackage{showframe}

\hypersetup{colorlinks}

\title[Homomorphism Problems]{Homomorphism Problems\\from Query Minimization in Graph Databases\\ to the Frontier of Decidability in Automatic Structures}
\author{Rémi Morvan}

% Biblio
\usepackage[
  style=alphabetic,
  autocite=footnote,
  backend=biber
]{biblatex}
\bibliography{bib-automatic-graphs}
\bibliography{bib-complexity}

%todo: add \citereset at end of each chapter

% Dependencies
\usepackage{xcolor}

% ---
% German color palette
% ---

% https://flatuicolors.com/palette/de
\definecolor{Desire}{HTML}{eb3b5a} % red
\definecolor{Boyzone}{HTML}{2d98da} % blue
\definecolor{Royal Blue}{HTML}{3867d6} % darker blue
\definecolor{NYC Taxi}{HTML}{f7b731} % yellow
\definecolor{Algal Fuel}{HTML}{20bf6b} % green
\definecolor{Innuendo}{HTML}{a5b1c2} % grey
\definecolor{Twinkle Blue}{HTML}{d1d8e0} % light grey
\definecolor{Gloomy Purple}{HTML}{8854d0} % dark purple

% ---
% Synonyms
% ---

\colorlet{cBlue}{Royal Blue}
\colorlet{cYellow}{NYC Taxi}
\colorlet{cGreen}{Algal Fuel}
\colorlet{cRed}{Desire}
\colorlet{cGrey}{Innuendo}
\colorlet{cLightGrey}{Twinkle Blue}
\colorlet{cPurple}{Gloomy Purple}

\colorlet{maincolor}{cRed!80!black}

% ---
% Colors for knowledge
% ---

\colorlet{KlDefn}{cRed!50!black} 
\colorlet{KlLink}{cBlue!40!black}
\colorlet{KlWarning}{cYellow!60!black}

% ---
% Colors for some graphs
% ---

\colorlet{c0}{cRed} 
\colorlet{c1}{cYellow}
\colorlet{c2}{cBlue}
\colorlet{c3}{cGrey}
\usepackage[UKenglish]{babel}

\usepackage{microtype}
\usepackage{booktabs}
\usepackage{graphicx}
\usepackage{lipsum}
\usepackage{stmaryrd}
\usepackage{multicol}


% Lettrines
% \usepackage{lettrine}
% \newfontface\gleaf{GothicLeaf-VEay}
% \renewcommand{\LettrineFontHook}{\gleaf}
% \setcounter{DefaultLines}{3}
\usepackage{tikz}
\usepackage{tikz-cd}

\usetikzlibrary{calc, arrows.meta, hobby, decorations.pathreplacing, decorations.markings, patterns, positioning}

\tikzset{
	node distance = 1.5em,
	line width = .75pt,
	>={Classical TikZ Rightarrow},
	font = \footnotesize,
	vertex/.style={
		draw,
		circle,
		line width=1pt,
		outer sep=2pt,
		inner sep=2pt
	},
	edge/.style={
		->,
		line width=1pt
	}
}
\usepackage[xcolor, hyperref, cleveref, notion, quotation, composition]{knowledge}
\usepackage{mathcommand}
\knowledgeconfigure{quotation, protect quotation={tikzcd}}
\knowledgeconfigure{diagnose line=true, diagnose bar=true}

\IfKnowledgePaperModeTF{
    %
}{
    % If we are NOT in paper mode (i.e. in composition mode or electronic mode)
    \knowledgestyle{intro notion}{color={KlDefn}, emphasize}
    \knowledgestyle{notion}{color={KlLink}}
    \hypersetup{
        colorlinks=true,
        breaklinks=true,
        linkcolor={KlLink}, % Links to sections, pages, etc.
        citecolor={KlCite}, % Links to bibliography
        filecolor={KlLink}, % Links to local file
        urlcolor={KlUrl}, % Links for URLs
    }
    \IfKnowledgeElectronicModeTF{
        %
    }{
        % If we are in composition mode, highlight unknown stuff (in yellow) and display the anchor point.
        \knowledgeconfigure{anchor point color={KlDefn}, anchor point shape=corner}
        \knowledgestyle{intro unknown}{color={KlWarning}, emphasize}
        \knowledgestyle{intro unknown cont}{color={KlWarning}, emphasize}
        \knowledgestyle{kl unknown}{color={KlWarning}}
        \knowledgestyle{kl unknown cont}{color={KlWarning}}
    }
}

% ---
% Correctly handling \mathop/\mathrel with \knowledgenewrobuscmd
% ---
\ExplSyntaxOn
\RenewDocumentCommand\withkl{mm}{
\int_gincr:N\knowledge_inner_modifier_count_int
\cs_gset:cpx
{\knowledge_inner_command:}
{\exp_not:N\cs_gset:Npn
\exp_not:c{\knowledge_inner_command:}
{\knowledge_inner_modifer_re_tl\knowledge_kl_modifiers_tl\exp_not:n{#1}}
\knowledge_kl_modifiers_tl\exp_not:n{#1}}
\knowledge_kl_modifiers_reset:
#2
\int_gdecr:N\knowledge_inner_modifier_count_int
}
\ExplSyntaxOff
\usepackage{amsthm, thmtools, thm-restate}

\declaretheoremstyle[bodyfont=\normalfont]{noitalics}
\declaretheoremstyle[headfont=\normalfont\itshape,bodyfont=\normalfont]{noitalicsminor}

\declaretheorem[name=Theorem, refname={Theorem,Theorems}, numberwithin=section, style=noitalics]{theorem}
\declaretheorem[name=Proposition, refname={Proposition,Propositions}, sibling=theorem, style=noitalics]{proposition}
\declaretheorem[name=Property, refname={Property,Properties}, sibling=theorem, style=noitalics]{property}
\declaretheorem[name=Definition, refname={Definition,Definitions}, sibling=theorem, style=noitalics]{definition}
\declaretheorem[name=Corollary, refname={Corollary,Corollaries}, sibling=theorem, style=noitalics]{corollary}
\declaretheorem[name=Conjecture, refname={Conjecture,Conjectures}, sibling=theorem, style=noitalics]{conjecture}
\declaretheorem[name=Question, refname={Question,Questions}, sibling=theorem, style=noitalics]{question}
\declaretheorem[name=Lemma, refname={Lemma,Lemmas}, sibling=theorem, style=noitalics]{lemma}
\declaretheorem[name=Example, refname={Example,Examples}, sibling=theorem, style=noitalics]{example}
\declaretheorem[name=Remark, refname={Remark,Remarks}, sibling=theorem, style=noitalics]{remark}
\declaretheorem[name=Fact, refname={Fact,Facts}, sibling=theorem, style=noitalics]{fact}
\declaretheorem[name=Claim, refname={Claim,Claims}, sibling=theorem, style=noitalicsminor]{claim}

% Macros and math commands
\newrobustcmd{\fancyand}{{\setmainfont{Tex Gyre Pagella}\textit{\&}}}

\newrobustcmd\decisionproblem[3]{
	\begin{center}\AP
	\fbox{%
	\begin{tabular}{rl}
	\multicolumn{2}{l}{#1} \\
	{\emph{Input}}: & \parbox[t]{.73\linewidth}{#2} \\
	{\emph{Question}}: & \parbox[t]{.73\linewidth}{#3}
	\end{tabular}%
	} 
	\end{center}
}
\newrobustcmd{\DPfont}[1]{\normalfont{\scshape #1}}

\usepackage{adforn}
\newrobustcmd{\proofcase}[1]{\adforn{39}~\emph{#1}~}

% ---
% Abstract and Chapter TOC 
% ---

\newenvironment{abstract}{\begin{center}\large\textsc{Abstract}\end{center}}{}
\newenvironment{acknowledgements}{\begin{center}\large\textsc{Acknowledgements}\end{center}}{}


\newrobustcmd{\chaptertoc}{
  \thispagestyle{plain}
  \begin{fullwidth}
	\null\vfill
    \startcontents
    \begin{center}
      \large\textsc{Contents}
    \end{center}
    \bigskip
	\printcontents{p}{1}{\setcounter{tocdepth}{2}}
	\vfill
  \end{fullwidth}
  \vfill\null
}

% ---
% Smash left/right arrow (used for "hom")
% ---
\newrobustcmd\smashxrightarrow[1]{%
  	\raisebox{-.04em}{$%
		\xrightarrow{\smash{\raisebox{-.1em}{%
	  		\tiny{#1}%
		}}}%
  	$}%
}%
\newrobustcmd\smashxleftarrow[1]{%
	\raisebox{-.04em}{$%
		\xleftarrow{\smash{\raisebox{-.1em}{%
			\tiny{#1}%
		}}}%
	$}%
}%

% ---
% Knowledge: intro in restatable
% Usage: instead of \intro use \introinrestatable in restable environments.
% Surround the main statement (where you want the notions to be introduced)
% with a `mainstatement` environment.
% ---

\newif\ifmainstatement
\mainstatementfalse

\newrobustcmd\introinrestatable[1]{%
	\ifmainstatement%
		\intro{#1}%
	\else%
		\kl{#1}%
	\fi%
}

\newenvironment{mainstatement}{
  \mainstatementtrue
}{
  \mainstatementfalse
}
\newrobustcmd\?{\symbf}
\newrobustcmd\+{\symcal}
\newrobustcmd\•{\symcal}

\newrobustcmd\card[1]{|1|}
\newcommand{\set}[1]{\{#1\}}
\newcommand{\tup}[1]{\langle#1\rangle}

\knowledgenewrobustcmd\pset[1]{\cmdkl{\symfrak{P}(#1)}} % powerset
\knowledgenewrobustcmd\psetp[1]{\cmdkl{\symfrak{P}_+(#1)}} % strict powerset

\newcommand{\dcup}{\sqcup} % disjoint union of sets
\newcommand{\bigdcup}{\bigsqcup} % disjoint union of sets

\knowledgenewrobustcmd\id[1][]{\cmdkl{\mathrm{id}_{#1}}} % identity function

\newrobustcmd\N{\symbb{N}} % natural numbers
\newrobustcmd\Np{\N_{>0}} % strictly positive nat. numbers
\newrobustcmd\Z{\symbb{Z}}
\knowledgenewrobustcmd\intInt[1]{\cmdkl{\lBrack} #1 \cmdkl{\rBrack}} % real numbers
\knowledgenewrobustcmd\ZnZ[1]{\cmdkl{\symbb{Z}/#1\symbb{Z}}}
\newrobustcmd\R{\symbb{R}} % real numbers
\knowledgenewrobustcmd\2{\cmdkl{\symbb{2}}} % binary alphabet 

\newrobustcmd\defeq{\mathrel{\hat{=}}} % equality by definition
\knowledgenewrobustcmd\pto{\mathrel{\cmdkl{\rightharpoonup}}} % partial function
\knowledgenewrobustcmd\surj{\mathrel{\cmdkl{\twoheadrightarrow}}} % surjection
\knowledgenewrobustcmd{\equivclass}[2][]{\cmdkl{[}#2\cmdkl{]^{#1}}} % equivalence class
\knowledgenewrobustcmd{\dom}{\cmdkl{\mathop{\textrm{dom}}}} % domain

\knowledgenewrobustcmd\restr[2]{{% function restriction
  #1 \cmdkl{\raisebox{-.1em}{\(\vert\)}{}_{#2}}
}}

\newrobustcmd{\poly}{\mathop{\textrm{poly}}} % polynomial
\knowledgenewrobustcmd\tower{\mathop{\cmdkl{\mathrm{tower}}}}

% ---
% Basic
% ---

\knowledgenewrobustcmd{\transition}[1]{\mathrel{\cmdkl{\smashxrightarrow{\ensuremath{#1}}}}}

\knowledgenewrobustcmd{\semFO}[2]{\cmdkl{\lBrack} #1 \cmdkl{\rBrack^{#2}}} % semantics of FO
\knowledgenewrobustcmd\FOmodels{\mathrel{\cmdkl{\models}}} % models relation

\knowledgenewrobustcmd\pmAlpha[1]{#1^{\pm}}

\newrobustcmd\Acc{\textrm{Acc}}
\newrobustcmd\Bcc{\textrm{Bcc}}
\knowledgenewrobustcmd\semTM[1]{\cmdkl{\lBrack}#1\cmdkl{\rBrack}} % semantics of Turing machine
\knowledgenewrobustcmd\statesTM[1]{\cmdkl{Q^{#1}}} % states of Turing machine

% ---
% Relational structures
% ---
\knowledgenewrobustcmd{\adjacency}[4]{\cmdkl{\+{Adj}_{\!#2}^{\smash{#3,#4}}(#1)}}
\knowledgenewrobustcmd{\ball}[3]{\cmdkl{\+B_{#1}^{#3}(#2)}}
\knowledgenewrobustcmd{\isom}{\mathrel{\cmdkl{\cong}}}

% ---
% Hom & core
% ---
\knowledgenewrobustcmd{\homto}{\mathrel{\cmdkl{\smashxrightarrow{hom}}}}
\newrobustcmd{\cohomto}{\mathrel{\kl[\homto]{\smashxleftarrow{hom}}}}
\RequirePackage{centernot}
\newrobustcmd{\nothomto}{\mathrel{\kl[\homto]{\centernot{\smashxrightarrow{hom}}}}}
\newrobustcmd{\notcohomto}{\mathrel{\kl[\homto]{\centernot{\smashxleftarrow{hom}}}}}

\knowledgenewrobustcmd\core[1]{\cmdkl{\check{{#1}}}}

% ---
% Constructions on structures
% --- 
\knowledgenewrobustcmd\disunion{\mathbin{\cmdkl{\uplus}}}
\knowledgenewrobustcmd\prodstruct{\mathbin{\cmdkl{\times}}}
\knowledgenewrobustcmd\powstruct[2]{\cmdkl{#1^{#2}}}
\knowledgenewrobustcmd\iterstruct[2]{#1^{\cmdkl{#2}}}


\knowledgenewrobustcmd{\atom}[1]{\mathrel{\smashxrightarrow{\ensuremath{#1}}}}
\knowledgenewrobustcmd{\coatom}[1]{\mathrel{\smashxleftarrow{\ensuremath{#1}}}}
\knowledgenewrobustcmd{\symatom}[1]{\mathrel{\smashxleftrightarrow{\ensuremath{#1}}}}

\colorlet{wrote}{cRed}
\colorlet{advised}{cBlue}
\newrobustcmd{\wrote}{\color{wrote}\scriptsize\text{wrote}}
\newrobustcmd{\advised}{\color{advised}\scriptsize\text{advised}}

% ---
% Alphabets, graphs
% ---

\knowledgenewrobustcmd{\A}{\mathbb{A}} % Finite alphabet of labels
\knowledgenewrobustcmd{\Aext}{\cmdkl{\mathbb{A}^\pm}} % Finite alphabet of labels
\knowledgenewrobustcmd\vertex[1]{\cmdkl{V}(#1)}
\knowledgenewrobustcmd\edges[1]{\cmdkl{E}(#1)}

\knowledgenewrobustcmd\qvar{\footnotesize\bullet} % quantified variable

% --- 
% Queries
% ---

\knowledgenewrobustcmd{\atoms}[1]{\cmdkl{\textnormal{Atoms}}(#1)}
\knowledgenewrobustcmd{\contained}{\mathrel{\cmdkl{\subseteqq}}}
\newrobustcmd{\strcontained}{
  \mathrel{\withkl{\kl[\contained]}{\cmdkl{%
    \subsetneqq
  }}}
}
\disablecommand\equiv
\suggestcommand\equiv{Use instead \semequiv for semantical equivalence.}
\knowledgenewrobustcmd{\semequiv}{\mathrel{\cmdkl{\LaTeXequiv}}}
\knowledgenewrobustcmd{\nbatoms}[2][]{\cmdkl{\|}#2\cmdkl{\|^{#1}_{\textrm{at}}}}
\knowledgenewrobustcmd{\size}[1]{\cmdkl{\|}#1\cmdkl{\|}}
\knowledgenewrobustcmd{\vars}{\cmdkl{\textrm{vars}}} % set of variables of a query
\newrobustcmd{\collapse}{\approx}
\knowledgenewrobustcmd{\Exp}{\cmdkl{\textnormal{Exp}}} % set of expansions

% --- 
% Classes of queries
% ---

\knowledgenewrobustcmd{\UCtwoRPQ}{\cmdkl{\textnormal{UC2RPQ}}}
\newrobustcmd{\UCRPQ}{\kl[\UCtwoRPQ]{\textnormal{UCRPQ}}}
\newrobustcmd{\CtwoRPQ}{%
  \withkl{\kl[\UCtwoRPQ]}{\cmdkl{%
    \textnormal{C2RPQ}
  }}
}

\knowledgenewrobustcmd{\UCRPQSRE}{\ensuremath{\cmdkl{\textup{UCRPQ}(\textup{SRE})}}}
\newrobustcmd{\CRPQSRE}{%
  \withkl{\kl[\UCRPQSRE]}{\cmdkl{%
    \textup{CRPQ}(\textup{SRE})%
  }}%
}%

\knowledgenewrobustcmd{\class}{\mathcal{C}}
\knowledgenewrobustcmd{\Tw}[1][k]{\cmdkl{\mathcal{T\hspace{-.15em}w}_{#1\!}}}
\knowledgenewrobustcmd{\Pw}[1][k]{\cmdkl{\mathcal{P\hspace{-.05em}w}_{#1\!}}}
\knowledgenewrobustcmd{\ContrPw}[1][k]{\cmdkl{\mathcal{C\hspace{-.05em}p\hspace{-.05em}w}_{#1\!}}}
\knowledgenewrobustcmd{\PwOneWay}[1][k]{\cmdkl{\mathcal{1P\hspace{-.05em}w}_{#1\!}}}
% \knowledgenewrobustcmd{\ContrPwOneWay}[1][k]{\cmdkl{\mathcal{C\hspace{-.05em}p\hspace{-.05em}w}^{\smash{1w}}_{#1\!}}}
\knowledgenewrobustcmd{\ContrPwOneWay}[1][k]{\cmdkl{\mathcal{1\hspace{-.08em}C\hspace{-.05em}p\hspace{-.05em}w}_{#1\!}}}
\knowledgenewrobustcmd{\ContrTw}[1][k]{\cmdkl{\mathcal{C\hspace{-.05em}t\hspace{-.05em}w}_{#1\!}}}
\knowledgenewrobustcmd{\TwOneWay}[1][k]{\cmdkl{\mathcal{1T\hspace{-.15em}w}_{#1\!}}}
% \knowledgenewrobustcmd{\ContrTwOneWay}[1][k]{\cmdkl{\mathcal{C\hspace{-.05em}t\hspace{-.05em}w}^{\smash{1w}}_{#1\!}}}
\knowledgenewrobustcmd{\ContrTwOneWay}[1][k]{\cmdkl{\mathcal{1\hspace{-.08em}C\hspace{-.05em}t\hspace{-.05em}w}_{#1\!}}}
\newrobustcmd\bw{\textrm{bw}}

% ---
% Refinements & Approximations
% ---

\newcommand{\anexpansion}{\xi}
\knowledgenewrobustcmd{\Refin}[1][]{\cmdkl{\textnormal{Ref}^{\smash{#1}}}}
\knowledgenewrobustcmd{\MUA}[2]{\cmdkl{\ensuremath{\textnormal{App}_{#2}(#1)}}}
\knowledgenewrobustcmd{\MUAHom}[2]{\cmdkl{\ensuremath{\textnormal{App}_{#2}^{\smash{\star}}(#1)}}}
\knowledgenewrobustcmd{\MUAHomBounded}[3]{\cmdkl{\ensuremath{\textnormal{App}_{#2}^{\smash{\star,#3}}(#1)}}}
\knowledgenewrobustcmd{\typeStw}{\cmdkl{\textnormal{type}}}
\knowledgenewrobustcmd{\Qapp}[1][k]{\cmdkl{\ensuremath{\textnormal{App}_{\Tw[#1]}^{\smash{\textup{zip}}}(\gamma)}}}
\knowledgenewrobustcmd{\contract}[1]{\cmdkl{[}#1\cmdkl{]}}

\knowledgenewrobustcmd\upquery[1]{#1^{\cmdkl{\uparrow}}}

% ---
% Semantic tw > tree decompositions
% ---

\knowledgenewrobustcmd\bagmap{\cmdkl{\mathbf{v}}}
\knowledgenewrobustcmd\tagmap{\cmdkl{\mathbf{t}}}
\knowledgenewrobustcmd\tagmappath[1]{\cmdkl{\mathbf{t}[#1]}}
\newrobustcmd\tagmappathprime[1]{%
  \withkl{\kl[\tagmappath]}{%
    \cmdkl{\mathbf{t}'[#1]}%
  }%
}


% ---
% Semantic tw > bounds
% ---

\disablecommand\ell
\suggestcommand\ell{Use instead \l for bound on the size of refinements.}
\knowledgerenewcommand{\l}{\cmdkl{\LaTeXell}}
\knowledgenewrobustcmd{\lOne}{\cmdkl{\LaTeXell_1}}
\knowledgenewrobustcmd{\avoid}{\textsf{avoid}}
\knowledgenewrobustcmd{\trap}{\textsf{trap}}

% ---
% Semantic tw > axioms
% ---

\knowledgenewrobustcmd\itemClosureInfCQ{\cmdkl{\textup{\textrm{(1)}}}}
\knowledgenewrobustcmd\itemClosureUCRPQ{\cmdkl{\textup{\textrm{(2)}}}}
\knowledgenewrobustcmd\itemClosureUCRPQSimple{\cmdkl{\textup{\textrm{(3)}}}}

% ---
% Semantic tw > misc
% ---

\knowledgenewrobustcmd{\pathl}{\cmdkl{\mathbf{P}_{\!l}}} % path-l approximation query
\knowledgenewrobustcmd\subaut[3]{#1\cmdkl{[#2,#3]}} % subautomaton
\knowledgenewrobustcmd{\fun}{\cmdkl{f}}
% ---
% Constructions on structures / presentations 
% --- 
\knowledgenewrobustcmd\prodpres{\mathbin{\cmdkl{\underline{\times}}}}

\renewrobustcmd{\ker}[1]{\withkl{\kl[\ker]}{\mathrel{\cmdkl{\mathrm{ker}_{#1}}}}} % ker of hom
	\knowledge{\ker}{notion}

% ---
% Idempotent core
% --- 
\knowledgenewrobustcmd\marked[1]{#1^{\cmdkl{\dag}}}
\knowledgenewrobustcmd\unarypred[1]{\cmdkl{P_{#1}}}
\knowledgenewrobustcmd\extendedSignature[2]{\cmdkl{#1_{#2}}}

% ---
% Hom-reg relation 
% ---
\knowledgenewrobustcmd{\homregto}{\mathrel{\cmdkl{\smashxrightarrow{reg hom}}}}
\newrobustcmd{\nothomregto}{\mathrel{\kl[\homregto]{\centernot{\smashxrightarrow{reg hom}}}}}

% ---
% Interpretations & Automatic presentations
% ---
\knowledgenewrobustcmd{\interpretation}[2]{\cmdkl{#1(#2)}}
\knowledgenewrobustcmd{\domainInter}[1]{\cmdkl{\mathrm{dom}_{#1}}}
\knowledgenewrobustcmd{\relInter}[2]{\cmdkl{#1_{\! #2}}}
\knowledgenewrobustcmd{\domainPres}[1]{\cmdkl{\mathrm{dom}_{#1}}}
\knowledgenewrobustcmd{\relPres}[2]{\cmdkl{#1_{\! #2}}}
\newrobustcmd{\smallpad}{%
	\hspace{.05em}\rule{.3em}{.075em}\hspace{.05em}
}
% \newcommand\vartextvisiblespace[1][.5em]{%
% 	\kern.07em
% 	\vrule width.12ex height.5ex
% 	\vrule width#1 height.12ex
% 	\vrule width.12ex height.5ex
% 	\kern.07em
% }
% \knowledgenewrobustcmd{\pad}{%
% 	\cmdkl{\vartextvisiblespace}
% }
\knowledgenewrobustcmd{\pad}{\cmdkl{\smallpad}}
\knowledgenewrobustcmd{\convol}{\mathbin{\cmdkl{\otimes}}} % convolution of two words of languages
\newrobustcmd\pair[2]{{\smaller\left(\begin{smallmatrix}%
	#1\vphantom{b}\\%
	#2\vphantom{b}%
\end{smallmatrix}\right)}} % pair of letters
\newrobustcmd\triple[3]{{\smaller\left(\begin{smallmatrix}%
	#1\vphantom{b}\\%
	#2\vphantom{b}\\%
	#3\vphantom{b}%
\end{smallmatrix}\right)}} % truple of letters
\knowledgenewrobustcmd{\convolAlpha}{\mathbin{\cmdkl{\otimes}}}
\knowledgenewrobustcmd\SigmaPair[1][\Sigma]{\cmdkl{#1^{\smash{2}}_{{\otimes}}}}
\knowledgenewrobustcmd{\convolRel}[1]{#1^{\cmdkl{\otimes}}} % convolution of a relation
\knowledgenewrobustcmd\WellFormed[1][\Sigma]{\cmdkl{\textsf{WellFormed}_{#1}}} % well-formed words
\knowledgenewrobustcmd\projWord{\cmdkl{\pi_{\textsf{word}}}} % projection of multitape word
\knowledgenewrobustcmd\projTape{\cmdkl{\pi_{\textsf{tape}}}}

% ---
% Hom problem & classes 
% --- 
\knowledgenewrobustcmd\Fin[1][\sigma]{\cmdkl{\mathrm{Fin}_{#1}}}
\knowledgenewrobustcmd\Aut[1][\sigma]{\cmdkl{\mathrm{Aut}_{#1}}}
\knowledgenewrobustcmd\FinPres[1][\sigma]{\cmdkl{\mathcal{F\!in}^{\smash{\textsf{pres}}}_{#1}}}
\knowledgenewrobustcmd\AutPres[1][\sigma]{\cmdkl{\mathcal{Aut}^{\smash{\textsf{pres}}}_{#1}}}

\newrobustcmd{\classStruct}{\mathcal{Cls}}

\newrobustcmd{\HomNoKl}[2]{\mathcal{H\!om}(#1,\,#2)}
\knowledgenewrobustcmd\Hom[2]{\cmdkl{\HomNoKl{#1}{#2}}}
\knowledgenewrobustcmd\HomFin[1]{\cmdkl{\HomNoKl{\mathrm{Fin}}{#1}}}
\knowledgenewrobustcmd\HomAut[1]{\cmdkl{\HomNoKl{\mathrm{Aut}}{#1}}}
\knowledgenewrobustcmd\HomAll[1]{\cmdkl{\HomNoKl{\mathrm{All}}{#1}}}

\newrobustcmd{\HomRegNoKl}[2]{\mathcal{H\!om}^{\smash{\textsf{reg}}}(#1,\,#2)}
\knowledgenewrobustcmd\HomReg[2]{\cmdkl{\HomRegNoKl{#1}{#2}}}
% \knowledgenewrobustcmd\HomRegFin[1]{\cmdkl{\HomRegNoKl{\mathrm{Fin}}{#1}}}
\knowledgenewrobustcmd\HomRegAut[1]{\cmdkl{\HomRegNoKl{\mathrm{Aut}}{#1}}}

% ---
% Concrete graphs / structures 
% ---
\knowledgenewrobustcmd\link[1]{\cmdkl{\?L_{#1}}}
\knowledgenewrobustcmd{\zigzag}[2]{\cmdkl{\?Z_{#2}^{(#1)}}}
\knowledgenewrobustcmd{\transitiveTournament}[1]{\cmdkl{\?T_{#1}}}
\knowledgenewrobustcmd{\pathGraph}[1]{\cmdkl{\?P_{#1}}}
\knowledgenewrobustcmd{\clique}[1]{\cmdkl{\?K_{#1}}}
\newcommandPIE{\omegaClique}{\withkl{\kl[\omegaClique]}{\cmdkl{\?K_{<\omega}#1#2#3}}}
	\knowledge{\omegaClique}{notion}
\newcommandPIE{\omegaCliquePres}{\withkl{\kl[\omegaCliquePres]}{\cmdkl{\+K_{\!<\omega}#1#2#3}}}
	\knowledge{\omegaCliquePres}{notion}

\knowledgenewrobustcmd\projHom[1]{\cmdkl{\pi_{#1}}} % projection homomorphism
\knowledgenewrobustcmd{\FederVardi}[1]{\cmdkl{\symfrak{U}(#1)}} % Feder-Vardi's construction
\knowledgenewrobustcmd\Myc{\mathop{\cmdkl{\symfrak{M}}}} % Mycielski's construction
\knowledgenewrobustcmd\MycInf{\mathop{\cmdkl{\?M}}} % Infinite Mycielski graph

% ---
% First-order logic for automatic relations
% ---
\knowledgenewrobustcmd{\lastLetter}[1]{\cmdkl{\+l_{#1}}}
\knowledgenewrobustcmd{\univStructSynchronous}[1]{\cmdkl{\?{#1^*}}}
\knowledgenewrobustcmd{\signatureSynchronous}[1]{\cmdkl{\sigma_{#1}^{\smash{\textrm{sync}}}}}

% ---
% Hyperedge consistency
% ---
\knowledgenewrobustcmd{\subsumed}{\mathrel{\cmdkl{\sqsubseteq}}}
\knowledgenewrobustcmd{\LatticeGuessFunctions}[2]{\cmdkl{\langle \pset{#2}^{#1},\, \subsumed \rangle}}
\knowledgenewrobustcmd{\topLatticeGuessFunctions}[1]{\cmdkl{\Lambda_{#1}}}

\newcommandPIE{\HCOperator}{\withkl{\kl[\HCOperator]}{\cmdkl{\symcal{H\!C}#1#2#3}}}
	\knowledge{\HCOperator}{notion}
\knowledgenewrobustcmd{\HCFixpoint}[2]{\cmdkl{H^{\,*}_{\smash{#1,#2}}}}

% ---
% Misc. dichotomy theorem
% ---
\knowledgenewrobustcmd{\unaryType}[2]{\cmdkl{\mu_{#2}(#1)}}
\knowledgenewrobustcmd{\structOfUnaryType}[1]{\cmdkl{\?1_{#1}}}
\knowledgenewrobustcmd{\ConstrUndecHom}[1]{#1^{\cmdkl{\star}}}
\knowledgenewrobustcmd{\rightquotient}[2]{#1\mathbin{\cmdkl{/}}#2}
\knowledgenewrobustcmd{\restrConnected}[1]{\cmdkl{\restr{#1}{\textrm{conn}}}}

\knowledgenewrobustcmd{\itemDTFinDual}{\cmdkl{\text{\bfseries\textup{(\textsf{DT})$_\textsf{fin-dual}$}}}}
\knowledgenewrobustcmd{\itemDTHomDec}{\cmdkl{\text{\bfseries\textup{(\textsf{DT})$_\textsf{hom-dec}$}}}}
\knowledgenewrobustcmd{\itemDTHomRegDec}{\cmdkl{\text{\bfseries\textup{(\textsf{DT})$_\textsf{hom-reg-dec}$}}}}
\knowledgenewrobustcmd{\itemDTEqual}{\cmdkl{\text{\bfseries\textup{(\textsf{DT})$_\textsf{equal}$}}}}
\knowledgenewrobustcmd{\itemDTFirstOrder}{\cmdkl{\text{\bfseries\textup{(\textsf{DT})$_\textsf{first-order}$}}}}

% ---
% Classes of relations
% ---
\knowledgenewrobustcmd{\RAT}{\cmdkl{\ensuremath{\textsc{Rat}}}}%
\knowledgenewrobustcmd{\AUT}{\cmdkl{\ensuremath{\textsc{Aut}}}}%
\knowledgenewrobustcmd{\SYNC}{\cmdkl{\ensuremath{\textsc{Sync}}}}% todo:disable
\knowledgenewrobustcmd{\REC}{\cmdkl{\ensuremath{\textsc{Rec}}}}%
\knowledgenewrobustcmd\kREC[1][k]{\cmdkl{\ensuremath{#1\textsc{-Rec}}}}
\knowledgenewrobustcmd\kPROD[1][k]{\cmdkl{\ensuremath{#1\textsc{-Prod}}}}

% ---
% Example of relations
% ---
\knowledgenewrobustcmd\sameParity{\mathrel{\cmdkl{\approx_{\textrm{mod~2}}}}}
\knowledgenewrobustcmd{\equalLength}{\mathrel{\cmdkl{\approx_{\textrm{len}}}}}

\knowledgenewrobustcmd{\prefix}{\mathrel{\cmdkl{\preccurlyeq_{\textrm{pref}}}}}
\knowledgenewrobustcmd{\suffix}{\mathrel{\cmdkl{\preccurlyeq_{\textrm{suff}}}}}
\knowledgenewrobustcmd{\subword}{\mathrel{\cmdkl{\preccurlyeq_{\textrm{subw}}}}}
\knowledgenewrobustcmd{\lex}{\mathrel{\cmdkl{\preccurlyeq_{\textrm{lex}}}}}


% ---
% Separation to Colourability
% ---
\knowledgenewrobustcmd{\Germ}[1]{\cmdkl{\textrm{Germ}_{#1}}} % germinal configurations
\newrobustcmd{\GermC}[2]{\kl[\Germ]{\color{#2}\textrm{Germ}_{#1}}} % variant with color
\knowledgenewrobustcmd{\Reach}[1]{\cmdkl{\textrm{Reach}_{#1}}}
\newrobustcmd{\ReachC}[2]{\kl[\Reach]{\color{#2}\textrm{Reach}_{#1}}} % variant with color

\knowledgenewrobustcmd{\incompGraph}[2]{\cmdkl{\+{I\mkern-1.5mu nc}_{#1,#2}}} % incompatibility graph
\knowledgenewrobustcmd{\compL}{\cmdkl{\textnormal{\small(\textsc{comp}${}_\ell$)}}}
\knowledgenewrobustcmd{\compR}{\cmdkl{\textnormal{\small(\textsc{comp}${}_\textit{r}$)}}}
\knowledgenewrobustcmd{\compLpr}{\cmdkl{\textnormal{\small(\textsc{comp}${}'_\ell$)}}}
\knowledgenewrobustcmd{\compRpr}{\cmdkl{\textnormal{\small(\textsc{comp}${}'_\textit{r}$)}}}

\knowledgenewrobustcmd{\Id}{\cmdkl{\+{I\mkern-1.5mu d}}} % identity relation

\newrobustcmd{\bSymb}{\textcolor{cBlue}{b}}
\newrobustcmd{\rSymb}{\textcolor{cRed}{r}}

\knowledgenewrobustcmd{\configs}[1][\+T]{\cmdkl{\textrm{Conf}_{#1}}}
\knowledgenewrobustcmd{\confGraph}[1][\+T]{\cmdkl{{\+{C\mkern-1.5mu on\!f}}_{\!#1}}}

\knowledgenewrobustcmd{\autequiv}[1][\+R]{\cmdkl{\sim_{\!#1}}}


% =---=
% Algebras for Automatic Relations
% =---=

% ---
% Categories
% ---
\newrobustcmd\cat[1]{\mathsf{#1}}
\knowledgenewrobustcmd\Set[1][\+S]{\cmdkl{\cat{Set}^{\smash{#1}}}}
\knowledgenewrobustcmd\Pos[1][\+S]{\cmdkl{\cat{Pos}^{\smash{#1}}}}
\knowledgenewrobustcmd\Dep[1][\+S]{\cmdkl{\cat{Dep}^{\smash{#1}}}}

% ---
% Types
% ---
\usepackage{xfrac}
\knowledgenewrobustcmd\LL{\cmdkl{\text{\sfrac{\footnotesize\textsc{l}\,}{\,\footnotesize\textsc{l}}}}\hspace{.1em}}
\knowledgenewrobustcmd\LB{\cmdkl{\text{\sfrac{\footnotesize\textsc{l}\,}{\,\footnotesize\textsc{b}}}}\hspace{.1em}}
\knowledgenewrobustcmd\BL{\cmdkl{\text{\sfrac{\footnotesize\textsc{b}\,}{\,\footnotesize\textsc{l}}}}\hspace{.1em}}

% ---
% Types & co
% ---
\knowledgenewrobustcmd\proj[1]{\cmdkl{\underline{#1}}} % Projection of relation
\knowledgenewrobustcmd\projP[1]{\cmdkl{\underline{#1}}}

\knowledgenewrobustcmd\typeW[2]{#1_{\cmdkl{#2}}}
\knowledgenewrobustcmd\types{\cmdkl{\mathcal{T}}\!}
\knowledgenewrobustcmd\typesEps{\cmdkl{\mathcal{T}_{1}}\!}
\knowledgenewrobustcmd\concattype{\cmdkl{\cdot}}
\knowledgenewrobustcmd\type[3][]{#2^{#1}_{\cmdkl{#3}}}

\knowledgenewrobustcmd\disunion{%
	\mathop{\cmdkl{\text{\raisebox{-2pt}{\LARGE$\uplus$}}}}%
}

\knowledgenewrobustcmd\dep[1][]{%
	\mathrel{\cmdkl{\asymp#1}}%
}

\knowledgenewrobustcmd\negrel{\cmdkl{\neg}}

% ---
% Free & Syntactic algebras
% ---
\knowledgenewrobustcmd\Sync[2][2]{\cmdkl{\?S_{#1}#2}}
\knowledgenewrobustcmd\SyncP[2][2]{\cmdkl{\?S^+_{#1}#2}}
\knowledgenewrobustcmd\consol[1]{\cmdkl{#1^{\smash{0}}}}

\knowledgenewrobustcmd\projtype{\cmdkl{\pi_{\smash{\textsf{type}}}}}
\knowledgenewrobustcmd\projval{\cmdkl{\pi_{\smash{\textsf{val}}}}}
\knowledgenewrobustcmd\typemap{\cmdkl{\mathrm{type}}}

\knowledgenewrobustcmd\inducedmor[1]{\cmdkl{\smash{\tilde{#1}}}}
\knowledgenewrobustcmd\inducedalg[1]{\cmdkl{\?A_{#1}}}
% Same but for semigroups
\knowledgenewrobustcmd\inducedmorSg[1]{\cmdkl{\smash{\tilde{#1}}}}
\knowledgenewrobustcmd\inducedalgSg[1]{\cmdkl{\?A_{#1}}}

\knowledgenewrobustcmd\SyntSA[1]{\cmdkl{\?A_{\smash{#1}}}}
\knowledgenewrobustcmd\SyntSAM[2][]{\cmdkl{\eta_{\smash{#2}}^{#1}\!}}
\knowledgenewrobustcmd\SyntSAP[1]{\cmdkl{\?A_{\smash{#1}}}}
\knowledgenewrobustcmd\SyntSAMP[2][]{\cmdkl{\eta_{\smash{#2}}^{#1}\!}}
\newrobustcmd\consolSyntSAP[1]{\kl[\SyntSAP]{\?A_{#1}^{\kl[\consol]{\,\smash{0}}}}}
\newrobustcmd\consolSyntSAMP[1]{\kl[\SyntSAMP]{\eta_{#1}^{\kl[\consol]{\,\smash{0}}}}}
\knowledgenewrobustcmd\sgpSyntSAMP[1]{\cmdkl{\eta_{#1}^{\smash{\textsf{sgp}}}}}

\knowledgenewrobustcmd\quotient[2]{\cmdkl{{#1}/{#2}}}
\knowledgenewrobustcmd\residual[2][]{#2^{\cmdkl{-1}}_{\kl[\type]{#1}}}

\newcommand{\asympbar}{\mathrel{\raisebox{-.1em}{\scalebox{.6}{\ensuremath{\setstackgap{S}{0pt}%
	\stackMath\mathrel{\stackon{\frown}{\stackon{\raisebox{.135em}{\rule{.6em}{.08em}}}{\smile}}}}}}}}

\knowledgenewrobustcmd\congr[1]{%
	\mathrel{\cmdkl{\asympbar_{#1}}}%
}
\newrobustcmd\notcongr[1]{%
	\mathrel{\kl[\congr]{\centernot{\asympbar_{#1}}}}
}

\knowledgenewrobustcmd\compos{%
	\mathbin{\cmdkl{\circ}}%
}

\newrobustcmd{\coslicesurj}{\!\!\text{\rotatebox[origin=c]{60}{$\relbar\!\joinrel\!\twoheadrightarrow$}}\!}
\newrobustcmd{\coslice}{\!/}

% ---
% Varieties
% ---
\knowledgenewrobustcmd\projL[1]{\cmdkl{#1^{\textrm{sync}}}}
\knowledgenewrobustcmd\projA[1]{\cmdkl{#1^{\textrm{sync}}}}

\knowledgenewrobustcmd\corrAR{%
	\mathop{\cmdkl{\to}}%
}
\knowledgenewrobustcmd\corrRA{%
	\mathop{\cmdkl{\to}}%
}
\knowledgenewrobustcmd\corr{%
	\mathop{\cmdkl{\leftrightarrow}}%
}

\knowledgenewrobustcmd\Nil{\cmdkl{\+{(\mkern-1mu c\mkern-1mu o\mkern-2mu )\mkern-1mu F\mkern-2mu i\mkern-1mu n}}}
\knowledgenewrobustcmd\NilS{\cmdkl{\B{N}\textrm{il}}}
\knowledgenewrobustcmd\LocTriv{\cmdkl{\+{L\mkern-1mu I}}}
\knowledgenewrobustcmd\LocTrivS{\cmdkl{\B{LI}}}

% -----
% Profinite words
% -----

\knowledgenewrobustcmd\idem{\cmdkl{\omega}}
\knowledgenewrobustcmd\dist{%
	\mathop{\cmdkl{\textbf{d}}}%
}
\knowledgenewrobustcmd\rist{%
	\mathop{\cmdkl{\textbf{r}}}%
}
\knowledgenewrobustcmd\typing{%
	\mathop{\cmdkl{\textbf{t}}}%
}

\knowledgenewrobustcmd\profSy[1]{\cmdkl{\smash{\widehat{#1}}}}
\knowledgenewrobustcmd\profSg[1]{\cmdkl{\smash{\widehat{#1}}}}

\knowledgenewrobustcmd\profeq{%
	\mathrel{\cmdkl{=}}%
}
\knowledgenewrobustcmd\profimp{%
	\mathrel{\cmdkl{\to}}%
}
\knowledgenewrobustcmd\profequiv{%
	\mathrel{\cmdkl{\leftrightarrow}}%
}
\knowledgenewrobustcmd\profdep{%
	\mathrel{\cmdkl{\asymp}}%
}

\knowledgenewrobustcmd\semprofdep[1]{\cmdkl{\lBrack} #1 \cmdkl{\rBrack}}

% Induced dependencies

\knowledgenewrobustcmd\substitWord[2][\typing]{\cmdkl{#2^{#1}}}
\knowledgenewrobustcmd\substitProf[2][\typing]{\cmdkl{#2^{#1}}}
\knowledgenewrobustcmd\inducedProfdep[1]{\cmdkl{#1^{\smash{\textsf{sync}}}}}

\knowledgenewrobustcmd\EqVar[1]{\cmdkl{\+E_{\smash{#1}}}}

% Free pro-V algebras
\knowledgenewrobustcmd\distvar[1][\B{V}]{%
	\mathop{\cmdkl{\textbf{d}}_{#1}}%
}
\knowledgenewrobustcmd\ristvar[1][\B{V}]{%
	\mathop{\cmdkl{\textbf{r}}_{#1}}%
}
\knowledgenewrobustcmd\simvar[1][\B{V}]{\mathrel{\cmdkl{\asympbar_{#1}}}}
\knowledgenewrobustcmd\freepro[2][\B{V}]{\cmdkl{\smash{\widehat{\mathbf{F}}}_{#1}#2}}
\knowledgenewrobustcmd\freeproproj[1][\B{V}]{\cmdkl{\pi_{#1}}}

\newrobustcmd\Pl[1][\B{V}]{%
	\mathop{%
		\withkl{\kl[\Pl]}{%
			\cmdkl{\mathcal{P\mkern-1mu l}_{\mkern-2mu #1}}
		}%
	}%
}
\knowledge{\Pl}{notion}

\newrobustcmd\Cl[2][\B{V}]{%
	\mathop{%
		\withkl{\kl[\Cl]}{%
			\cmdkl{\mathcal{C\mkern-1mu l}_{\mkern-2mu #1}^{\smash{#2}}}
		}%
	}%
}
\knowledge{\Cl}{notion}


% ---
% Monads 
% ---

\knowledgenewrobustcmd\MonadSync[1][2]{\cmdkl{\B{S}_{#1}}}
\knowledgenewrobustcmd\MonadSyncP[1][2]{\cmdkl{\B{S}^+_{#1}}}

\newrobustcmd\PictureTypedSet[5]{
	\begin{center}
		\begin{tikzpicture}
			\node[draw, circle] (ll) at (0, 0)  {$\LL$};
			\node[draw, circle] (lb) [above right= .7cm and 1.4cm of ll] {$\LB$};
			\node[draw, circle] (bl) [below right= .7cm and 1.4cm of ll] {$\BL$};
			
			\draw (ll) edge[->, loop left] node[left] {$#1$} (ll);
			\draw (ll) edge[->, dashed] node[midway, fill=white, opacity=.8] {$#2$} (lb);
			\draw (lb) edge[->, loop right] node[right] {$#3$} (lb);
			\draw (ll) edge[->, dashed] node[midway, fill=white, opacity=.8] {$#4$} (bl);
			\draw (bl) edge[->, loop right] node[right] {$#5$} (bl);
		\end{tikzpicture}
	\end{center}
}

\newrobustcmd{\sat}{\mathrel{\kl[\sat]{\vDash}}}

% Knowledge files
\input{kl/abbreviations.kl}
\input{kl/complexity.kl}
\input{kl/misc.kl}
\input{kl/structures.kl}
\input{kl/homomorphism.kl}
\input{kl/first-order.kl}
\input{kl/regular-languages.kl}
\input{kl/concrete-graphs.kl}

% Knowledge files for graph databases

% Knowledge files for automatic structures
\input{kl/automatic-struct/duality.kl}
\input{kl/automatic-struct/automatic-structures.kl}
\input{kl/automatic-struct/decision-problems.kl}
\input{kl/automatic-struct/turing-machine.kl}
\input{kl/automatic-struct/relations.kl}
\input{kl/automatic-struct/colouring.kl}
\input{kl/automatic-struct/misc-dichotomy.kl}


\setcounter{secnumdepth}{3}

\begin{document}

\frontmatter
\newgeometry{hmargin=3cm, top=4.5cm, bottom=3cm}
\begin{titlepage}
\begin{center}
  \Huge\scshape{%
  Homomorphism Problems}\\[.4cm]
  \Large\scshape{%
  from Query Minimization in Graph Databases\\[.2cm]to the Frontier of Decidability in Automatic Structures}\\
  % \vspace{5cm}
  % \makebox[0pt]{{%
  %   \fontsize{50}{0}\selectfont\color{black!11}%
  %   $\gamma \equiv^{?} \delta$%
  % }}{}\\
  \vspace{4em}
  \normalfont\LARGE{} Rémi Morvan\\[1em]
  \Large\scshape
  Ph.D. Thesis in Theoretical Computer Science\\
  \textcolor{maincolor}{Université de Bordeaux}
  \vfill
  \begin{center}
    \normalfont\Large\scshape Defended on XX June 2025.\\[1.5em]
    \normalfont
    \begin{tabular}{rll}
      Meghyn Bienvenu & \textsc{\small Université de Bordeaux} & \emph{president}\\[.5em]
      Mikołaj Bojańczyk & \textsc{\small Uniwersytet Warszawski} & \emph{reviewer}\\
      Wim Martens & \textsc{\small Universität Bayreuth} & \emph{reviewer} \\[.5em]
      Antoine Amarilli & \textsc{\small Inria Lille} & \emph{examiner}\\
      % Arnaud Durand & \textsc{\small Université Paris Cité} & \emph{examiner}\\
      Amélie Gheerbrant & \textsc{\small Université Paris Cité} & \emph{examiner}\\
      Martin Grohe & \textsc{\small RWTH Aachen} & \emph{examiner}\\
      % Howard Straubing & \textsc{\small Boston College} & \emph{examiner}\\[.5em]
      Bartek Klin & \textsc{\small University of Oxford} & \emph{examiner}\\
      Nicole Schweikardt & \textsc{\small Humboldt-Universität zu Berlin} & \emph{examiner}\\[.5em]
      Diego Figueira & \textsc{\small CNRS, Bordeaux} & \emph{supervisor}\\
      Nathanaël Fijalkow & \textsc{\small CNRS, Bordeaux} & \emph{co-supervisor}
      % Sophie Tisson
    \end{tabular}
  \end{center}
\end{center}
\end{titlepage}
\restoregeometry
\newpage
\thispagestyle{empty}
~
\newpage
\thispagestyle{empty}
% \setlength{\parskip}{\baselineskip}
\begin{fullwidth}
	\setlength{\parindent}{0pt}
	~\vfill
	\begin{center}
		\normalfont\Large\scshape Composition of the jury:\\[1.5em]
		\normalfont
		\begin{tabular}{r@{\hskip 1em}l@{\hskip 1em}l}
		  Anca Muscholl & \textsc{\small Université de Bordeaux} & \emph{president}\\[.5em]
		  Mikołaj Bojańczyk & \textsc{\small Uniwersytet Warszawski} & \emph{reviewer}\\
		  Wim Martens & \textsc{\small Universität Bayreuth} & \emph{\hphantom{revi}``} \\[.5em]
		  Antoine Amarilli & \textsc{\small Inria, Lille} & \emph{examiner}\\
		  Balder ten Cate & \textsc{\small Universiteit van Amsterdam} & \emph{\hphantom{revi}``}\\
		  Bartek Klin & \textsc{\small University of Oxford} & \emph{\hphantom{revi}``}\\
		  Sophie Tison & \textsc{\small Université de Lille} & \emph{\hphantom{revi}``}\\[.5em]
		  Diego Figueira & \textsc{\small CNRS, Bordeaux} & \emph{supervisor}\\
		  Nathanaël Fijalkow & \textsc{\small CNRS, Bordeaux} & \emph{co-supervisor}
		  % Sophie Tisson
		\end{tabular}
	\end{center}

	\vfill

	\par\textsc{Licensed under "CC BY 4.0".}
	\par\textit{Version of \today.}
\end{fullwidth}
% r.5 contents
\addcontentsline{toc}{chapter}{Contents}
\setcounter{tocdepth}{1}
\tableofcontents
% \listoffigures
% \listoftables

% % r.7 dedication
% \cleardoublepage
% ~\vfill
% \begin{doublespace}
% \noindent\fontsize{18}{22}\selectfont\itshape
% \nohyphenation
% Dedicated to those who appreciate \LaTeX{} 
% and the work of \mbox{Edward R.~Tufte} 
% and \mbox{Donald E.~Knuth}.
% \end{doublespace}
% \vfill
% \vfill

\mainmatter
\chapter*{Abstract}
\addcontentsline{toc}{chapter}{Abstract}

\chapter*{Résumé long}
\addcontentsline{toc}{chapter}{Résumé long}

\chapter*{Ideas/Todos}
\addcontentsline{toc}{chapter}{Ideas/Todos}

\begin{itemize}
	\item Change definition of CQ/CRPQ to have vertices. This allows for better duality,
		and also better statement for the Semantical Structure Theorem (in some prop we get
		``foo is an edge contraction of bla'' instead of ``is a minor of'').
		Update the definition of `one-way internal path' to allow for $n=0$.
\end{itemize}

\chapter{Introduction}

\part{Querying Graph Databases}

\chapter{Query Languages for Graph Databases}

\section{Elementary Notations}

\subsection{Blabla}

% \begin{figure}
% 	\centering
% 	\includegraphics[width=\linewidth]{fig/free_algebras.png}
% 	\caption{My super nice caption.}
% \end{figure}

\begin{table}
	\centering
	\begin{tabular}{cc}
		\toprule
		a & b \\ \midrule
		0 & 1 \\
		1 & 0 \\ \bottomrule
	\end{tabular}
	\caption{My super nice caption.}
\end{table}

\citereset
Hello this is a citation \cite{Bringhurst2005}.
Hello Q\sidenote{Qy first sidenote!} world and another test \cite{Bringhurst2005}.
\[\forall x \in \gamma,\, \exists y\in \delta, \delta \to \gamma\]
\[L \subseteq \Sigma^*\]
\[0 \in \symbb{N}\]
\[\symcal{AbcDefGhiJkl}\]
\[\symscr{AbcDefGhiJkl}\]
\[\symfrak{AbcDefGhiJkl}\]

\lipsum[1-2]

\begin{figure}[htb]
	\centering
	\includegraphics[width=\linewidth]{fig/landscape.png}
	\caption{The landscape of rationality.}
\end{figure}

\lipsum[3-10]

\chapter[Evaluation and Containment of Conjunctive Regular Path Queries]{Evaluation and Containment\\ of Conjunctive Regular Path Queries}

\chapter{Minimization of Conjunctive Regular Path Queries}

\chapter[{Semantic Tree-Width and Path-Width of Conjunctive Regular Path Queries}]{Semantic Tree-Width and Path-Width\\of Conjunctive Regular Path Queries}

% \chapter{Synthesis of Conjunctive Regular Path Queries}

\chapter{Conclusion \& Open Problems}


Extend the notion of `strong minimality` using the axioms used in the lower bound (`str-onto`): call such canonical databases / extensions `structurally minimal'.
Prove or disprove the following result: ``a "CRPQ" has "one-way semantic tree-width" at least $k$ "iff" it has an expansion that is structurally minimal and whose core has tree-width at least $k$''. In general: what about minor-closed classes?
Extend Grohe's Theorem to CRPQs (or maybe to a subclass of CRPQs, "eg" those whose
structurally minimal canonical databases characterize membership to minor-closed classes).

\part[The Frontier of Decidability in Automatic Structures]{The Frontier of Decidability\\in Automatic Structures}

\chapter{Finite-Word Relations and Automatic Structures}
\label{ch:preliminaries-automatic-structures}

\begin{chapterpresentation}
	\begin{abstract}
		TODO.
	\end{abstract}
	\medskip
	\begin{acknowledgements}
		Parts of \Cref{sec:preliminaries-automatic-structures-relations}
		come from \cite[\S~1]{Morvan2024Algebras}.
	\end{acknowledgements}
\end{chapterpresentation}
	
\chaptertoc

\epigraph{We do have a clear understanding of what a cow is, but not of what counts as a non-cow.}{Igor Walukiewicz}
% On the subject of ``asynchronous automata''.

\section{The Landscape of Rationality for Relations over Finite Words}
\label{sec:preliminaries-automatic-structures-relations}

\subsection{Regularity is Key...}

The class of "regular languages" is remarkably stable, and can either be characterized as the 
languages recognized by either:
\begin{itemize}
	\item deterministic or non-deterministic finite state automata,
	see "eg" \cite[Proposition~1.2.3, p.~7]{Pin2021FiniteAutomata},
	\item two-way finite state automata by Shepherdson-Rabin-Scott theorem
		\cite[Theorem~2, p.~198]{Shepherdson1959ReductionTwoWay}
		\cite[Theorem~15, p.~123]{RabinScott1959FiniteAutomata},
	\item rational expressions by Kleene's theorem,
		see "eg" \cite[Theorem~1.5.11, p.~34]{Pin2021FiniteAutomata},
	\item monadic second-order logic by Trakhtenbrot-Büchi-Elgot theorem,
		see "eg" \cite[Theorem~2.2, p.~32]{Bojanczyk2020MSO}, or
	\item finite monoids,
		see "eg" \cite[\S~1.4.2, p.~19]{Pin2021FiniteAutomata}.
\end{itemize}
Moreover, all transformations between these representations are effective---although some
models can be strictly more succinct.

These equivalences explain why the terms \emph{recognizable language}---meaning implicitly
``recognizable by a finite-state automaton'' or ``recognizable by a finite monoid''---and
\emph{rational language}---meaning ``described by a rational expression''---are used 
interchangeably. In fact, in this thesis as well as in most of the literature,
we will use the generic term "regular language".
However, in more complex settings, for instance subsets of non-free monoids,%
\footnote{Recall that a language is nothing else but a subset of a free
(usually finitely-generated) monoid.}
the equivalence between these classes no longer holds. \cite{Pin2021StackExchange}

The landscape of rationality for $k$-ary relations of finite words ($k \geq 2$) is far more complex than for languages,\footnote{Which can be seen as unary relations of finite words.} as depicted in \Cref{fig:landscape-rationality-relations}. We will briefly present these classes,
although this thesis will only deal with the two most restrictive ones, namely
"recognizable@@rel" and "automatic relations".%
\footnote{It should be noted that the names
of these classes were often coined independently of one another---sometimes defying common sense.
For instance, "automatic relations" are named this way because they correspond to
the relations recognized by some model of automata...
not unlike most of the classes in \Cref{fig:landscape-rationality-relations}.
They are also sometimes named ``regular relations'', which have nothing to do
with "regular functions" and do \emph{not} correspond to the intersection of "regular functions"
with "functional relations", as one would reasonably expect.}
We fix two alphabets $\Gamma$ and $\Sigma$. In the rest of this section, we focus
on relations $\+R \subseteq \Gamma^* \times \Sigma^*$, "aka" transductions.
\begin{figure}
	\centering
	\includegraphics[width=\linewidth]{fig/landscape-rationality-relations.png}
	\caption{
		\AP\label{fig:landscape-rationality-relations}
		The ``landscape of rationality'' for binary relations.
		Dashed regions are empty.
	}
\end{figure}
Often, in the literature, when the terminology ``transduction'' is used,
these relations are thought as functions from $\Gamma^*$ to $\pset{\Sigma^*}$---sometimes
assuming that the input is a singleton, sometimes not---,
and in this case $\Gamma$ (resp. $\Sigma$) plays the role of an ``input alphabet''
(resp. ``output alphabet''). On the other hand, when the terminology ``relation'' is used,
we often take $\Gamma = \Sigma$. This makes little difference since we can often
see a relation $\+R \subseteq \Gamma^* \times \Sigma^*$ as a subset of
$(\Gamma\sqcup \Sigma)^* \times (\Gamma\sqcup \Sigma)^*$
while preserving the notion of ``rationality'' at hand.
Note also that most notions can be trivially extended to $k$-ary relations ($k \geq 3$):
we simply work here with binary relations for the sake of simplicity.


\subsection{Recognizable Relations}

A relation $\+R \subseteq \Gamma^* \times \Sigma^*$ is \AP""recognizable@@rel""
if there exists a "finite monoid" $\?M$ together with a "monoid morphism"
\[
	f\colon \Gamma^* \times \Sigma^* \to \?M,
\]
as well as a subset $\Acc \subseteq M$ "st"
$\+R = f^{-1}[\Acc]$. We denote by \AP$\intro*\REC$ the class of "recognizable relations".%
\footnote{todo: Again, insert some set-theoretic bullshit.}

For instance the \AP""same parity relation""
\[
	\SameParity \defeq 
	\{
		\tup{u,v} \in \Gamma^* \times \Sigma^* \mid
		|u| = |v| \mod{2}
	\}
\]
is "recognizable". Indeed, letting $f\colon  \Gamma^* \times \Sigma^* \to \ZnZ{2}$,
then $\SameParity$ can be written as $f^{-1}[\{\bar 0\}]$.

These relations admit a remarkably simple characterization.\AP
\begin{proposition}[{""Mezei's theorem"", see "eg" \cite[Corollary~II.2.20, p.~254]{Sakarovitch2009Elements}}]
	\!\footnote{\cite[\S~2, ``Notes \& references'']{Sakarovitch2009Elements} mentions
	that this proposition is ``unanimously ascribed to G. Mezei (unpublished)''.}
	\label{prop:Mezei-theorem}
	A relation $\+R$ is "recognizable" "iff" there exists $n\in\N$,
	"regular languages" $\tup{K_i}_{i\in \lBrack 1,n\rBrack}$ over $\Gamma$
	and "regular languages" $\tup{L_i}_{i\in \lBrack 1,n\rBrack}$ over $\Sigma$
	"st"
	\[
		\+R = \bigcup_{i=1}^n K_i \times L_i.
	\]
\end{proposition}

In other words, "recognizable relations" are exactly the finite unions of
"Cartesian products" of "regular languages".
For instance, \[\SameParity =
(\Gamma\Gamma)^* \times (\Sigma\Sigma)^*
\cup \Gamma(\Gamma\Gamma)^* \times \Sigma(\Sigma\Sigma)^*.\]
We provide a slightly more general statement of "Mezei's theorem".

\begin{proposition}
	\label{prop:Mezei-theorem-generalization}
	Let $\B{V}$ be a "pseudovariety of finite monoids"
	and $\+V$ be the corresponding "pseudovariety of regular languages".
	Let $\+R \subseteq \Gamma^* \times \Sigma^*$ be a relation.
	The following are equivalent:
	\begin{enumerate}
		\item there exists a "finite monoid" $\?M \in \B{V}$, a "monoid morphism"
		$f\colon \Gamma^* \times \Sigma^* \to \?M$ and $\Acc \subseteq \?M$
		"st" $\+R = f^{-1}[\Acc]$;
		\item there exists $n \in \N$
		and $K_1,\hdots,K_n \in \+V_{\Gamma}$ and $L_1,\hdots,L_n \in \+V_{\Sigma}$
		"st" $\mathcal{R} = \bigcup_{i=1}^n K_i \times L_i$,
	\end{enumerate}
	in which case we say that $\+R$ is \AP""$\+V$-recognizable@@rel"".
\end{proposition}

When $\B{V}$ is the "pseudovariety@@reglang" of all "regular languages",
we get back \Cref{prop:Mezei-theorem}.

\begin{proof}
	\proofcase{From monoids to products.}
	Assume that $\+R$ is "recognizable@@rel". 
	Then by definition
	\[
		\+R = \bigcup_{z \in \Acc} f^{-1}[z].
	\]
	Observe then that $f(u,v) = f(\tup{u,\varepsilon}\cdot\tup{\varepsilon,v})
	= f(u,\varepsilon)\cdot f(\varepsilon,v)$ for all $u,v \in \Gamma^* \times \Sigma^*$, and hence:
	\[
		\+R = \bigcup_{\substack{x,y \in M\\ \text{"st" } x\cdot y \in \Acc}}
		\underbrace{\{u \in \Gamma^* \mid f(u, \varepsilon) = x\}}_{\defeq \;K_x}
		\times \underbrace{\{v \in \Sigma^* \mid f(\varepsilon, v) = y\}}_{\defeq \;L_y}.
	\]
	Since $M$ is finite, the union is finite, and moreover, each $K_x$ and $L_y$ is
	recognized by $M \in \B{V}$, and hence belong to $\+V$.

	\proofcase{From products to monoids.} If $\mathcal{R} = \bigcup_{i=1}^n K_i \times L_i$ 
		where all languages belong to $\+V$, then let $M_i,N_i \in \B{V}$
		be their "syntactic monoids", $g_i,h_i$ be their
		"syntactic morphism", and $\Acc_i,\Bcc_i$ be their accepting sets.
		Consider the monoid morphism
		\begin{center}
		\begin{tabular}{ccc}
			$\Gamma^* \times \Sigma^*$ & $\to$ & $\prod_{i}(M_i \times N_i)$ \\
			$\tup{u,v}$ & $\mapsto$ & $\tup{g_i(u), h_i(v)}_{i}$.
		\end{tabular}
		\end{center}
		Then $\+R$ is the preimage by this morphism of
		\[
			\bigcup_{i=1}^n \bigl(
			\cdots \times (M_{i-1}\times N_{i-1}) \times
			(\Acc_i \times \Bcc_i) \times (M_{i+1}\times N_{i+1}) \times \cdots \bigr).
		\]
		The conclusion follows from the fact that $\B{V}$ is closed under finite products.
\end{proof}

Both the algebraic definition of "recognizable relations" and
"Mezei's theorem" imply---informally!---that all reasonable problems on
"recognizable relations" are decidable. For instance, from \Cref{prop:Mezei-theorem-generalization},
we get that $\+V$ has decidable "membership" "iff" the class of "$\+V$-recognizable relations"
has decidable "membership".

On the other hand, \Cref{prop:Mezei-theorem} proves that "recognizable relations" are not very expressive.
\begin{corollary}
	\!\footnote{Of course, this property is far from being sufficient
	at characterizing "reflexive relations": note that the proof
	does not even use the "regularity@@lang" of the languages at hand...}%
	\footnote{Of course, another way of proving this result would be
	to apply "Raysey's infinite theorem" to $f\colon \Sigma^* \times \Sigma^* \to \?M$.
	Again, we do not use the fact that $f$ is a "monoid morphism", but simply
	that it is a finite-domain function.}
	\AP\label{coro:infinite-clique-recognizable}
	Let $\+R \subseteq \Sigma^* \times \Sigma^*$ be a "reflexive relation".
	Then $\+R$ contains an infinite clique, "ie"
	there exists an infinite language $L \subseteq \Sigma^*$ "st"
	$\tup{u,v} \in \+R$ for all $u,v \in L$.
\end{corollary}
\begin{proof}
	Indeed, by \Cref{prop:Mezei-theorem}, write $\+R$ as $\bigcup_{i=1}^n K_i \times L_i$.
	Given a word $u \in \Sigma^*$, define $f(u) \in \?2^{2n}$ where
	the $2i$-th (resp. $(2i+1)$-th) bit of $f(u)$ indicates if $u \in K_i$ (resp. $u \in L_i$)
	for all $i$.
	By pigeon-hole principle, there exists a bit-sequence in $\?2^{2n}$ whose preimage
	$L$ by $f$ is infinite. Then pick $u,v \in L$. Since $\+R$ is reflexive, then
	$\tup{u,u} \in \+R$ and so, since $f(u) = f(v)$, we
	have $u \in K_i$ "iff" $v \in K_i$ and
	$u \in L_i$ "iff" $v \in L_i$ for all $i$, and so $\tup{u,v} \in \+R$.
\end{proof}

In particular, this corollary implies that neither the \AP""prefix relation""
$\intro*{\prefixRel} \defeq \{
	\tup{u,v} \in \Sigma^*\times\Sigma^* \mid
	\text{$u$ is a prefix of $v$}
\}$, the \AP""suffix relation""
$\intro*{\prefixRel} \defeq \{
	\tup{u,v} \in \Sigma^*\times\Sigma^* \mid
	\text{$u$ is a prefix of $v$}
\}$, nor the equality relation are "recognizable@@rel".
A similar proof can show that the ""equal-length relation""
$\intro*\equalLength \defeq \{
	\tup{u,v} \in \Gamma^*\times\Sigma^* \mid
	|u| = |v|
\}$.%
\footnote{Note however that \Cref{coro:infinite-clique-recognizable} does not apply since
we assumed there that the input and output alphabets are equal.}


\subsection{Automatic Relations}

"Automatic relations" are a strictly larger class of relations,
and trade some decidability properties to gain in expressiveness.
This is precisely what makes this class interesting to us, and why we will focus
on both "automatic relations" and "automatic structures": while being relatively
expressive, some problems remain decidable.

Given a word $u \in \Gamma^*$ and $v\in \Sigma^*$, we define
its \AP""convolution@@word"" $u \intro*\convol v$ to be the word
$i \mapsto \tup{u_i, v_i}$ of length $\max{(|u|,|v|)}$,
with the convention that $u_i = \pad$ (resp. $v_i = \pad$)
if $u_i$ (resp. $v_i$) is undefined, where $\intro*\pad$ is a new letter called
""blank symbol"" or \reintro{padding symbol}. In other words,
$u \convol v$ is obtained by writing $u$ and $v$ on two left-align horizontal tapes,
adding "padding symbols" at the end of the shorter word if their length differ,
and then reading pairs of letters from left to right.
For this reason, the pair $\tup{a,b}$ is instead written $\pair{a}{b}$.
We let \AP$\Gamma \intro*\convolAlpha \Sigma$ denote the alphabet
\[\Gamma \times \Sigma \cup \Gamma \times \{\pad\} \cup \{\pad\} \times \Sigma,\]
and we let $\intro*\SigmaPair \defeq \Sigma \convolAlpha \Sigma$.

By construction, if $u \in \Gamma^*$ and $v\in \Sigma^*$, then $u\convol v \in (\Gamma\convolAlpha\Sigma)^*$.%
\footnote{When dealing with relations of higher arity, note that $\convol$ is associative up to a trivial
alphabet relabelling.}
For instance
\[
	aba \convol baa = \pair{a}{b}\pair{b}{a}\pair{a}{a}
	\quad\text{ and }\quad
	abab \convol cdd = \pair{a}{c}\pair{b}{d}\pair{a}{d}\pair{b}{\pad}.
\]
We denote by \AP$\intro*\convolRel{\+R}$ the language
$\{ u \convol v \mid \tup{u,v} \in \+R\} \subseteq (\Gamma\convolAlpha\Sigma)^*$.

A (finite-state) \AP""synchronous automaton"" $\+A$ over input alphabet $\Gamma$ and output alphabet $\Sigma$
is a finite-state "automaton" over $\Gamma\convolAlpha \Sigma$. We say that
$\+A$ accepts the pair of words $\tup{u,v} \in \Gamma^* \times \Sigma^*$ if
it accepts $u \convol v$ as a ``classical automaton''. Similarly, $\+A$ "recognizes@@sync"
$\+R$ if it recognizes $\convolRel{\+R}$ as a "classical automaton".
A relation is \AP""automatic@@rel"" if it is "recognized@@sync" by a finite-state "synchronous automaton".
We denote by \AP$\intro*\AUT$ the class of "automatic relations".

todo:example

\subsection{Todo}

\begin{itemize}
	\item todo:add polyregular functions.
	\item todo: rename ``synchronous relations'' to ``automatic relations''
	\item todo:the intersection of
	functional relations and two-way rational relations
	collapses to regular functions by
	\cite[Theorem 22, p.~243]{EH2001transduction}.
	\item cite Gaëtan's thesis (chapter 1 for finite words, chapter 8 for infinite words)
	\item mention pebble, marble and whatnot.
\end{itemize}
\section{Automatic Structures}
\label{sec:preliminaries-automatic-structures-automatic-structures}

\subsection{Definitions}

"Automatic structures" are a subclass of all "relational structures".
While some of them can be infinite, they are all finitely describable, by
automata. We will see that, this way, some problems on "finite structures"
remain decidable on this larger class of "automatic structures".

This notion has a remarkably eventful history:
it was introduced by Hodgson in his Ph.D. thesis in 1976
\cite{Hodgson1976PhD}\footnote{Unfortunately the manuscript is not available online.
See \cite{Hodgson1983Decidabilite} for the related journal article.}, but
the notion went largely unnoticed. The notion was rediscovered for groups
in the late 1980s \cite{Epstein1992Word}. 
Independently, Shapiro in 1992 \cite{Shapiro1992AutomaticStructures},
Khoussainov and Nerode in 1995 \cite{KhoussainovNerode1995AutomaticPresentations}
and Pélecq in his Ph.D. thesis from 1997
\cite[\S~3.1.3, p.~91]{Pelecq1997Isomorphismes}
reintroduced the notion of "automatic structures".%
\footnote{What we call "automatic structures"
correspond to ``synchronous automatic structures'' and ``structures automatiques synchrones'' in Shapiro's paper and Pélecq's thesis, respectively.
Pélecq gives credit for the definition to his advisor Géraud Sénizergues.
All three work by Shapiro, Khoussainov \& Nerode and Pélecq
claim to generalize the notion of "automatic groups" from \cite{Epstein1992Word}.}



We fix a finite "relational signature" $\sigma$.
An \AP""automatic presentation $\+A$ of a $\sigma$-structure"" consists of:
\begin{itemize}
	\item an alphabet $\Sigma$,
	\itemAP a "regular language" $\intro*\domainPres{\+A} \subseteq \Sigma^*$,
	\itemAP for every relation $\+R_{(k)} \in \sigma$, an
		"automatic relation" $\intro*\relPres{\+R}{\+A} \subseteq (\Sigma^*)^k$, and
	\itemAP an "automatic relation" $\reintro*\relPres{=}{\+A} \subseteq
		\Sigma^* \times \Sigma^*$ that must be an equivalence relation.%
	\footnote{Some readers already familiar with the notion of "automatic structures"
		might be somewhat surprised by the inclusion of this equality relation:
		we will soon see (\Cref{prop:making-presentations-injective})
		that in this specific setting, we do not need it.
		This is not true in general, see \Cref{footnote:omega-not-injective}.
	}
\end{itemize} 
The \AP\reintro{structure represented} $\?A$ by an "automatic presentation" $\+A$ has
$\domainPres{\+A}/{\relPres{=}{\+A}}$ as its domain, and the predicate $\+R_{(k)} \in \sigma$
is "interpreted@@pred" in such a way that a tuple $\tup{X_1,\dotsc,X_k}$ of equivalence classes
belongs to $\+R(\?A)$ if, and only if, there exists $\tup{u_i}_{1 \leq i \leq k} \in\tup{X_i}_{1 \leq i \leq k}$ "st" $\tup{u_1,\dotsc,u_k} \in \relPres{\+R}{\+A}$.
Given $u \in \domainPres{\+A}$, we denote by $\+A(s)$ the element of $\?A$ it represents,
namely the equivalence class of $u$ under $\relPres{=}{\+A}$.

We say that a "$\sigma$-structure" is \AP""automatic@@struct"" if it is
"represented@@struct" by an "automatic presentation".%
\footnote{Note that we can always assume "wlog" that $\relPres{\+R}{\+A} \subseteq
(\domainPres{\+A})^k$ since $(\domainPres{\+A})^k$ is "automatic@@rel" by
\Cref{prop:rec-implies-aut} and "automatic relations" are closed under intersection.}
For instance, $\langle \univStructSynchronous{\Sigma}, \langle\lastLetter{a}\rangle_{a \in \Sigma},\, \equalLength,\, \prefix \rangle$ is an "automatic structure".

Furthermore, the infinite binary tree can be represented by the "automatic presentation"
$\+B$ with $\domainPres{\+B} = \2^*$,
\[
	\relPres{\+E}{\+B} =
	\{\tup{u,u0} \mid u \in \2^*\} \cup \{\tup{u,u1} \mid u \in \2^*\},
\]
and $\relPres{=}{\+B}$ is equality, see \Cref{fig:infinite-binary-tree}.
\begin{marginfigure}
	\centering
	\begin{tikzpicture}
		\node[vertex] at (0,0) (eps) {};
		\node[vertex, below left=1em and 1.875em of eps] (a) {};
		\node[vertex, below right=1em and 1.875em of eps] (b) {};
		\node[vertex, below left=1em and .75em of a] (aa) {};
		\node[vertex, below right=1em and .75em of a] (ab) {};
		\node[vertex, below left=1em and .75em of b] (ba) {};
		\node[vertex, below right=1em and .75em of b] (bb) {};
		\foreach \n in {aa,ab,ba,bb} {
			\node[below left=1em and .375em of \n] (\n a) {};
			\node[below right=1em and .375em of \n] (\n b) {};
		}

		\draw[edge] (eps) to (a);
		\draw[edge] (eps) to (b);
		\draw[edge] (a) to (aa);
		\draw[edge] (a) to (ab);
		\draw[edge] (b) to (ba);
		\draw[edge] (b) to (bb);
		\foreach \n in {aa,ab,ba,bb} {
			\draw[edge] (\n) to (\n a);
			\draw[edge] (\n) to (\n b);
		}

		\node[right=-.25em of eps, font=\tiny, color=cDarkGrey] {$\varepsilon$};
		\node[right=-.25em of a, font=\tiny, color=cDarkGrey] {$0$};
		\node[right=-.25em of b, font=\tiny, color=cDarkGrey] {$1$};
		\node[right=-.25em of aa, font=\tiny, color=cDarkGrey] {$00$};
		\node[right=-.25em of ab, font=\tiny, color=cDarkGrey] {$01$};
		\node[right=-.25em of ba, font=\tiny, color=cDarkGrey] {$10$};
		\node[right=-.25em of bb, font=\tiny, color=cDarkGrey] {$11$};
	\end{tikzpicture}
	\caption{
		\AP\label{fig:infinite-binary-tree}
		An "automatic presentation" of the infinite binary tree.
	}
\end{marginfigure}
Presentations such as this one, where $\relPres{=}{\+B}$ is the equality relation,
are called ""injective presentations"".

\begin{example}
	\AP\label{ex:presburger}
	The structure $\tup{\N, +}$ is "automatic@@struct".%
	\footnote{We see $+$ as a ternary relation, given by $\{\tup{x,y,x+y} \mid x, y \in \N\}$.}
	We build an "automatic presentation" $\+N$ by using
	a binary encoding $\domainPres{\+N} = \2^*$. A word $w \in \2^*$ will
	represent the number $\sum_{i=0}^{|w|-1} w_i\cdot 2^{i-1}$---notice that 
	we write numbers with their least significant bit first. 
	Naturally, it follows that $\relPres{=}{\+N}$ puts two words $u$ and $v$ in relation
	if they are equal after removing the trailing zeroes.
	We then need to describe $\relPres{+}{\+N}$: the idea is to simulate addition, reading words
	from left to right, by using two states ($0$ and $1$) to remember the carry.
	\begin{marginfigure}
		\centering
		\begin{tikzpicture}[shorten >= 1pt, node distance = 1.8cm, on grid, baseline]
			\node[state, initial left, accepting] (q0) {0}; 
			\node[state, right=of q0] (q1) {1}; 
			\path[->]
				(q0) edge[loop above] node[font=\scriptsize, above left = 0em and -1.1em] {$\triple{0}{0}{0},
					\triple{0}{1}{1},\triple{1}{0}{1}$} (q0)
				(q0) edge[bend left] node[font=\scriptsize, above] {$\triple{1}{1}{0}$} (q1)
				(q1) edge[loop below] node[font=\scriptsize, below right = 0em and -1.1em] {$\triple{0}{1}{0},\triple{1}{0}{0},\triple{1}{1}{1}$} (q1)
				(q1) edge[bend left] node[font=\scriptsize, below] {$\triple{0}{0}{1}$} (q0);
		\end{tikzpicture}
		\caption{
			\AP\label{fig:automaton-presburger}
			"Synchronous automaton" describing the addition of
			natural numbers. For the sake of readability, transitions involving
			the "padding symbol" are not represented: $\pad$ is treated as a zero.
		}
	\end{marginfigure}
	For instance, the transition
	$\textcolor{c0}{0}
	\transition{\tup{\textcolor{c1}{0},\textcolor{c1}{1},\textcolor{c3}{1}}}
	\textcolor{c2}{0}$ can be understood as
	``when adding $\textcolor{c1}{0}$ and $\textcolor{c1}{1}$, with current carry $\textcolor{c0}{0}$, the results equals $\textcolor{c3}{1}$, with a carry of $\textcolor{c2}{0}$ for the next bit''.
	In general, we have a transition
	\[
		\textcolor{c0}{p}
		\transition{\tup{\textcolor{c1}{x},\textcolor{c1}{y},\textcolor{c3}{z}}}
		\textcolor{c2}{q}
		\quad\text{"iff"}\quad
		\textcolor{c1}{x} + \textcolor{c1}{y} + \textcolor{c0}{p}
		= \textcolor{c3}{z} + 2\textcolor{c2}{q}
	\]
	for all $p,q \in \2$ and
	$\tup{x,y,z} \in \2 \convolAlpha \2 \convolAlpha \2$.%
	\footnote{Of course, we identify $\pad$ with $0$.}
	This gives the automaton of figure \Cref{fig:automaton-presburger}.
	By construction, $\+N$ "represents@@struct" $\tup{\N, +}$.
\end{example}

Recall that, using \Cref{prop:automatic-first-order}, a relation is "automatic@@rel" 
if and only if it is "first-order definable" over $\signatureSynchronous{\Sigma}$.
This means that we can alternatively see "automatic presentations" as a
collection of first-order formulas---one for the domain, one for equality, and
one for each predicate. In turn, this view helps us prove the following result.


\begin{proposition}
	\!\footnote{\AP\label{footnote:omega-not-injective}%
	This proposition is not true on the more general class
	of "$\omega$-automatic structures"
	\cite[Theorem~6.6]{HjorthKhoussainovMontalbánNiesAutomaticStructures}.}
	\AP\label{prop:making-presentations-injective}
	Every "automatic structure" admits an "injective presentation".
\end{proposition}

\begin{proof}
	We prove this property using logic. Let $\+A$ be an arbitrary "automatic presentation"
	that "represents@@struct" $\?A$. The idea to build an "injective automatic presentation" $\+A'$
	of $\?A$ is to represent an equivalence class $\equivclass[\relPres{=}{\+A}]{u} \subseteq
	\domainPres{\+A}$ by its "length-lexicographic"-minimal element.%
	\footnote{The \AP""length-lexicographic order"" is defined
	by first comparing the length of the word, and in case they have the same length,
	using the "lexicographic order" to compare them. This order is a well-founded total order.}

	Concretely, $\+A'$ has the same alphabet as $\+A$.
	Then, we define $\domainPres{\+A'}$ as the set of minimal elements
	under the "length-lexicographic order" of
	$\equivclass[\relPres{=}{\+A}]{u}$ ($u \in \domainPres{\+A}$).
	This can be described by a "first-order formula" since $\+A$ is assumed to be "automatic@@pres",
	and since $\lex$ (and hence the "length-lexicographic order") is "automatic@@rel" by \Cref{ex:lexicographic-is-automatic}.
	Equality is interpreted as proper equality,
	and then
	\[
		\relPres{\+R}{\+A'}(x_1,\dotsc, x_k) \defeq
		\exists y_1.\,\dotsc \exists y_k.\,
		\Big( \bigwedge_{i=1}^k x_i \mathrel{\relPres{=}{\+A}} y_i \Big)
		\land \relPres{\+R}{\+A}(y_1,\dotsc, y_k),
	\]
	for any predicate $\+R_{(k)} \in \sigma$.
	Overall, $\+A'$ is an "injective presentation" that "represents@@struct" 
	exactly the same "structure" as $\+A$.
\end{proof}

\begin{proposition}
	\AP\label{prop:making-presentations-binary}
	Any "automatic structures" admits a "presentation" with an alphabet of size 2.
\end{proposition}

\begin{proof}
	The idea is to encode each letter of $\Sigma = \{a_1,\dotsc,a_n\}$
	over $\2$ in such a way that each letter $a_i$
	is encoded over a word $\widehat a_i \in \2^*$ "st"
	all $\widehat a_i$'s have the same length---this is necessary to preserve "automaticity@@struct".
	For instance, take $\widehat a_i$ to be the binary encoding of $i$
	over $k$ bits, for some fixed value of $k \gtrsim \log_2(n)$.
	Then, by \Cref{prop:automatic-closure-length-multiplying-morphism},
	the relations we obtain are still "automatic@@rel".
\end{proof}

Putting \Cref{prop:making-presentations-injective,prop:making-presentations-binary}
together, we get the following result, that says that $\univStructSynchronous{2}$ is ``universal'',
in the sense that it is not only automatic, but it ``contains'' all "automatic structures".
\begin{proposition}
	\AP\label{prop:universal-automatic-structure}
	Let $\?A$ be a $\sigma$-structure. The following are equivalent:
	\begin{enumerate}
		\itemAP\label{item:universal-automatic-structure-auto}
			$\?A$ is "automatic@@struct",
		\itemAP\label{item:universal-automatic-structure-injective-FO}
			$\?A$ is an "injective@@pres" one-dimensional "first-order interpretation" of
			$\univStructSynchronous{2}$,
		\itemAP\label{item:universal-automatic-structure-FO}
			$\?A$ is a "first-order interpretation" of $\univStructSynchronous{2}$.
	\end{enumerate}
\end{proposition}

Recall that $\univStructSynchronous{2}$ is the structure $\2^*$ equipped
with $\lastLetter{0}$, $\lastLetter{1}$, $\equalLength$ and $\prefix$: it can be
seen as the infinite binary tree, with unary relation saying if a node is a
left or right child, and two binary relations saying if two nodes are
at the same depth, and if one is an ancestor of the other.

\begin{proof}
	% \eqref{item:universal-automatic-structure-FO} $\Rightarrow$
	% \eqref{item:universal-automatic-structure-injective-FO} and
	% \eqref{item:universal-automatic-structure-injective-FO} $\Rightarrow$
	% \eqref{item:universal-automatic-structure-auto} are trivial.
	\eqref{item:universal-automatic-structure-auto} $\Rightarrow$
	\eqref{item:universal-automatic-structure-injective-FO} follows from
	\Cref{prop:making-presentations-injective,prop:making-presentations-binary}.
	\eqref{item:universal-automatic-structure-injective-FO} $\Rightarrow$
	\eqref{item:universal-automatic-structure-FO} is trivial.
	To prove \eqref{item:universal-automatic-structure-FO} $\Rightarrow$
	\eqref{item:universal-automatic-structure-auto}, consider a "$d$-dimensional interpretation".
	An element of the domain of this interpretation is a set of $d$-tuples of words of $\2^*$.
	The idea is to encode a $d$-tuple of words of $\2^*$
	as a word over the alphabet
	\[
		\underbrace{\2 \convolAlpha \cdots \convolAlpha \2}_{
			\text{$d$ times}
		}
	\]
	by transforming the tuple $\tup{u^1, \dotsc, u^d}$
	into $u^1 \convol \dotsc \convol u^d$.
	Formulas for equality and relations can be derived easily.
	We obtain an "automatic presentation" over the alphabet
	$\2 \convolAlpha \cdots \convolAlpha \2$.
\end{proof}

In the statement of \Cref{prop:universal-automatic-structure},
$\univStructSynchronous{2}$ can be replaced by any "automatic structure" $\?U$
"st" there is an injective one-dimensional "first-order interpretation" of
$\univStructSynchronous{2}$ in $\?U$. 
Another example of ``universal'' structure consists of the finite subsets of $\N$ equipped
with inclusion and the preorder $\preceq$ defined by $X \preceq Y$ "iff" $X$ and $Y$ are 
singletons, say $\{x\}$ and $\{y\}$, respectively, and $x \leq y$,
see \cite[Theorem~XII.2.3]{Blumensath2024MSOModelTheory}.

\begin{remark}
	In light of \Cref{fig:hierarchy-rational-relations}, "automatic relations"
	and "right-automatic relations" emerge as the two maximal classes of relations
	that admit both closure under Boolean operations and the decidability of
	its ``basic'' decision problems.
	While these two classes are orthogonal, it should be noted that
	``right-automatic structures'' and "automatic structures" are equally expressive:
	this can be shown by renaming any word $u_1\dotsc u_n$ to its reversal $u_n \dotsc u_1$.
	Hence, "automatic structures" are maximal in this sense: substituting ``automatic relations''
	for a notion of rationality of \Cref{sec:preliminaries-automatic-structures-relations}
	either gives a less expressive class ("recognizable relations"),
	an equally expressive class ("right-automatic relations"),
	or a class that is too expressive leading, both to a lack of closure properties
	and undecidability (all other cases).
\end{remark}

However, it should be noted that the notion of "automatic structures" can be generalized
by substituting finite words for more complex models. It leads to the notions of
\begin{itemize}
	\itemAP ""$\omega$-automatic structures"" by taking $\omega$-words,
	\itemAP ""tree-automatic structures"" by taking finite trees,
	\itemAP ""$\omega$-tree-automatic structures"" by taking $\omega$-trees,
\end{itemize}
together with suitable notions of automata for these objects.
We refer the reader to "eg" \cite[\S~XII]{Blumensath2024MSOModelTheory} for more details.

\begin{hypothesis}
	In light of \Cref{prop:making-presentations-injective},
	we will always assume the "automatic presentations"
	to be "injective@@pres", unless specified otherwise.
\end{hypothesis}

\subsection{Model-Checking}

One of the key interest of "automatic structures" is that, while they are infinite,
their model checking problem remains decidable.

\decisionproblem{""First-Order Model Checking of Automatic $\sigma$-Structures""}{
	A "first-order sentence" $\phi$ over $\sigma$ and
	an "automatic presentation" $\•A$ of an "automatic $\sigma$-structure" $\?A$.
}{
	Does $\?A \FOmodels \phi$?
}

The \AP""data complexity"" of this problem refers to the complexity
of the problem for a fixed $\phi$. More precisely,
we say that it is $\+C$-complete for a class $\+C$ when:
\begin{itemize}
	\item for every $\phi$, the problem over fixed $\phi$ is in $\+C$, and
	\item there is at least one $\phi$ for which it is $\+C$-hard.
\end{itemize}

\begin{proposition}[{""Hodgson's theorem""}]
	\!\footnote{Decidability was originally proven in \cite[Théorème~3.5]{Hodgson1983Decidabilite}.
	A slightly weaker form was independently reproved by Pélecq in his thesis
	\cite[Théorème~61, p.~107]{Pelecq1997Isomorphismes}.}
	%
	\AP\label{prop:first-order-model-checking-automatic-structures}
	"First-order model checking of automatic structures" is decidable,
	and in fact is "TOWER"-complete under polynomial-time reductions.
\end{proposition}

\Cref{prop:first-order-interpretation,prop:automatic-first-order}
prove this problem to be polynomial-time equivalent to its restriction
to the structure $\univStructSynchronous{2}$.%
\footnote{However, this does not work
for "data complexity" since, when reducing $\?A \FOmodels^? \phi$ to
$\univStructSynchronous{2} \FOmodels^? \phi^{\•A}$, the formula $\phi^{\•A}$ depends on $\•A$.}

\begin{proof}[Proof sketch]
	% Putting \Cref{prop:first-order-interpretation,prop:automatic-first-order},
	% from any "first-order sentence" $\phi$ over $\sigma$ and any "automatic presentation" $\•A$,
	% we can build a "first-order formula" $\phi^{\•A}$ over
	% $\signatureSynchronous{\2}$ "st"
	% \[
	% 	\univStructSynchronous{2} \FOmodels \phi^{\•A}
	% 	\quad\text{"iff"}\quad
	% 	\?A \FOmodels \phi.
	% \]
	% The size of $\phi^{\•A}$ is polynomial in the size of "first-order formulas"
	% for $\+A$---which are polynomial in the size of the automata for $\+A$ by \Cref{prop:automatic-first-order}---and in the size of $\phi$.
	% Then, we use the other implication of \Cref{prop:automatic-first-order}
	% to build a "synchronous ($0$-ary) automaton" for $\phi$. Checking whether
	% this automaton accepts or rejects is trivial---see the paragraph
	% after \Cref{prop:automatic-first-order}---say in logarithmic space.
	% Note however that the size of this automaton is
	% non-elementary---it is a tower of exponentials in the
	% number of quantifier alternation in $\phi^{\•A}$---and so we obtain a
	% "TOWER" algorithm.

	% For the "data complexity", we now assume that $\phi$ is fixed.
	\proofcase{"Tower" upper bound.}
	The construction is similar to the easy implication of \Cref{prop:automatic-first-order}:
	for any "automatic presentation" $\•A$, we build by induction on $\phi(x_1,\dotsc,x_k)$ a "synchronous ($k$-ary) automaton@synchronous automaton" $\+B^\phi$ "st"
	for all $u_1,\dotsc,u_k \in \Sigma^*$
	\[
		\+B^\phi \text{ accepts } u_1\convol\dotsc\convol u_k
		\quad\text{"iff"}\quad
		\tup{\?A, u_1, \dotsc, u_k} \FOmodels \phi(x_1,\dotsc,x_k).
	\]
	In the end, we get a "synchronous ($0$-ary) automaton@synchronous automaton" $\+B^\phi$ that 
	accepts "iff" $\?A \FOmodels \phi$.
	Notice that each quantifier alternation and negation implies to complement an automaton,
	"ie" to do a powerset construction, and so
	the size of $\+B^\phi$ is a tower of exponentials in the number of quantifier alternations and nested negations in $\phi$. Hence, we get a "Tower" algorithm.

	\proofcase{"Tower" lower bound.}%
	\footnote{We often found this result to be incorrectly credited to various papers in the literature.}
	In 1990, Compton and Henson
	proved that there exists a constant $c>0$ "st" "first-order model-checking"
	restricted to the structure $\univStructSynchronous{2}$ admits $n \mapsto \tower(cn)$ as
	a lower bound on the running time of a non-deterministic "Turing machine" solving the problem
	\cite[Example~8.3]{ComptonHenson1990UniformMethod}.
	In fact, applying \cite[Theorem~6.1.(iv)]{ComptonHenson1990UniformMethod}
	shows the problem to be hard under polynomial-time reductions for the class 
	of problems which can be solved in non-deterministic time at most $\tower(n^c)$ for some $c>0$.
	This corresponds to the class "Tower", and hence the problem is "Tower"-hard.
\end{proof}

\begin{proposition}[Folklore]
	\AP\label{prop:data-complexity-model-checking}
	The "data complexity" of the restriction
	of "first-order model checking of automatic structures"
	to "existential-positive sentences" is "NL"-complete.
\end{proposition}

\begin{proof}
	Let again $\•A$ be an "automatic presentation" and $\phi$ be a "first-order formula",
	but we now assume that $\phi$ is of the form
	\[
		\exists x_1.\,\dotsc \exists x_k.\, \psi(x_1,\dotsc,x_k),
	\]
	where $\psi$ is a "positive quantifier-free formula",
	the size of the automaton for $\psi$ is of the order $|\•A|^{|\psi|}$:
	conjunctions and disjunctions only require products but no powerset construction.
	Now instead of explicitly building an automaton for
	$\exists x_1.\,\dotsc \exists x_k.\, \psi(x_1,\dotsc,x_k)$,
	we test if the automaton for $\psi$ accepts any word.
	The answer is ``yes'' "iff" $\?A \FOmodels \phi$.
	Testing non-emptiness of the automaton amounts to checking if an accepting state can be
	reached from an initial state, which is "NL".
	Since $\phi$ is fixed, the automaton in question is of polynomial size.
	To argue that we can effectively obtain an "NL" algorithm, it suffices to notice that
	we do not have to explicitly build the automaton for $\psi$, but it suffices to
	work with pointers to the automata for $\•A$.
	The number of pointers required only depends on $\psi$.

	Lastly, the "NL" lower bound can be proven by a reduction from NFA non-emptiness, which is
	itself "NL"-hard by reduction from the "finite graph reachability problem".
\end{proof}

Recall that since "NL" is closed under complementation---see "eg" \cite[Corollary~9.23]{Immerman1998DescriptiveComplexity}---, we obtain the same complexity for
negations of "existential-positive sentences".

We refer the reader to \cite[\S~3]{BlumensathGradel2000AutomaticStructures} for the detailed 
complexity of other variations of "Hodgson's theorem". For instance,
the "data complexity" of its restriction to "existential sentences" (here negation is allowed)
is "NP"-complete \cite[Theorem~3.7]{BlumensathGradel2000AutomaticStructures}.

\begin{proposition}[Folklore]
	\AP\label{prop:first-order-reduction-preserve-automaticity}
	The image of an "automatic structure" by a "first-order reduction" is
	still an "automatic structure".
\end{proposition}
\begin{proof}
	This follows "eg" from \Cref{prop:automatic-first-order}.
\end{proof}

We define \AP""FOaut"" to be the class of all problems over "automatic structures"
which are "first-order definable". By \Cref{prop:first-order-reduction-preserve-automaticity},
this class is closed under "first-order reductions". Moreover,
by \Cref{prop:first-order-model-checking-automatic-structures}, we obtain an upper bound.

\begin{corollary}
	\!\footnote{Recall on the other hand that "FOfin" $\subseteq$ "L" (\Cref{prop:FO-in-L}).}
	"FOaut" $\subseteq$ "Tower".
\end{corollary}

The goal of the corollary above is to highlight the difference with \Cref{prop:FO-in-L}:
"first-order definable" problems on "automatic structures", while decidable, are not necessarily
in "L".

\begin{remark}[Presburger arithmetic]
	\AP""Presburger arithmetic"" is the first-order theory (over the signature $\tup{+, 0, 1}$)
	derived from the following axioms:
	\begin{itemize}
		\item $\forall x.\, 0 \neq x + 1$;
		\item $\forall x.\,\forall y.\, x + 1 = y + 1 \Rightarrow x = y$;
		\item $\forall x.\, x + 0 = x$;
		\item $\forall x.\,\forall y.\,\forall z.\, x + (y + z) = (x + y) + z$;
		\item $(\phi(0) \land (\forall x.\; \phi(x) \Rightarrow \phi(x+1))) \Rightarrow
			(\forall x.\; \phi(x))$, where $\phi(x)$ ranges over all "first-order formulas".
	\end{itemize}
	Note that $\tup{\N,+,0,1}$ is a model of this theory, which is moreover complete
	(see \cite{Stansifer1984Presburger}) and so, for a first-order sentence $\phi$, the following 
	are equivalent:
	\begin{itemize}
		\item $\phi$ belongs to the theory, "ie" it is a logical consequence of the axioms above,
		\item $\phi$ holds in all models satisfying the axioms above, and 
		\item $\tup{\N, +, 0, 1} \FOmodels \phi$.
	\end{itemize}
	Since $\tup{\N, +, 0, 1}$ is "first-order equivalent" to $\tup{\N, +}$, which
	is "automatic@@struct" by \Cref{ex:presburger},
	it follows that we can decide "Presburger arithmetic".
\end{remark}

The decidability result of
\Cref{prop:first-order-model-checking-automatic-structures} can be extended
to "$\omega$-tree-automatic structures".
In fact it can even be extended to so-called
\AP""higher-order automatic structures"", see
\cite[last remark of \S~XII.2]{Blumensath2024MSOModelTheory}.
Note that while "automatic structures" are always countable,
"higher-order automatic structures" have at most the cardinality of the continuum.
From this it follows that there exist "structures" with a decidable "first-order model-checking" that are not "automatic@@struct", and not even "higher-order automatic".
It suffices for instance to take any non-standard model of "Presburger arithmetic"
of cardinality strictly bigger than the continuum, which must exist by
"upward Löwenheim–Skolem theorem".%
\footnote{These models form a supclass of the non-standard models of Peano's arithmetic,
see \cite[Ex.~2, p.~36 \& \S~11.4]{Hodges1993ModelTheory}.}
The same argument works to show that ``elementary equivalence''
does not preserve the notion of "automaticity@@struct".

\paragraph*{Order-Invariant First-Order Logic.}
An \AP""order-invariant first-order formula""%
\footnote{This is a standard notion in model theory, see "eg"
\cite[Exercise~3.1.12]{Gradel2007FiniteModelTheory}.}
over $\sigma$
is any "first-order formula"\footnote{For the sake of simplicity we give all definitions
for "sentences@@FO", but they easily be generalized to handle free varirables.} $\phi$
over the signature $\sigma \dcup \{\leq\}$ "st" for any
"$\sigma$-structure" $\?A$, for any \emph{total orders} $\leq_1$, $\leq_2$ over $A$,
we have $\tup{\?A, \leq_1} \FOmodels \phi$ "iff" $\tup{\?A, \leq_2} \FOmodels \phi$.
In this case, we say that $\?A$ models $\phi$.
Similarly, we define \AP""$\omega$-order-invariant first-order formulas""
\cite[\S~3.2]{Rubin2008AutomataPresentingStructures}
by restricting the total orders to have order-type $\omega$---"ie" to be
isomorphic to $\tup{\N, \leq}$---or $|A|$ when $|A|$ is finite.

For instance, the formula
\[
	\forall x.\; \exists y.\; x \leq y \land x \neq y
\]
is "$\omega$-order-invariant@@FO": it expresses the fact that the model is infinite.
However, it is not "order-invariant@@FO".%
\footnote{Indeed, $\tup{\N, \leq}$ satisfies the formula but not $\tup{\N, \geq}$.}
\emph{A priori}, it is non-trivial to check if an "order-invariant first-order formula"
(resp. "$\omega$-order-invariant first-order formula") holds in an
"automatic $\sigma$-structure": how should this order be "interpreted@@pred"?
By "order-invariance@@FO", any total order will do, and moreover by
\Cref{ex:lexicographic-is-automatic}, the "length-lexicographic order" is always "automatic@@rel",
and has the required order-type!

\begin{proposition}[{\cite[\S~3.2]{Rubin2008AutomataPresentingStructures}}]
	\label{prop:order-invariance}
	Model-checking of "order-invariant first-order formulas"
	(resp. "$\omega$-order-invariant first-order formulas") over "automatic structures"
	is decidable.
\end{proposition}

A restricted form of this result was originally proved by Blumensath and Grädel
\cite[Corollary~5.4]{BlumensathGradel2004FinitePresentations} for the extension of
"first-order logic" with the quantifier ``there are infinitely many''.
Similarly, one can add counting quantifiers of the form ``the number of $x$'s "st" $\phi(x)$ holds
is congruent to $k$ mod $n$'' \cite[\S~3.2]{Rubin2008AutomataPresentingStructures}
while preserving decidability.

Note however that it is undecidable whether a "first-order formula" is
"order-invariant@@FO" or "$\omega$-order-invariant@@FO", be it on "finite structures"
or on all "structures". This follows from the undecidability of "first-order logic",
see "eg" \cite[Exercise~3.1.12]{Gradel2007FiniteModelTheory}.%
\footnote{Hence, \Cref{prop:order-invariance} should be understood as a promise problem:
we can decide the problem if we are given the promise that the input is
"order-invariant@@FO" (resp. "$\omega$-order-invariant@@FO").}

\paragraph*{Interpretations.}
One way of rephrasing---and proving---"Hodgson's theorem" would be to first observe
that $\univStructSynchronous{2}$ has decidability first-order theory,
and then than this property is preserved under "first-order interpretations".
The result would follow since all "automatic structures" are "first-order interpretations"
of $\univStructSynchronous{2}$ by \Cref{prop:universal-automatic-structure}.
Colcombet and Löding proposed in \cite{ColcombetLoding2007WMSO} an alternative vision
by showing that:
\begin{itemize}
	\item $\tup{\N, \textrm{succ}}$ has a decidable weak monadic second-order logic,
		where $\textrm{succ}$ denotes the successor relation,%
		\footnote{This is \emph{the} classical result of automata theory and the motivation
		for studying $\omega$-automata. Recall that weak monadic second-order 
		logic consists of "monadic second-order logic" where monadic quantifiers are
		restricted to finite sets.}
	\item if a "structure" $\?U$ has a decidable weak monadic second-order theory,
		then any "structure" $\?A$ that admits a
		``weak monadic second-order interpretation''%
		\footnote{This is the equivalent
		of "FO-interpretation" for weak monadic second-order logic. In particular,
		finite subsets elements of the domain of $\?U$ are interpreted as elements of
		the new structure. See \cite[\S~2.3]{ColcombetLoding2007WMSO} for a formal definition.}
		from $\?U$ has a decidable first-order theory
		\cite[Corollary~2.5]{ColcombetLoding2007WMSO};
	\item "automatic structures" are exactly the "structures" that can be obtained from
		$\tup{\N, \textrm{succ}}$ using WMSO-interpretations
		\cite[Proposition~3.1]{ColcombetLoding2007WMSO}.
\end{itemize}


\subsection{Problems on Automatic Structures}

From the notion of "automaticity@@struct" two questions naturally arise:
\begin{enumerate}
	\item What are the \emph{structural properties} of "automatic structures"?
		For instance, we have seen that all "automatic structures" are countable.
		This question has been somewhat extensively studied for algebraic structures.
	\item Given a \emph{decidable decision problem} on finite structures, is its generalization
		to "automatic structures" still decidable?
\end{enumerate}

\decisionproblem{""Isomorphism Problem for Automatic Structures""}{
	Two "automatic presentations" $\•A$ and $\•B$.
}{
	Is $\?A$ "isomorphic@@struct" to $\?B$?	
}
Blumensath and Grädel proved this problem
to be undecidable \cite[Theorem~5.15]{BlumensathGradel2004FinitePresentations}.
The problem was later shown to be complete for the first level
of the "analytical hierarchy" \cite[Theorem~5.9]{KhoussainovNiesRubinStephan2007Automatic}.

\paragraph*{Automatic Ordinals.}
"Automatic ordinals" are quite simple: using "eg" Cantor's normal form, it is straightforward
to prove that any ordinal strictly smaller than $\omega^\omega$ is "automatic@@struct".%
\footnote{We see an ordinal as a structure with a binary relation describing its order.}
Delhommé proved the converse property to be true:
an ordinal is "automatic@@struct" "iff" it is strictly smaller than $\omega^\omega$
\cite[Corollaire~2.2]{Delhommé2004AutomaticitéOrdinaux}.
Using again Cantor's normal form,
Khoussainov, Rubin \& Stephan proved that the isomorphism problem is decidable for
"automatic ordinals" \cite[Theorem~5.3]{KhoussainovRubinStephan2005AutomaticLinearOrders}.
These results have been generalized first to "$\omega$-tree-automatic ordinals"
by Finkel and Todor\v{c}ević \cite{FinkelTodorcevic2013AutomaticOrdinals}
and then to "tree-automatic ordinals" which
are moreover equipped with addition, by Jain, Khoussainov, Schlicht and Stephan
\cite{JainKhoussainovSchlichtStephan2019IsomorphismTreeAutomaticOrdinals}.

\paragraph*{Automatic Boolean Algebras.}
Recall that a Boolean algebra can be either seen as a partially ordered set
following a set of axioms ensuring among other the existence of a join, a meet
and a negation, or a set equipped these three operations, together with a minimal and maximal 
element. Both definitions are actually "first-order equivalent" and so considering one
or the other does not change the notion of "automaticity@@struct".

The "isomorphism problem" for "automatic Boolean algebras" is decidable
\cite[Corollary~3.5]{KhoussainovNiesRubinStephan2007Automatic} essentially
because there are very few "automatic Boolean algebras" \cite[Theorem~3.4]{KhoussainovNiesRubinStephan2007Automatic}.
However, when going to "$\omega$-tree-automaticity", not only is the "isomorphism problem"
undecidable, but Finkel and Todor\v{c}ević exhibited two somewhat simple-looking
"$\omega$-tree-automatic Boolean algebras" for which whether they were isomorphic
is independent from "ZFC" \cite[Theorem~6.1]{FinkelTodorcevic2010Isomorphism}.

\paragraph*{Automatic Groups.}
The literature on "automatic groups" is remarkably extensive.
Their study was introduced in the late 1980s,
by showing that they have a solvable word problem \cite[Theorem~2.1.9]{Epstein1992Word}.%
\footnote{Actually this last result is proved on the slightly bigger class of ``regularly generated groups''.}
Typical examples of infinite "automatic group" are the braid group
\cite[Theorem~9.3.1]{Epstein1992Word} and the finitely-generated free groups.
We refer the reader to \cite{Rees2022AutomaticGroups} for a recent and detailed account on
the history of the development of the theory of "automatic groups".
Remarkably, despite being a very active research area, the decidability of
the "isomorphism problem" remains open.

\begin{openproblem}
	Is the "isomorphism problem" decidable for "automatic groups"?
\end{openproblem}

\paragraph*{Automatic Semigroups.}
Following the success of "automatic groups",
"automatic semigroups" have been studied by Campbell,
Robertson, Ru\v{s}kuc and Thomas, who showed
that any finitely generated subsemigroup of a free semigroup
is "automatic@@struct"
\cite[Theorem~8.1]{CampbellRobertsonRuskucThomas2001AutomaticSemigroups}.

\paragraph*{Automatic Rings and Fields.}
Richer algebraic structures, like rings or fields actually often fail to be captured by
the notion of "automaticity@@struct". For instance, $\tup{\N,{+},{\cdot},0,1}$ (Peano's arithmetic)
is not "automatic@@struct".
This essentially follows from \Cref{prop:bound-automatic-structures}, applied to multiplication,
together with a counting argument, see \cite[Corollary~XII.8.11]{Blumensath2024MSOModelTheory}.
Khoussainov, Nies, Rubin \& Stephan moreover proved that no infinite field\footnote{In fact their result also holds for integral domains.} can be "automatic@@struct"
\cite[Theorem~3.10 \& Corollary~3.11]{JainKhoussainovSchlichtStephan2019IsomorphismTreeAutomaticOrdinals}.%
\footnote{Note that this contrasts with Tarski's result that the first-order theory
of $\tup{\R,+,\cdot,0,1}$, and more generally of any real closed fields is 
decidable since it admits quantifier elimination \cite[Theorem~8.4.4]{Hodges1993ModelTheory}.}


\subsection{Automatic Graphs}
\Cref{ch:dichotomy-theorem} will mostly focus on "automatic graphs", in which we will study
the question of colourability and the homomorphism problem.
One interesting source of undecidability for "automatic graphs" comes from the following construction.

\begin{example}
	\AP\label{ex:turing-machine-are-automatic}
	Consider a Turing machine $\+T = \tup{Q,\Gamma,\delta,q_0,\Acc}$, where $Q$ is the set of states, $\Gamma$ is tape alphabet,
	\[
		\delta\colon (Q \setminus \Acc) \times \Gamma_{\smallpad} \to \pset{Q \times \Gamma \times \set{\leftarrow, \downarrow, \rightarrow}}
	\]
	is the transition relation, $\Gamma_{\smallpad} \defeq \Gamma \dcup \set{\pad}$, and $q_0$ and $\Acc$ are the initial and set of final states, respectively.
	%
	We represent a "configuration@@TM" $\tup{u, q, v}$ by the word $uqv \in \Gamma^* Q \Gamma^*$:
	in light of this, we will henceforth denote by ``configuration'' any string from the set  \AP$\intro*\configs \defeq  \Gamma^* Q \Gamma^*$.\footnote{We will often write
	$uqv$ as the concatenation $u\cdot q \cdot v$ to emphasize
	the separation between all three words.}
	The \AP""configuration graph"" of $\+T$ is the infinite graph $\intro*\confGraph$ having $\configs$ as set of vertices and an edge from $\gamma$ to $\gamma'$, denoted $\gamma \rightarrow \gamma'$ if there is a one-step transition from $\gamma$ to $\gamma'$ in $\+T$. The "configuration graph" $\confGraph$ of any Turing machine $\+T$ is an effective "automatic graph".
\end{example}

As a consequence, by reduction from the halting problem of a universal Turing machine, 
we obtain the following result.
\begin{proposition}
	There exists a fixed "automatic graph" $\?G$ over the alphabet $\2$ "st"
	the problem of whether, given two words $u$ and $v \in \2^*$, there is a path from
	$u$ to $v$ in $\?G$ is undecidable.
\end{proposition}

Problems on automatic graphs have been mainly studied by Kuske and Lohrey, who studied the following problems over "automatic graphs".
\begin{itemize}
	\item \emph{Highly undecidable problems:} the existence of a Hamiltonian path
		is undecidable, in fact it is complete for the first level of the "\emph{analytical} hierarchy"
		\cite[Theorem~3.2]{KuskeLohrey2010AutomaticGraphs};
		the existence of an infinite path in "directed trees" shares the same complexity
		\cite[Theorem~3.6]{KuskeLohrey2010AutomaticGraphs};
		and so does the "isomorphism problem" \cite[Theorem~5.9]{KhoussainovNiesRubinStephan2007Automatic}.
	\item \emph{Moderately undecidable problems:}
		the existence of an Eulerian path is undecidable, but is only complete for
		the second level of the "\emph{arithmetical} hierarchy" "Pi0-2" \cite[Theorem~4.13]{KuskeLohrey2010AutomaticGraphs}.
\end{itemize}

However, as a consequence of \Cref{prop:order-invariance}, the 
fact that the graph contains infinitely many edges---or isolated vertices, or any "first-order
definable" property---, is decidable over "automatic graphs" 
as it can be expressed by an "$\omega$-order-invariant@@FO" property.

Some slightly more involved arguments, involving ``Ramsey quantifiers'',
can show that whether the graph contains an infinite "clique" (or "transitive tournament")
is decidable \cite[Corollary~5.5]{KuskeLohrey2010AutomaticGraphs}, 
see also \cite[Theorem~3.20]{Rubin2008AutomataPresentingStructures}.

\begin{proposition}[{\cite[Proposition~6.5]{Kocher2014AutomatischenGraphen}}]
	Whether an "automatic graph" is 2-colourable, or equivalently bipartite, is undecidable.
	More precisely it is "coRE"-complete.
\end{proposition}

For another survey on "automatic structures", see \cite{Gradel2020AutomaticStructures}.
Lastly, we want to note that \emph{recursive structures} have been extensively studied
since the late XXth century: they are defined analogously to "automatic structures",
but "synchronous automata" are replaced by Turing machines.
Unsurprisingly, all non-trivial problems are undecidable: hence, from a computability perspective,
the main question is to characterize the Turing degree of these problems.
The extent of the literature on this topic is too vast to be summarized here:
we refer the reader to the two-volume handbook \cite{ErshovGoncharovNerodeRemmelMarek1998Handbook1,ErshovGoncharovNerodeRemmelMarek1998Handbook2}.
\section{A Logical Excursion}
\label{sec:preliminaries-automatic-structures-logic}

\subsection{First-Order Interpretation}

A major construction in logic to restrict the scope of the class
of models---thereafter called ``universe''.
Formally, given a universe $\+U$ and a class of "queries@@sem" $\+Q$,%
\footnote{Recall that we see "queries@@sem" here in a purely semantical way: it is nothing but a subclass of $\+U$.}
given a subclass $\+V$ of $\+U$, we can consider the restriction
\[
	\restr{\+Q}{\+V} \defeq \{\phi \cap \+V \mid \phi \in \+Q\}.
\]
For instance, letting $\+U$ be the class of all "$\sigma$-structures" and $\+Q$ be the set of all 
"first-order sentences" over $\sigma$, by taking $\+V$ to be the set of all \emph{finite}
"$\sigma$-structures", then $\restr{\+Q}{\+V}$ precisely corresponds to
the "first-order sentences" over "finite structures". For instance, the class of all
\emph{finite} "$\sigma$-structures" belongs to $\restr{\+Q}{\+V}$ but not to
$\+Q$ (todo:addref).

However, this construction does not preserve any reasonable property on $\+Q$,
since $\+U'$ is an arbitrary subclass of $\+U$. The idea behind "first-order interpretation"
is precisely to circumvent this problem, by \emph{interpreting} a class of structures
inside another one by using "first-order logic", allowing this way to preserve some properties.

Let $\sigma$ and $\tau$ be "relational signatures".
A \AP ""$d$-dimensional first-order interpretation"" $\+I$ consists of the following tuple of
"first-order formulas" over $\sigma$:
\begin{itemize}
	\itemAP $\intro*\domainInter{\+I}(x^1,\hdots, x^d)$ (specifying the domain)
	\itemAP $\intro*\relInter{=}{\+I}(\tup{x^1,\hdots, x^d}, \tup{y^1,\hdots, y^d})$ (specifying equality)
	\itemAP $\reintro*\relInter{\+R}{\+I}(\tup{x^1_1,\hdots,x^d_1}, \hdots, \tup{x^1_k,\hdots,x^d_k})$
		for any predicate $\+R \in \tau$ of arity $k$,
\end{itemize}
such that $\relInter{=}{\+I}$ defines an equivalence relation on the domain of $\+I$.

Given a "$\sigma$-structure" $\?A$, the \AP""$\+I$-interpretation of $\?A$"" is the "$\tau$-structure" denoted by \AP$\intro*\interpretation{\+I}{\?A}$ defined as follows:%
\footnote{For the same of readability, we use $\bar a$ or $\bar x$ to
denote $d$-tuples of elements of $\?A$ and $d$-tuples of variables.}
\begin{itemize}
	\item its domain is the quotient of 
		$\{\bar a \in \?A^d \mid \?A, \bar a \models \domainInter{\+I}(\bar x)\}$
		by the equivalence relation
		$\set{\tup{\bar a, \bar b} \mid \?A, \bar a, \bar b \models \relInter{=}{\+I}(\bar x, \bar y)}$,
	\item for any predicate $\+R_{(k)} \in \tau$,
		we have
		\[
			\+R_{(k)}(\interpretation{\+I}{\?A}) =
			\Big\{\tup{\bar x_1, \hdots, \bar x_k} \mid
				\exists \bar y_1.\, \hdots.\, \bar y_k.\,
				\bigwedge_{i=1}^k \relInter{=}{\+I}(\bar x_i, \bar y_i)
				\land \relInter{\+R}{\+I}(\bar y_1, \hdots, \bar y_k)
			\Big\}.
		\]
\end{itemize}
For instance, letting $\sigma$ and $\tau$ be the "graph signature", consider
the "one-dimensional interpretation" $\+I$ where:
\begin{itemize}
	\item the domain formula keeps all vertices,
	\item the equality formula is proper equality,
	\item the formula for the edge predicate puts two variables $x$ and $y$ in relation
		if either there is an edge from $x$ to $y$, or if $x$ has no successor and $y$ has no predecessor.
\end{itemize}
Then, the $\+I$-interpretation of a directed path is a cycle,
see \Cref{fig:interpretation-path-into-cycle}.
\begin{marginfigure}
	\centering
	\begin{tikzpicture}
		\node[vertex, draw=c2, fill=c2, fill opacity=.4] at (0,0) (2) {};
		\node[vertex, left=of 2, draw=c1, fill=c1, fill opacity=.4] (1) {};
		\node[vertex, left=of 1, draw=c0, fill=c0, fill opacity=.4] (0) {};

		\draw[edge] (0) to (1);
		\draw[edge] (1) to (2);
	\end{tikzpicture}
	\qquad
	\begin{tikzpicture}
		\node[vertex, draw=c2, fill=c2, fill opacity=.4] at (0,0) (2) {};
		\node[vertex, left=of 2, draw=c1, fill=c1, fill opacity=.4] (1) {};
		\node[vertex, left=of 1, draw=c0, fill=c0, fill opacity=.4] (0) {};

		\draw[edge] (0) to (1);
		\draw[edge] (1) to (2);
		\draw[edge, bend right=60] (2) to (0);
	\end{tikzpicture}
	\caption{
		\AP\label{fig:interpretation-path-into-cycle}
		A directed path (left) and its "interpretation" by $\+I$ (right),
		that adds an edge from any vertex with no successor to any vertex with no predecessor.
	}
\end{marginfigure}

The idea behind "first-order interpretation" is that
"first-order formulas" about the interpretation $\interpretation{\+I}{\?A}$
can be translated into "first-order formulas" over $\?A$.
Formally, given a "first-order formula" $\phi(x_1,\hdots,x_k)$ over $\tau$,
we can define a "first-order formula" $\phi^{\+I}(\bar x_1,\hdots,\bar x_k)$ over $\sigma$
defined inductively by:
\begin{itemize}
	\item replacing each variable $x$ by a $d$-tuple $\bar x$,
	\item replacing every occurrence of
		$\+R(x_1,\hdots, x_k)$ (with $\+R_{(k)} \in \tau$)
		by the formula $\relInter{\+R}{\+I}(\bar x_1, \hdots, \bar x_k)$,
	\item relativizing quantification with respect to $\domainInter{\+I}$,
		meaning that
		\begin{align*}
			(\exists x.\, \phi(x))^{\+I} 
			& \defeq \exists \bar x.\, \domainInter{\+I}(\bar x) \land \phi^{\+I}(\bar x),
			\text{ and} \\ 
			(\forall x.\, \phi(x))^{\+I} 
			& \defeq \forall \bar x.\, \domainInter{\+I}(\bar x) \Rightarrow \phi^{\+I}(\bar x).
		\end{align*}
\end{itemize} 
\begin{proposition}
	\AP\label{prop:first-order-interpretation}
	For any "first-order formula" $\phi(x_1,\hdots,x_k)$ over $\tau$,
	for any "pointed $\sigma$-structure" $\tup{\?A, \bar a_1,\hdots, \bar a_k}$, we have:
	\[
		\tup{\interpretation{\+I}{\?A}, \equivclass{\bar a_1}, \hdots, \equivclass{\bar a_k}}
		\models \phi(x_1, \hdots, x_k)
		\quad\text{"iff"}\quad
		\tup{\?A, \bar a_1,\hdots, \bar a_k} \models \phi^{\+I}(\bar x_1,\hdots,\bar x_k).
	\]
\end{proposition}
And so, in particular, if $\+C$ is a class of structures, then
model checking (resp. satisfiability, resp. validity) over the class $\interpretation{\+I}{\+C}$
can be reduced to satisfiability (resp. satisfiability, resp. validity) over $\+C$.

\begin{marginfigure}
	\centering
	\begin{tikzpicture}
		\tikzset{every node/.style={fill opacity=.4}}

		\node[vertex] at (0,0) (0-0) {};
		\foreach \j in {1,2} {
			\pgfmathtruncatemacro{\prev}{\j - 1}
			\pgfmathtruncatemacro{\jc}{mod(\j,2) ? 2 : 1}
			\node[vertex, right=2em of 0-\prev] (0-\j) {};
		}
		\foreach \i in {1,2} {
			\pgfmathtruncatemacro{\prev}{\i - 1}
			\foreach \j in {0,1,2} {
				\pgfmathtruncatemacro{\jc}{mod(\i+\j,2) ? 2 : 1}
				\node[vertex, above=2em of \prev-\j] (\i-\j) {};
			}
		}

		\foreach \i in {0,1,2} {
			\foreach \j in {0,1} {
				\draw[edge] (\i-\j) to (\i-\the\numexpr\j+1\relax);
			}
		}
		\foreach \i in {0,1} {
			\foreach \j in {0,1,2} {
				\draw[edge] (\i-\j) to (\the\numexpr\i+1\relax-\j);
			}
		}
	\end{tikzpicture}
	\caption{
		\AP\label{fig:interpretation-path-into-grid}
		"Interpretation" of the directed path of
		\Cref{fig:interpretation-path-into-cycle} by the
		``block product interpretation''.
	}
\end{marginfigure}
\begin{example}
	The ``block product'' is a typical example of a 2-dimensional interpretation.
	Let $\sigma$ and $\tau$ be the "graph signature".
	We define a "two-dimensional interpretation" $\+I$ with trivial domain and equality%
	\footnote{Meaning formally that \[\domainInter{\+I}(x^1,x^2) \defeq \top,\text{ and}\]
	and $\relInter{=}{\+I}(\tup{x^1,x^2}, \tup{y^1,y^2})$ is
	the formula $x^1 = y^1 \land x^2 = y^2$.}
	and we put an edge from $\tup{x^1,x^2}$ to $\tup{y^1,y^2}$ if either
	there is an edge from $x^1$ to $y^1$ and $x^2 = y^2$, or if
	$x^1 = y^1$ and there is an edge from $x^2$ to $y^2$.
	Then the interpretation of a directed path is a directed grid,
	see \Cref{fig:interpretation-path-into-grid}.
\end{example}

\subsection{First-Order Reduction and First-Order Model Checking}

For the classical notion of \AP""first-order reductions"", see
\cite[Definition 2.11 \& Definition 1.26]{Immerman1998DescriptiveComplexity}.
Note that the complexity of a decision problem is traditionally measured
as a function of the size of the binary encoding of its input.
However, here, the notion of "first-order reductions" deals with
decision problems defined on "relational structures":
\begin{itemize}
	\item any ``classical'' decision problem, "ie" language $L \subseteq \2^*$
		can be seen as a class of "structure" over the "signature of binary strings";
	\item for any "relational signature" $\sigma$,
		there is an encoding of "finite $\sigma$-structures" as "finite structures"
		over the "signature of binary strings", "ie" as a ``classical''
		decision problem---see "eg" \cite[\S~2.2]{Immerman1998DescriptiveComplexity}.
\end{itemize}
Importantly, this last encoding is a "first-order reduction", proving that these reductions
make sense not only from a logical perspective, but also from a complexity-theoretic point of
view.

\begin{proposition}[Folklore]
	\label{prop:FO-in-L}
	"FOfin" $\subseteq$ "L".
\end{proposition}

\begin{proof}[Proof sketch]
	The naive recursive algorithm, recursing over the "first-order formula",
	works in logarithmic space: it suffices to keep one pointer to the "structure"
	for every variable of the "formula@@FO", but since this formula is fixed, we only
	require a constant number of pointers.
\end{proof}

Note that usually, "L"-hardness is defined using "first-order reductions".


\subsection{First-Order Interpretations}

todo.


\subsection{todo:move this around.}

What we said above work with the implicit assumptions that all structures at hand are 
"finite@@struct". However, part of this can be generalized to "automatic structures"---however, this comes to the cost of losing the inclusion in "L".

\begin{proposition}[Folklore]
	\AP\label{prop:first-order-reduction-preserve-automaticity}
	The image of an "automatic structure" by a "first-order reduction" is
	still an "automatic structure".
\end{proposition}
\begin{proof}
	TODO
\end{proof}

\decisionproblem{""First-Order Model Checking of Automatic Structures""}{
	A "first-order formula" $\phi(x_1,\hdots,x_k)$ over $\sigma$,
	an "automatic presentation" $\•A$ of an "automatic $\sigma$-structure" $\?A$,
	and words $u_1,\hdots,u_k \in \domainPres{\+A}$.
}{
	Does $\?A, u_1,\hdots, u_k \FOmodels \phi$?
}

\begin{proposition}
	\!\footnote{See "eg" \cite[Theorem XII.1.7]{Blumensath2024MSOModelTheory}.}
	%
	\label{prop:first-order-model-checking-automatic-structures}
	"First-order model checking of automatic structures" is decidable in todo.
	Its "data complexity" is "NL"-complete.
\end{proposition}

\begin{proof}
	TODO.

	For the "NL" lower bound, reduction from NFA non-emptiness.
\end{proof}

We define \AP""FOaut"" to be the class of all problems over "automatic structures"
which are "first-order definable". By \Cref{prop:first-order-reduction-preserve-automaticity},
this class is closed under "first-order reductions". Moreover,
by \Cref{prop:first-order-model-checking-automatic-structures}, we obtain an upper bound.

\begin{proposition}
	"FOaut" $\subseteq$ "NL".
\end{proposition}


\subsection{A Model-Theoretic Perspective}

Given an alphabet $\Sigma$, we define on $\Sigma^*$:
\begin{itemize}
	\itemAP a predicate $\intro*\lastLetter{a}$ indicating that the last letter of a word is $a$,
	\itemAP a binary relation $\reintro*\equalLength$ indicating that two words have the same length,
	\itemAP a binary relation $\intro*\prefix$ indicating that a word is a prefix of another.
\end{itemize} 
We denote by $\intro*\signatureSynchronous{\Sigma}$ the "signature" $\langle \langle\lastLetter{a}\rangle_{a \in \Sigma},\, \equalLength,\, \prefix \rangle$%
\footnote{For the sake of simplicity, we abusively use the same notations for
the "predicates" and their "interpretations@@predicate" in the "signature".} and
by \AP$\intro*\univStructSynchronous{\Sigma}$ the "$\signatureSynchronous{\Sigma}$-structure" over $\Sigma^*$ where
the "predicates" are "interpreted@@predicate" as above.

\begin{proposition}
	\AP\label{prop:automatic-first-order}
	If $\Sigma$ has at least two letters, then a relation over $\Sigma^*$
	is "automatic@@rel" "iff" it is "first-order definable" in
	$\univStructSynchronous{\Sigma}$.\footnote{When $\Sigma$ has a single letter,
	then the right-to-left implication holds, but not the converse one.
	For instance, it can be shown that $(aa)^*$ is not "first-order definable"
	in $\univStructSynchronous{\{a\}}$.}
\end{proposition}

todo:proof

\begin{example}
	\AP\label{ex:lexicographic-is-automatic}
	Given an ordered alphabet $\Sigma = \{a_1,\hdots,a_n\}$ with $a_1 < \hdots < a_n$,
	the \AP""lexicographic order"" $\intro*\lex$ defined by $u \lex v$ when there exists 
	a prefix $w$ both of $u$ and of $v$, "st" either $w$ has the same length as $u$,
	or the letter following $w$ in $u$ is strictly smaller than the letter following $w$ in $v$.
	
	Note that this can be defined by a "first-order formula" over $\signatureSynchronous{\Sigma}$
	since saying that the letter following $w$ in $u$ is $a$ amounts to
	guessing the smallest prefix $w'$ of $u$ that must strictly contain $w$ as a prefix,
	and checking that $w'$ ends with letter $a$.
	From \Cref{prop:automatic-first-order}, it follows that the "lexicographic order"
	is an "automatic relation".
\end{example}

\subsection{Logical Characterization of Other Classes of Relations}

todo.
\chapter{A Dichotomy Theorem for Automatic Structures}

\chapter{The Algebras for Automatic Relations and Beyond:\\the Relative Membership Problem for Pseudovarieties of Regular Languages}

\chapter{Conclusion \fancyand~Open Problems}

\backmatter
\begin{fullwidth}
	\chapter*{Bibliography}
	\addcontentsline{toc}{chapter}{Bibliography}
	\setlength{\columnsep}{8mm}
	\begin{multicols}{2}
		\printbibliography[heading=none]
	\end{multicols}
\end{fullwidth}

\end{document}

